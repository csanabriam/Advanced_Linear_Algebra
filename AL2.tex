\documentclass{book}

% \let\latexput\put
% \usepackage{pictex}
% \let\put\latexput

\usepackage{etex}

\usepackage{amsfonts}
\usepackage{amsthm}
\usepackage{amsmath}
\usepackage{amssymb} 
\usepackage{latexsym}

\usepackage[latin1]{inputenc}
\usepackage[spanish,es-noquoting]{babel}
\usepackage{varioref}

% \usepackage{color}
% \usepackage{psfrag}

\usepackage{fancyhdr}
\usepackage{makeidx}

\usepackage{tikz}
\usetikzlibrary{positioning, arrows, matrix, decorations.pathreplacing}

\usepackage[retainorgcmds]{IEEEtrantools}


\pagestyle{fancy}
\renewcommand{\chaptermark}[1]{\markboth{\chaptername\ \thechapter.\ #1}{}}
\renewcommand{\sectionmark}[1]{\markright{#1}{}}
\fancyhf{}
\fancyhead[LE,RO]{\bfseries \thepage}
\fancyhead[LO]{\bfseries \rightmark}
\fancyhead[RE]{\bfseries \leftmark}
\renewcommand{\headrulewidth}{0.5pt}
\renewcommand{\footrulewidth}{0pt}
\addtolength{\headheight}{1pt}
\fancypagestyle{plain}{\fancyhead{}\renewcommand{\headrulewidth}{0pt}}

\title{{\Huge Un curso de \'Algebra Lineal II}\\
\vspace{2cm}
Camilo Sanabria\\
\vspace{1cm}
Universidad de los Andes\\
Departamento de Matem\'aticas\\
Bogot\'a - Colombia.
}
\frenchspacing
\makeindex

\date{}

\DeclareMathOperator{\Sp}{Sp}
\DeclareMathOperator{\Hom}{Hom}
\DeclareMathOperator{\End}{End}
\DeclareMathOperator{\GL}{GL}
\DeclareMathOperator{\im}{im}
\DeclareMathOperator{\id}{id}
\DeclareMathOperator{\tr}{tr}
\DeclareMathOperator{\ord}{ord}
\DeclareMathOperator{\rea}{Re}
\DeclareMathOperator{\ima}{Im}
\DeclareMathOperator{\chara}{char}
\DeclareMathOperator{\sgn}{sgn}
\DeclareMathOperator{\Alt}{Alt}



\begin{document}
\frontmatter
\maketitle
\tableofcontents
\listoffigures 

\mainmatter

\theoremstyle{plain} \newtheorem{teo}{Teorema}[chapter]
\theoremstyle{plain} \newtheorem{pro}[teo]{Propiedad}
\theoremstyle{plain} \newtheorem{prop}[teo]{Proposici\'on}
\theoremstyle{plain} \newtheorem{coro}[teo]{Corolario}
\theoremstyle{plain} \newtheorem{lema}[teo]{Lema}
\theoremstyle{definition} \newtheorem{defn}[teo]{Definici\'on}
\theoremstyle{definition} \newtheorem{obs}[teo]{Observaci\'on}
\theoremstyle{definition} \newtheorem{nota}[teo]{Notaci\'on}
\theoremstyle{definition} \newtheorem{ejem}[teo]{Ejemplo}

\newcommand{\dem}{\noindent\emph{Dem}. }


\chapter{Espacios vectoriales y transformaciones lineales}

\section{Espacios vectoriales}

Sea $\mathbb{K}$ un cuerpo (ver la definición \ref{defcuerpo}). A los elementos de $\mathbb{K}$ los llamaremos escalares.

\begin{defn}\label{defespvec}
Un \emph{espacio vectorial sobre $\mathbb{K}$} es un conjunto $V$ junto con dos operaciones binarias
\[
\begin{array}{rclcrcl}
+:V\times V & \longrightarrow & V &\qquad& \cdot: \mathbb{K}\times V & \longrightarrow & V \\
(v,w) & \longmapsto & v+w &\qquad& (c,v) & \longmapsto & cv, 
\end{array}
\]
que llamamos respectivamente \emph{suma} y \emph{producto por escalar} (o \emph{adici\'on} y \emph{multiplicaci\'on por escalar}), y que contiene un elemento $O\in V$, que llamamos \emph{el origen}, los cuales satisfacen las siguientes propiedades.
\begin{enumerate}[(i)]
\item \emph{La estructura $(V,+,O)$ es un grupo abeliano}: para todo $u,v,w\in V$ se tiene
\[ v+w=w+v,\qquad u+(v+w)=(u+v)+w,\qquad v+O=v,\]
y para todo $v\in V$ existe $-v\in V$ que satisface la igualdad $-v+v=O$.
\item \emph{El producto por escalar es unitario y asociativo}: para todo $a,b\in \mathbb{K}$ y todo $v\in V$ se tiene
\[ 1v=v,\qquad a(bv)=(ab)v.\]
\item \emph{El producto por escalar se distribuye sobre la suma}: para todo $a,b\in \mathbb{K}$ y todo $v,w\in V$ se tiene
\[ a(v+w)=av+aw,\qquad (a+b)v=av+bv.\] 
\end{enumerate}
Un \emph{vector} es un elemento de un espacio vectorial.
\end{defn}

\begin{ejem}\label{ejem0}
Los siguientes espacios vectoriales sobre $\mathbb{K}$ son ejemplos imprescindibles.
\begin{enumerate}
\item \emph{Espacio cero-dimensional}: El conjunto $\{O\}$ junto con las \'unicas operaciones posibles.
\item \emph{Espacio uni-dimensional}: El conjunto $\mathbb{K}$ junto con las operaciones del cuerpo y $O=0$.
\item \emph{Espacio $n$-dimensional de coordenadas}: El conjunto $\mathbb{K}^n$ formado por el producto cartesiano de $n$ copias del cuerpo $K$, junto con las operaciones
\begin{eqnarray*}
(a_1,\ldots,a_n)+(b_1,\ldots,b_n) & = & (a_1+b_1,\ldots,a_n+b_n)\\
c(a_1,\ldots,a_n) & = & (ca_1,\ldots,ca_n)
\end{eqnarray*}
y $O=(0,\ldots,0)$.
\item \emph{Espacio de funciones con valor en $\mathbb{K}$}: Dado un conjunto $X$, el conjunto $\mathbb{K}^X$ de funciones $X\rightarrow \mathbb{K}$, junto con las operaciones
\begin{eqnarray*}
  f+g: x & \mapsto & f(x)+g(x)\\
  cf: x & \mapsto & cf(x)
\end{eqnarray*}
y el origen es la función $O: x\mapsto 0$.
\item \emph{Espacio de funciones con valor en $\mathbb{K}$, con casi todos los valores iguales a $0$}: Dado un conjunto $X$, el conjunto $\mathbb{K}^S_0$ de funciones $X\rightarrow \mathbb{K}$ para las que todos los valores, salvo para un n\'umero finito de elementos en $X$, son $0$, junto con las operaciones y el origen definidos para $\mathbb{K}^X$.
\item \emph{Espacio de polinomios con coeficientes en $\mathbb{K}$}: El conjunto $\mathbb{K}[t]$ de polinomios en la variable $t$ con coeficientes en $\mathbb{K}$ junto con las operaciones de suma y producto por escalar usuales y con el origen $O$ dado por el polinomio constante $0$ (ver la definición \ref{defpoly}).
\end{enumerate} 
\end{ejem}

\begin{pro}
Sea $V$ un espacio vectorial sobre $\mathbb{K}$
\begin{enumerate}[(i)]
  \item \emph{Ley de cancelaci\'on}: Dados $u,v,w\in V$, la igualdad $u+v=w+v$ implica $u=w$.
  \item \emph{Unicidad del origen}: Si $o\in V$ es tal que $v+o=v$ para algún $v\in V$ entonces $v=0$.
  \item \emph{Unicidad del opuesto}: Dado $v\in V$, si $w\in V$ es tal que $v+w=O$, entonces $w=-v$. 
\end{enumerate}
\end{pro}

\dem
\begin{enumerate}[(i)]
  \item A partir de la igualdad $u+v=w+v$, si sumamos $-v$ a ambos lados obtenemos $u=w$.
  \item Se sigue de la ley cancelativa aplicada a $v+o=v=v+O$.
  \item Se sigue de la ley cancelativa aplicada a $v+w=O=v+(-v)$.
\end{enumerate}

\begin{pro}
Sea $V$ es un espacio vectorial sobre $\mathbb{K}$.
\begin{enumerate}
\item Para todo $c\in \mathbb{K}$ y $v\in V$ tenemos $cO=0v=O$.
\item Para todo $v\in V$ tenemos $(-1)v=-v$.
\item Si tenemos $cv=O$, entonces $c=0$ \'o $v=O$.
\end{enumerate}
\end{pro}

\dem
\begin{enumerate}
\item Tenemos $cO+O=cO=c(O+O)=cO+cO$, luego, por la ley de cancelaci\'on, $cO=O$. Igualmente, tenemos $0v+O=0v=(0+0)v=0v+0v$, luego $0v=O$.
\item Por la unicidad del opuesto, basta ver que $v+(-1)v=O$, lo cual se sigue de las igualdades $v+(-1)v=1v+(-1)v=\left(1+(-1)\right)v=0v$.
\item Suponga que $cv=O$ con $c\ne 0$, entonces tenemos $O=c^{-1}O=c^{-1}(cv)=(c^{-1}c)v=1v=v$.
\end{enumerate}
\qed

\begin{defn}
Sea $V$ un espacio vectorial sobre $\mathbb{K}$ y $U\subseteq V$. Decimos que $U$ es un \emph{subespacio} de $V$ si $U$, con las operaciones heredadas de $V$, es un espacio vectorial.
\end{defn}

\begin{nota}
Usaremos los s\'imbolos $\le$, $<$, $\ge$ y $>$ para representar respectivamente \emph{subespacio de}, \emph{subespacio propio de}, \emph{superespacio de} y \emph{superespacio propio de}.
\end{nota}

\begin{pro}\label{subespsiysolosi}
Sea $V$ un espacio vectorial sobre $\mathbb{K}$. Si $U$ es un subconjunto de $V$, entonces $U\le V$ si y solo si $U$ satisface las siguientes propiedades.
\begin{enumerate}
\item $O\in U$.
\item \emph{El conjunto $U$ es cerrado bajo suma}: para todo $v,w\in U$, se tiene $v+w\in U$.
\item \emph{El conjunto $U$ es cerrado bajo multiplicaci\'on por escalar}: para todo $c\in \mathbb{K}$ y $v\in U$, se tiene $cv\in U$.
\end{enumerate} 
\end{pro}

\dem Suponga primero $U\le V$. Luego $U$ contiene un neutro respecto a la suma y, como la operaci\'on de suma es la de $V$, este es $O$. As\'i tenemos $O\in U$. Por otro lado, como $U$ es un espacio vectorial con las operaciones de $V$, la suma de dos elementos en $U$ est\'a en $U$ y producto por escalar de un elemento en $U$ tambi\'en est\'a en $U$.\\
Rec\'iprocamente, si $U$ contiene al origen y es cerrado bajo suma, la restricci\'on de la suma a $U\times U$ da una operaci\'on $+:U\times U\rightarrow U$ que cumple con el axioma (i) de la definici\'on \ref{defespvec}. Similarmente sucede con la restricci\'on del producto por escalar a $\mathbb{K}\times U$ y el axioma (ii). Finalmente, el axioma (iii) de distributividad se hereda de $V$.\qed

\begin{ejem}
Las soluciones a sistemas lineales homogéneos son subespacios. De hecho, tome $a_{i1},\ldots,a_{in}\in \mathbb{K}$, para $i\in\{1,\ldots,m\}$, con $m\le n$. El conjunto $U$ de soluciones $(x_1,\ldots,x_n)\in\mathbb{K}^n$ al sistema
$$\left\{
\begin{array}{lcr}
  a_{i1}x_1+\ldots+a_{in}x_n & = & 0\\
   & \vdots & \\
  a_{m1}x_1+\ldots+a_{mn}x_n & = & 0\\
\end{array}
\right.$$
contiene al origen, es cerrado mediante suma y producto por escalar, y es así un subespacio.
\end{ejem}

\begin{defn}
Sea $V$ un espacio vectorial sobre $\mathbb{K}$ y sean $v_1,\ldots,v_n\in V$. Una \emph{combinaci\'on lineal} de  $v_1,\ldots,v_n$ es un vector $v\in V$ tal que
\[
v=c_1v_1+\ldots+c_nv_n
\]
con $c_1,\ldots,c_n\in\mathbb{K}$. A los elementos $c_1,\ldots,c_n$ los llamamos los \emph{coeficientes} de la combinaci\'on lineal.
\end{defn}

\begin{prop}
Sea $V$ un espacio vectorial sobre $\mathbb{K}$. Dado un subconjunto no vac\'io $C$ de $V$, el conjunto formado por todas las combinaciones lineales de elementos en $C$ es un subespacio de $V$.
\end{prop}

\dem Sea $U$ el conjunto $\left\{\sum_{i=1}^nc_iv_i\Big|\ c_1,\ldots,c_n\in \mathbb{K},\ v_1,\ldots,v_n\in C \right\}$. Como $C$ no es vacío, para cualquier $c\in C$ se tiene $0c=O$ y así $U$ contiene al origen.
Para todo $v,w\in U$, existen $a_1,\ldots,a_n,b_1,\ldots,b_m\in \mathbb{K}$ y $v_1,\ldots,v_n,w_1,\ldots,w_m\in S$ tales que $v=\sum_{i=1}^n a_iv_i$ y $w=\sum_{j=1}^m b_jw_j$, y así
\[
v+w=a_1v_1+\ldots+a_nv_n+b_1w_1+\ldots+b_mw_m.
\]
Luego, $v+w$ es una combinaci\'on lineal de elementos de $C$ y $v+w$ pertenece a $U$. Similarmente, dado $c\in \mathbb{K}$, tenemos
\[
cv=ca_1v_1+\ldots+ca_nv_n,
\]
y as\'i $cv$ est\'a en $U$. Por lo tanto, $U$ es cerrado bajo suma y multiplicaci\'on por escalar, luego, por la propiedad \ref{subespsiysolosi}, $U$ es subespacio de $V$.\qed

\begin{defn}
Sea $V$ un espacio vectorial sobre $\mathbb{K}$ y sea $C$ un subconjunto de $V$ no vac\'io. Al subespacio formado por todas las combinaciones lineales de elementos de $C$ lo llamamos el \emph{espacio generado por $C$} y lo denotamos por $\langle C \rangle$. Por convenci\'on, definimos $\langle \emptyset\rangle=\{O\}$.
\end{defn}

\begin{prop}\label{propunion}
Sea $V$ un espacio vectorial sobre $\mathbb{K}$.
\begin{enumerate}
\item Si $\{U_i\}_{i\in I}$ es una familia de subespacios de $V$, entonces la intersecci\'on $\bigcap_{i \in I} U_i$ es un subespacio de $V$.
\item Si $C$ es un subconjunto de $V$, entonces $\langle C\rangle$ es el m\'inimo subespacio vectorial en $V$ que contiene a $C$. Es decir, si $W$ es un subespacio de $V$ que contiene a C, entonces $\langle C\rangle$ es un subespacio de $W$.
\end{enumerate}
\end{prop}

\dem
\begin{enumerate}
\item Sea $U$ el conjunto $\bigcap_{i\in I} U_i$. Para todo $i\in I$, el origen $O$ está en $U_i$ y así $O\in U$. Sea $v,w\in U$ y $a\in \mathbb{K}$. Para todo $i \in I$ tenemos $v,w\in U_i$, luego $v+w$ y $cv$ pertenecen a todo $U_i$ y as\'i tambi\'en est\'an en $U$.
\item Sea  $W$ un subespacio de $V$ que contiene a $C$. Como $W$ es cerrado bajo suma y bajo multiplicaci\'on por escalar, dados $v_1,\ldots,v_n\in C$ y $c_1,\ldots,c_n\in \mathbb{K}$, la combinaci\'on lineal $\sum_i^{n} c_iv_i$ est\'a en $W$. Es decir que toda combinaci\'on lineal de elementos de $C$ est\'a en $W$. Luego, tenemos $\langle C\rangle \subseteq W$. Pero como $\langle C\rangle$ es un espacio vectorial, obtenemos $\langle C\rangle \le W$.\qed
\end{enumerate}

\section{Base y dimensi\'on}

Sea $V$ un espacio vectorial sobre un cuerpo $K$.

\begin{nota}
Sea $I$ un conjunto de \'indices. Sea $S$ el conjunto $\{v_i\}_{i\in I}$, donde $v_i$ es un elemento de $V$ para $i\in I$. Las combinaciones lineales de elementos de $S$ las denotaremos por
$\sum_{i\in I} c_iv_i$, donde $c_i$ es un elemento en $K$ para todo $i\in I$, bajo la convenci\'on de que $c_i$ es $0$ para todo indice $i\in I$ salvo para una subcolecci\'on finita. Note que estas sumas son finitas pues los coeficientes que no son $0$ son finitos.
\end{nota}

\begin{defn}
Sea $\mathcal{B}$ el subconjunto $\{v_i\}_{i\in I}$ de $V$. Decimos que $\mathcal{B}$ es una \emph{base} de $V$ si para todo $v\in V$ existe una \'unica combinaci\'on lineal $\sum_{i\in I} c_iv_i$, donde $c_i$ es un elemento en $K$ para todo $i\in I$, que cumple $v=\sum_{i\in I} c_iv_i$. Dado $i\in I$, al coeficiente $c_i$ lo llamamos la \emph{coordenada $i$ de $v$ en la base $\mathcal{B}$}. Por convenci\'on, la base del espacio cero-dimensional es  $\emptyset$. 
\end{defn}

\begin{lema}\label{lemabas}
Sea $\mathcal{B}$ la base $\{v_i\}_{i\in I}$ de $V$. Suponga que $\mathcal{B}$ no es vac\'io. Si $\{c_i\}_{i\in I}$ es una colecci\'on de elementos en $K$ para la cual se tiene $O=\sum_{i\in I}c_iv_i$, entonces $c_i$ es igual a $0$ para todo $i\in I$.
\end{lema}

\dem La combinaci\'on lineal $\sum_{i\in I}0v_i$ es igual a $O$ y como es la \'unica combinaci\'on lineal de elementos en $\mathcal{B}$  igual al origen, se sigue el lema.\qed 


\begin{defn}
Decimos que $V$ tiene dimensi\'on finita si tiene una base finita. De lo contrario decimos que $V$ tiene dimensi\'on infinita.
\end{defn}

\begin{teo}[Teorema de la dimensi\'on] \label{basedim}
Si $V$ tiene dimensi\'on finita, el n\'umero de elementos de la base es independiente de la base escogida.
\end{teo}

\dem La afirmaci\'on para el caso cero-dimensional se sigue del hecho de que su \'unica base es de cero elementos. Ahora suponga por contradicci\'on que $V$ tiene dos bases $\{v_1,\ldots, v_n\}$ y $\{w_1,\ldots, w_m\}$ y que $n$ es diferente de $m$. Sin perdida de generalidad, podemos asumir $m>n$.\\
Para $j\in\{1,\ldots,m\}$ tenemos $$w_j=\sum_{i=1}^n a_{ij}v_i,$$ con $a_{ij}\in K$, para $i\in\{1,\ldots,n\}$. As\'i, si $v\in V$ es igual a $x_1w_1+\ldots+x_mw_m$, donde $x_1,\ldots,x_m$ son elemento en $K$, tendr\'iamos
$$v = \sum_{j=1}^m x_j\left(\sum_{i=1}^n a_{ij}v_i\right) = \sum_{i=1}^n\left(\sum_{j=1}^m a_{ij}x_j\right)v_i.$$
Por el lema anterior, se cumple $v=O$ si y solo si
\[
\sum_{j=1}^m a_{ij}x_j=0,
\]
para $i\in\{1,\ldots,n\}$. Estas $n$ igualdades forman un sistema homogeneo subdeterminado, esto es con m\'as variables que ecuaciones. En particular, este sistema tiene soluciones no triviales. Si $(x_1,\ldots,x_m)$ es una de ellas, entonces obtenemos el origen como una combinaci\'on lineal de los elementos de la base $\{w_1,\ldots, w_m\}$ con coeficientes no todos iguales a cero, lo cual contradice el lema. Por lo tanto tenemos $m=n$. \qed

\begin{defn}
Suponga que $V$ tiene dimensi\'on finita. Al n\'umero de elementos en una base de $V$ lo llamamos dimensi\'on de $V$ y lo denotamos por $\dim (V)$. Si $V$ tiene dimensi\'on infinita, escribimos $\dim(V)=\infty$, y usamos la convenci\'on $n<\infty$ para todo entero $n$.
\end{defn}

\begin{defn}
Sea $S$ un subconjunto de $V$. Decimos que $S$ es \emph{linealmente dependiente} si alg\'un elemento de $S$ es combinaci\'on lineal de los otros, es decir si existen $v_0,v_1,\ldots,v_n\in S$, tales que $v_0$ es distinto de $v_i$ para $i\in\{1,\ldots,n\}$, que satisfacen
\[
v_0=c_1v_1+\ldots+c_nv_n
\]
donde $c_1,\ldots,c_n$ son elementos en $K$. Si $S$ no es linealmente dependiente, decimos que $S$ es \emph{linealmente independiente}.
\end{defn}

\begin{prop}\label{proplinind}
Si $S$ es un subconjunto de $V$, entonces $S$ es linealmente independiente si y solo si
\[
O=c_1v_1+\ldots+c_nv_n,
\]
donde $v_1,\ldots,v_n$ son elementos en $S$ y $c_1,\ldots,c_n$ en $K$, implica $c_1=\ldots=c_n=0$. 
\end{prop}

\dem Establezcamos la contrapositiva: $S$ es linealmente dependiente si y solo si existen $v_1,\ldots,v_n\in S$ y $c_1,\ldots,c_n\in K$, con $c_i\ne 0$ para alg\'un $i\in\{1,\ldots, n\}$, tales que
\[
O=c_1v_1+\ldots+c_nv_n.
\]
La contrapositiva es inmediata pues la \'ultima igualdad es equivalente a
\[
v_i=\sum_{j=1,j\ne i}^n (-c_j/c_i)v_j,
\]
donde $c_i$ es invertible en $K$, es decir $c_i$ es distinto de $0$. \qed

\begin{prop}\label{defbase2}
Si $\mathcal{B}$ es un subconjunto no vac\'io de $V$, entonces $\mathcal{B}$ es una base de $V$ si y solo si satisface las siguientes dos propiedades.
\begin{enumerate}
\item El conjunto $\mathcal{B}$ es linealmente independiente.
\item El espacio $V$ es generado por $\mathcal{B}$.
\end{enumerate}
\end{prop}

\dem Es suficiente demostrar que si tenemos $\langle \mathcal{B}\rangle=V$ entonces $\mathcal{B}$ es una base si y solo si $\mathcal{B}$ es linealmente independiente. Con esto en mente, suponga primero que $\mathcal{B}$ es base, luego por el Lema \ref{lemabas}, $\mathcal{B}$ es linealmente independiente. Reciprocamente, suponga que $\mathcal{B}$ es linealmente independiente. Asuma por contradicci\'on que $\mathcal{B}$ no es base, luego existen dos combinaciones lineales con coeficientes distintos ambas iguales a alg\'un $v\in V$. La diferencia de estas dos combinaciones lineales dar\'ia una combinaci\'on lineal igual a $0$ con coeficientes no todos nulos, lo cual contradice la propiedad \ref{proplinind}.\qed

\begin{ejem}\label{ejembas0}
Por la propiedad \ref{defbase2}, se puede verificar que, para cada uno de los siguientes espacios $V$, el respectivo conjunto $\mathcal{B}$ es una base.
\begin{enumerate}
\item Sea $V=K^n$. Para $i\in\{1,\ldots, n\}$, defina $e_i\in V$ como el elemento con ceros en todas las entradas salvo en la $i$-\'esima donde tiene $1$. Sea $\mathcal{B}=\{e_1,\ldots,e_n\}$.
\item  Sea $I$ un conjunto finito y sea $V=K^I$. Para $i\in I$, defina la funci\'on $\delta_i: I\rightarrow K$ por $\delta_i(i)=1$ y $\delta_i(j)=0$ para $j\ne i$. Sea $\mathcal{B}=\{\delta_i\}_{i\in I}$. 
\item Sea $I$ un conjunto y sea $V=\left(K^I\right)_0$ (ver el ejemplo \ref{ejem0}.5). Para $i\in I$, defina la funci\'on $\delta_i: I\rightarrow K$ por $\delta_i(i)=1$ y $\delta_i(j)=0$ para $j\ne i$. Sea $\mathcal{B}=\{\delta_i\}_{i\in I}$. 
\item Para $V=K[t]$, sea $\mathcal{B}=\{1,t,t^2,\ldots\}=\{t^n\}_{n\ge 0}$ .
\end{enumerate}
\end{ejem}

\begin{defn}
Para $K^n$, la \emph{base can\'onica} es la base $\{e_1,\ldots,e_n\}$ definida en el ejemplo \ref{ejembas0}.1, la cual denotaremos por $\mathcal{C}$.
\end{defn}

\begin{lema}\label{inddep}
Sea $S$ un subconjunto de $V$ linealmente independiente. Si $v\in V$ es tal que $S\cup\{v\}$ es linealmente dependiente, entonces $v$ pertenece a $\langle S\rangle$.
\end{lema}

\dem Como $S\cup\{v\}$ es linealmente dependiente, por la proposici\'on \ref{proplinind} existen $a_1,\ldots,a_n,a_{n+1}\in K$, no todos iguales a cero, y $v_1,\ldots,v_n\in S$ para los cuales se tiene
\[
O=a_1v_1+\ldots+a_nv_n+a_{n+1}v.
\]
Note que $a_{n+1}$ es diferente de $0$, o de lo contrario tendr\'iamos una combinaci\'on lineal de $v_1,\ldots,v_n$ igual a $O$ con no todos los coeficientes iguales a cero, contradiciendo la independiencia lineal de $S$. Obtenemos
\[
v=\sum_{i=1}^n(-a_i/a_{n+1})v_i\in\langle v_1,\ldots,v_n\rangle,
\]
y por lo tanto $v$ pertenece a $\langle v_1,\ldots,v_n\rangle$, el cual es un subconjunto de $\langle S\rangle$. \qed

\begin{prop}\label{maximallinind}
Suponga que $V$ tiene dimensi\'on finita distinta de cero y sea $S$ un subconjunto finito de $V$. Si $S_0$ es un subconjunto linealmente independiente de $S$, entonces existe un subconjunto $S'$ de $S$ linealmente independiente maximal que contiene a $S_0$, es decir que si $S''$ es un subconjunto de $S$ linealmente independiente que contiene a $S'$, entonces $S'=S''$. M\'as a\'un $\langle S'\rangle$ es igual a $\langle S\rangle$.  
\end{prop}

\dem Sean $v_1,\ldots,v_n$ los elementos de $S$, enumerados de forma tal que $\{v_1,\ldots,v_m\}$ sea $S_0$ y $\{v_1,\ldots,v_m,v_{m+1},\ldots,v_n\}$ sea $S$. Definimos $m=0$ cuando $S_0=\emptyset$. Tenemos $m\le n$. Iterativamente, para $i\in\{1,\ldots,n-m\}$, definimos
$$
S_i = \left\{ \begin{array}{rl}
S_{i-1}&\textrm{ si } v_{m+i}\in\langle S_{i-1}\rangle\\
S_{i-1}\cup\{v_{m+i}\} &\textrm{ si } v_{m+i}\not\in\langle S_{i-1}\rangle
\end{array}\right. .
$$
Por el lema anterior, la independencia lineal de $S_{i-1}$ implica la de $S_i$. Tome $S'=S_{n-m}$. El conjunto $S'$ es linealmente independiente. Por construcci\'on, tenemos $S_0\subseteq S_1\subseteq\ldots\subseteq S_{n-m}= S'$. Veamos que $\langle S'\rangle$ es igual a $\langle S\rangle$. Si $j\in\{1,\ldots,n\}$ es tal que $v_j$ no pertenece a $S'$, entonces $v_j$ est\'a en $\langle S_{j-1}\rangle$ , el cual es un subconjunto de $\langle S'\rangle$, y as\'i $S'\cup\{v_j\}$ es linealmente dependiente. Luego $S'\subset S$ es m\'aximal respecto a las propiedades de contener a $S_0$ y ser linealmente independiente. El mismo argumento demuestra la contenencia $S\subseteq\langle S'\rangle$. La proposici\'on \ref{propunion}.2. implica $\langle S\rangle\le\langle S'\rangle$. La contenencia opuesta se sigue de $S'\subseteq S$, y por lo tanto $\langle S\rangle=\langle S'\rangle$.\qed

\begin{teo}[Extensi\'on de un conjunto linealmente independientes a una base]\label{extabase}
Suponga que $V$ tiene dimensi\'on finita. Si $\mathcal{B}_0$ es un subconjunto de $V$ finito y linealmente independiente, entonces existe una base $\mathcal{B}$ de $V$ que contiene a $\mathcal{B}_0$.
\end{teo}

\dem Como $V$ tiene dimensi\'on finita, existe una base finita $\mathcal{B}_1$ de $V$. Denote $S$ al conjunto $\mathcal{B}_0\cup \mathcal{B}_1$. Por la propiedad anterior, existe un subconjunto $\mathcal{B}$ de $S$ linealmente independiente maximal que contiene a $\mathcal{B}_0$. Tenemos $\langle \mathcal{B}\rangle=\langle S \rangle\ge\langle \mathcal{B}_1\rangle=V$. As\'i $\mathcal{B}$ es un conjunto linealmente independiente que genera a $V$, luego, por la proposici\'on \ref{defbase2}, $\mathcal{B}$ es una base.\qed

\begin{lema}
Si $V$ tiene dimensi\'on infinita, entonces para todo entero $n$ mayor o igual a $1$, existe un subconjunto $S$ de $V$ linealmente independiente con $n$ elementos.
\end{lema}

\dem Hacemos inducci\'on en $n$. Como $V$ tiene dimensi\'on infinita, tenemos $V\ne\{O\}$, y as\'i, para cualquier $v\in V$ distinto de $O$, el conjunto $\{v\}$ es linealmente independiente y  tiene $1$ elemento. Esto establece el caso base $n=1$. Para el paso inductivo, suponga que tenemos un subconjunto $\{v_1,\ldots,v_n\}$ de $V$ linealmente independiente. Como $V$ tiene dimensi\'on infinita, $\{v_1,\ldots,v_n\}$ no es una base de $V$, luego tampoco lo genera y por ende existe $v_{n+1}\in V$ fuera de $\langle v_1,\ldots,v_n\rangle$. Por lo tanto, el lema \ref{inddep} implica que el conjunto $\{v_1,\ldots,v_n,v_{n+1}\}$ es linealmente independiente.\qed

\begin{teo}[Monoton\'ia de la dimensi\'on]\label{monodim}
Si $U$ es un subespacio de $V$, entonces tenemos $\dim U\le \dim V$. M\'as a\'un, si $V$ tiene dimensi\'on finita, entonces $\dim U=\dim V$ si y solo si $U=V$.
\end{teo}

\dem Suponga primero $\dim(U)=\infty$. Asuma por contradicc\'on que $V$ tiene dimensi\'on finita. Por el lema anterior existe un subconunto $\mathcal{B}_0$ de $ U$ linealmente independiente, con $\dim(V)+1$ elementos. Por el teorema \ref{extabase}, existe una base $\mathcal{B}_1$ de $V$ que contiene a $\mathcal{B}_0$. Luego, $V$ tiene una base con m\'as elementos que su dimensi\'on, contradiciendo el teorema \ref{basedim}. Por lo tanto, tenemos $\dim(V)=\infty$, y as\'i $\dim(U)\le\dim(V)$.\\
Suponga ahora que $U$ tiene dimensi\'on finita. Si $V$ tiene dimensi\'on infinita, obtenemos $\dim(U)\le\dim(V)$. Finalmente, asumamos que $V$ tiene dimensi\'on finita. Tome una base $\mathcal{B}_0$ de $U$, la cual, por el teorema \ref{extabase}, podemos extender a una base $\mathcal{B}_1$ de $V$. La inclusi\'on $\mathcal{B}_0\subseteq \mathcal{B}_1$ implica $|\mathcal{B}_0|\le|\mathcal{B}_1|$, es decir $\dim(U)\le\dim(V)$.\\
Suponga que $V$ tiene dimensi\'on finita. Si $\dim(U)=\dim(V)$, entonces toda base $\mathcal{B}_0$ de $U$ es tambi\'en una base de $V$. De hecho, ya vimos que una base de $U$ se puede extender a una base de $V$, pero si esta extensi\'on contiene m\'as elementos, la dimensi\'on de $V$ ser\'ia mayor a la de $U$, contradiciendo la hip\'otesis $\dim(U)=\dim(V)$. Obtenemos as\'i $U=\langle \mathcal{B}_0\rangle=V$.\qed

\begin{obs}[Existencias de bases y dimensi\'on infinita]\label{basesinfty}
El teorema \ref{extabase} se puede usar para demostrar que, partiendo de un conjunto vacio $\mathcal{B}_0$,  todo espacio vectorial de dimensi\'on finita tiene una base. La cuesti\'on para dimensi\'on infinita toca la fibra de los fundamentos de la matem\'atica y, para demostrar la existencia de una base, requiere admitir el axioma de elecci\'on. Siendo m\'as precisos, usamos una versi\'on equivalente a este.
\begin{quote}
Lema de Zorn. Si $(P,\preccurlyeq)$ es un conjunto con un orden parcial en el que toda cadena admite una cota superior, entonces $P$ contiene al menos un elemento maximal.
\end{quote}
A partir de este axioma, suponga que $V$ tiene dimensi\'on infinita y tome un subconjunto $\mathcal{B}_0$ de $V$ linealmente independiente, \'o tome $\mathcal{B}_0=\emptyset$. Sea $P$ la colecci\'on de conjuntos linealmente independientes que contienen a $\mathcal{B}_0$, el cual ordenamos por contenencia. Dada una cadena en $P$, la uni\'on de todos sus elementos tambi\'en est\'a en $P$ y es una cota superior de ella. Por el lema de Zorn, $P$ contiene un m\'aximal $\mathcal{B}_1$. Usando un argumento similar al del lema\ref{inddep}, se demuestra que $\mathcal{B}_1$ es una base de $V$. De hecho, si existe $v\in V$fuera de $\langle \mathcal{B}_1\rangle$, el conjunto $\mathcal{B}_1\cup\{v\}$ ser\'ia un conjunto linealmente independiente que contiene estrictamente a $\mathcal{B}_1$ y a $\mathcal{B}_0$, contradiciendo la maximalidad de $\mathcal{B}_1$ en $P$.\\
De esta forma todo espacio vectorial tiene una base y a\'un m\'as, en cualquier espacio vectorial, todo conjunto linealmente independiente se puede extender a una. \\
Similarmente podemos extender la proposici\'on \ref{maximallinind} para concluir que si dentro de subconjunto $S$ de $V$, tomamos un subconjunto $S_0$ linealmente independiente, entonces existe un subconjunto $S'$ de $S$ linealmente independiente m\'aximal que contiene a $S_0$, y para un tal $S'$ se tiene $\langle S\rangle=\langle S'\rangle$. De hecho, tomamos, usando Lema de Zorn, un m\'aximal $S'$ en la colecci\'on, ordenada por inclusi\'on, de subconjunto linealmente indpendientes de $S$ que contienen a $S_0$. El lema \ref{inddep} implica la igualdad $\langle S\rangle=\langle S'\rangle$.
\end{obs}

\section{Transformaciones lineales}

Sean $U$, $V$ y $W$ espacios vectoriales sobre un cuerpo $K$.

\begin{defn}\label{deftrli}
Sea $f:V\rightarrow W$ una funci\'on. Decimos que $f$ es una \emph{transformaci\'on lineal} (o un \emph{morfismo de espacios vectoriales}) si satisface las siguientes propiedades.
\begin{enumerate}
\item \emph{Preserva sumas}: Para todo $v_1,v_2\in V$ se tiene $f(v_1+v_2)=f(v_1)+f(v_2)$.
\item \emph{Preserva productos por escalar}: Para todo $c\in K$ y todo $v\in V$ se tiene $f(cv)=cf(v)$. 
\end{enumerate}
Al conjunto de transformaciones lineales de $V$ en $W$ lo denotamos por $\Hom_K(V,W)$. A las transformaciones lineales de un espacio en \'el mismo las llamamos \emph{operadores} (o \emph{endomorfismos de espacios vectoriales}). Al conjunto $\Hom_K(V,V)$ de operadores lineales de $V$ lo denotamos por $\End_K(V)$.
\end{defn}

\begin{ejem}
\begin{enumerate}
\item \emph{Transforaci\'on lineal cero}: La funci\'on $\underline{O}:V\rightarrow W$ definida por $\underline{O}(v)=O$ para todo $v$ es una transformaci\'on lineal.
\item \emph{Operador identidad}: La funci\'on $\id_V:V\rightarrow V$ definida por $\id_V(v)=v$ para todo $v$ es un operador.
\end{enumerate}  
\end{ejem}

\begin{pro}\label{proptrlinbasicas}
Para toda $f\in\Hom_K(V,W)$ se tiene que
\begin{enumerate}
\item $f(O)=O$,
\item $f(-v)=-f(v)$ para todo $v\in V$, y
\item $f(c_1v_1+\ldots+c_nv_n)=c_1f(v_1)+\ldots+c_nf(v_n)$ para todo $c_1,\ldots,c_n\in K$ y todo $v_1,\ldots,v_n\in V$.
\end{enumerate}
\end{pro}

\dem 
\begin{enumerate}
\item Tenemos $f(O)=f(0O)=0f(O)=O$.
\item Dado $v\in V$, se tiene $f(v)+f(-v)=f\left(v+(-v)\right)=f(O)=O$, y as\'i, por la unicidad del opuesto, $f(-v)=-f(v)$.
\item Usaremos inducci\'on en $n$, siendo el caso base, $n=2$, cierto por el axioma 2 en la definici\'on \ref{deftrli} de transformaci\'on lineal. Ahora, si asumimos que la propiedad 3. es cierta para $n$, entonces para todo $c_1,\ldots,c_{n+1}\in K$ y todo $v_1,\ldots,v_{n+1}\in V$ tenemos
\begin{align*}
f(c_1v_1+\ldots+c_nv_n+c_{n+1}v_{n+1})& = f(c_1v_1+\ldots+c_nv_n)+f(c_{n+1}v_{n+1})\\
  & = c_1f(v_1)+\ldots+c_nf(v_n)+c_{n+1}f(v_{n+1}).
\end{align*}
Por lo tanto, es cierto para $n+1$ y la propiedad se sigue por inducci\'on.
\end{enumerate}\qed

\begin{obs}
Dados $f,g\in\Hom_K(V,W)$ y $c\in K$ definimos las transformaciones lineales $f+g$ y $cg$ en $\Hom_K(V,W)$ por $(f+g)(v)=f(v)+g(v)$ y $(cg)(v) = cg(v)$ para todo $v\in V$. El conjunto $\Hom_K(V,W)$ junto con estas operaciones y el origen dado por $\underline{O}$ es un espacio vectorial sobre $K$.
\end{obs}

\begin{prop}\label{compeslineal}
Si $f\in\Hom_K(V,W)$ y $g\in\Hom_K(W,U)$, entonces $(g\circ f)\in\Hom_K(V,U)$.
\end{prop}

\dem Para todo $u,v\in V$ tenemos
\[
g\circ f(u+v)=g\left( f(u)+f(v)\right)= g\circ f(u)+g\circ f(v).
\]
Para todo $v\in V$ y todo $c\in K$, tenemos
\[
g\circ f(cv)=g\left( c f(v)\right)=cg\circ f(v).
\]
\qed

\begin{prop}[Rigidez de las transformaciones lineales]\label{unitrlin}
Sean $v_1,\ldots,v_n\in V$ tales que $\langle v_1,\ldots,v_n\rangle=V$. Si $w_1,\ldots,w_n$ son elementos en $W$, entonces tenemos las siguientes dos propiedades.
\begin{enumerate}
\item Existe a lo sumo una transformaci\'on lineal $f\in\Hom_K(V,W)$ tal que, para $i\in\{1,\ldots,n\}$, $f(v_i)$ es $w_i$, 
\item Si $\{v_1,\ldots,v_n\}$ es linealmente independiente, entonces existe una transformaci\'on $f\in\Hom_K(V,W)$ tal que, para $i\in\{1,\ldots,n\}$, $f(v_i)$ es $w_i$.
\end{enumerate}
\end{prop}

\dem
\begin{enumerate}
\item Sea $f,g\in\Hom_K(V,W)$. Suponga que, para $i\in\{1,\ldots,n\}$ tenemos $f(v_i)=w_i=g(v_i)$. Dado $v\in V$, existen $c_1,\ldots,c_n\in K$ para los cuales tenemos $v=c_1v_1+\ldots+c_nv_n$ y 
\[
f(v)=f\left(\sum_{i=1}^nc_iv_i\right)=\sum_{i=1}^nc_if(v_i)=\sum_{i=1}^nc_iw_i=\sum_{i=1}^nc_ig(v_i)=g(v).
\]
Por lo tanto tenemos $f=g$.
\item Si $\{v_1,\ldots,v_n\}$ es linealmente independiente, por la proposici\'on \ref{defbase2}, se sigue que $\{v_1,\ldots,v_n\}$ es una base de $V$. Dado $v\in V$, existe un \'unico elemento $(c_1,\ldots, c_n)\in K^n$ para el que se tiene $v=c_1v_1+\ldots+c_nv_n$. Defina la funci\'on $f:V\rightarrow W$ por
\[
f(v)=c_1w_1+\ldots+c_nw_n.
\]
Veamos que $f$ es una transformaci\'on lineal. Dados $u,v\in V$, se tiene $u=a_1v_1+\ldots+a_nv_n$ y $v=b_1v_1+\ldots+b_nv_n$ con $(a_1,\ldots, a_n)$ y $(b_1,\ldots, b_n)$ en $K^n$. As\'i, obtenemos $u+v =  (a_1+b_1)v_1+\ldots+(a_n+b_n)v_n$ y
\begin{eqnarray*}
f(u+v) & = & (a_1+b_1)w_1+\ldots+(a_n+b_n)w_n\\
     & = & a_1w_1+\ldots+a_nw_n+ b_1w_1+\ldots+b_nw_n\\
     & = & f(u)+f(v).
\end{eqnarray*} 
Luego, $f$ respecta sumas.
Dado $v\in V$, se tiene $v=c_1v_1+\ldots+c_nv_n$ con $(c_1,\ldots, c_n)\in K^n$ . As\'i, obtenemos, para todo $c\in K$, $cv = (ca_1)v_1+\ldots+(ca_n)v_n$ y
\begin{eqnarray*}
f(cv_1) & = & (ca_1)w_1+\ldots+(ca_n)w_n\\
     & = & c(a_1w_1+\ldots+a_nw_n)\\
     & = & cf(v_1)
\end{eqnarray*}
Por lo tanto, $f$ tambi\'en respecta productos por escalar y se sigue que $f$ es una transformaci\'on lineal.
\end{enumerate}\qed

\begin{prop}
Sea $f\in\Hom_K(V,W)$. Si $f$ es biyectiva, entonces $f^{-1}$ es una transformaci\'on lineal. 
\end{prop}

\dem Sean $w_1,w_2\in W$. Como $f$ es sobreyectiva, tenemos $f(v_1)=w_1$ y $f(v_2)=w_2$ para algunos $v_1,v_2\in V$ y, as\'i, $f(v_1+v_2)=f(v_1)+f(v_2)=w_1+w_2$. Se siguen
\[
f^{-1}(w_1+w_2)=f^{-1}\left(f(v_1+v_2)\right)=v_1+v_2,
\] 
y
\[
f^{-1}(w_1)+f^{-1}(w_2)=f^{-1}\left(f(v_1)\right)+f^{-1}\left(f(v_2)\right)=v_1+v_2.
\]
Luego, para todo $w_1,w_2\in W$ tenemos  $f^{-1}(w_1+w_2)=f^{-1}(w_1)+f^{-1}(w_2)$.\\
Sea $w\in W$. Dado $v\in V$ para el cual $f(v)=w$, tenemos $f(cv)=cf(v)=cw$ para todo $c\in K$. Obtenemos
\[
f^{-1}(cw)=f^{-1}\left(f(cv)\right)=cv,
\]
y
\[
cf^{-1}(w)=cf^{-1}\left(f(v)\right)=cv.
\]
Luego, para todo $w\in W$ y todo $c\in K$ tenemos $f^{-1}(cw)=cf^{-1}(w)$. Por lo tanto, $f^{-1}$ respeta sumas y productos por escalar, y por consiguiente $f^{-1}$ es una transformaci\'on lineal.

\begin{defn}
Sea $f\in\Hom_K(V,W)$. Si $f$ es una biyecci\'on, entonces decimos que $f$ es un \emph{isomorfismo}. Si existe un isomorfismo $f\in\Hom_K(V,W)$ decimos que $V$ y $W$ son \emph{isomorfos} y lo denotamos por $V\simeq_K W$.
\end{defn}

\begin{ejem} Suponga que $V$ es unidimensional. Dado $\lambda\in K$, la funci\'on $f:V\rightarrow V$ definida por $f(v)=\lambda v$ para todo $v$ es una transformaci\'on lineal. Rec\'iprocamente, si $f:V\rightarrow V$ es un transformaci\'on lineal existe $\lambda\in K$ tal que $f(v)=\lambda v$ para todo $v$. De hecho, si $v_0\ne 0$ entonces $V=\langle v_0\rangle$ as\'i $f(v_0)=\lambda v_0$ para alg\'un $\lambda\in K$. Ahora, dado $v\in V$ existe $c\in K$ tal que $cv_0=v$, y tenemos que
\[
f(v)=cf(v_0)=c\lambda v_0=\lambda cv_0=\lambda v.
\]
Entonces, si $\Phi: K \rightarrow \Hom_K(V,V)$ es la funci\'on definida por $\Phi(\lambda)=m_\lambda$, donde $m_\lambda$ es tal que $m_\lambda(v)=\lambda v$ para todo $v\in V$, $\Phi$ es un isomorfismo y $\Hom_K(V,V)\simeq_K K$.
\end{ejem}

\begin{defn}
Sea $f\in\End_K(V)$. Si $f$ es una biyecci\'on, entonces decimos que $f$ es un \emph{automorfismo}. Al conjunto de automorfismos de $V$ lo denotamos por $\GL_K(V)$.
\end{defn}

\begin{teo}
Si que $V$ y $W$ tienen dimensi\'on finita, entonces $V\simeq_K W$ si y solo si $\dim(V)=\dim(W)$.
\end{teo}

\dem Primero establecemos la necesidad. Suponga que $V\simeq_K W$ y sea $f\in\Hom_K(V,W)$ un isomorfismo. Sea $\{v_1,\ldots,v_n\}$ una base de $V$, $n=\dim(V)$. Para $i\in\{1,\ldots,n\}$, sea $w_i=f(v_i)$. Veamos que $\{w_1,\ldots,w_n\}$ es base de $W$, para obtener que $\dim(W)=n=\dim(V)$. Si $c_1,\ldots,c_n\in K$ son tales que $c_1w_1+\ldots+c_nw_n=O$, entonces
\[
f(c_1v_1+\ldots+c_nv_n)=c_1f(v_1)+\ldots+c_nf(v_n)=c_1w_1+\ldots+c_nw_n=O.
\]
Como $f$ es inyectiva y $f(O)=O$, se sigue que $c_1v_1+\ldots+c_nv_n=O$. Como $\{v_1,\ldots,v_n\}$ es linealmente independiente, entonces $c_1=\ldots=c_n=0$. De esta forma, el conjunto $\{w_1,\ldots,w_n\}$ es linealmente independiente. Sea $w\in W$. Como $f$ es sobreyectiva, existe $v\in V$ tal que $f(v)=w$. Como $\{v_1,\ldots,v_n\}$ genera $V$, existen $c_1,\ldots,c_n\in K$ tales que $v=c_1v_1+\ldots+c_nv_n$. Luego
\[
w=f(v)=f(c_1v_1+\ldots+c_nv_n)=c_1f(v_1)+\ldots+c_nf(v_n)=c_1w_1+\ldots+c_nw_n.
\]
De donde $W=\langle w_1,\ldots,w_n\rangle$ y as\'i $\{w_1,\ldots,w_n\}$ es base de $W$. Esto completa la prueba de la necesidad.\\
Ahora establecemos la suficiencia. Suponga que $\dim(V)=\dim(W)=n$ y sean $\{v_1,\ldots,v_n\}$ y $\{w_1,\ldots,w_n\}$ respectivamente bases de $V$ y $W$. Por la proposici\'on \ref{unitrlin}.2, existe $f\in\Hom(V,W)$ tal que $f(v_i)=w_i$, para $i\in\{1,\ldots,n\}$. La prueba estar\'a completa cuando establezcamos que $f$ es biyectiva. Sean $u,v\in V$ tales que $f(u)=f(v)$. Existen $(a_1,\ldots,a_n)$ y $(b_1,\ldots,b_n)$ en $K^n$ tales que $u=a_1v_1+\ldots+a_nv_n$ y $v=b_1v_1+\ldots+b_nv_n$. Como $f(u-v)=f(u)-f(v)=O$ y
\[
u-v=(a_1-b_1)v_1+\ldots+(a_n-b_n)v_n,
\]
entonces
\[
O=f(u-v)=(a_1-b_1)w_1+\ldots+(a_n-b_n)w_n.
\]
La independencia lineal de $\{w_1,\ldots,w_n\}$ implica que, para $i\in\{1,\ldots,n\}$, $a_i-b_i=0$, o equivalentemente, $a_i=b_i$. De esta forma tenemos que $u=v$ y se sigue que $f$ es inyectiva. Sea $w\in W$ y sea $(c_1,\ldots, c_n)\in K^n$ tal que $w=c_1w_1+\ldots+c_nw_n$. Para $v=c_1v_1+\ldots+c_nv_n$,
\[
f(v)=f(c_1v_1+\ldots+c_nv_n)=c_1f(v_1)+\ldots+c_nf(v_n)=c_1w_1+\ldots+c_nw_n=w.
\]
Por lo cual, $f$ es sobreyectiva. Al ser inyectiva y sobreyectiva, $f$ es biyectiva.\qed

\begin{defn}
Sea $f\in\Hom_K(V,W)$.
\begin{enumerate}
\item El \emph{n\'ucleo} (o el \emph{kernel}) de $f$ es el conjunto $\{v\in V|\ f(v)=O\}$ y  lo denotamos por $\ker(f)$.
\item La \emph{imagen} de $f$ es el conjunto $\{w\in W|\ \exists v\in V: f(v)=w\}$ y la denotamos por
$\im(f)$.
\end{enumerate}
\end{defn}

\begin{pro}
Sea $f\in\Hom_K(V,W)$, entonces $\ker(f)$ e $\im(f)$ son respectivamente subespacios de $V$ y $W$.
\end{pro}

\dem Como $f(O)=O$, entonces $O\in\ker(f)$ y $O\in\im(f)$. Para todo $v_1,v_2\in \ker(f)$, tenemos $f(v_1+v_2)=f(v_1)+f(v_2)=O+O=O$ y  as\'i $u+v\in\ker(f)$. Para todo $v\in \ker(f)$ y todo $c\in K$, $f(cv)=cf(v)=cO=O$ y as\'i $cv\in\ker(f)$. Luego, de la propiedad \ref{subespsiysolosi}, se sigue que $\ker(f)$ es un subespacio de $V$. Para todo $w_1,w_2\in \im(f)$ existen $v_1,v_2\in V$ tales que $f(v_1)=w_1$ y $f(v_2)=w_2$. Luego $f(v_1+v_2)=f(v_1)+f(v_2)=w_1+w_2$ y as\'i $w_1+w_2\in\im(f)$. Para todo $w\in\im(f)$ existe $v\in V$ tal que $f(v)=w$. Luego, para todo $c\in K$, $f(cv)=cf(v)=cw$ y as\'i $cw\in\im(f)$. luego, de la propiedad \ref{subespsiysolosi}, se sigue que $\im(f)$ es un subespacio de $V$.\qed

\begin{prop}\label{inyectiva}
Sea $f\in\Hom_K(V,W)$, entonces $f$ es inyectiva si y solo si $\ker(f)=\{O\}$.
\end{prop}

\dem Suponga primero que $f$ es inyectiva, entonces, como $f(O)=O$, tenemos $\ker(f)=\{O\}$. Suponga ahora que $\ker(f)=\{O\}$. Si $u,v\in V$ son tales que $f(u)=f(v)$, entonces
\[
f(u-v)=f(u)-f(v)=O,
\]
De donde $u-v\in\ker(f)$, luego $u-v=O$, o equivalentemente $u=v$, y as\'i ,$f$ es inyecta.\qed

\begin{defn}
Suponga que $V$ tiene dimensi\'on finita. Sea $f\in\Hom_K(V,W)$.
\begin{enumerate}
\item La \emph{nulidad} de $f$, que denotamos $\nu(f)$ es la dimensi\'on de $\ker(f)$.
\item El \emph{rango} de $f$, que denotamos $\rho(f)$ es la dimensi\'on de $\im(f)$.
\end{enumerate}
\end{defn}

\begin{teo}[Teorema del rango]\label{teorango}
Suponga que $V$ tiene dimensi\'on finita. S $f\in\Hom_K(V,W)$, entonces
\[
\nu(f)+\rho(f)=\dim (V)
\]
\end{teo}

\dem Note que para el caso $\rho(f)=0$ el teorema se sigue inmediatamente. Asumamos que $\rho(f)>0$. Como $\ker(f)\le V$, por la monoton\'ia de la dimensi\'on tenemos $\nu(f)=\dim\left(\ker(f)\right)\le\dim(V)$. Sean $n=\nu(f)$ y $n+m=\dim (V)$. Sea $\mathcal{B}_0=\{v_1,\ldots,v_n\}\subseteq V$ una base de $\ker(f)$ (si $n=0$ tomamos $\mathcal{B}_0=\emptyset$). Extendemos $\mathcal{B}_0$ a una base $\mathcal{B}=\{v_1,\ldots,v_n,v_{n+1},\ldots,v_{n+m}\}$ de $V$.\\
Para $i\in\{1,\ldots,m\}$, sea $w_i=f(v_{n+i})$. Basta demostrar que $\{w_1,\ldots,w_m\}$ es una base de $\im(f)$, pues en tal caso tendr\'iamos que $m=\dim\left(im(f)\right)=\rho(f)$. Para establecerlo usaremos la proposici\'on \ref{defbase2}.\\
Veamos primero que $\{w_1,\ldots,w_m\}$ es linealmente independiente. Suponga que $a_1,\ldots,a_m\in K$ son tales que
\[
a_1w_1+\ldots+a_mw_m=O.
\]
Luego, si $v=a_1v_{n+1}+\ldots+a_mv_{n+m}$, entonces
\[
f(v)=a_1f(v_{n+1})+\ldots+a_mf(v_{n+m})=a_1w_1+\ldots+a_mw_m=O,
\]
y as\'i $v\in\ker(f)$. Pero como $\langle v_1,\ldots,v_n\rangle=\ker(f)$, entonces existe $(b_1,\ldots,b_n)\in K^n$ tal que
\[
v=b_1v_1+\ldots+b_nv_n
\]
es decir que $b_1v_1+\ldots+b_nv_n=v=a_1v_{n+1}+\ldots+a_mv_{n+m}$ y as\'i
\[
O=(-b_1)v_1+\ldots+(-b_n)v_n+a_1v_{n+1}+\ldots+a_mv_{n+m}.
\]
Finalmente, como $\{v_1,\ldots,v_n,v_{n+1},\ldots,v_{n+m}\}$ es linealmente independiente, la igualdad anterior implica que $a_1=\ldots=a_m=0$. La independencia lineal de $\{w_1,\ldots,w_m\}$ se sigue ahora de la  proposici\'on \ref{proplinind}.\\
Veamos que  $\{w_1,\ldots,w_m\}$ genera a $im(f)$. Como $w_1,\ldots,w_m\in \im(f)$ entonces $\langle w_1,\ldots,w_m\rangle\subseteq\im(f)$. Basta entonces establecer la otra inclusi\'on. Si $w\in\im(f)$, entonces existe $v\in V$ tal que $f(v)=w$. Sean $c_1,\ldots,c_n,c_{n+1},\ldots,c_{n+m}\in K$ tales que 
\[
v=c_1v_1+\ldots+c_nv_n+c_{n+1}v_{n+1}+\ldots+c_{n+m}v_{n+m},
\]
de forma que
\begin{eqnarray*}
w=f(v) & = & f(\underbrace{c_1v_1+\ldots+c_nv_n}_{\in\ \ker(f)})+c_{n+1}f(v_{n+1})+\ldots+c_{n+m}f(v_{n+m})\\
           & = & c_{n+1}w_1+\ldots+c_{n+m}w_m,
\end{eqnarray*}
y as\'i $w\in\langle w_1,\ldots,w_m\rangle$. De donde $\im(f)\subseteq\langle w_1,\ldots,w_m\rangle$.\qed

\begin{coro}\label{corteorango}
Suponga que $V$ tiene dimensi\'on finita y sea $f\in\Hom_K(V,W)$. Entonces las siguientes dos propiedades son equivalentes:
\begin{enumerate}
\item $\nu(f)=0$
\item $\rho(f)=\dim_K(V)$
\end{enumerate}
\end{coro}

\dem Se sigue inmediatamente del teorema del rango.

\section{Matrices y vectores de coordenadas}

Sea $K$ un cuerpo y sean $m,n,r,s$ enteros estrictamente positivos.

\begin{defn}
Sean $I$ y $J$ respectivamente los conjuntos $\{1,\ldots,m\}$ y $\{1,\ldots,n\}$. Una \emph{matriz $m\times n$ sobre $K$} (o una \emph{matriz $m\times n$ con entradas en $K$}) es una funci\'on $A\in K^{I\times J}$. Para todo $(i,j)\in I\times J$, sea $a_{ij}=A(i,j)$. Denotaremos $A=(a_{ij})$ \'o
$$ A=\left[\begin{array}{ccc}
a_{11} & \cdots & a_{1n}\\
\vdots & \ddots & \vdots\\
a_{m1} & \cdots & a_{mn}
\end{array}\right]$$
y llamamos a $a_{ij}$ la entrada $ij$ de $A$. La fila $i$ de $A$ es la matrix $1\times n$
$$\left[a_{i1}\ldots a_{in}\right]$$
y la columna $j$ de $A$ es la matrix $m\times 1$,
$$ \left[\begin{array}{c}
a_{1j}\\
\vdots\\
a_{mj}
\end{array}\right].$$

Al espacio vectorial formado por el conjunto de matrices $m\times n$ sobre $K$ junto con las operaciones de suma y multiplicaci\'on por escalar de $K^{I\times J}$ y el origen $\underline{O}$ lo denotamos por $M_{m\times n}(K)$.
\end{defn}

\begin{obs}
Los espacios vectoriales $M_{m\times n}(K)$  y $K^{I\times J}$ coinciden cuando $I=\{1,\ldots,m\}$ y $J=\{1,\ldots,n\}$.
\end{obs}

\begin{defn}
Sea $I$ el conjunto $\{1,\ldots,n\}$. Un \emph{vector de $n$ coordenadas sobre $K$}  (o un \emph{vector de coordenadas con entradas en $K$}) es una matriz $n\times 1$. Si $\overline{x}\in M_{n\times 1}(K)$ es tal que $\overline{x}(i,1)=x_i$, para $i\in I$, entonces denotaremos $\overline{x}=(x_i)$ \'o
$$ \overline{x}=\left[\begin{array}{c}
x_{1}\\
\vdots\\
x_{n}
\end{array}\right]$$
y llamamos a $x_i$ la coordenada $i$ de $\overline{x}$. Para simplificar, denotaremos $\overline{x}(i,1)$ por $\overline{x}(i)$.
\end{defn}

\begin{defn}
Sean $A$ y $B$ respectivamente matrices $m\times n$ y $n\times r$. Definimos el \emph{producto de $A$ y $B$}, que denotamos por $AB$, como la matriz $m\times r$ cuya entrada $ij$ es
$$\sum_{k=1}^n a_{ik}b_{kj}=a_{i1}b_{1i}+\ldots+a_{in}b_{nj}.$$
\end{defn}

\begin{pro}
La multiplicaci\'on matricial satisface las siguientes propiedades.
\begin{enumerate}
\item \emph{Commutatividad con escalares}: Para todo $c\in K$, toda matriz $A\in M_{m\times n}(K)$ y $B\in M_{n\times r}(K)$ tenemos $A(cB)=cAB=(cA)B$.
\item \emph{Distributividad}: Para toda matriz $A\in M_{m\times n}(K)$ y $B,C\in M_{n\times r}(K)$ tenemos $A(B+C)=AB+AC$.
\item \emph{Asociatividad}: Para toda matriz $A\in M_{m\times n}(K)$, $B\in M_{n\times r}(K)$ y $C\in M_{r\times s}(K)$ tenemos $A(BC)=(AB)C$.
\end{enumerate}
\end{pro}

\dem Sean $A=(a_{ij})$, $B=(b_{ij})$ y $C=(c_{ij})$.
\begin{enumerate}
\item La entrada $ij$ de $A(cB)$ es
$$\sum_{k=1}^n a_{ik}cb_{kj}=c\sum_{k=1}^n a_{ik}b_{kj}=\sum_{k=1}^n ca_{ik}b_{kj},$$
y as\'i es igual a la entrada $ij$ de $cAB$ y de $(cA)B$.
\item La entrada $ij$ de $A(B+C)$ es
$$\sum_{k=1}^n a_{ik}(b_{kj}+c_{kj})=\sum_{k=1}^n a_{ik}b_{kj}+\sum_{k=1}^n a_{ik}c_{kj},$$
y as\'i es igual a la entrada $ij$ de $AB+AC$.
\item La entrada $ij$ de $A(BC)$ es
$$\sum_{l=1}^n a_{il}\left(\sum_{k=1}^r b_{lk}c_{kj}\right) = \sum_{k=1}^r\sum_{l=1}^n a_{il}b_{lk}c_{kj} = \sum_{k=1}^r\left(\sum_{l=1}^n a_{il}b_{lk}\right)c_{kj},$$
y as\'i es igual a la entrada $ij$ de $(AB)C$.
\end{enumerate}
\qed

\begin{defn}
La \emph{matriz identidad $n\times n$} es la matriz $I_n\in M_{n\times n}(K)$ cuya entrada $ij$ es $1$ cuando $i=j$ y $0$ cuando $i\ne j$. En particular tenemos
$$I_n=\left[\begin{array}{ccc}
1 & \cdots & 0\\
\vdots & \ddots & \vdots\\
0 & \cdots & 1
\end{array}\right]$$
e $I_n=(\delta_{ij})$ donde definimos $\delta_{ij}=1$ cuando $i=j$ y $\delta_{ij}=0$ cuando $i\ne j$. Sea $I$ el conjunto $\{1,\ldots,n\}$. A la funci\'on $\delta\in K^{I\times I}$ definida por $\delta(i,j)=\delta_{ij}$ la llamamos \emph{la funci\'on delta de Kronecker}.  
\end{defn}

\begin{pro}
Para toda matriz $A\in M_{m\times n}(K)$, se tiene $I_mA=A=AI_n$.
\end{pro}

\dem Si $a_{ij}$ es la entrada $ij$ de $A$, entonces la entrada $ij$ de $I_mA$ es $\sum_{k=1}^m \delta_{ik}a_{kj}=a_{ij}$ y la de $AI_n$ es $\sum_{k=1}^na_{ik}\delta{kj}=a_{ij}$. \qed

\begin{defn}
Sea $A\in M_{n\times n}(K)$. Decimos que $A$ es una \emph{matriz invertible} si existe una matriz $A^{-1}\in M_{n\times n}(K)$ para la cual se tiene $AA^{-1}=I_n=A^{-1}A$. En tal caso, llamamos a $A^{-1}$ \emph{matriz inversa de $A$}.
\end{defn}

\begin{pro}[Unicidad de la inversa]
Sea $A\in M_{n\times n}(K)$ una matriz invertible. Si $B\in M_{n\times n}$ es tal que $BA$ \'o $AB$ es igual a $I_n$ entonces $B=A^{-1}$.
\end{pro}

\dem Si $BA$ es igual a $I_n$, entonces tenemos $B=BI_n=B(AA^{-1})=(BA)A^{-1}=I_nA^{-1}=A^{-1}$. Similarmente se establece $B=A^{-1}$ cuando $AB$ es igual a $I_n$.\qed

\subsection*{Matrices de transformaciones}

Sean $V$, $W$ y $U$ espacios vectoriales de dimensi\'on finita sobre un cuerpo $K$ y sean $n$, $m$ y $r$ sus respectivas dimensiones.

\begin{defn}
Sea $\mathcal{B}_V$ la base $\{v_1,\ldots,v_n\}$ de $V$. Sea $v\in V$ y sean $c_1,\ldots,c_n\in K$ tales que $v$ es igual a $c_1v_1+\ldots+c_nv_n.$
El \emph{vector de coordenadas $\Big[ v \Big]^{\mathcal{B}_V}$ de $v$ en la base $\mathcal{B}_V$} es el vector de $n$ coordenadas $(c_i)$.
\end{defn}

\begin{prop}
Sea $\mathcal{B}_V$ la base $\{v_1,\ldots,v_n\}$ de $V$ e $I=\{1,\ldots,n\}$. Entonces la funci\'on
\begin{eqnarray*}
\Big[ \bullet \Big]^{\mathcal{B}_V}: V & \longrightarrow & K^I\\
v & \longmapsto & \Big[ v \Big]^{\mathcal{B}_V}
\end{eqnarray*}
es un isomorfismo.
\end{prop}

\dem Tenemos $\left[ v \right]^{\mathcal{B}_V}=O$ si y solo si $v=0$, as\'i la proposici\'on se sigue del corolario \ref{corteorango} del teorema del rango. 

\begin{defn}
Sea $f\in\Hom_K(V,W)$. Sean $\mathcal{B}_V$ y $\mathcal{B}_W$ respectivamente las bases $\{v_1,\ldots,v_n\}$ y $\{w_1,\ldots,w_m\}$ de $V$ y $W$. Para $i\in\{1,\ldots,m\}$ y $j\in\{1,\ldots,n\}$, sea $a_{ij}\in K$ tal que $f(v_j)$ es igual a $a_{1j}w_1+\ldots+a_{mj}w_m$. La \emph{matriz de $f$ para las bases $\mathcal{B}_V$ y $\mathcal{B}_W$} es la matrix $m\times n$ cuya entrada $ij$ es $a_{ij}$, la cual denotamos $\Big[ f \Big]^{\mathcal{B}_W}_{\mathcal{B}_V}$.
\end{defn}

\begin{obs}
La columna $j$ de $\Big[ f \Big]^{\mathcal{B}_W}_{\mathcal{B}_V}$ es el vector de coordenadas $\Big[f(v_j)\Big]^{\mathcal{B}_W}$, en particular tenemos 
$$\Big[f\Big]^{\mathcal{B}_W}_{\mathcal{B}_V}=\Bigg[\Big[ f(v_1)\Big]^{\mathcal{B}_W}\Big|\ldots \Big| \Big[ f(v_n)\Big]^{\mathcal{B}_W}\Bigg]$$
y $\Big[\id_V\Big]^{\mathcal{B}_V}_{\mathcal{B}_V}=I_n$.
\end{obs}

\begin{pro}
Sea $f\in\Hom_K(V,W)$. Sean $\mathcal{B}_V$ y $\mathcal{B}_W$ respectivamente las bases $\{v_1,\ldots,v_n\}$ y $\{w_1,\ldots,w_m\}$ de $V$ y $W$. Para todo $v\in V$ se tiene
$$\Big[ f(v)\Big]^{\mathcal{B}_W}=\Big[ f \Big]^{\mathcal{B}_W}_{\mathcal{B}_V}\Big[ v \Big]^{\mathcal{B}_V}.$$
\end{pro}

\dem Sean $\Big[ f \Big]^{\mathcal{B}_W}_{\mathcal{B}_V}=(a_{ij})$ y $\Big[v\Big]^{\mathcal{B}_V}=(c_i)$. La coordenada $i$ del vector de coordenadas $\Big[ f \Big]^{\mathcal{B}_W}_{\mathcal{B}_V}\Big[ v \Big]^{\mathcal{B}_V}$ es $\sum_j a_{ij}c_j$, y as\'i, de las igualdades
\begin{align*}
f(v) & = f(c_1v_1+\ldots+c_nv_n)=c_1f(v_1)+\ldots+c_nf(v_n)\\
 & = c_1\sum_{i=1}^m a_{i1}w_i+\ldots+c_n\sum_{i=1}^m a_{in}w_i\\
 & = \sum_{j=1}^n\sum_{i=1}^m a_{ij}c_jw_i=\sum_{i=1}^m\left(\sum_{j=1}^n a_{ij}c_j\right)w_i,
\end{align*}
se sigue que la coordenada $i$ de $f(v)$ en la base $\mathcal{B}_W$ es igual a la coordenada $i$ del vector de coordenadas  $\Big[ f \Big]^{\mathcal{B}_W}_{\mathcal{B}_V}\Big[ v \Big]^{\mathcal{B}_V}$.\qed

\begin{pro}\label{compmult}
Sean $f\in\Hom_K(V,W)$ y $g\in\Hom_K(W,U)$. Si $\mathcal{B}_V$, $\mathcal{B}_W$ y $\mathcal{B}_U$ son respectivamente bases de $V$, $W$ y $U$, entonces se tiene
$$
\Big[ g\circ f \Big]^{\mathcal{B}_U}_{\mathcal{B}_V}=\Big[ g \Big]^{\mathcal{B}_U}_{\mathcal{B}_W}\Big[ f \Big]^{\mathcal{B}_W}_{\mathcal{B}_V}.
$$
\end{pro}

\dem Sean $\mathcal{B}_V=\{v_1,\ldots,v_n\}$, $\mathcal{B}_W=\{w_1,\ldots,w_m\}$ y $\mathcal{B}_U=\{u_1,\ldots,u_r\}$. Sean $\Big[ f \Big]^{\mathcal{B}_W}_{\mathcal{B}_V}=(a_{ij})$ y $\Big[ g \Big]^{\mathcal{B}_U}_{\mathcal{B}_W}=(b_{ij})$.
La entrada $ij$ de $\Big[ g \Big]^{\mathcal{B}_U}_{\mathcal{B}_W}\Big[ f \Big]^{\mathcal{B}_W}_{\mathcal{B}_V}$ es $\sum_k b_{ik}a_{kj}$, y as\'i, de las igualdades
\begin{align*}
g\circ f(v_j) &= g(a_{1j}w_1+\ldots+a_{mj}w_m)=a_{1j}g(w_1)+\ldots+a_{mj}g(w_m)\\
 & = a_{1j}\sum_{i=1}^rb_{i1}u_i+\ldots+a_{mj}\sum_{i=1}^rb_{im}u_i\\
 & = \sum_{k=1}^r\sum_{i=1}^rb_{ik}a_{kj}u_i=\sum_{i=1}^r\left(\sum_{k=1}^r b_{ik}a_{kj}\right)u_i,
\end{align*}
se sigue que la coordenada $i$ de $g\circ f(v_j)$ en la base $\mathcal{B}_U$ es igual a la entrada $ij$ de la matriz $\Big[ g \Big]^{\mathcal{B}_U}_{\mathcal{B}_W}\Big[ f \Big]^{\mathcal{B}_W}_{\mathcal{B}_V}$.\qed

\begin{prop}\label{homym}
Si $\mathcal{B}_V$ y $\mathcal{B}_W$ son respectivamente bases de $V$ y $W$, entonces el mapa $\Big[ \bullet \Big]^{\mathcal{B}_W}_{\mathcal{B}_V}$ definido por
\begin{eqnarray*}
\Big[ \bullet \Big]^{\mathcal{B}_W}_{\mathcal{B}_V}: \Hom_K(V,W) & \longrightarrow & M_{m\times n}(K)\\
f & \longmapsto & \Big[ f \Big]^{\mathcal{B}_W}_{\mathcal{B}_V}
\end{eqnarray*}
es un isomorfismo de espacios vectoriales sobre $K$. M\'as a\'un, $f$ es un isomorfismo si y solo si $\Big[ f \Big]_{\mathcal{B}_W}^{\mathcal{B}_V}$ es invertible, y en tal caso se tiene $\Big[ f^{-1} \Big]^{\mathcal{B}_W}_{\mathcal{B}_V}=\left(\Big[ f \Big]_{\mathcal{B}_V}^{\mathcal{B}_W}\right)^{-1}$.
\end{prop}

\dem  Sean $\mathcal{B}_V=\{v_1,\ldots,v_n\}$ y $\mathcal{B}_W=\{w_1,\ldots,w_n\}$. Sean $f,g\in\Hom_K(V,W)$. Como $(f+g)(v_j)=f(v_j)+g(v_j)$ y $(cf)(v_j)=cf(v_j)$ para todo $j\in\{1,\ldots,n\}$ y todo $c\in K$, entonces  $\Big[ \bullet \Big]^{\mathcal{B}_W}_{\mathcal{B}_V}$ es una transformaci\'on lineal. Para establecer que es una biyecci\'on basta notar que, por la proposici\'on \ref{unitrlin}, dada una matriz $A\in M_{m\times n}(K)$, con $A=(a_ij)$, existe una \'unica transformaci\'on lineal $f\in \Hom_K(V,W)$ tal que $f(v_j)=\sum_i a_{ij}w_i$.

Sea $\Big[ f \Big]^{\mathcal{B}_W}_{\mathcal{B}_V}=A$. Si $f$ es bijectiva y $B$ es la matriz $\Big[ f^{-1} \Big]_{\mathcal{B}_W}^{\mathcal{B}_V}$, entonces se tiene
$$BA=\Big[ f^{-1} \Big]_{\mathcal{B}_W}^{\mathcal{B}_V}\Big[ f \Big]^{\mathcal{B}_W}_{\mathcal{B}_V}= \Big[ f^{-1}\circ f \Big]^{\mathcal{B}_V}_{\mathcal{B}_V}=\Big[\id_V\Big]^{\mathcal{B}_V}_{\mathcal{B}_V}=I_n,$$
luego $A$ es invertible y $B=A^{-1}$. Si $A$ es invertible, entonces por la proposici\'on \ref{homym} existe una \'unica transformaci\'on lineal $g\in \Hom_K(W,V)$ tal que $A^{-1}$ es la matriz $\Big[ g \Big]^{\mathcal{B}_V}_{\mathcal{B}_W}$. Las igualdades
$$\Big[ g\circ f \Big]^{\mathcal{B}_V}_{\mathcal{B}_V}=\Big[ g \Big]^{\mathcal{B}_V}_{\mathcal{B}_W}\Big[ f \Big]^{\mathcal{B}_W}_{\mathcal{B}_V}=A^{-1}A=I_n=\Big[ \id_V \Big]^{\mathcal{B}_V}_{\mathcal{B}_V},$$
implican $g\circ f=\id_V$. Similarmente, tenemos $f\circ g=\id_W$. As\'i, $f$ es una biyecci\'on y $g$ es la inversa de $f^{-1}$.\qed

\begin{defn}
Sean $\mathcal{B}$ y $\mathcal{B}'$ respectivamente las bases $\{v_1,\ldots,v_n\}$ y $\{v'_1,\ldots,v'_n\}$ de $V$. La \emph{matriz de cambio de coordenadas de la base $\mathcal{B}$ a la base $\mathcal{B}'$} es la matriz $\Big[\id_V\Big]^{\mathcal{B}'}_\mathcal{B}$.
\end{defn}

\begin{obs}
La columna $j$ de $\Big[\id_V\Big]^{\mathcal{B}'}_{\mathcal{B}}$ es el vector de coordenadas $\Big[ v_j\Big]^{\mathcal{B}'}$, en particular tenemos
$$\Big[\id_V\Big]^{\mathcal{B}'}_{\mathcal{B}}=\Bigg[\Big[ v_1\Big]^{\mathcal{B}'}\Big|\ldots \Big| \Big[ v_n\Big]^{\mathcal{B}'}\Bigg].$$
\end{obs}

\begin{pro}
Sean $\mathcal{B}$ y $\mathcal{B}'$ respectivamente las bases $\{v_1,\ldots,v_n\}$ y $\{v'_1,\ldots,v'_n\}$ de $V$.
\begin{enumerate}
\item Para todo $v\in V$ se tiene $\Big[v\Big]^{\mathcal{B}'}=\Big[\id_V\Big]^{\mathcal{B}'}_\mathcal{B}\Big[v\Big]^{\mathcal{B}}.$
\item La matriz de cambio de coordenadas de una base a la otra y la matriz de cambio inverso son una la inversa de la otra, es decir que se tiene la igualdad
$$\Big[\id_V\Big]^{\mathcal{B}}_{\mathcal{B}'}=\left(\Big[\id_V\Big]^{\mathcal{B}'}_{\mathcal{B}}\right)^{-1}.$$
\end{enumerate}
\end{pro}

\dem \begin{enumerate}
\item Se sigue de las igualdades $\Big[v\Big]^{\mathcal{B}'}=\Big[\id_V(v)\Big]^{\mathcal{B}'}=\Big[\id_V\Big]^{\mathcal{B}'}_\mathcal{B}\Big[v\Big]^{\mathcal{B}}.$ 
\item Se sigue de la proposici\'on \ref{homym} aplicada a $\id_V$ y de la igualdad $\id_V^{-1}=\id_V.$
\end{enumerate}
\qed

\begin{pro}
Sean $\mathcal{B}_V,\mathcal{B}'_V$ bases de $V$ y sean $\mathcal{B}_W$, $\mathcal{B}'_W$ bases de $W$. Para toda $f\in\Hom_K(V,W)$, tenemos
\[
\Big[ f\Big]^{\mathcal{B}_W'}_{\mathcal{B}'_V}=\Big[\id_W\Big]^{\mathcal{B}'_W}_{\mathcal{B}_W}\Big[ f\Big]^{\mathcal{B}_W}_{\mathcal{B}_V}\Big[\id_V\Big]^{\mathcal{B}_V}_{\mathcal{B}'_V}
\]
\end{pro}

\dem La propiedad se sigue de la igualdad $f=\id_W\circ f\circ id_V$ y de la proposici\'on \ref{compmult}.\qed

\begin{obs}
Sean $\mathcal{B},\mathcal{B}'$ bases de $V$ y sea $f\in\End_K(V)$. Para
\[
A=\Big[ f\Big]^{\mathcal{B}}_{\mathcal{B}},\quad B=\Big[ f\Big]^{\mathcal{B}'}_{\mathcal{B}'},\textrm{ y}\quad C=\Big[\id_V\Big]^{\mathcal{B}}_{\mathcal{B}'}
\]
tenemos
\[
B=C^{-1}AC.
\]
\end{obs}

\begin{ejem}
Suponga que $\chara(K)$ es diferente de $2$, de forma que $-1\ne 1$ en $K$. Sea $f\in\Hom_K\left(K^2,K^2\right)$ el operador definido por
$$f(x,y)=(y,x).$$
Si $\mathcal{C}$ es la base  can\'onica $\left\{(1,0),(0,1)\right\}$ se tiene
$$\Big[ f\Big]^{\mathcal{C}}_{\mathcal{C}}=
\left[\begin{array}{rr}
0 & 1\\ 1 & 0
\end{array}\right]
$$
y si $\mathcal{B}$ es la base $\left\{(1,1),(1,-1)\right\}$ se tiene
$$\Big[ \id_{K^2}\Big]^{\mathcal{C}}_{\mathcal{B}}=
\left[\begin{array}{rr}
1 & 1\\ 1 & -1
\end{array}\right],
$$
luego obtenemos
\begin{align*}
\Big[ f\Big]^{\mathcal{B}}_{\mathcal{B}} &= \Big[ \id_{K^2}\Big]^{\mathcal{B}}_{\mathcal{C}}\Big[ f\Big]^{\mathcal{C}}_{\mathcal{C}}\Big[ \id_{K^2}\Big]^{\mathcal{C}}_{\mathcal{B}}\\
 &=\left[\begin{array}{rr}
1 & 1\\ 1 & -1
\end{array}\right]^{-1}
\left[\begin{array}{rr}
0 & 1\\ 1 & 0
\end{array}\right]
\left[\begin{array}{rr}
1 & 1\\ 1 & -1
\end{array}\right]\\
 &= \left[\begin{array}{rr}
1 & 0\\ 0 & -1
\end{array}\right].
\end{align*}
\end{ejem}

\begin{obs}[Vector de coordenadas en dimensi\'on infinita]\label{defnvectcoorinfty}
Dada una base $\mathcal{B}$ de $V$, con $\mathcal{B}=\{v_i\}_{i\in I}$, para cada $v\in V$ existe una \'unica combinaci\'on lineal de $\mathcal{B}$ igual a $v$. Para $i\in I$, sean $c_i\in K$ los coeficientes de esta combinaci\'on lineal, es decir tenemos $\sum_{i\in I} c_iv_i=v$.
La unicidad de esta combinaci\'on lineal nos permite identificar cada elemento en $v$ con el elemento $\Big[v\Big]^\mathcal{B}\in\left(K^I\right)_0$ definido por
\begin{eqnarray*}
\Big[v\Big]^\mathcal{B}: I & \longrightarrow & K\\
i & \longrightarrow & c_i.
\end{eqnarray*}
\end{obs}

\begin{obs}[Dimensi\'on infinita e isomorfismo]\label{dimiso}
La caracterizaci\'on de los espacios lineales, salvo isomorfismos, por su dimensi\'on se puede reescribir as\'i: si $\mathcal{B}_V$ y $\mathcal{B}_W$ son respectivamente bases $V$ y $W$, con $\mathcal{B}_V=\{v_j\}_{j\in J}$ y $\mathcal{B}_W=\{w_i\}_{i\in I}$, entonces $V\simeq_K W$ si y solo si existe una biyecci\'on $\phi:J\rightarrow I$.\\
Empecemos por establecer la suficiencia. Note que los mapas (ver notaci\'on en Observaci\'on \ref{defnvectcoorinfty})
\[
\begin{array}{rclcrcl}
V & \longrightarrow & \left(K^{J}\right)_0 &\qquad\textrm{y}\qquad& W & \longrightarrow & \left(K^I\right)_0 \\
v & \longmapsto & \Big[v\Big]^{\mathcal{B}_V} &\qquad& w & \longmapsto & \Big[w\Big]^{\mathcal{B}_W}, 
\end{array}
\]
son isomorfismos. Ahora, dada una biyecci\'on $\phi:J\rightarrow I$ entre los indices de las bases, la transformaci\'on lineal $\Phi\in\Hom_K\left(\left(K^J\right)_0,\left(K^I\right)_0\right)$ definida en la base $\{\delta_j\}_{j\in J}$ de $\left(K^J\right)_0$ (ver el ejemplo \ref{ejembas0}.3) por $\Phi(\delta_j)=\delta_{\phi(j)}$ es un isomorfismo. Tenemos as\'i $V\simeq_K\left(K^J\right)_0\simeq_K\left(K^I\right)_0\simeq_K W$.\\
Para establecer la necesidad necesitamos demostrar que si $V$ y  $W$ son isomorfos, entonces existe una biyecci\'on entre $J$ e $I$. Para esto basta demostrar que dos bases cualesquiera de $V$ est\'an en correspondencia biyectiva, pues si $f\in\Hom_K(W,V)$ es un isomorfismo, entonces la imagen $f(\mathcal{B}_W)$ es una base de $V$ y es un conjunto en biyecci\'on con $J$, y luego, si existe una biyecci\'on entre $\mathcal{B}_V$ y $f\left(\mathcal{B}_W\right)$, entonces hay una biyecci\'on entre $J$ e $I$. Supongamos entonces que $V$ tiene dimensi\'on infinita y sean $\mathcal{B}_V$ y $\mathcal{B}'_V$ respectivamente las bases $\{v_j\}_{j\in J}$ y $\{v_{j'}\}_{j'\in J'}$ de $V$. Para cada $j\in J$ denote por $J'_j$ al conjunto de indices $j'\in J'$ para los cuales $v_j$ tiene coordenada diferente de cero en la base $\mathcal{B}'$, es decir tenemos $J'_j=\left\{j'\in J'\ |\ \Big[v_j\Big]^{\mathcal{B}'_V}_{j'}\ne 0\right\}$. En particular $J'_j$ es la m\'inima colecci\'on de indices $j'\in J'$ con la propiedad $v_j\in\langle v_j'\rangle_{j'\in J'_j}$. Tenemos que, para $j\in J$, cada $J'_j$ es finito. Veamos que $\cup_{j\in J}J'_j=J'$. De hecho, en caso contrario, si existe $j'\in J'\setminus \bigcup_{j\in J}J'_j $ y $\{v_{j_1},\ldots,v_{j_n}\}\subseteq \mathcal{B}_V$ es tal que $v_{j'}$ est\'a en $\langle v_{j_1},\ldots,v_{j_n}\rangle$, entonces $v_{j'}$ est\'a en $\langle v_{k'}|\ k'\in \bigcup_{k=1}^n J'_{j_k} \rangle$ que es un subespacio de $\langle v_{k'}|\ k'\in J'\setminus\{j'\}\rangle$,
lo cual, por el lema \ref{inddep}, violar\'ia la independencia lineal de $\mathcal{B}'_V=\{v_{k'}\}_{k'\in J'}$. Defina $\mathcal{J}'$ como la uni\'on disyunta de todos los $J'_j$, para $j\in J$, es decir $\mathcal{J}'=\coprod_{j\in J} J'_j$.
Como $\mathcal{J}'$ es una uni\'on de conjuntos finitos y disyuntos indexada por $J$ y $J$ es infinito, entonces $J$ y $\mathcal{J}'$ son conjuntos biyectivos. Como tenemos $\cup_{j\in J}J'_j=J'$, en $\mathcal{J}'$ podemos inyectar a $J'$, y, as\'i tambi\'en, en $J$. Sim\'etricamente podemos inyectar $J$ en $J'$. Luego, por el Teorema de Schroeder-Bernstein, $J$ y $J'$ son biyectivos.
\end{obs}

\begin{obs}[Dimensi\'on arbitraria y descomposici\'on del dominio]
El teorema del rango se demostr\'o descomponiendo una base del dominio de la transformaci\'on lineal. Este resultado lo podemos generalizar a espacios de dimensi\'on infinita de la siguiente manera. Sea $f\in\Hom_K(V,W)$, entonces existe una base de $\mathcal{B}$ de $V$ tal que $\mathcal{B}$ es igual a la union de dos conjuntos disyuntos $\mathcal{B}_0$ y $ \mathcal{B}_1$, donde $\mathcal{B}_0$ es una base de $\ker(f)$ y $f\left(\mathcal{B}_1\right)$ es una base de $\im(f)$. De hecho, basta tomar una base $\mathcal{B}_0$ de $\ker(f)$ y extenderla a una de $V$ (ver Observaci\'on \ref{basesinfty}). El resto de detalles son similares a los de la demostraci\'on del teorema.
\end{obs}

\begin{obs}[Dimensi\'on infinita y transformaciones lineales]\label{unitrlinealinfty}
La proposici\'on \ref{unitrlin} de rigidez de las transformaciones lineales tambi\'en se puede generalizar a espacios de dimensi\'on infinita. Sea $S$ un conjunto generador de $V$. Dada una funci\'on $f_0: S\rightarrow W$, existe a los sumo una transformaci\'on lineal $f\in\Hom_K(V,W)$ para la cual se tiene $f(v)=f_0(v)$ para todo $v\in S$. Si adem\'as $S$ es linealmente independiente, entonces una tal transformaci\'on lineal $f$ existe. La demostraci\'on es fundamentalmente la misma que en el caso de base finita. Note que un caso particular de esta observaci\'on ya se us\'o en Observaci\'on \ref{dimiso}.
\end{obs}

\section{Suma y producto directo}

Sea $V$ un espacio vectorial sobre un cuerpo $K$.

\begin{defn}
Dados $V_1,\ldots,V_n\le V$, definimos su \emph{suma} como el conjunto $\left\{v_1+\ldots+v_n\in V\ |\ v_i\in V_i, i=1,\ldots,n \right\}$ que denotamos por $V_1+\ldots+V_n$ \'o por $\sum_{i=1}^n V_i$.
\end{defn}

\begin{pro}
Si $V_1,\ldots,V_n$ son subespacios de $V$, entonces $V_1+\ldots+V_n$ es un subespacion de $V$.
\end{pro}

\dem Usamos Propiedad \ref{subespsiysolosi}. Primero, note que $V_1+\ldots+V_n$ contiene al origen. Tome $v,v'\in  V_1+\ldots+V_n$ y $c\in K$. Para $i\in\{1,\ldots,n\}$, sean $v_i,v'_i\in V_i$ tales que se tiene $v=v_1+\ldots+v_n$ y $v'=v'_1+\ldots+v'_n$. Obtenemos as\'i $v+v'=(v_1+v'_1)+\ldots+(v_n+v'_n)$, y por ende, como $v_i+v'_i$ pertenece a $V_i$, para $i\in\{1,\ldots,n\}$, entonces $v+v'$ pertenece a $V_1+\ldots+V_n$. Igualmente, como tenemos $av_i\in V_i$, para $i\in\{1,\ldots,n\}$, de la igualdad $cv=cv_1+\ldots+cv_n$ vemos que $ v$ est\'a en $V_1+\ldots+V_n$.\qed

\begin{teo}\label{sumaint}
Si $V_1$ y $V_2$ son subespacios de $V$ de dimensi\'on finita, entonces $V_1\cap V_2$ y $V_1+V_2$ tambi\'en lo son. M\'as a\'un se tiene
\[
\dim(V_1+V_2)=\dim(V_1)+\dim(V_2)-\dim(V_1\cap V_2),
\]
o equivalentemente, $\dim(V_1)+\dim(V_2)=\dim(V_1+V_2)+\dim(V_1\cap V_2)$.
\end{teo}

\dem Como $V_1$ tiene dimensi\'on finita, y $V_1\cap V_2$ es un subespacio de $V_1$, entonces $V_1\cap V_2$ tambi\'en tiene dimensi\'on finita, por el teorema \ref{monodim}. Sean $n_1=\dim(V_1)$, $n_2=\dim(V_2)$, $p=\dim(V_1\cap V_2)$ y $\{v_1,\ldots,v_p\}$ una base de $V_1\cap V_2$. Extendemos esta base de $V_1\cap V_2$ a una base $\mathcal{B}_1$ de $V_1$ constituida por los vectores $v_1,\ldots,v_p,v'_{p+1},\ldots,v'_{n_1}$, y a una base $\mathcal{B}_2$ de $V_2$ constituida por los vectores $v_1,\ldots,v_p,v''_{p+1},\ldots,v''_{n_2}$. As\'i, el conjunto $\{v_1,\ldots,v_p,v'_{p+1},\ldots,v'_{n_1},v''_{p+1},\ldots,v''_{n_2}\}$ genera a $V_1+V_2$, luego este subespacio tiene dimensi\'on finita. Sea $$\mathcal{B}=\{v_1,\ldots,v_p,v'_{p+1},\ldots,v'_{n_1},v''_{p+1},\ldots,v''_{n_2}\}.$$ El teorema se sigue si demostramos que $\mathcal{B}$ es una base, para esto nos hace falta demostrar que es linealmente independiente y lo haremos usando la proposici\'on \ref{proplinind}. Suponga que $a_1,\ldots,a_p,a'_{p+1},\ldots,a'_{n_1},a''_{p+1},\ldots,a''_{n_2}$ son tales que se tiene
\[
0=\sum_{i=1}^p a_iv_i+\sum_{i=p+1}^{n_1}a'_iv'_i+\sum_{i=p+1}^{n_2} a''_iv''_i.
\]
Luego, si $v$ es el vector $\sum_{i=1}^p a_iv_i+\sum_{i=p+1}^{n_1}a'_iv'_i$, entonces $v$ est\'a en $V_1$ y tenemos $v=\sum_{i=p+1}^{n_2} a''_iv''_i$, y as\'i $v$ est\'a tambi\'en en $V_2$, y por ende $v$ est\'a en $V_1\cap V_2$. Sean $b_1\ldots,b_p\in K$ para los cuales se tiene
\[
v=b_1v_1+\ldots+b_pv_p,
\]
entonces obtenemos
\[
0=v-v=\sum_{i=1}^p (a_i-b_i)v_i+\sum_{i=p+1}^{n_1}a'_iv'_i.
\]
Por la independencia lineal de $\mathcal{B}_1$ tenemos $a'_{p+1}=\ldots=a'_{n_1}=0$ y
\[
0=\sum_{i=1}^p a_iv_i+\sum_{i=p+1}^{n_2} a''_iv''_i.
\]
Por la independencia lineal de $\mathcal{B}_2$, tenemos $a_1=\ldots=a_p=a''_{p+1}=\ldots=a''_{n_2}=0$. \qed

\begin{obs}
Note que si $i,s,n_1,n_2$ son tales que $i\le n_1\le n_2\le s$ y $s$ es menor que la dimensi\'on de $V$, entonces existen $V_1,V_2\le V$ con $\dim(V_1)=n_1$, $\dim(V_2)=n_2$, $\dim(V_1\cap V_2)=i$ y $\dim(V_1+V_2)=s$, siempre que
\[
n_1+n_2=s+i.
\]
De hecho si $\{v_1,\ldots,v_s\}$ es una es una colecci\'on de $s$ vectores linealmente independientes en $V$ basta tomar
$V_1=\langle v_1,\ldots,v_{n_1}\rangle$ y  $V_2=\langle v_1,\ldots,v_i,v_{n_1+1},\ldots,v_s\rangle$.
M\'as a\'un, si $V'_1$ y $V'_2$ son subespacios de $V$ para los se tiene $\dim(V'_1)=n_1$, $\dim(V'_2)=n_2$, $\dim(V'_1\cap V'_2)=i$ y $\dim(V'_1+V'_2)=s$, entonces existe un automorfismo $f\in\End_K(V)$ tal que $f(V_1)=V'_1$ y $f(V_2)=V'_2$.
\end{obs}

\begin{defn}
Sean $V_1,V_2\le V$. Decimos que $V_1$ y $V_2$ est\'an en \emph{posici\'on general} si $\dim(V_1+V_2)$ es tan grande y $\dim(V_1\cap V_2)$ es tan peque\~no como lo es posible.
\end{defn}

\begin{ejem}
Dos subespacios bidimensional de un espacio tridimensional est\'an en posici\'on general si su intersecci\'on es un espacio unidimensional. Dos subespacio cuatridimensional de un espacio sexadimensional est\'an en posici\'on general si su intersecci\'on es un espacio bidimensional. Dos subespacios tridimensionales en un espacio septadimensional est\'an en posici\'on general si su intersecci\'on es trivial.
\end{ejem}

\begin{defn}
Sean $V_1,\ldots,V_n\le V$, decimos que $V$ es la \emph{suma directa} de $V_1,\ldots,V_n$, lo cual denotamos por
\[
V=V_1\oplus\ldots\oplus V_n=\bigoplus_{i=1}^n V_i
\]
si para cada $v\in V$ existe un \'unico $(v_1,\ldots,v_n)\in V_1\times\ldots\times V_n$ que cumple
\[
v=v_1+\ldots+v_n
\]
\end{defn}

\begin{pro}\label{sumadirsiysolosi}
Sean $V_1,\ldots,V_n\le V$. Se tiene $V=\bigoplus_{i=1}^n V_i$ si y solo si $V_1,\ldots,v_n$ satisfacen las siguientes dos propiedades.
\begin{enumerate}
\item La suma $\sum_{i=1}^n V_i$ es igual a $V$.
\item Para todo $i\in\{1,\ldots,n\}$ se tiene $V_i\cap\sum_{j\ne i} V_j=\{O\}$.
\end{enumerate}
\end{pro}

\dem Suponga primero que tenemos $V=\bigoplus_{i=1}^n V_i$, luego por definici\'on tenemos $V=\sum_{i=1}^n V_i$. Por otro lado, sea $i\in\{1,\ldots,n\}$ y tome $v\in V_i\cap\sum_{j\ne i} V_j$. As\'i, existe $(v_1,\ldots,v_n)\in V_1\times\ldots\times V_n$ para el cual se cumple $v=-v_i=\sum_{j\ne i} v_j$ y por lo cual se tiene $O=v_1+\ldots+v_n$. Pero por otro lado, para $(O,\ldots,O)\in V_1\times\ldots\times V_n$ se tiene $O=O+\ldots+O$, luego, por unicidad de esta descomposici\'on se siguen las igualdades $v_1=\ldots=v_n=O$, $v=O$ y $V_i\cap\sum_{j\ne i} V_j=\{O\}$.\\
Rec\'iprocamente, suponga que $\sum_{i=1}^n V_i$ es igual a $V$ y que para todo $i\in\{1,\ldots,n\}$ se tiene $V_i\cap\sum_{j\ne i} V_j=\{O\}$. Sea $v\in V$. Entonces existe $(v_1,\ldots,v_n)\in V_1\times\ldots\times V_n$ para el cual se tiene $v=v_1+\ldots+v_n$. Veamos que esta descomposici\'on es \'unica. De hecho, si $(v'_1,\ldots,v'_n)\in V_1\times\ldots\times V_n$ es tal que se tiene $v=v'_1+\ldots+v'_n$, dado $i\in\{1,\ldots,n\}$, se obtiene
\[
\underbrace{v_i-v'_i}_{\in\ V_i}=\underbrace{\sum_{j\ne i} (v'_j-v_j)}_{\in\ \sum_{j\ne i} V_j}.
\]
Luego tenemos $v_i-v'_i\in V_i\cap\sum_{j\ne i} V_j=\{O\}$, es decir $v_i-v'_i=O$ y as\'i $v_i=v'_i$.\qed

\begin{prop}
Sean $V_1,\ldots, V_n\le V$ para los cuales se tiene $V=\sum_{i=1}^n V_i$ y suponga que $V$ tiene dimensi\'on finita. Entonces la siguientes propiedades son equivalentes.
\begin{enumerate}
\item Para todo $i\in\{1,\ldots,n\}$ se tiene $V_i\cap\sum_{j\ne i} V_j=\{0\}$.
\item Se tiene $\sum_{i=1}^n \dim(V_i)=\dim(V)$.
\end{enumerate}
\end{prop}

\dem Suponga primero que tenemos $V_i\cap\sum_{j\ne i} V_j=\{0\}$, para todo $i\in\{1,\ldots,n\}$. Por el teorema \ref{sumaint} se obtiene
$$\dim(V)=\dim(V_1)+\dim\left(\sum_{j>1} V_j\right).$$
La inclusi\'on $\left(V_2\cap\sum_{j>2} V_j\right)\subseteq \left(V_2\cap\sum_{j\ne 2} V_j\right)$ implica la igualdad $\left(V_2\cap\sum_{j> 2} V_j\right)=\{O\}$. Por el teorema \ref{sumaint} se obtiene
$$\dim\left(\sum_{j>1} V_j\right)=\dim(V_2)+\dim\left(\sum_{j>2} V_j\right).$$
Inductivamente, obtenemos 
\begin{eqnarray*}
\dim(V) & = & \dim(V_1)+\dim\left(\sum_{j>1} V_j\right)\\
             & = & \dim(V_1)+\dim(V_2)+\dim\left(\sum_{j>2} V_j\right)\\
             & \vdots & \\
             & = & \dim(V_1)+\ldots+\dim(V_n)          
\end{eqnarray*}
Suponga ahora que se tiene $\sum_{i=1}^n \dim(V_i)=\dim(V)$. Por el teorema \ref{sumaint} se tienen las desigualdades
\begin{eqnarray*}
\dim(V) & \le & \dim(V_1)+\dim\left(\sum_{j>1} V_j\right)\\
             & \le & \dim(V_1)+\dim(V_2)+\dim\left(\sum_{j>2} V_j\right)\\
             & \vdots & \\
             & \le & \sum_{i=1}^n \dim(V_i).      
\end{eqnarray*}
Pero se tiene $\sum_{i=1}^n \dim(V_i)=\dim(V)$, luego estas desigualdades son igualdades y en particular obtenemos $\dim(V)=\dim(V_1)+\dim\left(\sum_{j>1} V_j\right)$. De donde, por el mismo teorema \ref{sumaint}, se tiene $\dim\left(V_1\cap\sum_{j\ne 1} V_j\right)=0$, es decir $V_1\cap\sum_{j\ne 1} V_j=\{O\}$. Reordenando los subespacios $V_i$, $i=1,\ldots,n$, obtenemos $V_i\cap\sum_{j\ne i} V_j=\{0\}$, para todo $i\in\{1,\ldots,n\}$. \qed

\begin{defn}
Sea $p\in\End_K(V)$. Decimos que $p$ es una \emph{proyecci\'on} si se tiene $p\circ p=p$.
\end{defn}

\begin{obs}
Si $p\in\Hom_K(V,V)$ es una proyecci\'on y $V_0$ es la imagen de $p$ entonces se tiene $p(v_0)=v_0$ para todo $v_0\in V_0$. De hecho si $v_0$ pertenece a $V_0$, existe $v\in V$ que satisface $p(v)=v_0$, luego obtenemos $p(v_0)=p\circ p(v)=p(v)=v_0$.
\end{obs}

\begin{obs}
Suponga que tenemos $V=V_1\oplus V_2$, y defina los operadores $p_1,p_2\in\End_K(V)$ por $p_1(v)=v_1$ y $p_2(v)=v_2$ si se tiene $v=v_1+v_2$ con $(v_1,v_2)\in V_1\times V_2$. Note que $p_1$ y $p_2$ son proyecciones que cumplen $p_1\circ p_2=p_2\circ p_1=\underline{O}$ y $p_1+p_2=\id_V$. Similarmente, si tenemos $V=\bigoplus_{i=1}^{n}V_i$, podemos definir $n$ proyecciones $p_1,\ldots,p_n$ que satisfacen $p_i(V)=V_i$, para cada $i\in\{1,\ldots,n\}$, $\sum_{i=1}^n p_i=\id_V$, y $p_i\circ p_j=\underline{O}$ si $i\ne j$. Esto nos sugiere otra forma de caracterizar sumas directas, como lo sugiere el siguiente teorema.
\end{obs}

\begin{teo}\label{proysumadir}
Sean $p_1,\ldots,p_n\in\End_K(V)$ proyecciones y $V_1,\ldots,V_n$ sus respectivas im\'agenes. Si se tiene $\sum_{i=1}^n p_i=\id_V$ y $p_i\circ p_j=0$ para $i\ne j$, entonces se tiene $V=\bigoplus_{i=1}^n V_i$.
\end{teo}

\dem Usamos la propiedad \ref{sumadirsiysolosi} para establecer este teorema. Para ver que tenemos $V=\sum_{i=1}^nV_i$, dado $v\in V$, definimos $v_i\in V_i$ por $v_i=p_i(v)$, para $i\in\{1,\ldots,n\}$, y obtenemos
\[
v=\id_V(v)=\sum_{i=1}^n p_i(v)=\sum_{i=1}^n v_i.
\]
Para establecer $V_i\cap\sum_{j\ne i} V_j=\{O\}$, tome $i\in\{1,\ldots,n\}$ y $v\in V_i\cap\sum_{j\ne i} V_j$ y veamos que se sigue $v=O$. Como $v$ pertenece a $\sum_{j\ne i} V_j$, tenemos $v=\sum_{j\ne i} v_j$ para algunos $v_j\in V_j$, con $j\ne i$. En particular, existe $v'_j\in V$ que satisface $v_j=p_j(v'_j)$, para cada $j\ne i$, y as\'i obtenemos $v=\sum_{j\ne i} p_j(v_j)$. Por otro lado, como $v$ pertenece a $V_i$, existe $v'_i\in V$ que satisface $v=p_i(v'_i)$. Por ende, se siguen las igualdades
\[
v=p_i(v'_i)=p_i\circ p_i (v'_i)=p_i(v)=p_i\left(\sum_{j\ne i} p_j(v_j)\right)=\sum_{j\ne i} p_i\circ p_j(v_j)=O.
\]
\qed


\begin{obs}
Para terminar est\'a secci\'on, vamos a definir dos generalizaciones de la suma directa, que son la suma directa externa y el producto directo. En estas definiciones combinamos una colecci\'on, no necesariamente finita, de espacios para obtener un nuevo espacio. Cuando combinamos una colecci\'on finita de espacios, obtenemos espacio isomorfos.
\end{obs}

\begin{defn}
Sea $I$ una colecci\'on de indices y $\left\{V_i\right\}_{i\in I}$ una familia de espacios vectoriales sobre $K$.
\begin{enumerate}
\item El \emph{producto directo} de $\left\{V_i\right\}_{i\in I}$ es el espacio
\[
\prod_{i\in I} V_i=\left\{\phi: I\rightarrow \coprod_{i\in I}V_i\ \Big|\ \phi(i)\in V_i\right\},
\]
el cual es un espacio vectorial sobre $K$ bajo las operaciones
\[
(\phi+\psi)(i)=\phi(i)+\psi(i)\qquad (a\phi)(i)=a\psi(i)
\]
para todo $\phi,\psi\in \prod_{i\in I} V_i$ y $a\in K$; y,
\item la \emph{suma directa externa} de $\left\{V_i\right\}_{i\in I}$ por
\[
\bigoplus_{i\in I} V_i=\left\{\phi\in \prod_{i\in I} V_i \Big|\ \phi(i)\ne 0\textrm{ \'unicamente para finitos indices } i\in I\right\},
\]
el cual es un subespacio de $\prod_{i\in I} V_i$.
\end{enumerate} 
Si adem\'as $\left\{W_i\right\}_{i\in I}$ es otra familia de espacios vectoriales sobre $K$, y para cada $i\in I$ tenemos un $f_i\in\Hom_K(V_i,W_i)$, definimos:
\begin{enumerate}
\item el \emph{producto externo} de $\left\{f_i\right\}_{i\in I}$ por
\begin{eqnarray*}
\prod_{i\in I} f_i: \prod_{i\in I} V_i & \longrightarrow & \prod_{i\in I} W_i\\
                           \phi & \longmapsto & \left(\prod_{i\in I} f_i\right) (\phi): i\mapsto f_i\left(\phi(i)\right),
\end{eqnarray*}
el cual es una transformaci\'on lineal; y,
\item la \emph{suma directa externa} de $\left\{f_i\right\}_{i\in I}$ por
\begin{eqnarray*}
\bigoplus_{i\in I} f_i: \bigoplus_{i\in I} V_i & \longrightarrow & \bigoplus_{i\in I} W_i\\
                           \phi & \longmapsto & \left(\prod_{i\in I} f_i\right) (\phi): i\mapsto f_i\left(\phi(i)\right),
\end{eqnarray*}
la cual es la transformaci\'on lineal inducida por  $\prod_{i\in I} f_i$ entre los subespacios $\bigoplus_{i\in I} V_i$ y $\bigoplus_{i\in I} W_i$.
\end{enumerate}
\end{defn}

\begin{obs}
Note que si $\left\{V_i\right\}_{i\in I}$ es una colecci\'on de espacios vectoriales sobre $K$ indexada por los indices $i\in I$; y, $f_i\in\Hom_K(V,V_i)$ y $g_i\in\Hom_K(V_i,V)$, para todo $i\in I$, podemos definir las transformaciones lineales
\[
\begin{array}{rclcrcl}
f:V& \longrightarrow & \prod_{i\in I} V_i &\quad& g: \bigoplus_{i\in I} V_i & \longrightarrow & V \\
  v & \longmapsto & f(v):i\mapsto f_i(v) &\quad& \phi & \longmapsto & \sum_{i\in I} g_i\left(\phi(i)\right). 
\end{array}
\]
Note que la suma $\sum_{i\in I} g_i\left(\phi(i)\right)$ es finita pues $\phi(i)=0$ para todos los $i\in I$ salvo un n\'umero finito de indices.
\end{obs}

\begin{obs}
Note que, si $\mathcal{B}=\{v_i\}_{i\in I}\subseteq V$ es una base, entonces
\[
V\ \simeq_K\ \left(K^I\right)_0\ \simeq_K\ \bigoplus_{i\in I}K,
\]
y
\[
K^I\ \simeq_K\ \prod_{i\in I} K
\]
\end{obs}

\section{Espacios cocientes}

Sea $K$ un cuerpo y $V$, $W$ espacios vectoriales sobre $K$.

\begin{defn}
Sean $V_0\le V$ y $v\in V$. Definimos la \emph{translaci\'on de $V_0$ por $v$} como el conjunto
\[
v+V_0=\{v'\in V\ |\ v'=v+v_0,\ v_0\in V\}.
\]
\end{defn}

\begin{obs}
Tenemos $v+V_0=v'+V_0$ si y solo si $v-v'\in V_0$. De hecho, si $v+V_0=v'+V_0$, como $v\in v+V_0=v'+V_0$, existe $v_0\in V_0$ tal que $v=v'+v_0$, es decir $v-v'=v_0\in V_0$; rec\'iprocamente, si $v_0=v-v'\in V_0$, cualquier $w\in v+V_0$ es de la forma $w=v+w_0$ para alg\'un $w_0\in V_0$, en particular $w=v'+(v_0+w_0)\in v'+V_0$, y cualquier $w'\in v'+V_0$ es de la forma $w'=v'+w'_0$ para alg\'un $w'_0\in V_0$, en particular $w'=v+(w'_0-v_0)\in v+V_0$.
\end{obs}

\begin{defn}
Sea $V_0\le V$, el \emph{espacio cociente $V$ m\'odulo $V_0$} es el conjunto de traslaciones de $V_0$:
\[
V/V_0=\{v+V_0\ |\ v\in V\}
\] 
\end{defn}

\begin{prop}
Sean $V_0\le V$, $v,w,v',w'\in V$ y $a\in K$. Si $v+V_0=w+V_0$ y $v'+V_0=w'+V_0$ entonces $(v+v')+V_0=(w+w')+V_0$ y $av+V_0=aw+V_0$.
\end{prop}

\dem $v+V_0=w+V_0$ y $v'+V_0=w'+V_0$ si y solo si $v-w\in V_0$ y $v'-w'\in V_0$, en tal caso $(v+v')-(w+w')=(v-w)+(v'-w')\in V_0$, es decir $(v+v')+V_0=(w+w')+V_0$, y $av-aw=a(v-w)\in V_0$, es decir $av+V_0=aw+V_0$.\qed

\begin{pro}
Sea $V_0\le V$. El espacio cociente $V/V_0$ es un espacio vectorial sobre $K$ bajo las operaciones
\[
\left(v+V_0\right)+\left(v'+V_0\right)=(v+v')+V_0\qquad a\left(v+V_0\right)=av+V_0,
\]
y su origen es $0+V_0=V_0$. El mapa
\begin{eqnarray*}
\pi_{V_0}: V & \longrightarrow & V/V_0 \\
                v & \longmapsto & v+V_0
\end{eqnarray*}
es una transformaci\'on lineal sobreyectiva con $\ker(\pi_{V_0})=V_0$
\end{pro}

\dem La proposici\'on anterior garantiza que tales operaciones est\'an bien definidas, las propiedades de estas en Definici\'on \ref{defespvec} se heredan de las de $V$. La misma proposici\'on implica la linearidad de $\pi_{V_0}$. Por definici\'on de $V/V_0$, $\pi_{V_0}$ es sobreyectiva. Por \'ultimo, $v\in\ker(\pi_{V_0})$ si y solo si $\pi_{V_0}(v)=V_0$, es decir si y solo si $v+V_0=V_0$, o si y solo si $v\in V_0$. \qed

\begin{pro}
Sea $V_0\le V$ y suponga que $V$ tiene dimensi\'on finita, entonces
\[
\dim(V/V_0)=\dim(V)-\dim(V_0)
\]
\end{pro}

\dem Se sigue inmediatamente de Teorema \ref{teorango} y de la propiedad anterior.

\begin{teo}
Sean $f\in\Hom_K(V,W)$ y $V_0=\ker(f)$. Entonces existe una \'unica transformaci\'on lineal $f_{V_0}\in\Hom_K(V/V_0,W)$ tal que $f=f_{V_0}\circ\pi_{V_0}$. La transformaci\'on $f_{V_0}$ es inyectiva, y, si $f$ es adem\'as sobreyectiva, $f_{V_0}$ es un isomorfismo.
\end{teo}

\dem Note que $f(v)=f(v')$ si y solo si $v-v'\in V_0$, es decir si y solo si $v+V_0=v'+V_0$. Defina entonces
\begin{eqnarray*}
f_{V_0}: V/V_0 & \longrightarrow & W\\
            v+V_0 & \longmapsto      & f(v).
\end{eqnarray*}
As\'i, $f_{V_0}$ es lineal pues $f$ lo es, y adem\'as es inyectiva pues $f(v)=f(v')$ si y solo si $v+V_0=v'+V_0$. Por contrucci\'on $f=f_{V_0}\circ\pi_{V_0}$. Ahora si $f$ es sobreyectiva, entonces $f_{V_0}$ es biyectiva y as\'i un isomorfismo.\qed
 
\chapter{Estructura de las transformaciones lineales}

Sea $K$ un cuerpo y $V$, $W$ espacios vectoriales sobre $K$.

\begin{nota}
Suponga que $V$ y $W$ tienen dimensi\'on finita y denote $n=\dim(V)$, $m=\dim(W)$. Sean $\mathcal{B}_V=\{v_1,\ldots,v_n\}\subseteq V$ y $\mathcal{B}_W=\{w_1,\ldots,w_m\}\subseteq W$ bases. Dada $f\in\Hom_K(V,W)$,  la matriz $\{1,\ldots,m\}\times \{1,\ldots,n\}$
\[
A=\Big[f\Big]^{\mathcal{B}_W}_{\mathcal{B}_V},
\]
que representa a $f$ respecto a las bases $\mathcal{B}_V$ y $\mathcal{B}_W$, se denota por un arreglo rectangular $m\times n=|\mathcal{B}_W|\times|\mathcal{B}_V|$, con entradas en $K$, cuya $ij$-\'esima entrada es
\[
a_{ij}=\Big[f(v_j)\Big]^{\mathcal{B}_W}_i.
\]
De forma que
\[
f(v_j)=\sum_{i=1}^n a_{ij}w_i.
\]
En tal caso identificaremos a la matriz $A$ con el arreglo
\[
\left[\begin{array}{ccc}
a_{11} & \cdots & a_{1n}\\
\vdots & \ddots & \vdots\\
a_{m1} & \cdots & a_{mn}
\end{array}\right]
\]
Igualmente, a las matrices $\{1,\ldots,n\}\times\{*\}$ y $\{1,\ldots,m\}\times\{*\}$ de coordenadas en las bases $\mathcal{B}_V$ y $\mathcal{B}_W$ las identificaremos con los arreglos $n\times 1$ y $m\times 1$ con entradas en $K$, de tal forma que para $v\in V$ y $w\in W$ escribimos
\[
\Big[v\Big]^{\mathcal{B}_V}=\left[\begin{array}{c}
c_1\\
\vdots\\
c_n
\end{array}\right],
\qquad
\Big[w\Big]^{\mathcal{B}_W}=\left[\begin{array}{c}
d_1\\
\vdots\\
d_m
\end{array}\right],
\]
cuando $v=\sum_{j=1}^n c_jv_j$ y $w=\sum_{i=1}^m d_iw_i$. En particular
\[
\Big[f(v)\Big]^{\mathcal{B}_W}=\Big[f\Big]^{\mathcal{B}_W}_{\mathcal{B}_V}\Big[v\Big]^{\mathcal{B}_V}=
\left[\begin{array}{ccc}
a_{11} & \cdots & a_{1n}\\
\vdots & \ddots & \vdots\\
a_{m1} & \cdots & a_{mn}
\end{array}\right]
\left[\begin{array}{c}
c_1\\
\vdots\\
c_n
\end{array}\right]=
\left[\begin{array}{c}
\sum_j a_{1j}c_j\\
\vdots\\
\sum_j a_{mj}c_j
\end{array}\right].
\]
A los arreglos $m\times n$ los llamaremos tambi\'en \emph{matrices $m\times n$} y el espacio de estas lo denotamos por $M_{m\times n}(K)$. 
\end{nota}

\begin{obs}
Sean $A=(a_{ij})_{i,j=1}^n$ y $C=(c_{ij})_{i,j=1}^n$ matrices $n\times n$, entonces
\[
\tr(AC)=\sum_{i=1}^n\sum_{j=1}^na_{ij}c_{ji}=\sum_{j=1}^n\sum_{i=1}^nc_{ij}a_{ji}=\sum_{i=1}^n\sum_{j=1}^nc_{ij}a_{ji}=\tr(CA).
\]
Ahora, si $V$ tiene dimensi\'on finita igual a $n$, y $f\in\Hom_K(V,V)$, dadas dos bases $\mathcal{B},\mathcal{B}'\subseteq V$, tenemos dos matrices $n\times n$ que representan a $f$, $A=\Big[f\Big]^{\mathcal{B}}_{\mathcal{B}}$ y $B=\Big[f\Big]^{\mathcal{B}'}_{\mathcal{B}'}$. Entonces, si adem\'as $C=\Big[\id_V\Big]_{\mathcal{B}'}^{\mathcal{B}}$,
\[
B=C^{-1}AC,
\]
y
\begin{eqnarray*}
\tr(B) & = &\tr(C^{-1}AC)=\tr(ACC^{-1})\\
         & = &\tr(A)\\
\det(B) & = & \det(C^{-1}AC)=\det(C)^{1}\det(A)\det(C)\\
           & = &\det(A)
\end{eqnarray*}
Es decir la traza y el determinante de una matriz de representaci\'on de un operador lineal, respecto a la misma base para el dominio y el rango, es independiente de la base escogida.
\end{obs}

\begin{defn}
Suponga que $V$ tiene dimensi\'on finita, sean $f\in\Hom_K(V,V)$ y $\mathcal{B}\subseteq V$ una base. Definimos el \emph{determinante} y la \emph{traza} de $f$ respectivamente por
\[
\det(f)=\det\left(\Big[f\Big]^{\mathcal{B}}_{\mathcal{B}}\right)\qquad \tr(f)=\tr\left(\Big[f\Big]^{\mathcal{B}}_{\mathcal{B}}\right).
\]
\end{defn}

\section{Descomposici\'on directa}

\begin{defn}
Sean $V_1,V_2\le V$, decimos que $V_1$ y $V_2$ forman una \emph{descomposici\'on directa} de $V$ si $V=V_1\oplus V_2$.
\end{defn}

\begin{teo}
Sea $f\in\Hom_K(V,W)$. Entonces existe descomposiciones directas $V=V_0\oplus V_1$ y $W=W_1\oplus W_2$ tales que $\ker(f)=V_0$, $\im(f)=W_1$. En particular $f$ induce un isomorfismo entre $V_1$ y $W_1$.
\end{teo}

\dem Sea $\mathcal{B}_0$ una base de $V_0=\ker(f)$, la cual extendemos a una base $\mathcal{B}=\mathcal{B}_0\cup \mathcal{B}_1$ de $V$. Defina $V_1=\langle \mathcal{B}_1\rangle$. As\'i pues $V_0+V_1=V$ y $V_0\cap V_1=\{0\}$, en particular $V=V_0\oplus V_1$. Por otro lado, si $v,v'\in V_1$ son tales que $f(v)=f(v')$, entonces $v-v'\in \ker(f)=V_0$, luego $v-v'\in V_0\cap V_1=\{0\}$, luego $v=v'$. Es decir la restricci\'on de $f$ a $V_1$ es inyectiva.\\
Sea $\mathcal{B}'_1=f(\mathcal{B}_1)$. Como $f$ es inyectiva en $V_1$, es decir $f(v)=0$ con $v\in V_1$ si y solo si $v=0$, $\mathcal{B}'_1$ es linealmente independiente. Defina $W_1=\langle \mathcal{B}'_1\rangle$, de forma que $\mathcal{B}'_1$ es una base de $W_1$ y $W_1=f(V_1)$. Por construcci\'on $\im(f)=W_1$; pues, dado $w\in\im(f)$, existe $v\in V$ tal que $w=f(v)$, si $v=v_0+v_1$ con $(v_0,v_1)\in V_0\times V_1$, $w=f(v)=f(v_0)+f(v_1)=f(v_1)$. Finalmente, extienda $\mathcal{B}'_1$ a una base $\mathcal{B}'=\mathcal{B}'_1\cup \mathcal{B}'_2$ de $W$. Si $W_2=\langle \mathcal{B}'_2\rangle$, $V=V_0\oplus V_1$ y $W=W_1\oplus W_2$ son las descomposiciones directas buscadas. Como $f$ es inyectiva en $V_1$ y $f(V_1)=W_1$, $f$ induce un isomorfismo entre $V_1$ y $W_1$. \qed

\begin{coro}
Suponga que $V$ y $W$ tienen dimensi\'on finita, sea $f\in\Hom_K(V,W)$, y denote $n=\dim(V)$, $m=\dim(W)$ y $r=\dim(\im(f))$. Entonces existen bases $\mathcal{B}=\{v_j\}_{j=1}^n\subseteq V$ y $\mathcal{B}'=\{w_i\}_{i=1}^m\subseteq W$ tales que, si
\[
A=\Big[f\Big]^{\mathcal{B}'}_{\mathcal{B}}=(a_{ij}),
\]
$a_{ii}=1$ si $0\le i\le r$ y $a_{ij}=0$ si $i\ne j$, o si $r<i$ e $i=j$. Es decir
\[
A=\left[\begin{array}{c|c}
I_r & 0\\
\hline
0   & 0
\end{array}\right]
\]
donde $I_r$ denota la matriz $r\times r$ con unos en diagonal y ceros en el resto de entradas y $0$ los or\'igenes de $M_{r\times (n-r)}(K)$, $M_{(m-r)\times r}(K)$ y $M_{(m-r)\times(n-r)}(K)$.
\end{coro}

\dem Tome $\mathcal{B}_0$, $\mathcal{B}_1$, $\mathcal{B}'_1$ y $\mathcal{B}'_2$ como en la prueba del teorema, y denote $v_1,\ldots,v_n\in V$ y $w_1,\ldots,w_m\in W$ de forma que
\[
\mathcal{B}_1=\{v_1,\ldots,v_r\}, \mathcal{B}_0=\{v_{r+1},\ldots,v_n\}, \mathcal{B}'_1=\{w_1,\ldots,w_r\}, \mathcal{B}'_2=\{w_{r+1},\ldots,v_m\}.
\]
Las bases $\mathcal{B}=\{v_1,\ldots,v_n\}$ y $\mathcal{B}'=\{w_1,\ldots,w_m\}$ son tales que $\Big[f\Big]^{\mathcal{B}'}_{\mathcal{B}}$ tiene la forma buscada.\qed

\section{Espacios invariantes y espacios propios}

\begin{defn}
Sean $f\in\Hom_K(V,V)$ y $V_0\le V$. Decimos que $V_0$ es \emph{invariante bajo $f$} si $f(V_0)\subseteq V_0$. La restricci\'on de $f$ a $V_0$ la denotamos $f_{V_0}$, es decir $f_{V_0}\in\Hom_K(V_0,V_0)$ es el operador definido por:
\begin{eqnarray*}
f_{V_0}: V_0 & \longrightarrow & V_0\\
 v_0 & \longmapsto & f(v_0)
\end{eqnarray*}
\end{defn}

\begin{defn}
Sean $I$ un conjunto y $A\in M_{I\times I}(K)$. Decimos que $A$ es \emph{diagonal} si $A(i,j)=0$ siempre que $i\ne j$. Sea $f\in\Hom_K(V,V)$, decimos que $f$ es diagonalizable si $\Big[f\Big]^{\mathcal{B}}_{\mathcal{B}}$ es diagonal para alguna base $\mathcal{B}$ de $V$.
\end{defn}

\begin{teo}\label{diagosiysolosi}
Sea $f\in\Hom_K(V,V)$. Entonces $f$ es diagonalizable si y solo si existe una familia $\{V_i\}_{i\in I}$ de subespacios unidimensional de $V$, invariantes bajo $f$, tal que $V=\bigoplus_{i\in I}V_i$.
\end{teo}

\dem Note primero que si $\mathcal{B}=\{v_i\}_{i\in I}\subseteq V$ es una base, entonces
\[
V=\bigoplus_{i\in I}\langle v_i\rangle.
\]
Suponga primero que $f$ es diagonalizable y sea $\mathcal{B}=\{v_i\}_{i\in I}\subseteq V$ base tal que $\Big[f\Big]^{\mathcal{B}}_{\mathcal{B}}$ es diagonal. Para cada $i\in I$ defina $V_i=\langle v_i\rangle$. Ahora, dados $i,j\in I$,
\[
\Big[f(v_j)\Big]^{\mathcal{B}}_i=\sum_{l\in I}\Big[f\Big]^{\mathcal{B}}_{\mathcal{B},(i,l)}\Big[v_j\Big]^{\mathcal{B}}_l=\Big[f\Big]^{\mathcal{B}}_{\mathcal{B},(i,j)}.
\]
As\'i, como $\Big[f\Big]^{\mathcal{B}}_{\mathcal{B}}$ es diagonal,
\[
f(v_j)=\sum_{i\in I}\Big[f(v_j)\Big]^{\mathcal{B}}_{i}v_i=\sum_{i\in I}\Big[f\Big]^{\mathcal{B}}_{\mathcal{B},(i,j)}v_i=\Big[f\Big]^{\mathcal{B}}_{\mathcal{B},(j,j)}v_j,
\]
es decir que si $\lambda_j=\Big[f\Big]^{\mathcal{B}}_{\mathcal{B},(j,j)}$, entonces $f(v_j)=\lambda_j v_j\in V_j$, luego $V_j$ es invariante bajo $f$. De donde
\[
V=\bigoplus_{j\in I}V_j
\]
es una descomposici\'on de $V$ en espacios unidimensional invariantes bajo $f$.\\
Suponga ahora que $V=\bigoplus_{i\in I}V_i$, donde $\{V_i\}_{i\in I}$ es una familia subespacios unidimensional de $V$ invariantes bajo $f$. Para cada $i\in I$ sea $v_i\in V_i$, con $v_i\ne 0$, de tal forma que $V_i=\langle v_i\rangle$. Luego
\[
\mathcal{B}=\left\{v_i\right\}_{i\in I},
\]
es una base de $V$; y, adem\'as, como cada $V_i$ es invariante bajo $f$ y unidimensional, $f(v_i)=\lambda_iv_i$ para alg\'un $\lambda_i\in K$. As\'i pues
\[
\Big[f\Big]^{\mathcal{B}}_{\mathcal{B},(i,j)}=\Big[f(v_j)\Big]^{\mathcal{B}}_i=
\left\{\begin{array}{rl} \lambda_i &\textrm{ si } i=j\\ 0 &\textrm{ si } i\ne j \end{array}\right.,
\]
es decir $\Big[f\Big]^{\mathcal{B}}_{\mathcal{B}}$ es diagonal.\qed

\begin{defn}
Sea $f\in\Hom_K(V,V)$ y $V_0\le V$ con $\dim(V_0)=1$. Decimos que $V_0$ es un \emph{espacio propio} de $f$ si $V_0$ es invariante bajo $f$. En tal caso, a los elementos en $V_0$ diferentes del origen los llamamos \emph{vectores propios} de $f$. Dado un vector propio $v$ en $V_0$, existe $\lambda\in K$ tal que $f(v)=\lambda v$; a este $\lambda$ lo llamamos \emph{valor propio} (asociado a $V_0$ o a $v$) de $f$. Igualmente en tal caso, decimos que $V_0$ es un espacio propio (\'o $v$ es un vector propio) asociado a $\lambda$.
\end{defn}

\begin{obs}
Del mismo modo en que definimos arreglos $m\times n$, donde $n$ y $m$ son enteros positivos, con entradas en $K$, podemos definir arreglos $m\times n$ con entradas en conjunto de polinomios con coeficientes en $K$ en la variable $t$. A este conjunto lo denotaremos $M_{m\times n}(K[t])$. Los elementos en $K[t]$ se pueden multiplicar y sumar entre si en base a operaciones de multiplicaci\'on y suma de $K$. De esta forma podemos igualmente hablar del determinante y de la traza de un matriz $n\times n$ con entradas en $K[t]$, los cuales ser\'an igualmente polinomios en $K[t]$. 
\end{obs}

\begin{obs}
Sea $n\in\mathbb{Z}_{>0}$ y $A\in M_{n\times n}(K)$. Dada cualquier $C\in M_{n\times n}(K)$, invertible, tenemos
\[
\det(t I_n-A)=\det\Big(C^{-1}(t I_n-A)C\Big)=\det(t I_n-C^{-1}AC)
\]
donde $t I_n-A,t I_n-C^{-1}AC\in M_{n\times n}(K[t])$. Esta observaci\'on nos permite formular la siguiente definici\'on.
\end{obs}

\begin{defn}
Suponga que $V$ tiene dimensi\'on finita y denote $n=\dim(V)$. Dado $f\in\Hom_K(V,V)$, definimos el \emph{polinomio carater\'istico} de $f$ por
\[
P_f(t)=\det(t I_n-A)\in K[t]
\]
donde $A=\Big[f\Big]^{\mathcal{B}}_{\mathcal{B}}$, y $\mathcal{B}\subseteq V$ es una base.
\end{defn}

\begin{teo}
Suponga que $V$ tiene dimensi\'on finita. Sean $f\in\Hom_K(V,V)$ y $\lambda\in K$. Entonces, $\lambda$ es un valor propio de $f$ si y solo si $P_f(\lambda)=0$.
\end{teo}

\dem Sea $\mathcal{B}\subseteq V$ una base. El escalar $\lambda\in K$ es un valor propio de $f$ si y solo si existe $v\in V$, con $v\ne 0$, tal que $f(v)=\lambda v$, o, equivalentemente, tal que $\left(\lambda\id_V-f\right)(v)=0$. Es decir $\lambda\in K$ es un valor propio de $f$ si y solo si $\lambda\id_V-f$ no es inyectiva, lo que equivale a
\[
0=\det(\lambda\id_V-f)=\det\left(\lambda I_n-\Big[f\Big]^{\mathcal{B}}_{\mathcal{B}}\right)=P_f(\lambda).
\]
\qed

\begin{defn}
Sean $P(t)\in K[t]$ y $f\in\Hom_K(V,V)$. Definimos el operador $P(f)\in\Hom_K(V,V)$ por
\[
P(f)=a_nf^n+a_{n-1}f^{n-1}+\ldots+a_1f+a_0\id_V
\]
cuando $P(t)=a_nt^n+a_{n-1}t^{n-1}+\ldots+a_1t+a_0$, donde para todo entero positivo $k$
\[
f^k=\underbrace{f\circ\ldots\circ f}_{k-\textrm{veces}}.
\]
\end{defn}

\begin{obs}
\begin{enumerate}
\item Sea $C\in M_{m\times n}(K[t])$, donde $m$ y $n$ son enteros positivos, cuya $ij$-\'esima entrada denotamos $c_{ij}(t)$. Dado $f\in\Hom_K(V,V)$ definimos la transformaci\'on lineal
\begin{eqnarray}
C_f:\underbrace{V\times\ldots\times V}_{n-\textrm{veces}} &\longrightarrow & \underbrace{V\times\ldots\times V}_{m-\textrm{veces}}
\end{eqnarray}
por
\[
C_f(v_1,\ldots,v_n)=\left(\sum_{j=1}^nc_{1j}(f)(v_j),\ldots,\sum_{j=1}^nc_{mj}(f)(v_j)\right).
\]
\item Note que si $C_1\in M_{m\times n}(K[t])$ y $C_2\in M_{l\times m}(K[t])$, donde $l$, $m$ y $n$ son enteros positivos, dado $f\in\Hom_K(V,V)$, 
\[
\left(C_1C_2\right)_f=C_{1 f}\circ C_{2 f}. 
\]
\item Dado $B\in M_{n\times n}(K[t])$, donde $n$ es un entero positivo, cuya $ij$-\'esima entrada es $b_{ij}(t)$, denotamos por $\tilde{B}$ su matriz de cofactores, es decir la matriz $n\times n$ con entradas en $K[t]$ cuya $ij$-\'esima entrada es
\[
\tilde{b}_{ij}=(-1)^{i+j}\det(B_{ij})
\]
donde $B_{ij}$ es el arreglo $(n-1)\times(n-1)$ que se obtiene a partir de $B$ eliminando la $i$-\'esima fila y la $j$-\'esima columna. De tal forma que
\[
B\tilde{B}^\intercal=\left[\begin{array}{cccc}
\det(B) & 0 & \cdots & 0\\
0 & \det(B) & \cdots & 0\\
\vdots & \vdots & \ddots &\vdots\\
0 & 0 &\cdots & \det(B)
\end{array}\right]=(B\tilde{B}^\intercal)^\intercal=\tilde{B}B^\intercal
\]
donde $\tilde{B}^\intercal$ es la transpuesta de $\tilde{B}$, es decir la matriz $n\times n$ cuya $ij$-\'esima entrada es la entrada $ji$-\'esima de $\tilde{B}$; similarlmente para $B^\intercal$ y $(B\tilde{B}^\intercal)^\intercal$.
\end{enumerate}
\end{obs}

\begin{teo}[Caley-Hamilton]
Suponga que $V$ tiene dimensi\'on finita y sea $f\in\Hom_K(V,V)$. Entonces $P_f(f)=0$.
\end{teo}

\dem Sean $n=\dim(V)$ y $\mathcal{B}=\{v_1,\ldots,v_n\}\subseteq V$ una base. Defina
\[
\Big[f\Big]^{\mathcal{B}}_\mathcal{B}=A=(a_{ij})_{i,j=1}^n
\]
de forma que
\[
f(v_j)=\sum_{i=1}^na_{ij}v_i.
\]
Considere la matriz $B=tI_n-A\in M_{n\times n}(K[t])$. Entonces $\tilde{B}B^\intercal=P_f(t)I_n$. Ahora
\begin{eqnarray*}
\left(B^\intercal\right)_f(v_1,\ldots, v_n)& = & \left(f(v_1)-\left(\sum_{j=1}^na_{j1}v_j\right),\ldots, f(v_n)-\left(\sum_{j=1}^na_{jn}v_j\right)\right)\\
   & = & \left(0,\ldots, 0\right),
\end{eqnarray*}
por un lado; pero, por el otro
\begin{eqnarray*}
\left(P_f(f)(v_1),\ldots, P_f(f)(v_n)\right) & = & \left(P_f(t)I_n\right)_f(v_1,\ldots,v_n)\\
    & = & \left( \tilde{B}B^T \right)_f (v_1,\ldots, v_n)\\
    & = & \tilde{B}_f\circ\left(B^\intercal\right)_f(v_1,\ldots,v_n)\\
    & = & \tilde{B}_f(0,\ldots, 0)\\
    & = & (0,\ldots, 0).
\end{eqnarray*}
Luego $\mathcal{B}\subseteq \ker\left(P_f(f)\right)$ y as\'i $P_f(f)=0$.\qed

\begin{obs}\label{polyconmutan}
Note que si $P_1(t),P_2(t)\in K[t]$, $P(t)=P_1(t)P_2(t)$ y $f\in\Hom_K(V,V)$, entonces $P(f)=P_1(f)\circ P_2(f)=P_2(f)\circ P_1(f)$, pues $\left(af^m\right)\circ\left( bf^n\right)=\left( bf^n\right)\circ \left(af^m\right)$ para todo $n,m\in\mathbb{Z}_{\ge 0}$ y $a,b\in K$.
\end{obs}

\begin{pro}
Sea $P(t)\in K[t]$ y $f\in\Hom_K(V,V)$, entonces $V_0=\ker\left(P(f)\right)$ es invariante bajo $f$.
\end{pro}

\dem Sea $v\in V_0$, luego $P(f)\left(f(v)\right)=P(f)\circ f(v)=f\circ P(f)(v)=f(0)=0$. Es decir $f(v)\in\ker\left(P(f)\right)=V_0$.\qed

\begin{pro}
Sean $P(t)\in K[t]$ y $f\in\Hom_K(V,V)$ tales que $P(f)=0$. Si $P_1(t),P_2(t)\in K[t]$ son tales que $P(t)=P_1(t)P_2(t)$ y $\left(P_1(t),P_2(t)\right)=1$, entonces
\[
V=V_1\oplus V_2
\] 
donde $V_1=\ker\left(P_1(f)\right)$ y $V_2=\ker\left(P_2(f)\right)$. M\'as a\'un $V_1$ y $V_2$ son invariantes bajo $f$ y existen polinomios $\Pi_1(t),\Pi_2(t)\in K[t]$, tales que
\[
\Pi_1(f)=p_1\qquad\textrm{y}\qquad\Pi_2(f)=p_2
\]
son las proyecciones en $V_1$ y $V_2$.
\end{pro}

\dem Sean $Q_1,Q_2\in K[t]$ tales que $Q_1(t)P_1(t)+P_2(t)Q_2(t)=1$, luego
\[
Q_1(f)\circ P_1(f)+P_2(f)\circ Q_2(f)=\id_V
\]
en particular, dado $v\in V$
\[
\begin{array}{rcccc}
v & = & \underbrace{Q_1(f)\circ P_1(f)(v)} & + & \underbrace{P_2(f)\circ Q_2(f)(v)}\\
  & = & v_2 & + & v_1.
\end{array}
\]
Ahora
\[
P_2(f)(v_2)=P_2(f)\circ Q_1(f)\circ P_1(f)(v)=Q_1(f)\circ P_1(f)\circ P_2(f) (v)=Q_1(f)\circ P(f)(v)=0
\]
y
\[
P_1(f)(v_1)=P_1(f)\circ Q_2(f)\circ P_2(f)(v)=Q_2(f)\circ P_1(f)\circ P_2(f) (v)=Q_2(f)\circ P(f)(v)=0
\]
luego $v_2\in V_2$ y $v_1\in V_1$. As\'i $V=V_1+V_2$. Ahora si asumimos que $v\in V_1\cap V_2$, $P_1(f)(v)=0=P_2(f)(v)$, entonces $v_1=0=v_2$, luego $v=0$.\\
Por la propiedad anterior $V_1$ y $V_2$ son invariantes bajo $f$. Finalmente si $\Pi_1(t)=Q_2(t)P_2(t)$ y $\Pi_2(t)=Q_1(t)P_1(t)$, tenemos
\[
\Pi_2(t)+\Pi_1(t)=1,
\]
y
\[
\Pi_2(f)+\Pi_1(f)=\id_V.
\]
Ahora,
\[
\Pi_1(t)\Pi_2(t)=Q_2(t)P_2(t)Q_1(t)P_1(t)=Q_2(t)Q_1(t)P(t)
\]
luego
\[
\Pi_1(f)\circ\Pi_2(f)=0,
\]
y, como
\[
\Pi_2(t)=\Pi_2(t)\left(\Pi_2(t)+\Pi_1(t)\right)=\left(\Pi_2(t)\right)^2+\Pi_2(t)\Pi_1(t)
\]
entonces
\[
\Pi_2(f)=\left(\Pi_2(f)\right)^2.
\]
Similarmente obtenemos
\[
\Pi_1(f)=\left(\Pi_1(f)\right)^2.
\]
Luego, si $\Pi_1(f)=p_1$ y $\Pi_2(f)=p_2$, por Teorema \ref{proysumadir}, $p_1$ y $p_2$ son proyecciones sobre $V_1$ y $V_2$ respectivamente.\qed

\begin{ejem}
Sea $p\in\Hom_K(V,V)$ una proyecci\'on, es decir $p^2=p$. Si $P(t)=t^2-t$ entonces $P(p)=p^2-p=0$ y $P(t)=t(t-1)=P_1(t)P_2(t)$ donde $P_1(t)=t-1$ y $P_2(t)=t$. Note que $\left(P_1(t),P_2(t)\right)=1$ y $$-P_1(t)+P_2(t)=1.$$
As\'i, por la demostraci\'on de la propiedad anterior obtenemos que si \begin{align*}
V_1 & =\ker\left(P_1(p)\right)=\ker\left(p-\id_V\right)\\
V_2 & =\ker\left(P_2(p)\right)=\ker\left(p\right)
\end{align*}
entonces $V=V_1\oplus V_2$ y si
\begin{align*}
\Pi_1(t) & =P_2(t)=t\\
\Pi_2(t) & =-P_1(t)=1-t
\end{align*}
entonces $p_1=\Pi_2(p)=p$ y $p_2=\Pi_1(p)=\id_V-p$ son proyecciones respectivamente sobre $V_1$ y $V_2$ tales que $p_1+p_2=\id_V$.
\end{ejem}

\begin{ejem}
Suponga que $\chara(K)\ne 2$, de forma que $-1\ne 1$. Sea $f\in\Hom_K\left(K^2,K^2\right)$ el operador definido por
$$f(x,y)=(y,x).$$
Si $\mathcal{C}=\left\{(1,0),(0,1)\right\}$ es la base can\'onica entonces
$$\left[ f\right]^{\mathcal{C}}_{\mathcal{C}}=
\left[\begin{array}{rr}
0 & 1\\ 1 & 0
\end{array}\right]
$$
y $P_f(t)=t^2-1=(t-1)(t+1)$. Por el teorema del Caley-Hamilton $P_f(f)=0$, entonces si $P_1(t)=t-1$ y $P_2(t)=t+1$, por la propiedad anterior, $K^2=V_1\oplus V_2$ donde $V_1=\ker\left(f-\id_{K^2}\right)$ y $V_2=\ker\left(f+\id_{K^2}\right)$. Como
$$-\frac{1}{2}P_1(t)+\frac{1}{2}P_2(t)=1$$
entonces
$$p_1=\frac{1}{2}\left(f+\id_{K^2}\right)\quad\textrm{ y }\quad p_2=-\frac{1}{2}\left(f-\id_{K^2}\right)$$
son las proyecciones sobre $V_1$ y $V_2$. Expl\'icitamente
$$p_1(x,y)=\frac{1}{2}(x+y,x+y)\quad\textrm{ y }\quad p_2(x,y)=\frac{1}{2}(x-y,y-x).$$
\end{ejem}

\begin{obs}
Note que bajo las condiciones de la propiedad anterior, si denotamos por $f_i\in\Hom_K(V_i,V_i)$, para $i=1,2$ la restricci\'on de $f$ a $V_i$, es decir $f_i(v_i)=f(v_i)\in V_i$ para todo $v_i\in V_i$, tenemos que $P_i(f_i)=0$, pues $V_i=\ker\left(P_i(f)\right)$ as\'i que $P_i(f_i)(v_i)=P_i(f)(v_i)=0$. As\'i, inductivamente, podemos aplicar la propiedad a cualquier descomposici\'on de $P_i(t)$ en factores primos relativos para obtener el siguiente resultado. 
\end{obs}

\begin{pro}\label{prodescomp}
Sean $P(t)\in K[t]$ y $f\in\Hom_K(V,V)$ tales que $P(f)=0$. Si $P_1(t),P_2(t),\ldots,P_n(t)\in K[t]$ son tales que $P(t)=P_1(t)P_2(t)\ldots P_n(t)$ y $\left(P_i(t),P_j(t)\right)=1$ siempre que $i\ne j$, entonces
\[
V=V_1\oplus V_2\oplus\ldots\oplus V_n
\] 
donde $V_i=\ker\left(P_i(f)\right)$, $i=1,\ldots,n$. M\'as a\'un cada $V_i$ es invariante bajo $f$ y existen polinomios $\Pi_1(t),\ldots,\Pi_n(t)\in K[t]$, tales que
\[
\Pi_1(f)=p_1\qquad,\ldots,\qquad\Pi_n(f)=p_n
\]
son las proyecciones sobre $V_1,\ldots,V_n$.
\end{pro}

\dem Falta mostrar la existencia de $\Pi_1(t),\ldots,\Pi_n(t)\in K[t]$. De hecho, para $i=1,\ldots,n$ sea
\[
R_i(t)=\prod_{j\ne i}P_j(t), 
\]
Entonces $(R_1(t),\ldots,R_n(t))=1$. Tome $Q_1(t),\ldots, Q_n(t)\in K[t]$ tales que
\[
Q_1(t)R_1(t)+\ldots Q_n(t)R_n(t)=1.
\]
De forma que, si $\Pi_i(t)=Q_i(t)R_i(t)$, $i=1,\ldots,n$. Entonces, similarmente a la demostraci\'on anterior obtenemos
\[
\Pi_1(f)+\ldots+\Pi_n(f)=\id_V,
\]
$\Pi_i(f)\circ\Pi_j(f)=0$, si $i\ne j$, y $\left(\Pi_i(f)\right)^2=\Pi_i(f)$. El resultado se sigue de Teorema \ref{proysumadir}.\qed

\begin{ejem}
Sea $f\in\textrm{Hom}_{\mathbb{Q}}(\mathbb{Q}^4,\mathbb{Q}^4)$ el operador definido por
$$f(x,y,z,w)=(x-y+w,-x-z+2w,2x-y-z-w,2x-y)$$
Si $\mathcal{C}$ la base can\'onica de $\mathbb{Q}^4$ entonces:
$$\Big[f\Big]_\mathcal{C}^\mathcal{C}=\left[\begin{array}{rrrr}
1 & -1 & 0 & 1\\
-1 & 0 & -1 & 2\\
2 & -1 & -1 & -1\\
2 & -1 & 0 & 0
\end{array}\right]$$
y $P_f(t)=P_1(t)P_2(t)P_3(t)$ donde $P_1(t)=(t+1)$, $P_2(t)=(t-1)$, $P_3(t)=(t^2-2)$. Luego, por la propiedad anterior y el teorema de Caley-Hamilton, si para $i=1,2,3$ definimos $V_i=\ker(P_i(f))$, cada uno de estos espacios es invariante bajo $f$ y tenemos la descomposici\'on:
$$\mathbb{Q}^4=V_1\oplus V_2\oplus V_3.$$
Si usamos la misma notaci\'on de la demostraci\'on anterior, tenemos $R_1(t)=P_2(t)P3(t)=(t-1)(t^2-2)$, $R_2(t)=P_1(t)P_3(t)=(t+1)(t^2-2)$, $R_3=(t-1)(t+1)$, y como
$$\dfrac{1}{2}R_1(t)-\dfrac{1}{2}R_2(t)+R_3(t)=1,$$
si
\begin{align*}
\Pi_1(t) & =\dfrac{1}{2}R_1(t)=\dfrac{(t-1)(t^2-2)}{2},\\
\Pi_2(t) & =-\dfrac{1}{2}R_2(t)=-\dfrac{(t+1)(t^2-2)}{2},\textrm{ y }\\
\Pi_3(t) & =R_3(t)=(t-1)(t+1),
\end{align*}
entonces $p_i=\Pi_i(f)$, para $i=1,2,3$, definen las respectivas proyecciones sobre $V_i$ de acuerdo a nuestra descomposici\'on de $\mathbb{Q}^4$. Las representaciones matriciales en la base can\'onica de estas proyecciones son:
$$\Big[ p_1\Big]_\mathcal{C}^\mathcal{C}=\left[\begin{array}{rrrr}
1 & 0 & 0 & -1\\
2 & 0 & 0 & -2\\
1 & 0 & 0 & -1\\
0 & 0 & 0 & 0
\end{array}\right]$$
$$\Big[ p_2\Big]_\mathcal{C}^\mathcal{C}=\left[\begin{array}{rrrr}
-3 & 2 & -1 & 2\\
-3 & 2 & -1 & 2\\
0 & 0 & 0 & 0\\
-3 & 2 & -1 & 2
\end{array}\right] $$
$$\Big[ p_3\Big]_\mathcal{C}^\mathcal{C}=\left[\begin{array}{rrrr}
3 & -2 & 1 & -1\\
1 & -1 & 1 & 0\\
-1 & 0 & 1 & 1\\
3 & -2 & 1 & -1
\end{array}\right] $$
as\'i pues $V_1=\im(p_1)=\langle (1,2,1,0)\rangle$, $V_2=\im(p_2)=\langle (1,1,0,1)\rangle$ y $V_3=\im(p_3)=\langle (1,1,1,1),(1,0,-1,1)\rangle$. Sea $\mathcal{B}=\{(1,2,1,0),(1,1,0,1),(1,1,1,1),(1,0,-1,1)\}$, de forma que la representaci\'on matricial de $f$ en esta base
$$\Big[ f\Big]_\mathcal{B}^\mathcal{B}=\left[\begin{array}{r|r|rr}
-1 & 0 & 0 & 0\\
\hline
0 & 1 & 0 & 0\\
\hline
0 & 0 & 0 & 2\\
0 & 0 & 1 & 0
\end{array}\right].$$
es una matriz diagonal por bloques, donde cada bloque describe la restricci\'on de $f$ a cada uno de los subespacios invariantes en la descomposici\'on. 
\end{ejem}

\section{Operadores nilpotentes, espacios c\'iclicos y forma de Jordan}

Sea $f\in\Hom_K(V,V)$ un operador.

\begin{obs}
Note que si $V$ tiene dimensi\'on finita y tomamos $f\in\Hom_K(V,V)$, $P_f(f)=0$. Ahora suponga que $P_f(t)$ se descompone en factores lineales
\[
P_f(t)=(t-\lambda_1)^{m_1}(t-\lambda_2)^{m_2}\ldots(t-\lambda_n)^{m_n}, \quad \lambda_1,\lambda_2,\ldots,\lambda_n\in K.
\]
con $\lambda_i\ne\lambda_j$ si $i\ne j$. De esta forma, si $V_i=\ker\left((f-\lambda_i\id_V)^{m_i}\right)$, para $i=1,\ldots,n$,
\[
V=V_1\oplus V_2\oplus\ldots\oplus V_n
\]
y si adem\'as denotamos $g_i\in\Hom_K(V_i,V_i)$ a la restricci\'on de $f-\lambda_i\id_V$ a $V_i$, tenemos $g_i^{m_i}=0$. Este tipo de operadores, cuya potencia se anula, motivan la siguiente definici\'on.
\end{obs}

\begin{defn}
Decimos que $f$ es nilpotente si existe $r\in\mathbb{Z}_{>0}$ tal que $f^r=0$, y al m\'inimo entre estos lo llamamos el grado de $f$.
\end{defn}

\begin{pro}
Suponga que $f$ es nilpotente de grado $r$, y $V\ne\{0\}$, entonces tenemos una cadena de contenencias estrictas
\[
\{0\}<\ker(f)<\ker(f^2)<\ldots<\ker(f^r)=V.
\]
En particular si $V$ tiene dimensi\'on finita, $r\le\dim(V)$.
\end{pro}

\dem Note primero que para todo $i\in\mathbb{Z}_{>0}$, si $v\in V$ es tal que $f^i(v)=0$, entonces $f^{i+1}(v)=0$, luego $\ker(f^i)\le\ker(f^{i+1})$.\\
Si $r=1$, no hay nada que demostrar pues $f=0$ y as\'i la cadena corresponde a $\{0\}<V$. Ahora suponga que $r>1$, luego $f^{r-1}\ne 0$ y as\'i existe $v\in V$ tal que $f^{r-1}(v)\ne 0$. Note que para $i=1,\ldots,r-1$
\begin{align*}
f^{i-1}\left(f^{r-i}(v)\right) & =f^{r-1}(v)\ne 0,\textrm{ y }\\
  f^i\left(f^{r-i}(v)\right) & =f^r(v)=0
\end{align*}
as\'i $f^{r-i}(v)\in \ker(f^i)\setminus \ker(f^{i-1})$ y tenemos una contenencia estricta $\ker(f^{i-1})<\ker(f^i)$.\\
Suponga ahora que $V$ tiene dimensi\'on finita y denote, para $i=1,\ldots,r$, $n_i=\dim(\ker(f^i))$. Entonces
\[
0<n_1<n_2<\ldots<n_r=\dim(V)
\]
es una cadena de $r+1$ enteros estrictamente creciente que arranca en $0$, luego $1\le n_1$, $2\le n_2$, $\ldots$, $r\le n_r=\dim(V)$.\qed

\begin{ejem}\label{ejnil1}
Sea $f\in\Hom_K(K^4,K^4)$ definido por
$$f(x,y,z,w)=(y,z,w,0).$$
As\'i
\begin{align*}
f^2(x,y,z,w) & = (z,w,0,0),\\
f^3(x,y,z,w) & = (w,0,0,0),\\
f^4(x,y,z,w) & = (0,0,0,0)
\end{align*}
y si $n_i=\dim(\ker(f^i))$ entonces
$$n_1=1,\quad n_2=2,\quad n_3=3,\quad n_4=4.$$
La representaci\'on matricial de $f$ en la base can\'onica $\mathcal{C}$ es
$$
\left[f\right]^{\mathcal{C}}_{\mathcal{C}}=\left[\begin{array}{rrrr}
0 & 1 & 0 & 0\\
0 & 0 & 1 & 0\\
0 & 0 & 0 & 1\\
0 & 0 & 0 & 0
\end{array}\right]
$$
y el polinomio caracter\'istico es $P_f(t)=t^4$.
\end{ejem}

\begin{ejem}\label{ejnil2}
Sea $f\in\Hom_K(K^4,K^4)$ definido por
$$f(x,y,z,w)=(y,z,0,0).$$
As\'i
\begin{align*}
f^2(x,y,z,w) & = (z,0,0,0),\\
f^3(x,y,z,w) & = (0,0,0,0)
\end{align*}
y si $n_i=\dim(\ker(f^i))$ entonces
$$n_1=2,\quad n_2=3,\quad n_3=4.$$
La representaci\'on matricial de $f$ en la base can\'onica $\mathcal{C}$ es
$$
\left[f\right]^{\mathcal{C}}_{\mathcal{C}}\left[\begin{array}{rrr|r}
0 & 1 & 0 & 0\\
0 & 0 & 1 & 0\\
0 & 0 & 0 & 0\\
\hline
0 & 0 & 0 & 0
\end{array}\right]
$$
y el polinomio caracter\'istico es $P_f(t)=t^4$.
\end{ejem}

\begin{ejem}\label{ejnil3}
Sea $f\in\Hom_K(K^4,K^4)$ definido por
$$f(x,y,z,w)=(y,0,w,0).$$
As\'i
\begin{align*}
f^2(x,y,z,w) & = (0,0,0,0)
\end{align*}
y si $n_i=\dim(\ker(f^i))$ entonces
$$n_1=2,\quad n_2=4$$
La representaci\'on matricial de $f$ en la base can\'onica $\mathcal{C}$ es
$$
\left[f\right]^{\mathcal{C}}_{\mathcal{C}}\left[\begin{array}{rr|rr}
0 & 1 & 0 & 0\\
0 & 0 & 0 & 0\\
\hline
0 & 0 & 0 & 1\\
0 & 0 & 0 & 0
\end{array}\right]
$$
y el polinomio caracter\'istico es $P_f(t)=t^4$.
\end{ejem}

\begin{ejem}\label{ejnil4}
Sea $f\in\Hom_K(K^4,K^4)$ definido por
$$f(x,y,z,w)=(y,0,0,0).$$
As\'i
\begin{align*}
f^2(x,y,z,w) & = (0,0,0,0)
\end{align*}
y si $n_i=\dim(\ker(f^i))$ entonces
$$n_1=3,\quad n_2=4$$
La representaci\'on matricial de $f$ en la base can\'onica $\mathcal{C}$ es
$$
\left[f\right]^{\mathcal{C}}_{\mathcal{C}}\left[\begin{array}{rr|r|r}
0 & 1 & 0 & 0\\
0 & 0 & 0 & 0\\
\hline
0 & 0 & 0 & 0\\
\hline
0 & 0 & 0 & 0
\end{array}\right]
$$
y el polinomio caracter\'istico es $P_f(t)=t^4$.
\end{ejem}

\begin{defn}
Sea $v\in V$, si existe $k\in\mathbb{Z}_{>0}$ tal que $f^k(v)=0$, al m\'inimo entre estos los llamamos el orden de $v$ bajo $f$ y lo denotamos por $\ord_f(v)$.
\end{defn}

\begin{pro}
Sea $v\in V$, $v\ne 0$, y suponga que $k=\ord_f(v)$, entonces $S=\{v,f(v),\ldots,f^{k-1}(v)\}$ es linealmente independiente.
\end{pro}

\dem Suponga que $a_0,a_1,\ldots,a_{k-1}\in K$ son tales que
\[
a_0v+a_1f(v)+\ldots+a_{k-1}f^{{k-1}}(v)=0.
\]
Aplicando $f^{k-1}$ a esta igualdad obtenemos $a_0f^{k-1}(v)=0$, pero $f^{k-1}(v)\ne 0$ luego $a_0=0$. Inductivamente, si hemos establecido que $a_0=a_1=\ldots=a_{i-1}=0$ para $0<i<k-1$, aplicando $f^{k-i-1}$ a la misma igualdad, obtenemos $a_if^{k-1}(v)=0$, luego $a_i=0$. As\'i $a_0=a_1=\ldots=a_{k-1}=0$.\qed

\begin{obs}\label{obsformajordannil}
Suponga que $V$ tiene dimensi\'on finita y $f$ es nilpotente de grado $r=\dim(V)$. Si $v\in V$ es tal que $v\not\in\ker(f^{r-1})$ entonces $\ord_f(v)=r$, luego si $v_i=f^{r-i}(v)$ para $i=1,\ldots,r$, por la propiedad anterior
\[
\mathcal{B}=\{v_1,\ldots,v_r\}=\{f^{r-1}(v),\ldots,f(v),v\}
\]
es una base de $V$; m\'as a\'un
\[
\Big[f\Big]^{\mathcal{B}}_{\mathcal{B}}=\left[\begin{array}{ccccc}
0 & 1 & 0 &\cdots & 0\\
0 & 0 & 1 &\cdots & 0\\
\vdots & \vdots & \vdots &\ddots & \vdots\\
0 & 0 & 0 & \cdots & 1\\
0 & 0 & 0 & \cdots & 0
\end{array}\right].
\]
\end{obs}

\begin{defn}
Decimos que $V$ es c\'iclico bajo $f$ si
\[
S=\left\{f^i(v)\right\}_{i\in\mathbb{Z}_{>0}}
\]
genera a $V$, es decir $\langle S\rangle=V$, para alg\'un $v\in V$ . En tal caso decimos que $v$ es un \emph{vector c\'iclico} relativo a $f$.
\end{defn}

\begin{obs}
Si $V$ tiene dimensi\'on finita y $f\ne 0$ es nilpotente de grado $r=\dim(V)$, la observaci\'on anterior explica que $V$ es c\'iclico bajo $f$. 
\end{obs}

\begin{defn}
Suponga que $V$ tiene dimensi\'on finita y $f\ne 0$ es nilpotente de grado $r=\dim(V)$, una base de la forma
\[
\mathcal{B}=\{v_1,\ldots,v_r\},\qquad v_i=f^{r-i}(v_r)
\]
se llama una \emph{base de Jordan de $V$ relativa a $f$}.  
\end{defn}

\begin{obs}
En caso de que $f$ sea nilpotente de grado inferior, $V$ no es c\'iclico, pero se puede descomponer en subespacios invariantes bajo $f$ y c\'iclicos bajo la restricci\'on de $f$ a ellos. Esto es el contenido del siguiente teorema.
\end{obs}

\begin{teo}\label{formajordannil}
Suponga que $V$ tiene dimensi\'on finita y que $f\ne 0$ es nilpotente de grado $r$. Sea $n=\dim\left(\ker(f)\right)$. Entonces existen $n$ subespacios invariantes bajo $f$, $V_1,\ldots, V_n$ tales que  
\[
V=V_1\oplus\ldots\oplus V_n
\]
y si $f_i\in\Hom_K(V_i,V_i)$ es la restricci\'on de $f$ a $V_i$, para $i=1,\ldots,n$, entonces $V_i$ c\'iclico bajo $f_i$.
\end{teo}

\dem
Denotemos $K_i=\ker(f^i)$, de forma que $K_0=\{0\}$ y $K_r=V$. Note que para $i=1,\ldots,r-1$, $K_i<K_{i+1}$. Podemos as\'i descomponer para cada $i=2,\ldots r$
\[
K_i=K_{i-1}\oplus K'_i.
\]
De forma que si $v\in K'_i$, $v\ne 0$, entonces $\ord_f(v)=i$. Por lo tanto
\[
f\left(K'_i\right)\le K'_{i-1}.
\]
Tenemos entonces
\begin{eqnarray*}
V & = & K_r\\
   & = & K_{r-1}\oplus K'_r\\
   & \vdots &\\
   & = & K_1\oplus K'_2\oplus\ldots\oplus K'_r
\end{eqnarray*}
y
\[
K'_r\overset{f}\longrightarrow K'_{r-1}\overset{f}\longrightarrow\ldots\overset{f}\longrightarrow K'_2\overset{f}\longrightarrow K_1\overset{f}\longrightarrow \{0\}
\]
Se trata entonces de escoger una base de $V$ que sea compatible con esta descomposici\'on y esta cadena de im\'agenes bajo $f$. Denote $n_i=\dim(K_i)$ y $n'_i=\dim(K'_i)$, para $i=2,\ldots,r$, y $n_1=n=\dim(K_1)$ de forma que
$$n_i=n'_i+n_{i-1}$$
y
\begin{align*}
\dim(V)& =n_r\\
 & =n'_r+n_{r-1}\\
 & \vdots\\
 & =n'_r+n'_{r-1}+\ldots+n_2\\
 & =n'_r+n'_{r-1}+\ldots+n'_2+n_1\\
 & =n'_r+n'_{r-1}+\ldots+n'_2+n
\end{align*}
Sea $\mathcal{B}_r=\{v_{r,1},\ldots,v_{r,n'_r}\}\subseteq V$ una base de $K'_r$. Para $i=1,\ldots n'_r$, sea $$v_{r-1,i}=f(v_{r,i}).$$ Note que $f(\mathcal{B}_r)=\{v_{r-1,1},\ldots,v_{r-1,n'_r}\}\subseteq K'_{r-1}$ es linealmente independiente. De hecho si
\[
a_1v_{r-1,1}+\ldots+a_{n'_r}v_{r-1,n'_r}=0,
\]
entonces
\[
a_1f(v_{r,1})+\ldots+a_{n'_r}f(v_{r,n'_r})=0
\]
luego $a_1v_{r,1}+\ldots+a_{n'_r}v_{r,n'_r}\in K_1\cap K'_r$; por lo tanto $a_1v_{r,1}+\ldots+a_{n'_r}v_{r,n'_r}=0$ y $a_1=\ldots=a_{n'_r}=0$.\\
Sea $\mathcal{B}_{r-1}=\{v_{r-1,1},\ldots,v_{r-1,n'_{r-1}}\}$ un base de $K'_{r-1}$ que contiene a $f(\mathcal{B}_r)$. Para $i=1,\ldots n'_{r-1}$, sea $$v_{r-2,i}=f(v_{r-1,i}).$$ Similarmente, note que $f(\mathcal{B}_{r-1})=\{v_{r-2,1},\ldots,v_{r-2,n'_{r-1}}\}\subseteq K'_{r-2}$ es linealmente independiente.\\
Iterativamente obtenemos bases $\mathcal{B}_1,\mathcal{B}_2,\ldots,\mathcal{B}_r$ respectivamente de $K_1,K'_2,\ldots,K'_r$ con $f(\mathcal{B}_{i+1})\subseteq\mathcal{B}_i$ para $i=1,\ldots,r-1$. En particular
\[
\mathcal{B}=\mathcal{B}_1\cup \mathcal{B}_2\cup\ldots\cup \mathcal{B}_r
\]
es una base de $V$. Defina (ver Figura \ref{edificios})
\begin{eqnarray*}
V_1 & = & \langle v_{j,1}\in \mathcal{B}\ |\ 1\le \dim(K'_j)\rangle \\
V_2 & = & \langle v_{j,2}\in \mathcal{B}\ |\ 2\le \dim(K'_j)\rangle \\
       & \vdots & \\
V_n & = & \langle v_{j,n}\in \mathcal{B}\ |\ n\le \dim(K'_j)\rangle
\end{eqnarray*}
de esta forma por construcci\'on cada $V_i$, $i=1,\ldots, n$, son invariantes bajo $f$ y c\'iclicos bajo $f_i$.\qed 

\begin{figure}[!hbp]
\centering
\frame{
\begin{tikzpicture}[auto, node distance=1.1cm,>=latex']
    \node (Kr) {$K'_r$};
    \node (Kr-1) [below of=Kr] {$K'_{r-1}$};
    \node (Kdots) [below of=Kr-1] {$\vdots$};
    \node (K2) [below of=Kdots] {$K'_2$};
    \node (K1) [below of=K2] {$K_1$};
    \node (0) [below of=K1] {$\{0\}$};
    
    \node (v1r) [right of=Kr] {$v_{r,1}$};    
    \node (v2r) [right of=v1r] {$v_{r,2}$};
    \node (vdotsr) [right of=v2r] {$\cdots$};
    \node (vnrr) [right of=vdotsr] {$v_{r,n'_r}$};    
        
    \node (v1r-1) [right of=Kr-1] {$v_{r-1,1}$};    
    \node (v2r-1) [right of=v1r-1] {$v_{r-1,2}$};
    \node (vdotsr-1) [right of=v2r-1] {$\cdots$};
    \node (vnrr-1) [right of=vdotsr-1] {$v_{r-1,n'_r}$};
    \node (vdots2r-1) [right of=vnrr-1] {$\cdots$};
    \node (vnr-1r-1) [right of=vdots2r-1] {$v_{r-1,n'_{r-1}}$};
    
    \node (v1dots) [right of=Kdots] {$\vdots$};    
    \node (v2dots) [right of=v1dots] {$\vdots$};
    \node (vdotsdots) [right of=v2dots] {$\cdots$};
    \node (vnrdots) [right of=vdotsdots] {$\vdots$};
    \node (vdots2dots) [right of=vnrdots] {$\cdots$};
    \node (vnr-1dots) [right of=vdots2dots] {$\vdots$};

    \node (v12) [right of=K2] {$v_{2,1}$};    
    \node (v22) [right of=v12] {$v_{2,2}$};
    \node (vdots2) [right of=v22] {$\cdots$};
    \node (vnr2) [right of=vdots2] {$v_{2,n'_r}$};
    \node (vdots22) [right of=vnr2] {$\cdots$};
    \node (vnr-12) [right of=vdots22] {$v_{2,n'_{r-1}}$};
    \node (vdots32) [right of=vnr-12] {$\cdots$};
    \node (vn22) [right of=vdots32] {$v_{2,n'_2}$};

    \node (v11) [right of=K1] {$v_{1,1}$};    
    \node (v21) [right of=v11] {$v_{1,2}$};
    \node (vdots1) [right of=v21] {$\cdots$};
    \node (vnr1) [right of=vdots1] {$v_{1,n'_r}$};
    \node (vdots21) [right of=vnr1] {$\cdots$};
    \node (vnr-11) [right of=vdots21] {$v_{1,n'_{r-1}}$};
    \node (vdots31) [right of=vnr-11] {$\cdots$};
    \node (vn21) [right of=vdots31] {$v_{1,n'_2}$};
    \node (vdots41) [right of=vn21] {$\cdots$};
    \node (vn11) [right of=vdots41] {$v_{1,n}$};
    
    \node (v10) [below of=v11] {$0$};    
    \node (v20) [below of=v21] {$0$};
    \node (vdots0) [below of=vdots1] {$\cdots$};
    \node (vnr0) [below of=vnr1] {$0$};
    \node (vdots20) [below of=vdots21] {$\cdots$};
    \node (vnr-10) [below of=vnr-11] {$0$};
    \node (vdots30) [below of=vdots31] {$\cdots$};
    \node (vn20) [below of=vn21] {$0$};
    \node (vdots40) [below of=vdots41] {$\cdots$};
    \node (vn10) [below of=vn11] {$0$};
    
    \node (V1) [below of=v10] {$V_1$};    
    \node (V2) [below of=v20] {$V_2$};
    \node (Vdots) [below of=vdots0] {$\cdots$};
    \node (Vnr) [below of=vnr0] {$V_{n'_r}$};
    \node (Vdots2) [below of=vdots20] {$\cdots$};
    \node (Vnr-1) [below of=vnr-10] {$V_{n'_{r-1}}$};
    \node (Vdots3) [below of=vdots30] {$\cdots$};
    \node (Vn2) [below of=vn20] {$V_{n'_2}$};
    \node (Vdots4) [below of=vdots40] {$\cdots$};
    \node (Vn1) [below of=vn10] {$V_n$};
    
    \path[->] (Kr) edge node {} (Kr-1);
    \path[->] (v1r) edge  node {} (v1r-1);
    \path[->] (v2r) edge  node {} (v2r-1);
    \path[->] (vnrr) edge  node {} (vnrr-1);
    
    \path[->] (Kr-1) edge node {} (Kdots);
    \path[->] (v1r-1) edge  node {} (v1dots);
    \path[->] (v2r-1) edge  node {} (v2dots);
    \path[->] (vnrr-1) edge  node {} (vnrdots);
    \path[->] (vnr-1r-1) edge  node {} (vnr-1dots);

    \path[->] (Kdots) edge node {} (K2);
    \path[->] (v1dots) edge  node {} (v12);
    \path[->] (v2dots) edge  node {} (v22);
    \path[->] (vnrdots) edge  node {} (vnr2);
    \path[->] (vnr-1dots) edge  node {} (vnr-12);
  
    \path[->] (K2) edge node {} (K1);
    \path[->] (v12) edge  node {} (v11);
    \path[->] (v22) edge  node {} (v21);
    \path[->] (vnr2) edge  node {} (vnr1);
    \path[->] (vnr-12) edge  node {} (vnr-11);
    \path[->] (vn22) edge  node {} (vn21);
    
    \path[->] (K1) edge node {} (0);
    \path[->] (v11) edge  node {} (v10);
    \path[->] (v21) edge  node {} (v20);
    \path[->] (vnr1) edge  node {} (vnr0);
    \path[->] (vnr-11) edge  node {} (vnr-10);
    \path[->] (vn21) edge  node {} (vn20);
    \path[->] (vn11) edge node {} (vn10);    
\end{tikzpicture}
}
\caption{Edificios colapsando}
\label{edificios}
\end{figure}

\begin{ejem}
Si $f\in\Hom_K(K^4,K^4)$ est\'a definido como en Ejemplo \ref{ejnil1}
$$f(x,y,z,w)=(y,z,w,0),$$
entonces
$$n=1,\quad n'_2=1,\quad n'_3=1,\quad n'_4=1.$$
\end{ejem}

\begin{ejem}
Si $f\in\Hom_K(K^4,K^4)$ est\'a definido como en Ejemplo \ref{ejnil2}
$$f(x,y,z,w)=(y,z,0,0),$$
entonces
$$n=2,\quad n'_2=1,\quad n'_3=1.$$
\end{ejem}

\begin{ejem}
Si $f\in\Hom_K(K^4,K^4)$ est\'a definido como en Ejemplo \ref{ejnil3}
$$f(x,y,z,w)=(y,0,w,0),$$
entonces
$$n=2,\quad n'_2=2.$$
\end{ejem}

\begin{ejem}
Si $f\in\Hom_K(K^4,K^4)$ est\'a definido como en Ejemplo \ref{ejnil4}
$$f(x,y,z,w)=(y,0,0,0),$$
entonces
$$n=3,\quad n'_2=1.$$
\end{ejem}

\begin{obs}\label{bloquesjordannil}
Bajo la hip\'otesis del teorema, y usando la notaci\'on en \'el, obtenemos que para cada $V_i$, $i=1,\ldots,n$, tenemos una base de Jordan $\mathcal{B}_i$ relativa a $f_i$. De esta forma la uni\'on de ella forma una base $\mathcal{B}$ de $V$. La representaci\'on matricial de $f$ en la base $T$ es una matriz diagonal por bloques:
\[
\Big[f\Big]^\mathcal{B}_\mathcal{B}=\left[\begin{array}{c|c|c|c}
J_1 & 0 & \cdots & 0\\
\hline
0 & J_2 & \cdots & 0\\
\hline
\vdots & \vdots & \ddots & \vdots\\
\hline
0 & 0 & \cdots & J_n
\end{array}\right] 
\]
donde cada $J_i=\Big[f\Big]^{\mathcal{B}_i}_{\mathcal{B}_i}$ es una matriz $\dim(V_i)\times\dim(V_i)$ de la forma en Observaci\'on \ref{obsformajordannil}.
\end{obs}

\begin{obs}
Como corolario de la prueba del teorema tenemos que cuando $V$ tiene dimensi\'on finita y $f$ es nilpotente, la informaci\'on subministrada por las cantidades 
\begin{eqnarray*}
\dim(K_1) & = & n\\
\dim(K_i)-\dim(K_{i-1}) & = & n'_i,\qquad i=2,\ldots,r
\end{eqnarray*}
son tales que  $n\ge n'_2\ge \ldots \ge n'_r$ y determinan univocamente la transformaci\'on $f$, salvo cambio de coordenadas. De hecho dadas dos transformaciones con igual informaci\'on, para cada una podemos encontrar una base de $V$ que arrojan la misma representaci\'on matricial. Espec\'ificamente, $n$ indica el n\'umero de bloques de Jordan y $n'_i$ el n\'umero de bloques de Jordan de tama\~no mayor o igual a $i$.
\end{obs}

\begin{defn}
Se le llama \emph{matriz en bloque de Jordan} a una matriz cuadrada $n\times n$ de la forma
\[
J_{\lambda,n}=\left[\begin{array}{ccccc}
\lambda & 1 & 0 &\cdots & 0\\
0 & \lambda & 1 &\cdots & 0\\
\vdots & \vdots & \ddots &\ddots & \vdots\\
0 & 0 & 0 & \cdots & 1\\
0 & 0 & 0 & \cdots & \lambda
\end{array}\right].
\]
\end{defn}

\begin{lema}
Suponga que $V$ tiene dimensi\'on finita y que $P(t)=(t-\lambda)^m\in K[t]$, es tal que $P(f)=0$. Entonces existe una base $\mathcal{B}$ de $V$ tal que la representaci\'on matricial de $f$ en esta base es
es una matriz diagonal por bloques:
\[
\Big[f\Big]^{\mathcal{B}}_{\mathcal{B}}=\left[\begin{array}{c|c|c|c}
J_1 & 0 & \cdots & 0\\
\hline
0 & J_2 & \cdots & 0\\
\hline
\vdots & \vdots & \ddots & \vdots\\
\hline
0 & 0 & \cdots & J_n
\end{array}\right] 
\]
donde cada $J_i$, $i=1,\ldots,n$ es una matriz en bloque de Jordan.
\end{lema}

\dem Tenemos $P(f)=(f-\lambda\id_V)^m=0$. Luego el operador $g=f-\lambda\id_V$ es nilpotente. Por Teorema \ref{formajordannil},
\[
V=V_1\oplus V_2\oplus\ldots\oplus V_n
\]
donde para cada $V_i$, $i=1,\ldots, n$, hay una base de la forma $\mathcal{B}_i=\{v_{1,i},\ldots,v_{m_i,i}\}$, con $\dim(V_i)=m_i$ y
\[
\begin{array}{rcccl}
v_{m_i-1,i} & = & g(v_{m_i,i}) & = & f(v_{m_i,i})-\lambda v_{m_i,i}\\
v_{m_i-2,i} & = & g(v_{m_i-1,i}) & = & f(v_{m_i-1,i})-\lambda v_{m_i-1,i}\\
 & \vdots & & \vdots & \\
v_{1,i} & = & g(v_{2,i}) & = & f(v_{2,i})-\lambda v_{2,i}\\
0 & = & g(v_{1,i}) & = & f(v_{1,i})-\lambda v_{1,i}.
\end{array}
\]
As\'i,
\begin{eqnarray*}
f(v_{m_i,i}) & = & v_{m_i-1,i}+\lambda v_{m_i,i}\\
f(v_{m_i-1,i}) & = & v_{m_i-2,i}+\lambda v_{m_i-1,i}\\
 & \vdots & \\
f(v_{2,i}) & = & v_{1,i}+\lambda v_{2,i}\\
f(v_{1,i}) & = & \lambda v_{1,i}.
\end{eqnarray*}
En particular, cada $V_i$ es invariante bajo $f$, luego, si $f_i\in\Hom_K(V_i,V_i)$ denota la restricci\'on de $f$ a $V_i$,
$\Big[f_i\Big]^{\mathcal{B}_i}_{\mathcal{B}_i}=J_i$ es una matriz en bloque de Jordan. De esto, si $\mathcal{B}=\mathcal{B}_1\cup\ldots\cup \mathcal{B}_n$, la representaci\'on matricial $\Big[f\Big]^{\mathcal{B}}_{\mathcal{B}}$ tiene la forma buscada.\qed

\begin{teo}[Teorema de Jordan]
Suponga que $V$ tiene dimensi\'on finita y que
\[
P_f(t)=(t-\lambda_1)^{m_1}(t-\lambda_2)^{m_2}\ldots(t-\lambda_r)^{m_r}, \quad \lambda_1,\lambda_2,\ldots,\lambda_r\in K.
\]
Entonces existe una base $\mathcal{B}$ de $V$ tal que la representaci\'on matricial de $f$ en esta base es
es una matriz diagonal por bloques de Jordan. 
\end{teo}

\dem Sin perdida de generalidad podemos asumir que $\lambda_i\ne\lambda_j$ si $i\ne j$. As\'i
\[
\left( (t-\lambda_i)^{m_i},(t-\lambda_j)^{m_j}\right)=1
\]
si $i\ne j$. Por el teorema de Caley-Hamilton $P_f(f)=0$, luego por Propiedad \ref{prodescomp},
\[
V=V_1\oplus \ldots \oplus V_r
\]
donde cada $V_i=\ker\left((f-\lambda_i\id_V)^{m_i}\right)$, $i=1,\ldots,r$, es invariante bajo $f$. En particular, si $f_i\in\Hom_K(V_i,V_i)$ es la restricci\'on de $f$ a $V_i$, $i=1,\ldots,r$, $P_i(f_i)=0$, donde $P_i(t)=(t-\lambda_i)^{m_i}$. Por lo tanto, el lema implica que existe una base $\mathcal{B}_i$ de $V_i$ para la cual $\Big[f_i\Big]^{\mathcal{B}_i}_{\mathcal{B}_i}$ es una matriz diagonal por bloques de Jordan. Finalmente si $\mathcal{B}=\mathcal{B}_1\cup\ldots\cup \mathcal{B}_n$, la representaci\'on matricial $\Big[f\Big]^{\mathcal{B}}_{\mathcal{B}}$ tiene la forma afirmada.\qed

\begin{defn}
Generalizamos la definici\'on anterior de base de Jordan. Si $V$ tiene dimensi\'on finita, decimos que una base de $V$ es una \emph{base de Jordan relativa a $f$} si la representaci\'on matricial de este operador en aquella base es diagonal en bloques de Jordan.
\end{defn}

\begin{lema}
Sea $f\in\Hom_K(V,V)$. Si $\lambda_1,\ldots,\lambda_n\in K$ son valores propios, todos distintos, de $f$, y, para $i=1,\ldots,n$, $v_i\in V$ es un vector propio de $\lambda_i$, entonces $\{v_1,\ldots,v_n\}$ es linealmente independiente. 
\end{lema}

\dem Por inducci\'on en $n$, siendo el caso base $n=1$ inmediato, pues $\{v_1\}$ es linealmente independiente si $v_1\ne 0$, la cual se cumple pues $v_1$ es vector propio. Para el paso inductivo, si $a_1,\ldots,a_n$ son tales que $a_1v_1+\ldots+a_nv_n=0$,por contradicci\'on podemos asumir que cada $a_i\ne 0$, o de lo contrario, por hipotesis de inducci\'on, si alg\'un $a_i$  es $0$ el resto tambi\'en lo son. Entonces
\[
0=(f-\lambda_n\id_V)(a_1v_1+\ldots+a_nv_n)=a_1(\lambda_1-\lambda_n)v_1+\ldots+a_{n-1}(\lambda_{n-1}-\lambda_n)v_{n-1};
\]
y as\'i, por hip\'otesis de inducci\'on, para $i=1,\ldots,n-1$, $a_i(\lambda_i-\lambda_n)=0$. Pero $a_i\ne 0$ y $\lambda_i-\lambda_n\ne 0$,  si $i\in\{1,\ldots,n-1\}$ , lo cual es una contradicci\'on.\qed

\begin{lema}
Suponga que $f\in\Hom_K(V,V)$ es diagonalizable, entonces:
\begin{enumerate}
\item Si $V_0\ne\{0\}$ es invariante bajo $f$, su restricci\'on a $V_0$, $f_0\in\Hom_K(V_0,V_0)$, tambi\'en es diagonalizable. 
\item Si  $g\in\Hom_K(V,V)$ es diagonalizable y $f\circ g=g\circ f$, entonces existe una familia $\{V_i\}_{i\in I}$ de espacios propios simult\'aneamente de $f$ y $g$ tal que $V=\bigoplus_{i\in I}V_i$. En particular si $v$ es un vector propio simult\'aneamente de $f$ y $g$, $v$ es un vector propio de $af+bg$ para todo $a,b\in K$.
\end{enumerate}
\end{lema} 

\dem \begin{enumerate}
\item Dado un valor propio $\lambda\in K$ de $f$, definimos $E_\lambda\le V$ como el subespacio generado por los vectores propios de $f$ asociados a $\lambda$, es decir $E_\lambda=\ker(f-\lambda\id_V)$, y $F_\lambda=V_0\cap E_\lambda$. Note que, como $f$ es diagonalizable, por el lema anterior,
\[
V=\bigoplus_{i\in I} E_{\lambda_i}
\]
donde $\{\lambda_i\}_{i\in I}$ es la colecci\'on de valores propios de $f$. Sea $v\in V_0$, $v\ne 0$, y
\[
v=v_1+\ldots+v_n
\]
una descomposici\'on en vectores propios asociados respectivamente a valores propios $\lambda_1, \ldots , \lambda_n\in K$. Por inducci\'on en $n$ veamos que $v_1,\ldots,v_n\in V_0$ siendo el caso base $n=1$ inmediato pues en tal caso $v_0=v_1$. Para el paso inductivo, como $V_0$ es invariante bajo $f$
\[
(f-\lambda_n\id_V)(v)=(\lambda_1-\lambda_n)v_1+\ldots+(\lambda_{n-1}-\lambda_{n})v_{n-1}
\]
tambi\'en pertenece a $V_0$. Luego por hip\'otesis inductiva , $(\lambda_1-\lambda_n)v_1, \ldots, (\lambda_{n-1}-\lambda_{n})v_{n-1}\in V_0$, as\'i pues $v_1, \ldots, v_{n-1}\in V_0$ y $v_n=v-v_1-\ldots-v_{n-1}\in V_0$. De donde
\[
V_0=\bigoplus_{i\in J} F_{\lambda_i},
\]
donde $J$ es la colecci\'on de $i\in I$ tales que $F_{\lambda_i}\ne\{0\}$. Entonces $f_0$ es diagonalizable tomando bases de cada $F_{\lambda_i}$, $i\in J$.
\item Usando la notaci\'on de la demostraci\'on de la primera afirmaci\'on del lema, si $v\in E_{\lambda_i}$, $i\in I$,
\[
f\left(g(v)\right)=g\left(f(v)\right)=\lambda_ig(v),
\]
luego $E_{\lambda_i}$ es invariante bajo $g$. Por la primera parte del lema, la restricci\'on de $g$ a $E_{\lambda_i}$, $g_i\in\Hom_K (E_{\lambda_i},E_{\lambda_i})$ es diagonalizable. Luego $g$ es diagonalizable tomando bases de cada $E_{\lambda_i}$, $i\in I$. Los espacios propios generados por cada uno de estos elementos de estas bases forman una colecci\'on de espacios propios simult\'aneos cuya suma es una suma directa igual a $V$. \qed 
\end{enumerate}   

\begin{teo}[Descomposici\'on de Jordan-Chevalley]\label{descjorche}
Suponga que $V$ tiene dimensi\'on finita y que
\[
P_f(t)=(t-\lambda_1)^{m_1}(t-\lambda_2)^{m_2}\ldots(t-\lambda_r)^{m_r}, \quad \lambda_1,\lambda_2,\ldots,\lambda_r\in K.
\]
donde $\lambda_i\ne\lambda_j$ si $i\ne j$. Entonces existen operadores $f_N,f_D\in\Hom_K(V,V)$, tales que
\begin{enumerate}
\item $f_D$ es diagonalizable y $f_N$ es nilpotente;
\item $f_D+f_N=f$; y,
\item $f_D\circ f_N=f_N\circ f_D$.
\end{enumerate}
M\'as a\'un, esta descomposici\'on es \'unica respecto a estas tres propiedades. Adem\'as existen polinomios $P_D(t),P_N(t)\in K[t]$ tales que $f_N=P_N(f)$ y $f_D=P_D(f)$.
\end{teo}

\dem Defina $P_i(t)=(t-\lambda_i)^{m_i}$, $i=1,\ldots,n$. Por Propiedad \ref{prodescomp}, existen $\Pi_1(t),\ldots,\Pi_n(t)\in K[t]$ tales que $\Pi_i(f)=p_i$, $i=1,\ldots,n$, son las proyecciones sobre $V_i=\ker\left((f-\lambda_i\id_V)^{m_i}\right)$ respecto a la descomposici\'on 
\[
V=V_1\oplus \ldots \oplus V_r.
\]
Defina $P_D(t)=\lambda_1\Pi_1(t)+\ldots+\lambda_n\Pi_n(t)$, y $f_D=P_D(f)$. De esta forma, si $v_i\in V_i$,
\[
f_D(v_i)=\lambda_1p_1(v_i)+\ldots+\lambda_np_n(v_i)=\lambda_iv_i,
\]
y as\'i $f_D$ es diagonalizable por Teorema \ref{diagosiysolosi}. Defina $P_N(t)=t-P_D(t)$ y $f_N=P_N(f)=f-f_D$. De esta forma, $f_D+f_N=f$, y si $v_i\in V_i$
\[
f_N(v_i)=f(v_i)-f_D(v_i)=f(v_i)-\lambda_i(v_i)=\left(f-\lambda_i\id_V\right)(v_i),
\]
luego la restricci\'on de $f_N$ a $V_i$ es nilpotente de grado $\le m_i$. De donde $f_N$ es nilpotente de grado $\le\max\{m_1,\ldots,m_n\}$. Finalmente,
\[
f_D\circ f_N=P_D(f)\circ P_N(f)=P_N(f)\circ P_D(f)=f_N\circ f_D.
\]
Si $f'_D,f'_N\in\Hom_K(V,V)$ conmutan y son respectivamente diagonalizable y nilpotente tales que $f=f'_D+f'_N$, entonces
\[
f\circ f'_D=(f'_D+f'_N)f'_D=f'_D\circ f'_D+f'_N\circ f'_D=f'_D\circ f'_D+f'_D\circ f'_N=f'_D\circ f,
\]
es decir $f$ y $f'_D$ conmutan. Por lo cual, $P_D(f)=f_D$ y $f'_D$ tambi\'en lo hacen.\\
Entonces $f_D$ y $f'_D$ son diagonalizables y conmutan. Ahora, si $v$ es un vector propio com\'un, entonces $v$ es un vector propio de $f_D-f'_D$. Pero $f_D-f'_D=f'_N-f_N$, y, como $f'_N$ y $f_N$ igualmente conmutan, $f'_N-f_N$ es igualmente nilpotente. As\'i $f_D-f'_D$ es diagonalizable y, a su vez, nilpotente, el valor propio asociado a $v$ es $0$. Por el lema anterior existe una base de $V$ de vectores propios simult\'aneos de $f_D$ y $f'_D$, luego todos los valores propios de $f_D-f'_D$ son $0$. Es decir $f_D-f'_D=0=f'_N-f_N$; y, $f'_D=f_D$ y $f'_N=f_N$.\qed

\section{Polinomio minimal y transformaciones semi-simples} 
\chapter{Espacio dual}

Sea $K$ un cuerpo y $V$, $W$ espacios vectoriales sobre $K$.

\begin{nota}
Dada una colecci\'on de indices $I$, definimos para cada $i,j\in I$ el s\'imbolo \emph{delta de Kronecker}:
\[
\delta_{ij}=\left\{\begin{array}{rl}1 & \textrm{si } i=j\\ 0 & \textrm{si } i\ne j\end{array}\right.
\]
\end{nota}

\section{Funcionales lineales}

\begin{defn}
El \emph{espacio dual de $V$} es el espacio vectorial $V^*=\Hom_K(V,K)$, es decir la colecci\'on de transformaciones lineales
\[
\lambda: V\longrightarrow K.
\]
A los elementos $\lambda\in V^*$ los llamamos \emph{funcionales lineales}. 
\end{defn}

\begin{prop}
Si $V$ tiene dimensi\'on finita $\dim(V)=\dim(V^*)$.
\end{prop}

\dem Sea $\{v_1,\ldots,v_n\}$ una base de $V$, donde $n=\dim(V)$. Por Proposici\'on \ref{unitrlin}.2, $\lambda\in V^*$ est\'a un\'ivocamente por los valores $\lambda(v_1),\ldots,\lambda(v_n)$. Defina $\lambda_1,\ldots,\lambda_n\in V*$ por
\[
\lambda_i(v_j)=\delta_{ij}.
\]
Veamos que $\{\lambda_1,\ldots,\lambda_n\}$ es una base de $V^*$ probando que es linealmente independiente y que genera a $V^*$; y as\'i obtenemos $\dim(V^*)=n$. Para la independencia lineal, tome $a_1,\ldots,a_n\in K$ tales que
\[
\sum_{i=1}^na_i\lambda_i=0.
\]
De forma que, para $j=1,\ldots,n$,
\[
0=\left(\sum_{i=1}^na_i\lambda_i\right)(v_j)=\sum_{i=1}^na_i\lambda_i(v_j)=\sum_{i=1}^ja_i\delta_{ij}=a_j.
\]
Para ver que $V^*=\langle\lambda_1,\ldots,\lambda_n\rangle$, dado $\lambda\in V^*$, defina $a_i=\lambda(v_i)$ y sea 
\[
\mu=\sum_{i=1}^na_i\lambda_i.
\]
De forma que, para $j=1,\ldots,n$,
\[
\mu(v_j)=\left(\sum_{i=1}^na_i\lambda_i\right)(v_j)=\sum_{i=1}^ja_i\delta_{ij}=a_j=\lambda(v_j),
\]
luego $\mu=\lambda$.\qed

\begin{defn}
Suponga que $V$ tiene dimensi\'on finita y sea $\mathcal{B}=\{v_1,\ldots,v_n\}$ una base de $V$, donde $n=\dim(V)$. A la base $\mathcal{B}^*=\{\lambda_1,\ldots,\lambda_n\}$ de $V^*$ donde
\[
\lambda_i(v_j)=\delta_{ij}.
\]
la llamamos \emph{base dual} de $\mathcal{B}$.
\end{defn}

\begin{obs}\label{basedualinfinita}
Si $V$ tiene dimensi\'on finita, $\mathcal{B}=\{v_1,\ldots,v_n\}$ es una base de $V$ y $\mathcal{B}^*=\{\lambda_1,\ldots,\lambda_n\}$ es la base dual, entonces para todo $v\in V$
\[
v=\sum_{i=1}^n\lambda_i(v)v_i.
\]
De hecho si $a_1,\ldots,a_n\in K$ son tales que $a_1v_1+\ldots+a_nv_n=v$,
\[
\lambda_i(v)=\lambda_i\left(\sum_{j=1}^na_jv_j\right)=\sum_{j=1}^n a_j\delta_{ij}=a_i.
\]
Es decir, $\lambda_i$ arroja la coordenada en $v_i$.
\end{obs}

\begin{obs}\label{basedualinfinita}
Si $V$ tiene dimensi\'on infinita y $\{v_i\}_{i\in I}$ es una base de $V$, igualmente podemos definir la colecci\'on $\{\lambda_i\}_{i\in I}\subseteq V^*$ por
\[
\lambda_i(v_j)=\delta_{ij}.
\]
e igualmente tenemos que para todo $v\in V$
\[
v=\sum_{i\in I}\lambda_i(v)v_i.
\]
Note que $\lambda_i(v)=0$ para todo $i\in I$ salvo para una subcolecci\'on finita de indices. La diferencia con el caso en dimensi\'on infinita es que $\{\lambda_i\}_{i\in I}$ no es una base de $V^*$, pues en tal caso el funcional lineal $\lambda$ definido por $\lambda(v_i)=1$, para todo $i\in I$, no es una combinaci\'on lineal de $\{\lambda_i\}_{i\in I}$.
\end{obs}

\begin{defn}
Sean $S\subseteq V$ y $L\subseteq V^*$, definimos:
\begin{enumerate}
\item el \emph{anulador} de $S$ por
\[
S^0=\{\lambda\in V^*\ |\ \lambda(v)=0, \forall v\in S\};
\]
\item el \emph{cero} de $L$ por
\[
L_0=\{v\in V\ |\ \lambda(v)=0, \forall \lambda\in L\}.
\]
\end{enumerate} 
\end{defn}

\begin{pro}
Sean $S\subseteq V$ y $L\subseteq V^*$. Tenemos:
\begin{enumerate}
\item $S^0\le V^*$ y $L_0\le V$;
\item si $S_1,S_2\subseteq V$ y $L_1,L_2\subseteq V^*$ son tales que
\[
S_1\subseteq S_2\qquad\textrm{ y }\qquad L_1\subseteq L_2,
\]
entonces
\[
S_2^0\le S_1^0\qquad\textrm{ y }\qquad \left(L_2\right)_0\le\left(L_1\right)_0;
\]
\item $\langle S^0\rangle_0=\Sp(S)$;
\item si $V_1,V_2\le V$ y $V_1^*,V_2^*\le V^*$, entonces
\[
\left(V_1+V_2\right)^0=V_1^0\cap V_2^0,\textrm{ y } \left(V_1^*+V_2^*\right)_0=\left(V_1^*\right)_0\cap \left(V_2^*\right)_0;
\]
\item si $V$ tiene dimensi\'on finita,
\[
\dim\left(\langle S\rangle\right)+\dim(S^0)=\dim(V),\textrm{ y } \dim(L_0)+\dim\left(\langle L\rangle\right)=\dim(V^*).
\]
\end{enumerate}
\end{pro}

\dem
\begin{enumerate}
\item Si $\lambda_1,\lambda_2\in S^0$ y $c\in K$, entonces para todo $v\in S$
\[
(\lambda_1+\lambda_2)(v)=\lambda_1(v)+\lambda_2(v)=0,\textrm{ y } (c\lambda_1)(v)=c\lambda_1(v)=0, 
\]
es decir $\lambda_1+\lambda_2\in S^0$ y $c\lambda_1\in S^0$. Luego $S_0$ es un subespacio de $V^*$. Similarmente $L_0$ es un subespacio de $V$.
\item Sea $\lambda\in S_2^0$, luego, si $v\in S_1$, como $v\in S_2$, entonces $\lambda(v)=0$; en particular $\lambda\in S_1^0$. Similarmente, sea $v\in\left(L_2\right)^0$, luego, si $\lambda\in L_1$, como $\lambda\in L_2$, entonces $\lambda(v)=0$; en particular $v\in\left(L_1\right)^0$.
\item Sea $v\in\langle S\rangle$, entonces existen $v_1,\ldots,v_m\in S$ y $a_1,\ldots,a_m\in K$ tales que
\[
v=a_1v_1+\ldots+a_mv_m
\]
as\'i, si $\lambda\in S^0$, $\lambda(v)=a_1\lambda(v_1)+\ldots+a_m\lambda(v_m)=0$. Luego $\langle S\rangle\le \left(S^0\right)_0$.\\
Tome ahora un subconjunto $S'\subseteq S$ linealmente independiente tal que $\langle S'\rangle=\langle S\rangle$, el cual extendemos a una base $\mathcal{B}=\{v_i\}_{i\in I}$ de $V$. Defina, para cada $i\in I$, $\lambda_i\in V^*$ por
\[
\lambda_i(v_j)=\delta_{ij}
\]
para todo $j\in I$. De esta forma $\lambda_i\in S^0$ si y solo si $v_i\not\in S'$. Sea $J\subset I$ la subcolecci\'on de indices definida por $j\in J$ si $v_j\in S'$. Entonces $L=\{\lambda_i\}_{i\in I\setminus J}\subseteq S^0$ y $\left(S^0\right)_0\le L_0$. Ahora si $v\in V$, como
\[
v=\sum_{i\in J}\lambda_i(v)v_i+\sum_{i\in I\setminus J}\lambda_i(v)v_i
\] 
entonces $v\in L_0$ si y solo si $v\in\langle S'\rangle$, es decir $L_0=\langle S'\rangle$. Luego $\left(S^0\right)_0\le\langle S'\rangle=\langle S\rangle$.
\item Suponga que $\lambda\in (V_1+V_2)^0$, luego, si $v\in V_i$, con $i=1$ \'o $i=2$, entonces $v\in V_1+V_2$ y $\lambda(v)=0$, en particular $\lambda\in V_1^0\cap V_2^0$. Rec\'iprocamente, si $\lambda\in V_1^0\cap V_2^0$ y $v\in V_1+V_2$, con $v=v_1+v_2$, $v_1\in V_1$ y $v_2\in V_2$, $\lambda(v)=\lambda(v_1)+\lambda(v_2)=0$, en particular $\lambda\in (V_1+V_2)^0$.\\
Similarmente, suponga que $v\in\left(V_1^*+V_2^*\right)_0$, luego, si $\lambda\in V_i^*$, con $i=1$ \'o $i=2$, entonces $\lambda\in V_1^*+V_2^*$ y $\lambda(v)=0$, en particular $v\in \left(V_1^*\right)_0\cap \left(V_2^*\right)_0$. Rec\'iprocamente, si $v\in \left(V_1^*\right)_0\cap \left(V_2^*\right)_0$ y $\lambda\in V_1^*+V_2^*$, con $\lambda=\lambda_1+\lambda_2$, $\lambda_1\in V_1$ y $\lambda_2\in V_2$, $\lambda(v)=\lambda_1(v)+\lambda_2(v)=0$, en particular $v\in \left(V_1^*+V_2^*\right)_0$.
\item Tome un subconjunto $\{v_1,\ldots,v_k\}=S'\subseteq S$ linealmente independiente tal que $\langle S'\rangle=\langle S\rangle$, el cual extendemos a una base $\mathcal{B}=\{v_1\ldots,v_n\}$ de $V$. Sea $\mathcal{B}^*=\{\lambda_1\ldots,\lambda_n\}$ la base dual a $\mathcal{B}$. Defina $f\in\Hom_K(V,V)$ por
\[
f(v)=\sum_{i=1}^k\lambda_i(v)v_i.
\]
Por construcci\'on, $\im(f)=\langle S'\rangle=\langle S\rangle$ y $\ker(f)=\langle\lambda_{k+1},\ldots,\lambda_n\rangle_0$. Pero $\langle\lambda_{k+1}\ldots\lambda_{n}\rangle=S^0$, luego $\dim\left(\langle S\rangle\right)+\dim(S^0)=n$.\\
Similarmente, tome un subconjunto $\{\lambda_1,\ldots,\lambda_k\}=L'\subseteq L$ linealmente independiente tal que $\langle L'\rangle=\langle L\rangle$, el cual extendemos a una base $\mathcal{B}^*=\{\lambda_1,\ldots,\lambda_n\}$ de $V$. Sea $\{v_1,\ldots,v_n\}$ una base de $V$. Defina $f\in\Hom_K(V,V)$ por
\[
f(v)=\sum_{i=1}^k\lambda_i(v)v_i.
\]
Por construcci\'on, $\ker(f)=L'_0=L_0$ y $\im(f)=\langle v_1,\ldots,v_k\rangle$. Luego $\dim\im(f)=\dim\left(\langle L\rangle\right)$ y $\dim(L_0)+\dim\left(\langle L\rangle\right)=n$.\qed
\end{enumerate}

\begin{prop}
Sea $V_1\le V$ entonces $\left(V/V_1\right)^*=V_1^0$.
\end{prop}

\dem Tome $\pi_{V_1}:V\rightarrow V/V_1$ con $\pi_{V_1}(v)=v+V_1$; y, defina la transformaci\'on lineal $f:\left(V/V_1\right)^*\rightarrow V^*$ por $f(\lambda)=\lambda\circ\pi_{V_1}$. Veamos que $f$ es un isomorfismo. Primero es inyectiva pues si $\lambda\circ\pi_{V_1}=0$, entonces, para todo $v+V_1\in V/V_1$, $\lambda(v+V_1)=\lambda\circ\pi_{V_1}(v)=0$. Es decir $\lambda=0$. Por otro lado $f$ es sobreyectiva, pues dado $\mu\in V_1^0$, si $v-v'\in V_1$ entonces $\mu(v)-\mu(v')=\mu(v-v')=0$, luego la funci\'on $\lambda:V/V_1\rightarrow K$ tal que $\lambda(v+V_1)=\mu(v)$ es un funcional lineal tal que $f(\lambda)=\mu$. \qed

\begin{teo}\label{dualdual}
Existe una transformaci\'on lineal can\'onica inyectiva
\begin{eqnarray*}
\widehat{\bullet}: V & \longrightarrow & \left(V^*\right)^*\\
                            v &\longmapsto &\widehat{v}:\lambda\mapsto\lambda(v).
\end{eqnarray*}
Si $V$ tiene dimensi\'on finita, $\widehat{\bullet}$ es un isomorfismo.
\end{teo}

\dem Por definici\'on $\widehat{\bullet}$ es lineal, pues 
\[
\widehat{v_1+v_2}(\lambda)=\lambda(v_1+v_2)=\lambda(v_1)+\lambda(v_2)=\left(\widehat{v_1}+\widehat{v_2}\right)(\lambda).
\]
Ahora sea $\{v_i\}_{i\in I}$ una base de $V$, y $\{\lambda_i\}_{i\in I}\subseteq V^*$ la colecci\'on tal que $\lambda_i(v_j)=\delta_{ij}$. Como para todo $v\in V$, $v=\sum_{i\in I}\lambda_i(v)v_i$, si $\widehat{v}=0$ entonces $\lambda_i(v)=0$ para todo $i$ y as\'i $v=0$. Luego $\widehat{\bullet}$ es inyectiva.\\
Si $V$ tiene dimensi\'on finita, $\dim(V)=\dim(V^*)=\dim\left(\left(V^*\right)^*\right)$, entonces $\widehat{\bullet}$ es un isomorfismo.\qed

\section{Transformaci\'on dual}

\begin{defn}
Sea $f\in\Hom_K(V,W)$. Definimos la \emph{transformaci\'on dual}, $f^*\in\Hom_K(W^*,V^*)$, por
\[
f^*(\lambda)=\lambda\circ f
\]
para todo $\lambda\in W^*$.
\end{defn}

\begin{figure}[!hbp]
\centering
\begin{tikzpicture}[auto, node distance=2cm,>=latex']
    \node (V) {$V$};
    \node (W) [right of=V] {$W$};
    \node (K) [below of=W] {$K$};
    
    \path[->] (V) edge node {$f$} (W);
    \path[->] (W) edge  node {$\lambda$} (K);
    \path[dashed,<-] (K) edge  node {$f^*(\lambda)$} (V);
\end{tikzpicture}
\caption{Transformaci\'on dual}
\label{trdual}
\end{figure}

\begin{obs}
La linearidad del mapa $f^*$ se sigue de las siguientes igualdades, validas para todo $f\in\Hom_K(V,W)$, $\lambda_1,\lambda_2\in W^*$, $c\in K$:
\begin{eqnarray*}
(\lambda_1+\lambda_2)\circ f & = & \lambda_1\circ f+\lambda_2\circ f\\
(c\lambda_1)\circ f & = & c(\lambda_1\circ f)
\end{eqnarray*}
\end{obs}

\begin{pro}
Sean $U$ un espacio vectorial sobre $K$ y $f\in\Hom_K(V,W)$ y $g\in\Hom_K(W,U)$, entonces
\[
(g\circ f)^*=f^*\circ g^*
\]
\end{pro}

\dem Para $\lambda\in U^*$, tenemos
\[
(g\circ f)^*\lambda=\lambda\circ (g\circ f)=(\lambda\circ g)\circ f=g^*(\lambda)\circ f=f^*\circ g^* (\lambda)
\]
\qed

\begin{pro}
Sea $f\in\Hom_K(V,W)$, entonces
\begin{enumerate}
\item Si $f$ es sobreyectiva, $f^*$ es inyectiva; y,
\item Si $f$ es inyectiva, $f^*$ es sobreyectiva. 
\end{enumerate}
\end{pro}

\dem
\begin{enumerate}
\item Sea $\lambda\in W^*$ tal que $f^*(\lambda)=0$, entonces, dado $w\in W$, como $f$ es sobreyectiva, existe $v\in V$ tal que $w=f(v)$, as\'i
\[
\lambda(w)=\lambda\left(f(v)\right)=f^*(\lambda)(v)=0
\]
luego $\lambda=0$ y $f^*$ es inyectiva.
\item Sea $W_1,W_2\le W$ tales que $W=W_1\oplus W_2$ y $W_1=f(V)$. Tome $\mu\in V$, y  defina $\lambda: W \rightarrow K$ por
\[
\lambda(w)=\mu(v_1)
\]
donde $w=w_1+w_2$, $w_1\in W_1$ y $w_2\in W_2$ y $f(v_1)=w_1$. Como $f$ es inyectiva, $v_1$ es \'unico, y la funci\'on $\lambda$ est\'a bien definida. Como la descomposici\'on de $w=w_1+w_2$ es lineal y $\mu$ y $f$ son lineales, entonces $\lambda$ lineal, es decir $\lambda\in W^*$. Por contrucci\'on $\mu=f^*\lambda$ pues
\[
f^*(\lambda)(v_1)=\lambda(f(v_1))=\lambda(w_1)=\mu(v_1),
\]
luego $f^*$ es sobreyectiva.\qed
\end{enumerate}

\begin{pro}
Sean $V_1,V_2\le V$ tales que $V=V_1\oplus V_2$; y, $\pi_1:V\rightarrow  V_1$ y $\pi_2:V\rightarrow V_2$ respectivamente las proyecciones sobre $V_1$ y $V_2$ dadas por la descomposici\'on $V=V_1\oplus V_2$. Entonces
\[
V^*=\pi_1^*(V^*_1)\oplus\pi_2^*(V^*_2)
\]
\end{pro}

\dem Dado $\lambda\in V^*$, defina $\lambda_1\in V_1^*$ y $\lambda_2\in V_2$ por
\[
\lambda_1(v_1)=\lambda(v_1)\qquad\lambda_2(v_2)=\lambda(v_2).
\]
De tal forma que si $v=v_1+v_2\in V$ con $v_1=\pi_1(v)$ y $v_2=\pi_2(v)$, entonces
\begin{eqnarray*}
\lambda(v) & = & \lambda(v_1)+\lambda(v_2)\\
 & = & \lambda_1\left(\pi_1(v)\right)+\lambda_2\left(\pi_2(v)\right)\\
 & = & \left(\pi_1^*(\lambda_1)+\pi_2^*(\lambda_2)\right)(v)
\end{eqnarray*}  
luego $V^*=\pi_1^*(V_1)+\pi_2^*(V_2)$. Ahora, si $\lambda\in \pi_1^*(V_1)\cap\pi_2^*(V_2)$, existen $\lambda_1\in V_1^*$ y $\lambda_2\in V_2^*$ tales que $\lambda=\pi_1^*(\lambda_1)=\pi_2^*(\lambda_2)$, de esta forma, para todo $v=v_1+v_2\in V$ con $v_1=\pi_1(v)$ y $v_2=\pi_2(v)$
\[
\lambda(v)=\lambda(v_1)+\lambda(v_2)=\lambda_2\left(\pi_2(v_1)\right)+\lambda_1\left(\pi_1(v_2)\right)=\lambda_2(0)+\lambda_1(0)=0.
\]
Luego $\pi_1^*(V_1)\cap\pi_2^*(V_2)=\{0\}$.\qed

\begin{teo}
El mapa
\begin{eqnarray*}
\bullet^*:\Hom_K(V,W) & \longrightarrow & \Hom_K(W^*,V^*)\\
f & \longmapsto & f^*
\end{eqnarray*}
es una transformaci\'on lineal inyectiva. Si $W$ tiene dimensi\'on finita entonces es un isomorfismo.
\end{teo}

\dem Sean $f,g\in\Hom_K(V,W)$, y $c\in K$. Dado $\lambda\in W^*$ tenemos:
\begin{eqnarray*}
(f+g)^*(\lambda) & = & \lambda\circ(f+g)\\
   & = & \lambda\circ f+\lambda\circ g\\
   & = & f^*(\lambda)+g^*(\lambda)\\
   & = & (f^*+g^*)(\lambda)\\
(cf)^*(\lambda) & = &  \lambda\circ (cf)\\
   & = & c(\lambda\circ f)\\
   & = & cf^*(\lambda).
\end{eqnarray*}
Es decir $(f+g)^*=f^*+g^*$ y $(cf)^*=cf^*$, y as\'i $\bullet^*$es lineal.\\
Ahora suponga que $f^*=0$, es decir $f^*(\lambda)=0$ para todo $\lambda\in W^*$. Tomamos una base $\{w_i\}_{i\in I}$ de $W$ y definimos $\{\lambda_i\}_{i\in I}\subseteq W^*$ por
\[
\delta_i(v_j)=\delta_{ij}
\]
As\'i, como en Observaci\'on \ref{basedualinfinita}, tenemos para todo $v\in V$
\[
f(v)=\sum_{i\in I}\lambda_i\left(f(v)\right)w_i=\sum_{i\in I} f^*(\lambda_i)(v)w_i=\sum_{i\in I} 0w_i=0.
\]
Es decir $f=0$ y as\'i $\bullet^*$ es inyectiva.\\
Si adem\'as asumimos que $W$ tiene dimensi\'on finita, entonces $\{\lambda_i\}_{i\in I}$ es la base de $W^*$, dual de $\{w_i\}_{i\in I}$. En particular $\phi\in\Hom^*(W^*,V^*)$ est\'a un\'ivocamente determinado por la im\'agen $\{\phi(\lambda_i)\}_{i\in I}\subseteq V^*$. Defina $f\in\Hom_K(V,W)$ por la suma finita
\[
f(v)=\sum_{j\in I}\left[\phi(\lambda_j)(v)\right]w_j,
\]
de forma tal que para todo $i\in I$, $v\in V$,
\begin{eqnarray*}
f^*(\lambda_i)(v) & = & \lambda_i(f(v))\\
 & = & \lambda_i\left(\sum_{j\in I}\left[\phi(\lambda_j)(v)\right]w_j\right)\\
 & = & \sum_{j\in I}\left[\phi(\lambda_j)(v)\right]\lambda_i(w_j)\\
 & = & \sum_{i\in I}\left[\phi(\lambda_j)(v)\right]\delta_{ij}=\phi(\lambda_i)(v).
\end{eqnarray*}
Es decir $f^*(\lambda_i)=\phi(\lambda_i)$, para todo $i\in I$. Luego $f^*=\phi$, de donde $\bullet^*$ es tambi\'en sobreyectiva, as\'i es un isomorfismo.\qed

\begin{obs}
Suponga que $W$ tiene dimensi\'on finita, sea $\mathcal{B}_W=\{w_1,\ldots,w_m\}$ una base de $W$ y $\lambda\in W^*=\Hom_K(W,K)$. Si tomamos la base $\{1\}$ de $K$, entonces la representaci\'on matricial de $\lambda$ respecto a las base $\mathcal{B}_W$ y $\{1\}$ es
\[
\Big[\lambda\Big]^{\{1\}}_{\mathcal{B}_W}=\Big[\lambda(w_1)\cdots\lambda(w_m)\Big].
\]
Ahora, si $\mathcal{B}_V$ es una base de $V$ y $f\in\Hom_K(V,W)$, tenemos
\[
\Big[f^*(\lambda)\Big]^{\{1\}}_{\mathcal{B}_V}=\Big[\lambda\circ f\Big]^{\{1\}}_{\mathcal{B}_V}=\Big[\lambda\Big]^{\{1\}}_S\Big[f\Big]^{\mathcal{B}_V}_{\mathcal{B}_W}.
\]
Por otro lado, podemos tomar las coordenadas de $\lambda$ en la base $\mathcal{B}_W^*=\{\lambda_1,\ldots,\lambda_m\}$ de $W^*$ dual de $\mathcal{B}_W$:
\[
\Big[\lambda\Big]^{\mathcal{B}_W^*}=\left[\begin{array}{c} \lambda(w_1)\\ \vdots \\ \lambda(w_m)\end{array}\right]
\]
Si ademas asumimos que $V$ tiene tambi\'en dimensi\'on finita y tomamos la base $\mathcal{B}_V^*$  de $V^*$ dual de $T$, entonces
\[
\Big[f^*(\lambda)\Big]^{\mathcal{B}_V^*}=\Big[f^*\Big]^{\mathcal{B}_V^*}_{\mathcal{B}_W^*}\Big[\lambda\Big]^{\mathcal{B}_W^*}.
\]
La pregunta inmediata es: ?`Cu\'al es la relaci\'on entre $\Big[f\Big]^{\mathcal{B}_W}_{\mathcal{B}_V}$ y $\Big[f^*\Big]^{\mathcal{B}_V^*}_{\mathcal{B}_W^*}$?
\end{obs}

\begin{defn}
Sean $I,J$ conjuntos y $A\in M_{I\times J}(K)$, definimos la \emph{matriz traspuesta} de $A$ por $A^\intercal\in M_{J\times I}(K)$ tal que
\[
A^\intercal(j,i)=A(i,j)
\]
para todo $(j,i)\in J\times I$. Es decir el valor en $(j,i)$ de $A^\intercal$ es el valor en $(i,j)$ de $A$. Similarmente si $m,n\in\mathbb{Z}_{>0}$, y $A\in M_{m\times n}(K)$, definimos su traspuesta por $A^\intercal\in M_{n\times m}(K)$ tal que
\[
A^\intercal(j,i)=A(i,j)
\]
Sea $A\in M_{I\times I}(K)$, o $A\in M_{n\times n}(K)$, decimos que $A$ es \emph{sim\'etrica} si $A^\intercal=A$.
\end{defn}

\begin{teo}
Suponga que $V$ y $W$ tienen dimensi\'on finita, $n=\dim(V)>0$ y $m=\dim(W)>0$, y $f\in\Hom_K(V,W)$. Sean $\mathcal{B}_W=\{w_1,\ldots,w_m\}$ y $\mathcal{B}_V=\{v_1,\ldots,v_n\}$ respectivamente bases de $W$ y $V$; y, sean $\mathcal{B}_W^*=\{\lambda_1,\ldots,\lambda_m\}$ y $\mathcal{B}_V^*=\{\mu_1,\ldots,\mu_n\}$ respectivamente las bases de $W^*$ y $V^*$ duales de $S$ y $T$. Sea $A\in M_{m\times n}(K)$ la representaci\'on matricial de $f$ respecto a las bases $\mathcal{B}_V$ y $\mathcal{B}_W$, entonces $A^\intercal\in M_{n\times m}(K)$ es la representaci\'on matricial de $f^*$ respecto a las bases $\mathcal{B}_V^*$ y $\mathcal{B}_W^*$. Es decir,
\[
\Big[f^*\Big]^{\mathcal{B}_V^*}_{\mathcal{B}_W^*}=\left(\Big[f\Big]^{\mathcal{B}_W}_{\mathcal{B}_V}\right)^\intercal 
\] 
\end{teo}

\dem Si $a_{ij}\in K$ es la $ij$-\'esima entrada de $A=\Big[f\Big]^{\mathcal{B}_W}_{\mathcal{B}_V}$, entonces
\[
a_{ij}=\lambda_i\left(f(v_j)\right)=f^*(\lambda_i)(v_j);
\]
y,
\[
f^*(\lambda_i)=\sum_{l=1}^{n}\left[f^*(\lambda_i)(v_l)\right]\mu_l,
\]
luego la $ji$-esima entrada de $\Big[f^*\Big]^{\mathcal{B}_V^*}_{\mathcal{B}_W^*}$ es $f^*(\lambda_i)(v_j)=a_{ij}$.\qed
\chapter{Espacios eucl\'ideos}

Sea $V$ un espacio vectorial sobre $\mathbb{R}$.

\section{Producto interno}

\begin{defn}
Un \emph{producto interno} en $V$ es una funci\'on
\begin{eqnarray*}
\langle\bullet,\bullet\rangle: V\times V & \longrightarrow & \mathbb{R}\\
(v_1,v_2) & \longmapsto & \langle v_1;v_2\rangle
\end{eqnarray*}
tal que:
\begin{enumerate}
\item \emph{es bilineal}: para todo $v,v_1,v_2\in V$ y $c\in\mathbb{R}$
\begin{eqnarray*}
\langle v_1+v_2;v\rangle & = & \langle v_1;v\rangle+\langle v_2;v\rangle\\
\langle cv_1;v_2\rangle & = & c\langle v_1;v_2\rangle\\
\langle v;v_1+v_2\rangle & = & \langle v;v_1\rangle + \langle v;v_2\rangle\\
\langle v_1;cv_2\rangle & = & c\langle v_1;v_2\rangle;
\end{eqnarray*}
\item \emph{es sim\'etrica}: para todo $v_1,v_2\in V$
\[
\langle v_2;v_1\rangle=\langle v_1;v_2\rangle;
\]
\item \emph{es definitivamente positiva} para todo $v\in V$, $v\ne 0$,
\[
\langle v;v\rangle>0.
\]
\end{enumerate}
Un \emph{espacio eucl\'ideo} es un espacio vectorial sobre $\mathbb{R}$ provisto de un producto interno. 
\end{defn}


\begin{obs}
Se sigue que $\langle v;v \rangle=0$ si y solo si $v=0$.
\end{obs}

\begin{ejem}
\begin{enumerate}
\item Sobre $V=\mathbb{R}^n$,
\[
\langle (x_1,\ldots,x_n);(y_1,\ldots,y_n)\rangle =\sum_{i=1}^n x_iy_i.
\]
\item Sobre $V=M_{n\times n}(\mathbb{R})$,
\[
\langle A; B\rangle=\tr(B^\intercal A).
\]
\item Sea $[a,b]\subseteq\mathbb{R}$ un intervalo cerrado. Sobre $V=C^0[a,b]$, el conjunto de funciones continuas $[a,b]\rightarrow\mathbb{R}$,
\[
\langle f;g \rangle=\int_a^bf(x)g(x)dx.
\]
\end{enumerate}
\end{ejem}

\begin{defn}
Dado un espacio eucl\'ideo $V$, definimos la norma de $v\in V$ por $\|v\|=\sqrt{\langle v;v\rangle}$.
\end{defn}

\begin{pro}
Sea $V$ un espacio eucl\'ideo, entonces:
\begin{enumerate}
\item $\|cv\|=|c|\|v\|$, para todo $c\in\mathbb{R}$ y $v\in V$;
\item \emph{Desigualdad de Cauchy-Schwarz}: $|\langle v_1;v_2\rangle|\le\|v_1\|\|v_2\|$, para todo $v_1,v_2\in V$, m\'as a\'un se tiene $|\langle v_1;v_2\rangle|=\|v_1\|\|v_2\|$ \'unicamente cuando $\{v_1,v_2\}$ es linealmente dependiente; y,
\item \emph{Desigualdad triangular}: $\|v_1+v_2\|\le \|v_1\|+\|v_2\|$, para todo $v_1,v_2\in V$, m\'as a\'un se tiene $\|v_1+v_2\|= \|v_1\|+\|v_2\|$ si y solo si $av_1=bv_2$ con $a,b\ge 0$.
\end{enumerate}
\end{pro}

\dem\begin{enumerate}
\item Dados $c\in\mathbb{R}$ y $v\in V$
\[
\|cv\|=\sqrt{\langle cv;cv\rangle}=\sqrt{c^2\langle v;v\rangle}=|c|\sqrt{\langle v;v\rangle}=|c|\|v\|.
\]
\item Si $v_1=0$ o $v_2=0$ tenemos $0=|\langle v_1,v_2\rangle|=\|v_1\|\|v_2\|$. En el caso general, para todo $a,b\in\mathbb{R}$,
\begin{eqnarray*}
0 \le \|av_1-bv_2\|^2 & = & \langle av_1-bv_2 , av_1-bv_2\rangle\\
   & = & a^2\langle v_1, v_1\rangle-ab\langle v_2, v_1\rangle-ab\langle v_1, v_2\rangle+b^2\langle v_2, v_2\rangle\\
   & = & a^2\|v_1\|^2-2ab\langle v_1, v_2\rangle+b^2\|v_2\|^2.
\end{eqnarray*}
Si suponemos ahora que $v_2\ne 0$, es decir $\|v_2\|^2>0$, y $a\ne 0$, dividiendo por $a^2$, obtenemos as\'i una expresi\'on cuadr\'atica en $b/a$, p\'ositiva o nula para todo valor, luego su discriminante satisface
\[
4\langle v_1,v_2\rangle^2-4\|v_1\|^2\|v_2\|^2\le 0
\]
Lo cual implica la desigualdad deseada. Igualmente, esta expresi\'on cuadr\'atica se anula, es decir $av_1-bv_2=0$ para alg\'un $a,b\in\mathbb{R}$, si y solo si su discrimante es cero, es decir si y solo si $\langle v_1,v_2\rangle^2=\|v_1\|^2\|v_2\|^2$.
\item Tomando $a=1$ y $b=-1$ en la expresi\'on arriba, obtenemos
\[
\|v_1+v_2\|^2=\|v_1\|^2+2\langle v_1; v_2\rangle+\|v_2\|^2;
\]
y la desigualdad de Cauchy-Schwarz implica
\[
\|v_1+v_2\|^2\le\|v_1\|^2+2\|v_1\|\|v_2\|+\|v_2\|^2=(\|v_1\|+\|v_2\|)^2
\]
que es equivalente a la desigualdad afirmada. Se obtiene una igualdad en la desigualdad triangular si y solo si $\langle v_1, v_2\rangle=\|v_1\|\|v_2\|$, lo cual equivale a $av_1-bv_2=0$ donde $a=\|v_2\|$ y $b=\|v_1\|$.\qed
\end{enumerate}

\begin{obs}
En vista de la desigualdad de Cauchy-Schwarz, es usual definir el \'angulo entre $v_1$ y $v_2$ por $\theta=\arccos\left(\langle v_1;v_2\rangle/\|v_1\|\|v_2\|\right)$, siempre que $v_1\ne 0$ y $v_2\ne 0$. La direcci\'on, o el signo, del \'angulo no se puede definir intr\'insecamente a partir del producto interno, hace falta definir una orientaci\'on. Con est\'a definici\'on del \'angulo entre dos elementos distintos del origen en un espacio eucl\'ideo, se empata con la definici\'on cl\'asica, seg\'un la cual
\[
\langle v_1;v_2\rangle=\|v_1\|\|v_2\|\cos\theta.
\]
cuando $v_1\ne 0$ y $v_2\ne 0$. El caso particular de mayor inter\'es es naturalmente cuando existe una relaci\'on de ortogonalidad entre $v_1$ y $v_2$.
\end{obs}

\begin{defn}
Sea $V$ un espacio eucl\'ideo y $S\subseteq V$. Decimos que $S$ es \emph{ortogonal} si $\langle v_1;v_2\rangle=0$ para todo $v_1,v_2\in T$ tales que $v_1\ne v_2$. Si adem\'as $\|v\|=1$ para todo $v\in S$, decimos que $S$ es \emph{ortonormal}.
\end{defn}

\begin{obs}\label{ortokro}
Note que $S=\{v_i\}_{i\in I}$ es ortonormal si y solo si
\[
\langle v_i;v_j\rangle=\delta_{ij}.
\]
para todo $i,j\in I$.
\end{obs}

\begin{pro}\label{ortlinind}
Sea $V$ un espacio eucl\'ideo. Si $S\subset V$ es ortogonal, y $0\not\in S$, entonces $S$ es linealmente independiente.
\end{pro}

\dem Suponga que $v_1,\ldots,v_n\in S$ y $a_1,\ldots,a_n\in S$ son tales que
\[
a_1v_1+\ldots+a_nv_n=0.
\]
Entonces, para $i=1,\ldots,n$,
\[
0=\langle 0;v_i\rangle=\langle a_1v_1+\cdots+a_nv_n;v_i\rangle=a_1\langle v_1;v_i\rangle+\ldots+a_n\langle v_n;v_i\rangle=a_i\|v_i\|^2,
\]
pero $v_i\ne 0$ pues $0\not\in S$, luego $\|v_i\|^2\ne 0$, y as\'i $a_i=0$.\qed

\begin{teo}[Ortogonalizaci\'on de Gram-Schmidt]\label{gramsch}
Sea $V$ un espacio eucl\'ideo. Suponga que $V$ tiene dimensi\'on finita, entonces $V$ tiene una base ortonormal. M\'as a\'un, si $\{v_1,\ldots,v_n\}$ es una base de $V$, existe una base ortonormal $\{u_1,\ldots,u_n\}$ de $V$ tal que para $k=1,\ldots,n$
\[
\langle v_1,\ldots,v_k\rangle=\langle u_1,\ldots,u_k\rangle.
\]
\end{teo}

\dem Sea $\{v_1,\ldots,v_n\}$ una base de $V$, definimos recursivamente $\{v'_1,\ldots,v'_n\}$ por
\begin{eqnarray*}
v'_1 & = & v_1\\
v'_{k+1} & = & v_{k+1}-\sum_{i=1}^k\frac{\langle v_{k+1};v'_i\rangle}{\|v'_i\|^2} v'_i.
\end{eqnarray*}
Veamos que $\{v'_1,\ldots,v'_n\}$ es ortogonal. Para esto, basta establecer por inducci\'on que, si $1\le k< n$, entonces $\langle v'_{k+1};v'_j\rangle=0$ para $1\le j\le k$. Para el caso base, $k=1=j$ y
\begin{eqnarray*}
\langle v'_2;v'_1\rangle & = & \langle v_2;v'_1\rangle-\frac{\langle v_2;v'_1\rangle}{\|v'_1\|^2} \langle v'_1;v'_1\rangle\\
  & = & \langle v_2,v'_1\rangle-\langle v_2;v'_1\rangle\\
  & = & 0.
\end{eqnarray*}
Para el paso inductivo, asumimos que $\langle v'_i;v'_j\rangle=0=\langle v'_j;v'_i\rangle$, siempre que $1\le i<j\le k$, de tal forma que si $1\le j\le k$,
\begin{eqnarray*}
\langle v'_{k+1};v'_j\rangle & = & \langle v_{k+1}; v'_j\rangle-\sum_{i=1}^k\frac{\langle v_{k+1};v'_i\rangle}{\|v'_i\|^2}\langle v'_i; v'_j\rangle\\
 & = & \langle v_{k+1}; v'_j\rangle-\frac{\langle v_{k+1};v'_j\rangle}{\|v'_j\|^2}\langle v'_j; v'_j\rangle\\
 & = & \langle v_{k+1}; v'_j\rangle-\langle v_{k+1}; v'_j\rangle\\
 & = & 0
\end{eqnarray*}
Adem\'as recursivamente vemos que
\begin{eqnarray*}
\langle v_1\rangle & = & \langle v'_1\rangle \\
\langle v_1,\ldots,v_{k+1}\rangle & = & \langle v'_1,\ldots,v'_{k+1}\rangle
\end{eqnarray*}
pues $v_{k+1}-v'_{k+1}\in\langle v'_1,\ldots,v'_k\rangle=\langle v_1,\ldots,v_k\rangle$. Note que, como $\{v_1,\ldots,v_n\}$ es linealmente independiente, entonces $v_{k+1}\not\in\langle v_1,\ldots,v_k\rangle=\langle v'_1,\ldots,v'_k\rangle$, luego $v'_{k+1}\not\in\langle v'_1,\ldots,v'_k\rangle$ y as\'i, para $i=1,\ldots,n$, $v'_i\ne 0$. De donde, como $\{v'_1,\ldots,v'_n\}$ es ortogonal y no contiene al origen, es un conjunto linealmente independiente con el mismo n\'umero de elementos que la dimensi\'on de $V$, entonces es una base de $V$.\\
Finalmente, para obtener la base ortonormal basta tomar, para $i=1,\ldots,n$,
\[
u_i=\frac{1}{\|v'_i\|}v'_i.
\]
Tenemos para $k=1,\ldots,n$
\[
\langle v_1,\ldots,v_k\rangle=\langle v'_1,\ldots,v'_k\rangle=\langle u_1,\ldots,u_k\rangle.
\]
\qed

\begin{pro}\label{coorortonor}
Sea $V$ un espacio eucl\'ideo. Suponga que $V$ tiene dimensi\'on finita y $\{u_1,\ldots,u_n\}$ es una base ortonormal de $V$, entonces para todo $v\in V$
\[
v=\sum_{i=1}^n\langle v;u_i\rangle u_i.
\]
En particular, si $v_1,v_2\in V$ son tales que
\[
v_1=\sum_{i=1}^nx_iu_i\qquad v_2=\sum_{i=1}^ny_iu_i,
\]
entonces
\[
\langle v_1;v_2\rangle=\sum_{i=1}^nx_iy_i.
\]
\end{pro}

\dem Como $\{u_1,\ldots,u_n\}$ es base, existen $a_1,\ldots,a_n\in K$ tal que $v=\sum_{i=1}^na_iu_i$. De esta forma, para $j=1,\ldots,n$, 
\[
\langle v;u_j\rangle=\sum_{i=1}^na_i\langle u_i;u_j\rangle=\sum_{i=1}^n a_i\delta_{ij}=a_j
\]
(ver Observaci\'on \ref{ortokro}). Finalmente,
\[
\langle v_1;v_2\rangle=\langle\sum_{i=1}^nx_iu_i;\sum_{j=1}^ny_ju_j\rangle=\sum_{i=1}^n\sum_{j=1}^nx_iy_j\delta_{ij}=\sum_{i=1}^nx_iy_i.
\]
\qed

\begin{obs}\label{prodort}
Si $\mathcal{B}=\{u_1,\ldots,u_n\}$ es una base ortonormal de $V$, la propiedad anterior implica que para todo $v_1,v_2\in V$ tenemos
$$\langle v_1;v_2\rangle=\left(\big[v_2\big]^{\mathcal{B}}\right)^\intercal\big[v_1\big]^{\mathcal{B}}$$
\end{obs}

\begin{defn}
Sean $V$ un espacio eucl\'ideo y $S\subseteq V$, el \emph{conjunto ortogonal} a $S$ est\'a definido por
\[
S^\perp=\{v\in V|\ \langle v;u\rangle=0, \textrm{ para todo } u\in S\}
\]
\end{defn}

\begin{pro}\label{ortsubesp}
Sean $V$ un espacio eucl\'ideo y $S\subseteq V$, entonces $S^\perp\le V$.
\end{pro}

\dem Como $\langle 0,v\rangle=0$ para todo $v\in V$, $0\in S^\perp$. Tome ahora $v_1,v_2,v\in S^\perp$ y $a\in\mathbb{R}$, entonces para todo $u\in S$
\begin{eqnarray*}
\langle v_1+v_2;u\rangle & = & \langle v_1;u\rangle+\langle v_2;u\rangle=0\\
\langle av;u\rangle & = & a\langle v;u\rangle=0 
\end{eqnarray*}
luego $v_1+v_2\in S^\perp$ y $av\in S^\perp$; as\'i, $S^\perp$ es un subespacio de $V$.\qed

\begin{teo}\label{complort}
Sea $V$ un espacio eucl\'ideo. Suponga que $V$ tiene dimensi\'on finita y sea $U\le V$, entonces
\[
V=U\oplus U^\perp
\]
\end{teo}

\dem Sean $n=\dim(V)$, $m=\dim(U)$ y $\{v_1,\ldots,v_n\}$ una base de $V$ tal que $\{v_1,\ldots,v_m\}$ es una base de $U$. Sea $\{u_1,\ldots,u_n\}$ la base obtenida mediante ortogonalizaci\'on a partir de $\{v_1,\ldots,v_n\}$. En particular $U=\langle u_1,\ldots,u_m\rangle$ y $\langle u_{m+1},\ldots,u_n\rangle\le U^\perp$. Tome $v\in U^\perp$, tenemos
\[
v=\sum_{i=1}^n\langle v;u_i\rangle u_i=\sum_{i=m+1}^n\langle v;u_i\rangle u_i,
\]
luego $v\in\langle u_{m+1},\ldots,u_n\rangle$. Entonces $U^\perp=\langle u_{m+1},\ldots,u_n\rangle$ y $V=U\oplus U^\perp$.\qed

\begin{defn}
Sea $V$ un espacio eucl\'ideo de dimensi\'on finita y $U\le V$. Llamamos a $U^\perp$ el \emph{complemento ortogonal de $U$}. A la proyecci\'on
\[
p^\perp_U:V\longrightarrow V
\]
sobre $U$, definida por la descomposici\'on $V=U\oplus U^\perp$ la llamamos \emph{proyecci\'on ortogonal sobre $U$}.
\end{defn}

\begin{pro}
Sea $V$ un espacio eucl\'ideo de dimensi\'on finita, $\dim(V)=n$,  y $\langle v_1,\ldots,v_m\rangle=U\le V$. Si $\mathcal{B}$ es una base ortonormal de $V$ y $\{v_1,\ldots,v_m\}$ es linealmente independiente entonces
$$ \Big[p^\perp_U\Big]^{\mathcal{B}}_{\mathcal{B}}=A(A^\intercal A)^{-1}A^\intercal $$
donde $A\in M_{n\times m}(\mathbb{R})$ es la matrix cuya $j$-\'esima columna es $\Big[v_j\Big]^\mathcal{B}$, $j=1,\ldots,n$.
\end{pro}

\dem Para todo $v\in V$ tenemos $p^\perp_U(v)\in U$ luego existe un \'unico $\overline{c}_v\in M_{m\times 1}(\mathbb{R})$ tal que
$$\Big[p^\perp_U(v)\Big]^\mathcal{B}=Ac_v$$
Si $P=\Big[p^\perp_U\Big]^\mathcal{B}$, y $\overline{x}=\Big[v\Big]^\mathcal{B}$ entonces
$$P\overline{x}=A\overline{c}_v.$$
Ahora como $v-p^\perp_U(v)\in U^\perp$ entonces $\langle v-p^\perp_U(v);v_j\rangle=0$ para $j=1,\ldots,n$ luego (ver Observaci\'on \ref{prodort})
$$\left(\Big[v_j\Big]^\mathcal{B}\right)^\intercal (\overline{x}-A\overline{c}_v)=0$$
y
\begin{align*}
0 &= A^\intercal(\overline{x}-A\overline{c}_v)\\
 &= A^\intercal\overline{x}-A^\intercal A\overline{c}_v
\end{align*}
Veamos que $A^\intercal A\in M_{m\times m}(\mathbb{R})$ es invertible. De hecho, si $\{u_1,\ldots,u_m\}$ es una base ortonormal de $U$ y $c_{ij}\in\mathbb{R}$, $i,j=1,\ldots,m$, son tales que
$$u_j=c_{1j}v_1+\ldots+c_{mj}v_m,\quad j=1,\ldots,m$$
y $C=(c_{ij})$ entonces $C\in M_{m\times m}(\mathbb{R})$ es invertible y la $j$-\'esima columna de $AC$ es $\Big[u_j\Big]^\mathcal{B}$. As\'i
$$I_m=(AC)^\intercal AC=C^\intercal A^\intercal A C$$
y $A^\intercal A=(CC^\intercal)^{-1}$, luego $A^\intercal A$ es invertible. Tenemos entonces
\begin{align*}
\overline{c}_v & = (A^\intercal A)^{-1}A\overline{x}\\
A\overline{c}_v & =  A(A^\intercal A)^{-1}A\overline{x}\\
P\overline{x} & =  A(A^\intercal A)^{-1}A\overline{x}\\
\end{align*}
y se sigue $P=A(A^\intercal A)^{-1}A^\intercal$.\qed
\begin{pro}\label{proyautoadj}
Sea $V$ un espacio eucl\'ideo. Suponga que $V$ tiene dimensi\'on finita y sea $U\le V$. Tenemos para todo $v_1,v_2\in V$
\[
\langle p^\perp_U(v_1);v_2\rangle=\langle v_1;p^\perp_U(v_2)\rangle.
\]
Si $\{u_1,\ldots,u_m\}$ es una base ortonormal de $U$ entonces para todo $v\in V$
\[
p^\perp_U(v)=\sum_{i=1}^m\langle v;u_i\rangle u_i.
\]
\end{pro}

\dem Sea $v'_1,v'_2\in U^\perp$ tales que
\[
v_1=p^\perp_U(v_1)+v'_1 \qquad v_2=p^\perp_U(v_2)+v'_2.
\]
Entonces
\begin{eqnarray*}
\langle p^\perp_U(v_1);v_2\rangle & = & \langle p^\perp_U(v_1);p^\perp_U(v_2)\rangle+\langle p^\perp_U(v_1);v'_2\rangle=\langle p^\perp_U(v_1);p^\perp_U(v_2)\rangle\\
\langle v_1;p^\perp_U(v_2)\rangle & = & \langle p^\perp_U(v_1);p^\perp_U(v_2)\rangle+\langle v'_1;p^\perp_U(v_2)\rangle=\langle p^\perp_U(v_1);p^\perp_U(v_2)\rangle.
\end{eqnarray*}
Finalmente, si completamos la base ortonormal $\{u_1,\ldots,u_m\}$ de $U$ a una base ortonormal $\{u_1,\ldots,u_n\}$ de $V$,
\[
v=\underbrace{\sum_{i=1}^m\langle v;u_i\rangle u_i}_{\in U}+\underbrace{\sum_{i=m+1}^n\langle v;u_i\rangle u_i}_{\in U^\perp}.
\]
\qed

\begin{obs}
Los operadores sobre un espacio eucl\'ideo que, como la proyecci\'on ortogonal, pasan de un lado al otro del producto interno tienen varias propiedades, la m\'as importante de ellas es que son diagonalizables, el cual es el contenido del Teorema Espectral. Antes de establecer este resultado, necesitamos elaborar la teor\'ia de los operadores adjuntos.
\end{obs}

\section{Operador adjunto}

Sea $V$ un espacio eucl\'ideo y $f\in\Hom_\mathbb{R}(V,V)$ un operador.

\begin{defn}
Sea $g\in\Hom_\mathbb{R}(V,V)$, decimos que $g$ es un \emph{operador adjunto de $f$} si para todo $v_1,v_2\in V$
\[
\langle g(v_1);v_2 \rangle=\langle v_1;f(v_2)\rangle.
\]
Decimos que $f$ es \emph{auto-adjunto} si $f$ es un operador adjunto de $f$. 
\end{defn}

\begin{obs}
Note que si $g$ es adjunto de $f$, entonces $f$ es adjunto de $g$. De hecho
\[
\langle f(v_1);v_2\rangle= \langle v_2;f(v_1)\rangle= \langle g(v_2);v_1\rangle= \langle v_1;g(v_2)\rangle
\]
\end{obs}

\begin{prop}\label{adjtras}
Suponga que $V$ tiene dimensi\'on finita, entonces existe un \'unico operador $g\in\Hom_{R}(V,V)$ adjunto de $f$. M\'as a\'un, si $\mathcal{B}=\{u_1,\ldots,u_n\}$ es una base ortonormal de $V$, entonces
\[
\Big[g\Big]^\mathcal{B}_\mathcal{B}=\left(\Big[f\Big]^\mathcal{B}_\mathcal{B}\right)^\intercal
\]
\end{prop}

\dem Defina el operador $g\in\Hom_\mathbb{R}(V,V)$ por la imagen de la base $\mathcal{B}$:
\[
g(u_j)=\sum_{i=1}^n\langle u_j;f(u_i)\rangle u_i.
\]
De esta forma
\[
\langle g(u_j);u_i\rangle=\langle u_j;f(u_i)\rangle
\]
y por bilinearidad del producto interno, $g$ es adjunto de $f$. Por otro lado si, $h\in\Hom_\mathbb{R}(V,V)$ es adjunto de $f$, por Propiedad \ref{coorortonor},
\begin{eqnarray*}
h(u_j) & = & \sum_{i=1}^n \langle h(u_j);u_i\rangle u_i\\
         & = & \sum_{i=1}^n \langle u_j;f(u_i)\rangle u_i\\
         & = & g(u_j),
\end{eqnarray*}
luego $h=g$.\\
Ahora, para ver que la representaci\'on matricial de $g$ respecto a $\mathcal{B}$ es la traspuesta de la de $f$ basta observar que
\begin{eqnarray*}
\Big[g\Big]^{\mathcal{B}}_{\mathcal{B},(j,i)} & = & \Big[g(u_i)\Big]^{\mathcal{B}}_j\\
  & = & \langle g(u_i);u_j \rangle\\
  & = & \langle u_i;f(u_j) \rangle\\
  & = & \langle f(u_j);u_i\rangle\\
  & = & \Big[f(u_j)\Big]^{\mathcal{B}}_i\\
  & = & \Big[f\Big]^{\mathcal{B}}_{\mathcal{B},(i,j)}
\end{eqnarray*}
\qed

\begin{nota}
Si $V$ tiene dimensi\'on finita, a la adjunta de $f$ la denotaremos por $f^*$.
\end{nota}

\begin{obs}
En particular $(f^*)^*=f$. Note que la notaci\'on de la adjunta es la misma que la notaci\'on para transformaci\'on dual. Mientras que la adjunta sigue siendo un operador de $V$, el dual de un operador es un operador en $V^*$. Confundir las dos notaciones tiene su fundamento en la siguiente propiedad.
\end{obs}

\begin{pro}
El mapa
\begin{eqnarray*}
\imath: V & \longrightarrow & V^*\\
            v & \longmapsto & \imath_v=\langle \bullet; v\rangle: v'\mapsto\langle v';v\rangle
\end{eqnarray*}
es una transformaci\'on lineal inyectiva, la cual es un isomorfismo si $V$ tiene dimensi\'on finita.
\end{pro}

\dem Por la linearidad en el segundo factor del producto interno, $\imath$ es una transformaci\'on lineal. Suponga ahora que $\imath_v=0$, en particular $0=\imath_v(v)=\langle v,v \rangle=\|v\|^2$, luego $v=0$ y as\'i $\imath$ es inyectiva. Finalmente como $\dim(V)=\dim(V^*)$ cuando $V$ tiene dimensi\'on finita, entonces $\imath$ en este caso tambi\'en es sobreyectiva, y es un isomorfismo.

\begin{prop}
Suponga que $V$ tiene dimensi\'on finita y sea
\begin{eqnarray*}
\widehat{\bullet}: V & \longrightarrow & \left(V^*\right)^*\\
                            v &\longmapsto &\widehat{v}:\lambda\mapsto\lambda(v).
\end{eqnarray*}
el isomorfismo can\'onico. Entonces para todo $v\in V$
\[
\imath^*(\widehat{v})=\imath(v).
\]
\end{prop}

\dem Para todo $v'\in V$
\begin{eqnarray*}
\imath^*(\widehat{v})(v')  & = & \widehat{v}(\imath(v'))\\
  & = & \imath(v')(v)\\
  & = & \langle v;v'\rangle\\
  & = & \langle v';v\rangle\\
  & = & \imath(v)(v').
\end{eqnarray*}
\qed

\begin{obs}
Note que si $V$ tiene dimensi\'on finita, para todo $v'\in V$,
\begin{eqnarray*}
f^*(\imath_v)(v') & = & \imath_v\left(f(v')\right)\\
   & = & \langle f(v');v \rangle\\
   & = & \langle v';f^*(v)\rangle\\
   & = & \imath_{f^*(v)}(v')
\end{eqnarray*}
luego $f^*(\imath_v)=\imath_{f^*(v)}$, es decir
\[
f^*\circ \imath=\imath \circ f^*,
\]
lo cual justifica la confusi\'on entre las dos notaciones de dual y adjunto (en la \'ultima igualdad, $f^*$ a la izquierda es el dual de $f$, mientras que a la derecha es el adjunto); pues, a trav\'es del isomorfismo $\imath$, ambos conceptos coinciden.
\end{obs}

\begin{obs}
Suponga que $V$ tiene dimensi\'on finita y sean $\mathcal{B}=\{v_1,\ldots,v_m\}$ una base de $V$ y $\mathcal{B}^*=\{\lambda_1,\ldots,\lambda_m\}$ la base de $V^*$ dual de $\mathcal{B}$. Tomamos la imagen de $\mathcal{B}$ mediante el isomorfismo can\'onico $V\mapsto \left(V^*\right)^*$, la cual es la base $\widehat{\mathcal{B}}=\{\widehat{v_1},\ldots,\widehat{v_m}\}$ de $\left(V^*\right)^*$ dual de $\mathcal{B}^*$. La proposici\'on anterior implica que si tomamos las representaciones matriciales en $M_{m\times m}(\mathbb{R})$
\[
A=\Big[\imath\Big]^{\mathcal{B}^*}_{\mathcal{B}},\textrm{ y } B=\Big[\imath^*(\widehat{\bullet})\Big]^{\mathcal{B}^*}_{\mathcal{B}}=\Big[\imath^*\Big]^{\mathcal{B}^*}_{\widehat{\mathcal{B}}}\Big[\widehat{\bullet}\Big]^{\widehat{\mathcal{B}}}_\mathcal{B}=\Big[\imath^*\Big]^{\mathcal{B}^*}_{\widehat{\mathcal{B}}}, 
\]
entonces $B=A$, pero por otro lado $B=A^\intercal$, luego $A^\intercal=A$.
Es decir, la representaci\'on matricial de $\imath$ respecto a una base y su dual es sim\'etrica.
\end{obs}

\begin{obs}
Note que si $V$ tiene dimensi\'on finita, $f^*\circ f$ es auto-adjunta, de hecho para todo $v_1,v_2\in V$
\[
\langle f^*\circ f(v_1);v_2\rangle=\langle f(v_1);f(v_2)\rangle=\langle v_1;f^*\circ f(v_2)\rangle
\]
\end{obs}

\begin{prop}
Si $V$ tiene dimensi\'on finita, las siguientes dos propiedades son equivalentes:
\begin{enumerate}
\item $f$ es auto-adjunta; y,
\item la representaci\'on matricial de $f$ respecto a toda base ortonormal es sim\'etrica.
\end{enumerate}
\end{prop}

\dem Proposici\'on \ref{adjtras} implica que si $f$ es auto-adjunta, su representaci\'on matricial respecto a una base ortogonal es sim\'etrica. Para obtener el converso, tomamos una base ortonormal de $V$, $\mathcal{B}=\{u_1,\ldots,u_n\}$ y asumimos que $\Big[f\Big]^\mathcal{B}_\mathcal{B}$ es sim\'etrica, es decir para todo $i,j\in\{1,\ldots,n\}$
\[
\langle f(u_j);u_i \rangle= \Big[f\Big]^\mathcal{B}_{\mathcal{B},(i,j)}= \Big[f\Big]^\mathcal{B}_{\mathcal{B},(j,i)}= \langle f(u_i);u_j \rangle,
\]
luego
\[
\langle f(u_j);u_i \rangle=\langle f(u_i);u_j \rangle=\langle u_j;f(u_i) \rangle
\]
lo cual, por bilinearidad del producto interno, implica que $f$ es auto-adjunta.\qed

\begin{obs}
Nos disponemos ahora a estudiar la descomposici\'on de Jordan-Chevalley de los operadores auto-adjuntos. El ingrediente fundamental ser\'a establecer que las partes diagonalizable y nilpotente son en este caso tambi\'en auto-adjuntos. 
\end{obs}

\begin{lema} Suponga que $f$ es auto-adjunto, entonces
\begin{enumerate}
\item Para todo $P(t)\in\mathbb{R}[t]$, $P(f)$ es auto-adjunto;
\item si $f$ es nilpotente, $f=0$; y,
\item si $v_1,v_2\in V$ son vectores propios asociados a valores propios distintos, entonces $\langle v_1,v_2\rangle=0$.
\end{enumerate}
\end{lema}

\dem
\begin{enumerate}
\item Note primero que para todo $v_1,v_2\in V$, recursivamente establecemos que para $k\in\mathbb{Z}_{\ge 0}$,
\[
\langle f^i(v_1);v_2\rangle = \langle v_1;f^i(v_2)\rangle.
\]
Ahora, si $P(t)=\sum_{k=0}^na_kt^k$ entonces para todo $v_1,v_2\in V$
\begin{eqnarray*}
\langle P(f)(v_1);v_2 \rangle & = & \langle \sum_{k=0}^na_kf^k(v_1); v_2 \rangle\\
 & = & \sum_{k=0}^n a_k\langle f^k(v_1);v_2\rangle\\
 & = & \sum_{k=0}^n a_k\langle v_1;f^k(v_2)\rangle\\
 & = & \langle v_1;\sum_{k=0}^na_kf^k(v_2)\rangle\\
 & = & \langle v_1;P(f)(v_2)\rangle.
\end{eqnarray*}
\item Sea $r\in\mathbb{Z}_{>0}$ el grado de nilpotencia de $f$. Asuma por contradicci\'on que $r>1$, luego $r\ge 2$, y existe $v\in V$ tal que $f^{r-1}(v)\ne 0$; pero en tal caso
\[
\| f^{r-1}(v)\|^2=\langle f^{r-1}(v);f^{r-1}(v) \rangle=\langle f^r(v);f^{r-2}(v)\rangle=\langle 0;f^{r-2}(v)\rangle=0
\]
luego $f^{r-1}(v)=0$, lo cual contradice la elecci\'on de $v$. Luego $r=1$ y as\'i $f=0$.
\item Sean $\lambda_1,\lambda_2\in\mathbb{R}$, $\lambda_1-\lambda_2\ne 0$, tales que $f(v_1)=\lambda_1v_1$ y $f(v_2)=\lambda_2v_2$. As\'i
\[
\lambda_1\langle v_1;v_2\rangle=\langle\lambda_1v_1;v_2\rangle=\langle f(v_1);v_2\rangle=\langle v_1;f(v_2)\rangle=\langle v_1;\lambda_2v_2\rangle=\lambda_2\langle v_1;v_2\rangle,
\]
luego
\[
(\lambda_1-\lambda_2)\langle v_1;v_2\rangle=0,
\]
pero como $\lambda_1-\lambda_2\ne 0$, $\langle v_1;v_2\rangle=0$.\qed
\end{enumerate}

\begin{teo}[Teorema Espectral]\label{teoesp} Suponga que $V$ tiene dimensi\'on finita, $f$ es auto-adjunta y que
\[
P_f(t)=(t-\lambda_1)^{m_1}(t-\lambda_2)^{m_2}\ldots(t-\lambda_r)^{m_r}, \quad \lambda_1,\lambda_2,\ldots,\lambda_r\in \mathbb{R}.
\]
entonces existe una base ortonormal $\mathcal{B}=\{u_1,\ldots,u_n\}$ de $V$ tal que $\Big[f\Big]^\mathcal{B}_\mathcal{B}$ es diagonal.
\end{teo}

\dem Por Teorema \ref{descjorche} existen $P_D(t),P_N(t)\in\mathbb{R}[t]$, tales que si $f_D=P_D(f)$ y $f_N=P_N(f)$ entonces $f=f_D+f_N$ es la descomposici\'on de Jordan-Chevalley, es decir $f_D$ es diagonalizable y $f_N$ nilpontente y estas conmutan. Ahora, por el lema, $f_N$ es auto-adjunta y as\'i, como es nilpotente, $f_N=0$. Luego $f=f_D$ es diagonalizable. Para $i=1,\ldots,r$, denote $V_i$ el espacio generado por los vectores propios de $f$ asociados a $\lambda_i$, es decir
\[
V_i=\{v\in V|\ f(v)=\lambda_iv\},
\]
de forma que, como $f$ es diagonalizable,
\[
V=V_1\oplus\ldots\oplus V_r.
\]
Por el lema tambi\'en sabemos que si $v_i\in V_i$ y $v_j\in V_j$, $i\ne j$, tenemos $\langle v_i;v_j \rangle=0$, luego si $\mathcal{B}_1,\ldots,\mathcal{B}_r$ son respectivamente bases ortonormales de $V_1,\ldots,V_r$,
\[
\mathcal{B}=\mathcal{B}_1\cup\ldots\cup \mathcal{B}_r
\]
es una base ortonormal de $V$ formada por vectores propios de $f$, en particular, $\Big[f\Big]^\mathcal{B}_\mathcal{B}$ es diagonal.\qed

\begin{obs}
M\'as adelante, cuando estudiemos la versi\'on compleja del teorema espectral, veremos que el polinomio caracter\'istico de un operador auto-adjunto sobre un espacio eucl\'ideo de dimensi\'on finita siempre se puede factorizar en factores lineales en $\mathbb{R}[t]$, lo cual a su vez implicar\'a que estos operadores son siempre diagonalizables mediante una base ortonormal. Estudiemos ahora con m\'as detalle este tipo bases.
\end{obs}

\section{Operadores ortogonales}

Sea $V$ un espacio eucl\'ideo y $f\in\Hom_\mathbb{R}(V,V)$ un operador.

\begin{defn}
Decimos que $f$ es un \emph{operador ortogonal} si para todo $v_1,v_2\in V$
\[
\langle f(v_1);f(v_2) \rangle =\langle v_1;v_2\rangle.
\]
\end{defn}

\begin{obs}
Tenemos
\begin{eqnarray*}
\|v_1+v_2\|^2 & = & \langle v_1+v_2;v_1+v_2 \rangle\\
 & = & \langle v_1;v_1\rangle+\langle v_2;v_1\rangle+\langle v_1;v_2\rangle+\langle v_2;v_2\rangle\\
 & = & \|v_1\|^2+2\langle v_1;v_2\rangle+\|v_2\|^2,
\end{eqnarray*}
de forma que el producto interno se puede expresar en t\'erminos de la norma:
\[
\langle v_1;v_2\rangle=\frac{\|v_1+v_2\|^2-\left(\|v_1\|^2+\|v_2\|^2\right)}{2}.
\]
De esto podemos concluir que $f$ es ortogonal si y solo $f$ preserva la norma, es decir $\|f(v)\|=\|v\|$ para todo $v\in V$.
\end{obs}

\begin{prop}\label{equivorto}
Si $V$ tiene dimensi\'on finita, las siguiente propiedades son equivalentes
\begin{enumerate}
\item $f$ es ortogonal;
\item $f$ preserva la norma;
\item $f^*\circ f=\id_V$; y,
\item la imagen por $f$ de una base ortonormal es una base ortonormal.
\end{enumerate}
\end{prop}

\dem La observaci\'on muestra la equivalencia entre las dos primeras propiedades. Veamos ahora la equivalencia entre la primera y la tercera. Suponga primero que $f$ es ortogonal y sea $\mathcal{B}=\{u_1,\ldots,u_n\}$ una base ortonormal de $V$. Para todo $v\in V$
\begin{eqnarray*}
f^*\circ f(v) & = & \sum_{i=1}^n\langle f^*\circ f(v);u_i\rangle u_i\\
 & = & \sum_{i=1}^n\langle f(v);f(u_i)\rangle u_i\\
 & = & \sum_{i=1}^n\langle v;u_i\rangle u_i\\
 & = & v.
\end{eqnarray*}
Suponga ahora que $f^*\circ f=\id_V$, entonces para todo $v_1,v_2\in V$
\[
\langle f(v_1);f(v_2)\rangle=\langle f^*\circ f(v_1);v_2\rangle=\langle v_1;v_2\rangle.
\]
Finalmente establecemos la equivalencia entre la primera propiedad y la cuarta. Si $f$ es ortogonal y $\mathcal{B}=\{u_1,\ldots u_n\}$ entonces
\[
\langle f(u_i);f(u_j)\rangle=\langle u_i;u_j\rangle=\delta_{ij}
\]
para todo $i,j\in\{1,\ldots,n\}$, as\'i $f(\mathcal{B})$ es una base ortonormal (ver Observaci\'on \ref{ortokro}). Rec\'iprocamente, asuma que $f$ env\'ia una base ortonormal $\mathcal{B}=\{u_1,\ldots,u_n\}$, a la base ortonormal $f(\mathcal{B})=\{f(u_1),\ldots,f(u_n)\}$. Luego, para todo $v_1,v_2\in V$, si $x_1,\ldots,x_n,y_1,\ldots,y_n\in\mathbb{R}$ son tales que
\[
v_1=\sum_{i=1}^nx_iu_i\qquad v_2=\sum_{i=1}^ny_iu_i,
\]
entonces
\[
f(v_1)=\sum_{i=1}^nx_if(u_i)\qquad v_2=\sum_{i=1}^ny_if(u_i)
\]
y, por Propiedad \ref{coorortonor},
\[
\langle f(v_1);f(v_2)\rangle=\sum_{i=1}^nx_iy_i=\langle v_1;v_2\rangle.
\]
\qed

\begin{obs}
Note que impl\'icitamente la tercera propiedad est\'a diciendo que los operadores ortogonales sobre espacios eucl\'ideos de dimensi\'on finita son invertibles, pues su inversa es su adjunta; y la cuarta que esta inversa es tambi\'en ortogonal, pues tambi\'en env\'ia una base ortogonal en una base ortogonal. 
\end{obs}

\begin{defn}
Decimos que $A\in M_{n\times n}(\mathbb{R})$ es una \emph{matriz ortogonal} si
\[
A^\intercal A=I_n
\]
donde $I_n$ denota la matriz con unos en la diagonal y ceros en el resto de entradas.
\end{defn}

\begin{prop}
Si $V$ tiene dimensi\'on finita y $\mathcal{B}=\{u_1,\ldots,u_n\}$ es una base ortonormal de $V$, $f$ es ortogonal si y solo si $\Big[f\Big]^\mathcal{B}_\mathcal{B}$ es ortogonal. 
\end{prop}

\dem La proposici\'on se sigue del hecho que si $A=\Big[f\Big]^\mathcal{B}_\mathcal{B}$, entonces $A^\intercal=\Big[f^*\Big]^\mathcal{B}_\mathcal{B}$ y
\[
A^\intercal A=\Big[f^*\Big]^\mathcal{B}_\mathcal{B} \Big[f\Big]^\mathcal{B}_\mathcal{B}=\Big[f^*\circ f\Big]^\mathcal{B}_\mathcal{B}.
\]
Entonces $A^\intercal A=I_n$ si y solo si $f^*\circ f=\id_V$.\qed

\begin{teo}\label{ortotorsor}
Si $V$ tiene dimensi\'on finita igual a $n$, la colecci\'on de operadores ortogonales de $V$ est\'a en correspondencia biyectiva con la colecci\'on de $n$-tuplas $(v_1,\ldots,v_n)\in V\times\ldots\times V$ tales que $\{v_1,\ldots,v_n\}$ es una base ortonormal de $V$.
\end{teo}

\dem Sea $\mathcal{B}=\{u_1,\ldots u_n\}$ una base ortonormal de $V$. Dado un operador ortogonal $g\in\Hom_\mathbb{R}(V,V)$, le asociamos la $n$-tupla
\[
\left(g(u_1),\ldots,g(u_n)\right).
\]
Dada la $n$-tupla $(v_1,\ldots,v_n)\in V\times\ldots\times V$ tal que $\{v_1,\ldots,v_n\}$ es una base ortonormal de $V$, le asociamos el operador definido sobre la base $\mathcal{B}$ por
\[
u_1\mapsto v_1,\ldots,u_n\mapsto v_n.
\]
Las equivalencias en Proposici\'on \ref{equivorto} implican que $\left(g(u_1),\ldots,g(u_n)\right)$ es una $n$-tupla cuyas componentes forman una base ortonormal de $V$ y que el operador tal que $u_1\mapsto v_1, \ldots, u_n\mapsto v_n$ es ortogonal. Las dos asociaciones son una la inversa de la otra.\qed

\begin{obs}
Note que la correspondencia descrita en la prueba del teorema depende de la escogencia de la base $\mathcal{B}=\{u_1,\ldots u_n\}$ y del ordenamiento de los elementos que la conforman. Vale la pena subrayar el hecho que los operadores ortogonales se pueden componer entre si y obtener un tercer operador ortogonal, y tambi\'en invertir y obtener otro operador ortogonal. Conjuntos con estas propiedades se les llama grupos. Una vez se fija una base, junto con un ordenamiento de sus elementos, la correspondencia del teorema respeta la estructura de grupo.\\ Formalmente, si $G$ denota la colleci\'on de operadores ortogonales de $V$, $X$ la de $n$-tuplas cuyas componentes form\'an bases ortogonales y
\begin{eqnarray*}
\Phi_T: G & \longrightarrow & X\\
 g & \longmapsto & \left(g(u_1),\ldots,g(u_n)\right)
\end{eqnarray*}
es la correspondencia biyectiva definida por $T$ (junto con un ordenamiento), entonces
\begin{eqnarray*}
\Phi_T(\id_V) & = & (u_1,\ldots u_n)\\
\Phi_T(g\circ h) & = & g\left(\Phi_T(h)\right)
\end{eqnarray*}
para todo $g,h\in G$, donde definimos $g(v_1,\ldots,v_n)=\left(g(v_1),\ldots,g(v_n)\right)$ para todo $(v_1,\ldots,v_n)\in X$. M\'as general
\begin{eqnarray*}
\Phi: X\times G &\longmapsto & X\times X \\
\left(x,g\right) & \longmapsto & \left(x,gx\right)
\end{eqnarray*}
es una biyecci\'on, en la cual, si fijamos el primer factor en $x=(u_1,\ldots u_n)$, obtenemos en el segundo la biyeccion $\phi_T$.
\end{obs}
\chapter{Espacios unitarios}

Sea $V$ un espacio vectorial sobre $\mathbb{C}$.

\begin{nota}
Dado un escalar $c\in\mathbb{C}$, denotamos por $\overline{c}$ a su conjugado el cual est\'a definido por
\[
\overline{c}=a-bi,\textrm{ si } c=a+bi,\ a,b\in\mathbb{R},
\]
por $|c|=\sqrt{a^2+b^2}=\sqrt{c\overline{c}}$ a su norma, por $\rea(c)=a=(c+\overline{c})/2$ su parte real y por $\ima(c)=b=(c-\overline{c})/2i$ su parte imaginaria.
\end{nota}

\begin{obs}
Bastantes elementos elaborados en este cap\'itulo se establecen por argumentos similares a los expuestos durante el desarrollo de la teor\'ia de espacios eucl\'ideos. En tales casos, dejaremos la verificaci\'on de los detalles al lector. 
\end{obs}

\section{Producto herm\'itico}

\begin{defn}
Un \emph{producto herm\'itico} en $V$ es una funci\'on
\begin{eqnarray*}
\langle\bullet,\bullet\rangle: V\times V & \longrightarrow & \mathbb{C}\\
(v_1,v_2) & \longmapsto & \langle v_1;v_2\rangle
\end{eqnarray*}
tal que:
\begin{enumerate}
\item \emph{es sesquilineal}: para todo $v,v_1,v_2\in V$ y $c\in\mathbb{C}$
\begin{eqnarray*}
\langle v_1+v_2;v\rangle & = & \langle v_1;v\rangle+\langle v_2;v\rangle\\
\langle cv_1;v_2\rangle & = & c\langle v_1;v_2\rangle\\
\langle v;v_1+v_2\rangle & = & \langle v;v_1\rangle + \langle v;v_2\rangle
\end{eqnarray*}
\item \emph{es herm\'itica}: para todo $v_1,v_2\in V$
\[
\langle v_2;v_1\rangle=\overline{\langle v_1;v_2\rangle};
\]
\item \emph{es definitivamente positiva} para todo $v\in V$, $v\ne 0$,
\[
\langle v;v\rangle>0.
\]
\end{enumerate}
Un \emph{espacio unitario} es un espacio vectorial sobre $\mathbb{C}$ provisto de un producto herm\'itico. 
\end{defn}


\begin{obs}
Se sigue que $\langle v;v \rangle=0$ si y solo si $v=0$ y que para todo $v,v_1,v_2\in V$ y $c\in\mathbb{C}$
\[
\langle v_1;cv_2\rangle  =  \overline{c}\langle v_1;v_2\rangle
\]
\end{obs}

\begin{ejem}
\begin{enumerate}
\item Sobre $V=\mathbb{C}^n$,
\[
\langle (z_1,\ldots,z_n);(w_1,\ldots,w_n)\rangle =\sum_{i=1}^n z_i\overline{w_i}.
\]
\item Sobre $V=M_{n\times n}(\mathbb{C})$,
\[
\langle A; B\rangle>=\tr(B^* A).
\]
donde $B^*=\overline{B}^\intercal$ es la matriz cuyas entradas son las conjugadas de las entradas de las matriz traspuesta de $B$.
\item Sea $[a,b]\subseteq\mathbb{R}$ un intervalo cerrado. Sobre $V=C^0_{\mathbb{C}}[a,b]$, el conjunto de funciones continuas $[a,b]\rightarrow\mathbb{C}$,
\[
\langle f;g \rangle=\int_a^bf(x)\overline{g(x)}dx.
\]
\end{enumerate}
\end{ejem}

\begin{defn}
Dado un espacio herm\'itico $V$, definimos la norma de $v\in V$ por $\|v\|=\sqrt{\langle v;v\rangle}$.
\end{defn}

\begin{pro}
Sea $V$ un espacio unitario, entonces:
\begin{enumerate}
\item $\|cv\|=|c|\|v\|$, para todo $c\in\mathbb{C}$ y $v\in V$;
\item \emph{Desigualdad de Cauchy-Schwarz}: $|\langle v_1;v_2\rangle|\le\|v_1\|\|v_2\|$, para todo $v_1,v_2\in V$, m\'as a\'un se tiene $\langle v_1,v_2\rangle=\|v_1\|\|v_2\|$ \'unicamente cuando $\{v_1,v_2\}$ es linealmente dependiente; y,
\item \emph{Desigualdad triangular}: $\|v_1+v_2\|\le \|v_1\|+\|v_2\|$, para todo $v_1,v_2\in V$, m\'as a\'un se tiene $\|v_1+v_2\|\le \|v_1\|+\|v_2\|$ si y solo si $av_1=bv_2$ con $a,b\ge 0$.
\end{enumerate}
\end{pro}

\dem\begin{enumerate}
\item Dados $c\in\mathbb{C}$ y $v\in V$
\[
\|cv\|=\sqrt{\langle cv;cv\rangle}=\sqrt{c\overline{c}\langle v;v\rangle}=|c|\sqrt{\langle v;v\rangle}=|c|\|v\|.
\]
\item Si $v_1=0$ o $v_2=0$, se tiene $0=\langle v_1;v_2\rangle=\|v_1\|\|v_2\|$ y se sigue la desigualdad. En general, para todo $a,b\in\mathbb{C}$,
\begin{eqnarray*}
0 \le \|av_1-bv_2\|^2 & = & \langle av_1-bv_2 , av_1-bv_2\rangle\\
   & = & a\overline{a}\langle v_1; v_1\rangle-b\overline{a}\langle v_2; v_1\rangle-a\overline{b}\langle v_1; v_2\rangle+b\overline{b}\langle v_2; v_2\rangle\\
   & = & |a|^2\|v_1\|^2-\left(\overline{a}b\overline{\langle v_1; v_2\rangle}+a\overline{b}\langle v_1; v_2\rangle\right)+|b|^2\|v_2\|^2.
\end{eqnarray*}
En particular si $a=\|v_2\|^2$ y $b=\langle v_1; v_2\rangle$, obtenemos
\begin{eqnarray*}
0 & \le & \|v_2\|^4\|v_1\|^2-\|v_2\|^2|\langle v_1, v_2\rangle|^2,
\end{eqnarray*}
lo cual, si suponemos que $v_2\ne 0$ (e.d. $|v_2|^2\ne 0$), implica la desigualdad deseada. Ahora bien, remontando las igualdades, $|\langle v_1,v_2\rangle|=\|v_1\|\|v_2\|$ equivale a $\|av_1-bv_2\|^2=0$, donde  $a=\|v_2\|^2$ y $b=\langle v_1; v_2\rangle$.
\item Tomando $a=1$ y $b=-1$, obtenemos
\begin{eqnarray*}
\|v_1+v_2\|^2 & = & \|v_1\|^2+\left(\overline{\langle v_1; v_2\rangle}+\langle v_1; v_2\rangle\right)+\|v_2\|^2\\
 & = & \|v_1\|^2+2\rea\left(\langle v_1; v_2\rangle\right)+\|v_2\|^2\\
 & \le & \|v_1\|^2+2|\langle v_1; v_2\rangle|+\|v_2\|^2
\end{eqnarray*}
La desigualdad de Cauchy-Schwarz implica
\[
\|v_1+v_2\|^2\le\|v_1\|^2+2\|v_1\|\|v_2\|+\|v_2\|^2=(\|v_1\|+\|v_2\|)^2
\]
que es equivalente a la desigualdad afirmada. La igualdad se obtiene si y solo si  $\rea\left(\langle v_1, v_2\rangle\right)=\langle v_1, v_2\rangle=\|v_1\|\|v_2\|$, lo cual equivale a $av_1-bv_2=0$ donde $a=\|v_2\|$ y $b=\|v_1\|$ pues
\[
\|av_1-bv_2\|^2=|a|^2\|v_1\|^2-2\rea\left(a\overline{b}\langle v_1; v_2\rangle\right)+|b|^2\|v_2\|^2.
\]
\qed
\end{enumerate}

\begin{defn}
Sea $V$ un espacio unitario y $S\subseteq V$. Decimos que $S$ es \emph{ortogonal} si $\langle v_1,v_2\rangle=0$ para todo $v_1,v_2\in T$ tales que $v_1\ne v_2$. Si adem\'as $\|v\|=1$ para todo $v\in S$, decimos que $S$ es \emph{ortonormal}.
\end{defn}

\begin{obs}\label{ortokroher}
Note que $S=\{v_i\}_{i\in I}$ es ortonormal si y solo si
\[
\langle v_i;v_j\rangle=\delta_{ij}.
\]
para todo $i,j\in I$.
\end{obs}

\begin{pro}
Sea $V$ un espacio unitario. Si $S\subset V$ es ortogonal, y $0\not\in S$, entonces $S$ es linealmente independiente.
\end{pro}

\dem El argumento es similar al de la demostraci\'on de Propiedad \ref{ortlinind}.\qed

\begin{teo}[Ortogonalizaci\'on de Gram-Schmidt]
Sea $V$ un espacio unitario. Suponga que $V$ tiene dimensi\'on finita, entonces $V$ tiene una base ortonormal. M\'as a\'un, si $\{v_1,\ldots,v_n\}$ es una base de $V$, existe una base ortonormal $\{u_1,\ldots,u_n\}$ de $V$ tal que para $k=1,\ldots,n$
\[
\langle v_1,\ldots,v_k\rangle=\langle u_1,\ldots,u_k\rangle.
\]
\end{teo}

\dem El argumento es similar al de la demostraci\'on de Teorema \ref{gramsch}.\qed

\begin{pro}\label{coorortonorher}
Sea $V$ un espacio unitario. Suponga que $V$ tiene dimensi\'on finita y $\{u_1,\ldots,u_n\}$ es una base ortonormal de $V$, entonces para todo $v\in V$
\[
v=\sum_{i=1}^n\langle v;u_i\rangle u_i.
\]
En particular, si $v_1,v_2\in V$ son tales que
\[
v_1=\sum_{i=1}^nz_iu_i\qquad v_2=\sum_{i=1}^nw_iu_i,
\]
entonces
\[
\langle v_1;v_2\rangle=\sum_{i=1}^nz_i\overline{w_i}.
\]
\end{pro}

\dem Como $\{u_1,\ldots,u_n\}$ es base, existen $a_1,\ldots,a_n\in K$ tal que $v=\sum_{i=1}^na_iu_i$. De esta forma, para $j=1,\ldots,n$, 
\[
\langle v;u_j\rangle=\sum_{i=1}^na_i\langle u_i,u_j\rangle=\sum_{i=1}^n a_i\delta_{ij}=a_j
\]
(ver Observaci\'on \ref{ortokroher}). Finalmente,
\[
\langle v_1;v_2\rangle=\langle\sum_{i=1}^nz_iu_i,\sum_{j=1}^nw_ju_j\rangle=\sum_{i=1}^n\sum_{j=1}^nz_i\overline{w_j}\delta_{ij}=\sum_{i=1}^nz_i\overline{w_i}.
\]
\qed

\begin{defn}
Sean $V$ un espacio unitario y $S\subseteq V$, el \emph{conjunto ortogonal} a $S$ est\'a definido por
\[
S^\perp=\{v\in V|\ \langle v,u\rangle=0, \textrm{ para todo } u\in S\}
\]
\end{defn}

\begin{pro}
Sean $V$ un espacio unitario y $S\subseteq V$, entonces $S^\perp\le V$.
\end{pro}

\dem El argumento es similar al de la demostraci\'on de Propiedad \ref{ortsubesp}.\qed

\begin{teo}
Sea $V$ un espacio unitario. Suponga que $V$ tiene dimensi\'on finita y sea $U\le V$, entonces
\[
V=U\oplus U^\perp
\]
\end{teo}

\dem El argumento es similar al de la demostraci\'on de Teorema \ref{complort}.\qed

\begin{defn}
Sea $V$ un espacio unitario. Suponga que $V$ tiene dimensi\'on finita y sea $U\le V$. Llamamos a $U^\perp$ el \emph{complemento ortogonal de $U$}. A la proyecci\'on
\[
p^\perp_U:V\longrightarrow V
\]
sobre $U$, definida por la descomposici\'on $V=U\oplus U^\perp$ la llamamos \emph{proyecci\'on ortogonal sobre $U$}.
\end{defn}

\begin{pro}
Sea $V$ un espacio unitario. Suponga que $V$ tiene dimensi\'on finita y sea $U\le V$. Tenemos para todo $v_1,v_2\in V$
\[
\langle p^\perp_U(v_1);v_2\rangle=\langle v_1;p^\perp_U(v_2)\rangle
\]
Si $\{u_1,\ldots,u_m\}$ es una base ortonormal de $U$ entonces para todo $v\in V$
\[
p^\perp_U(v)=\sum_{i=1}^m\langle v;u_i\rangle u_i.
\]
\end{pro}

\dem El argumento es similar al de la demostracci\'on de Propiedad \ref{proyautoadj}.\qed

\section{Operador adjunto}

Sea $V$ un espacio unitario y $f\in\Hom_\mathbb{C}(V,V)$ un operador.

\begin{defn}
Sea $g\in\Hom_\mathbb{C}(V,V)$, decimos que $g$ es un \emph{operador adjunto de $f$} si para todo $v_1,v_2\in V$
\[
\langle g(v_1);v_2 \rangle=\langle v_1;f(v_2)\rangle.
\]
Decimos que $f$ es \emph{auto-adjunto} si $f$ es un operador adjunto de $f$. 
\end{defn}

\begin{obs}
Note que si $g$ es adjunto de $f$, entonces $f$ es adjunto de $g$. De hecho
\[
\langle f(v_1);v_2\rangle= \overline{\langle v_2;f(v_1)\rangle}= \overline{\langle g(v_2);v_1\rangle}= \langle v_1;g(v_2)\rangle
\]
\end{obs}

\begin{defn}
Sean $I,J$ conjuntos y $A\in M_{I\times J}(\mathbb{C})$, definimos la \emph{matriz adjunta} de $A$ por $A^*\in M_{J\times I}(\mathbb{C})$ tal que
\[
A^*(j,i)=\overline{A(i,j)}
\]
para todo $(j,i)\in J\times I$. Es decir el valor en $(j,i)$ de $A^*$ es el conjugado del valor en $(i,j)$ de $A$. Similarmente si $m,n\in\mathbb{Z}_{>0}$, y $A\in M_{m\times n}(\mathbb{C})$, definimos su adjunta por $A^*\in M_{n\times m}(\mathbb{C})$ tal que
\[
A^*(j,i)=\overline{A(i,j)}
\]
Sea $A\in M_{I\times I}(K)$, o $A\in M_{n\times n}(K)$, decimos que $A$ es \emph{herm\'itica} si $A^*=A$.
\end{defn}

\begin{prop}\label{adjtrasher}
Suponga que $V$ tiene dimensi\'on finita, entonces existe un \'unico operador $g\in\Hom_{\mathbb{C}}(V,V)$ adjunto de $f$. M\'as a\'un, si $\mathcal{B}=\{u_1,\ldots,u_n\}$ es una base ortonormal de $V$, entonces
\[
\Big[g\Big]^\mathcal{B}_\mathcal{B}=\left(\Big[f\Big]^\mathcal{B}_\mathcal{B}\right)^*
\]
\end{prop}

\dem Defina el operador $g\in\Hom_\mathbb{C}(V,V)$ por la imagen de la base $\mathcal{B}$:
\[
g(u_j)=\sum_{i=1}^n\langle u_j;f(u_i)\rangle u_i.
\]
De esta forma
\[
\langle g(u_j);u_i\rangle=\langle u_j;f(u_i)\rangle
\]
y por Propiedad \ref{coorortonorher}
\begin{eqnarray*}
\langle g(v_1); v_2\rangle & = & \sum_{i=1}^n\langle g(v_1);u_i\rangle\overline{\langle v_2;u_i\rangle}\\
 & = & \sum_{i,j=1}^n\langle v_1;u_j\rangle\langle g(u_j);u_i\rangle \overline{\langle v_2;u_i\rangle}\\
 & = & \sum_{i,j=1}^n\langle v_1;u_j\rangle\langle u_j;f(u_i)\rangle \overline{\langle v_2;u_i\rangle}\\
 & = & \sum_{j=1}^n\langle v_1;u_j\rangle\langle u_j;f(v_2)\rangle\\
 & = & \sum_{j=1}^n\langle v_1;u_j\rangle\overline{\langle f(v_2);u_j\rangle}\\
 & = & \langle v_1; f(v_2)\rangle
\end{eqnarray*}
Por otro lado si, $h\in\Hom_\mathbb{C}(V,V)$ es adjunto de $f$, por Propiedad \ref{coorortonorher},
\begin{eqnarray*}
h(u_j) & = & \sum_{i=1}^n \langle h(u_j);u_i\rangle u_i\\
         & = & \sum_{i=1}^n \langle u_j;f(u_i)\rangle u_i\\
         & = & g(u_j),
\end{eqnarray*}
luego $h=g$.\\
Ahora, para ver que la representaci\'on matricial de $g$ respecto a $\mathcal{B}$ es la adjunta de la de $f$ basta observar que
\begin{eqnarray*}
\Big[g\Big]^\mathcal{B}_{\mathcal{B},(j,i)} & = & \Big[g(u_i)\Big]^\mathcal{B}_j\\
  & = & \langle g(u_i);u_j \rangle\\
  & = & \langle u_i;f(u_j) \rangle\\
  & = & \overline{\langle f(u_j);u_i\rangle}\\
  & = & \overline{\Big[f(u_j)\Big]^\mathcal{B}_i}\\
  & = & \overline{\Big[f\Big]^T_{\mathcal{B},(i,j)}}
\end{eqnarray*}
\qed

\begin{nota}
Si $V$ tiene dimensi\'on finita, a la adjunta de $f$ la denotaremos por $f^*$.
\end{nota}

\begin{pro}
El mapa
\begin{eqnarray*}
h: V & \longrightarrow & V^*\\
            v & \longmapsto & h_v=\langle \bullet; v\rangle: v'\mapsto\langle v';v\rangle
\end{eqnarray*}
es \emph{semilineal}, es decir para todo $v_1,v_2,v\in V$ y $c\in\mathbb{C}$
\begin{eqnarray*}
h(v_1+v_2) & = & h(v_1)+h(v_2)\\
h(cv) & = & \overline{c}h(v),
\end{eqnarray*}
e inyectivo. Si adem\'as $V$ tiene dimensi\'on finita, $h$ es biyectivo.
\end{pro}

\dem Para todo $v'\in V$, dados $v_1,v_2,v\in V$ y $c\in\mathbb{C}$
\begin{eqnarray*}
h(v_1+v_2)(v') & = & \langle v';v_1+v_2\rangle\\
  & = & \langle v';v_1\rangle + \langle v';v_2\rangle\\
  & = & h(v_1)(v')+h(v_2)(v')\\
  & = & \left(h(v_1)+h(v_2)\right)(v), \textrm{ y}\\
h(cv)(v') & = & \langle v';cv \rangle\\
  & = & \overline{c}\langle v';v\rangle\\
  & = & \overline{c}h(v)(v')
\end{eqnarray*}
Suponga ahora que $v_1,v_2\in V$ son tales $h(v_1)=h(v_2)$, es decir $h(v_1-v_2)=0$, en particular $0=h(v_1-v_2)(v_1-v_2)=\langle v_1-v_2;v_1-v_2 \rangle=\|v_1-v_2\|^2$, luego $v_1-v_2=0$ y as\'i$v_1=v_2$, es decir $\imath$ es inyectiva. Finalmente suponga que $V$ tiene dimensi\'on finita y sea $\mathcal{B}=\{u_1,\ldots,u_n\}$ una base de $V$, entonces $h(T)=\{h(u_1),\ldots,h(u_n)\}$ es la base de $V^*$ dual de $V$, pues
\[
h(u_j)(u_i)=\langle u_i;u_j\rangle=\delta_{ij}.
\] 
En particular dado $\lambda\in V^*$, si $a_1,\ldots,a_n\in\mathbb{C}$ son tales que $\lambda=\sum_{i=1}^na_ih(u_i)$ entonces $h(\sum_{i=1}^n\overline{a_i}u_i)=\lambda$, luego $h$ es tambi\'en sobreyectiva y as\'i biyectiva.\qed

\begin{obs}
Note que si $V$ tiene dimensi\'on finita, para todo $v'\in V$,
\begin{eqnarray*}
f^*(h_v)(v') & = & h_v\left(f(v')\right)\\
   & = & \langle f(v');v \rangle\\
   & = & \langle v';f^*(v)\rangle\\
   & = & h_{f^*(v)}(v')
\end{eqnarray*}
luego $f^*(h_v)=h_{f^*(v)}$, es decir
\[
f^*\circ h=h \circ f^*,
\]
donde, en esta igualdad, $f^*$ a la izquierda es el dual de $f$, mientras que a la derecha es el adjunto. A trav\'es de la biyecci\'on semilineal $h$, ambos conceptos coinciden.
\end{obs}

\begin{obs}
Note que si $V$ tiene dimensi\'on finita, $f^*\circ f$ es auto-adjunta, de hecho para todo $v_1,v_2\in V$
\[
\langle f^*\circ f(v_1);v_2\rangle=\langle f(v_1);f(v_2)\rangle=\langle v_1;f^*\circ f(v_2)\rangle
\]
\end{obs}

\begin{prop}
Si $V$ tiene dimensi\'on finita, las siguientes dos propiedades son equivalentes:
\begin{enumerate}
\item $f$ es auto-adjunta; y,
\item la representaci\'on matricial de $f$ respecto a una base ortonormal es herm\'itica.
\end{enumerate}
\end{prop}

\dem Proposici\'on \ref{adjtrasher} implica que si $f$ es auto-adjunta, su representaci\'on matricial respecto a una base ortogonal es herm\'itica. Para obtener el converso, tomamos una base ortonormal de $V$, $\mathcal{B}=\{u_1,\ldots,u_n\}$ y asumimos que $\Big[f\Big]^\mathcal{B}_\mathcal{B}$ es herm\'itica, es decir para todo $i,j\in\{1,\ldots,n\}$
\[
\langle f(u_j);u_i \rangle= \Big[f\Big]^\mathcal{B}_{\mathcal{B},(i,j)}= \overline{\Big[f\Big]^\mathcal{B}_{\mathcal{B},(j,i)}}= \overline{\langle f(u_i);u_j \rangle}=\langle u_j;f(u_i) \rangle,
\]
luego si $v_1=\sum_{i=1}^{n}z_iu_i$ y $v_2=\sum_{j=1}^{n}w_ju_j$
\begin{eqnarray*}
\langle f(v_1);v_2 \rangle & = & \sum_{i,j=1}^nz_i\overline{w_j}\langle f(u_i);u_j\rangle\\
  & = & \sum_{i,j=1}^nz_i\overline{w_j}\langle u_i;f(u_j)\rangle\\
  & = & \langle v_1;f(v_2)\rangle
\end{eqnarray*}
\qed

\begin{obs}
La descomposici\'on de Jordan-Chevalley de los operadores auto-adjuntos sobre un espacio unitario empata con la descomposici\'on sobre espacios euclideos pues tienen la particularidad de tener todos sus valores propios reales. Esto nos permitir\'a relajar las hip\'otesis del teorema espectral ya demostrado. 
\end{obs}

\begin{lema} Suponga que $f$ es auto-adjunto, entonces
\begin{enumerate}
\item Para todo $P(t)\in\mathbb{R}[t]$, $P(f)$ es auto-adjunto;
\item si $f$ es nilpotente, $f=0$;
\item los valores propios de $f$ son reales; y,
\item si $v_1,v_2\in V$ son vectores propios asociados a valores propios distintos, entonces $\langle v_1;v_2\rangle=0$.
\end{enumerate}
\end{lema}

\dem
\begin{enumerate}
\item Note primero que para todo $v_1,v_2\in V$, recursivamente establecemos que para $k\in\mathbb{Z}_{\ge 0}$,
\[
\langle f^i(v_1);v_2\rangle = \langle v_1;f^i(v_2)\rangle.
\]
Ahora, si $P(t)=\sum_{k=0}^na_kx^k$, para $k=1,\ldots,n$ $a_k=\overline{a_k}$, y entonces para todo $v_1,v_2\in V$
\begin{eqnarray*}
\langle P(f)(v_1);v_2 \rangle & = & \langle \sum_{k=0}^na_kf^k(v_1); v_2 \rangle\\
 & = & \sum_{k=0}^n a_k\langle f^k(v_1);v_2\rangle\\
 & = & \sum_{k=0}^n a_k\langle v_1;f^k(v_2)\rangle\\
 & = & \langle v_1;\sum_{k=0}^n\overline{a_k}f^k(v_2)\rangle\\
 & = & \langle v_1;P(f)(v_2)\rangle.
\end{eqnarray*}
\item Sea $r\in\mathbb{Z}_{>0}$ el grado de nilpotencia de $f$. Asuma por contradicci\'on que $r>1$, luego $r\ge 2$, y existe $v\in V$ tal que $f^{r-1}(v)\ne 0$; pero en tal caso
\[
\| f^{r-1}(v)\|^2=\langle f^{r-1}(v);f^{r-1}(v) \rangle=\langle f^r(v);f^{r-2}(v)\rangle=\langle 0;f^{r-2}(v)\rangle=0
\]
luego $f^{r-1}(v)=0$, lo cual contradice la elecci\'on de $v$. Luego $r=1$ y as\'i $f=0$.
\item Sean $\lambda\in\mathbb{C}$ y $v\in V$, $v\ne 0$, tales que $f(v)=\lambda v$. As\'i
\[
\lambda\langle v;v\rangle=\langle\lambda v;v\rangle=\langle f(v);v\rangle=\langle v;f(v)\rangle=\langle v;\lambda v\rangle=\overline{\lambda}\langle v;v\rangle,
\]
de donde $(\lambda-\overline{\lambda})\|v\|^2=0$, pero $v\ne 0$, entonces, $\lambda=\overline{\lambda}$, es decir $\lambda\in\mathbb{R}$.
\item Sean $\lambda_1,\lambda_2\in\mathbb{R}$, $\lambda_1-\lambda_2\ne 0$, tales que $f(v_1)=\lambda_1v_1$ y $f(v_2)=\lambda_2v_2$. As\'i
\[
\lambda_1\langle v_1;v_2\rangle=\langle\lambda_1v_1;v_2\rangle=\langle f(v_1);v_2\rangle=\langle v_1;f(v_2)\rangle=\langle v_1;\lambda_2v_2\rangle=\lambda_2\langle v_1;v_2\rangle,
\]
luego
\[
(\lambda_1-\lambda_2)\langle v_1;v_2\rangle=0,
\]
pero como $\lambda_1-\lambda_2=\ne 0$, $\langle v_1;v_2\rangle=0$.\qed
\end{enumerate}

\begin{teo}[Teorema Espectral] Suponga que $V$ tiene dimensi\'on finita y $f$ es auto-adjunta entonces existe una base ortonormal $\mathcal{B}=\{u_1,\ldots,u_n\}$ de $V$ tal que $\Big[f\Big]^\mathcal{B}_\mathcal{B}$ es diagonal.
\end{teo}

\dem Como $\mathbb{C}$ es algebraicamente cerrado,
\[
P_f(t)=(t-\lambda_1)^{m_1}(t-\lambda_2)^{m_2}\ldots(t-\lambda_r)^{m_r}, \quad \lambda_1,\lambda_2,\ldots,\lambda_r\in \mathbb{C}.
\]
Como $\lambda_1,\lambda_2,\ldots,\lambda_r$ son valores propios, por el lema son valores reales y $P_f(t)\in\mathbb{R}[t]$. En particular, como $t-\lambda_1,t-\lambda_2,\ldots,t-\lambda_r\in\mathbb{R}[t]$, por la demostraci\'on de Teorema \ref{descjorche} existen $P_D(t),P_N(t)\in\mathbb{R}[t]$, tales que si $f_D=P_D(f)$ y $f_N=P_N(f)$ entonces $f=f_D+f_N$ es la descomposici\'on de Jordan-Chevalley, es decir $f_D$ es diagonalizable y $f_N$ nilpontente y estas conmutan. Ahora, por el lema, $f_N$ es auto-adjunta y as\'i, como es nilpotente, $f_N=0$. Luego $f=f_D$ es diagonalizable. Para $i=1,\ldots,r$, denote $V_i$ el espacio generado por los vectores propios de $f$ asociados a $\lambda_i$, es decir
\[
V_i=\{v\in V|\ f(v)=\lambda_iv\},
\]
de forma que, como $f$ es diagonalizable,
\[
V=V_1\oplus\ldots\oplus V_r.
\]
Por el lema tambi\'en sabemos que si $v_i\in V_i$ y $v_j\in V_j$, $i\ne j$, tenemos $\langle v_i,v_j \rangle=0$, luego si $\mathcal{B}_1,\ldots,\mathcal{B}_r$ son respectivamente bases ortonormales de $V_1,\ldots,V_r$,
\[
\mathcal{B}=\mathcal{B}_1\cup\ldots\cup \mathcal{B}_r
\]
es una base ortonormal de $V$ formada por vectores propios de $f$, en particular, $\Big[f\Big]^\mathcal{B}_\mathcal{B}$ es diagonal.\qed

\begin{coro}
Sea $V$ un espacio eucl\'ideo de dimensi\'on finita y $f\in\Hom_{\mathbb{R}}(V,V)$ un operador auto-adjunto, entonces una base ortonormal $\mathcal{B}=\{u_1,\ldots,u_n\}$ de $V$ tal que $\Big[f\Big]^\mathcal{B}_\mathcal{B}$ es diagonal.
\end{coro}

\dem Sea $\mathcal{B}$ una base ortonormal de $V$ y $A\in M_{n\times n}(\mathbb{R})$ la representaci\'on de $f$ respecto a la base $\mathcal{B}$ donde $n=\dim(V)$. Entonces $A$ es una matriz sim\'etrica con entradas reales. Sea
\begin{eqnarray*}
f_A:\mathbb{C}^n & \longrightarrow & \mathbb{C}^n\\
(z_1,\ldots,z_n) & \longmapsto & \left(\sum_{k=1}^na_{1k}z_k,\ldots,\sum_{k=1}^na_{nk}z_k\right)
\end{eqnarray*}
donde la $kl$-\'esima entrada de es $A(k,l)=a_{kl}$. De forma que la representaci\'on matricial de $f_A$ respecto a la base can\'onica de $\mathbb{C}^n$ es $A$, la cual es herm\'itica pues es sim\'etrica con entradas reales, luego $f_A$ es auto-adjunta respecto al producto herm\'itico
\[
\langle (z_1,\ldots,z_n),(w_1,\ldots,w_n)\rangle =\sum_{k=1}^nz_k\overline{w_k}.
\]
As\'i
\[
P_f(t)=P_{f_A}(t)=(t-\lambda_1)^{m_1}(t-\lambda_2)^{m_2}\ldots(t-\lambda_r)^{m_r}, \quad \lambda_1,\lambda_2,\ldots,\lambda_r\in \mathbb{R}.
\]
y la conclusi\'on se sigue ahora de Teorema \ref{teoesp}.\qed

\section{Operadores unitarios}

Sea $V$ un espacio unitario y $f\in\Hom_\mathbb{C}(V,V)$ un operador.

\begin{defn}
Decimos que $f$ es un \emph{operador unitario} si para todo $v_1,v_2\in V$
\[
\langle f(v_1);f(v_2) \rangle =\langle v_1;v_2\rangle.
\]
\end{defn}

\begin{obs}
Tenemos
\begin{eqnarray*}
\|v_1+v_2\|^2 & = & \langle v_1+v_2;v_1+v_2 \rangle\\
 & = & \langle v_1;v_1\rangle+\langle v_2;v_1\rangle+\langle v_1;v_2\rangle+\langle v_2;v_2\rangle\\
 & = & \langle v_1;v_1\rangle+\overline{\langle v_1;v_2\rangle}+\langle v_1;v_2\rangle+\langle v_2;v_2\rangle\\
 & = & \|v_1\|^2+2\rea\left(\langle v_1;v_2\rangle\right)+\|v_2\|^2,\textrm{ y }\\
\|v_1+iv_2\|^2 & = & \langle v_1+iv_2;v_1+iv_2 \rangle\\
 & = & \langle v_1;v_1\rangle+i\langle v_2;v_1\rangle-i\langle v_1;v_2\rangle+\langle v_2;v_2\rangle\\
 & = & \langle v_1;v_1\rangle-i\left(\langle v_1;v_2\rangle-\overline{\langle v_1;v_2\rangle}\right)+\langle v_2;v_2\rangle\\
 & = & \|v_1\|^2+2\ima\left(\langle v_1;v_2\rangle\right)+\|v_2\|^2,
\end{eqnarray*}
de forma que el producto interno se puede expresar en t\'erminos de la norma:
\[
\langle v_1;v_2\rangle=\frac{\|v_1+v_2\|^2-\left(\|v_1\|^2+\|v_2\|^2\right)}{2} + \frac{\|v_1+iv_2\|^2-\left(\|v_1\|^2+\|v_2\|^2\right)}{2}i.
\]
De esto podemos concluir que $f$ es unitario si y solo $f$ preserva la norma, es decir $\|f(v)\|=\|v\|$ para todo $v\in V$.
\end{obs}

\begin{prop}
Si $V$ tiene dimensi\'on finita, las siguiente propiedades son equivalentes
\begin{enumerate}
\item $f$ es unitario;
\item $f$ preserva la norma;
\item $f^*\circ f=\id_V$; y,
\item la imagen por $f$ de una base ortonormal es una base ortonormal.
\end{enumerate}
\end{prop}

\dem El argumento es similar al de la demostraci\'on de Proposici\'on \ref{equivorto}
\qed

\begin{obs}
Como en el caso de los operadores ortogonales sobre espacios eucl\'ideos, los operadores unitarios son invertibles y env\'ian bases ortogonales en bases ortogonales. 
\end{obs}

\begin{defn}
Decimos que $A\in M_{n\times n}(\mathbb{C})$ es una \emph{matriz unitaria} si
\[
A^* A=I_n
\]
donde $I_n$ denota la matriz con unos en la diagonal y ceros en el resto de entradas.
\end{defn}

\begin{prop}
Si $V$ tiene dimensi\'on finita y $\mathcal{B}=\{u_1,\ldots,u_n\}$ es una base ortonormal de $V$, $f$ es unitario si y solo si $\Big[f\Big]^\mathcal{B}_\mathcal{B}$ es unitaria. 
\end{prop}

\dem La proposici\'on se sigue del hecho que si $A=\Big[f\Big]^\mathcal{B}_\mathcal{B}$, entonces $A^*=\Big[f^*\Big]^\mathcal{B}_\mathcal{B}$ y
\[
A^* A=\Big[f^*\Big]^\mathcal{B}_\mathcal{B} \Big[f\Big]^\mathcal{B}_\mathcal{B}=\Big[f^*\circ f\Big]^\mathcal{B}_\mathcal{B}.
\]
Entonces $A^* A=I_n$ si y solo si $f^*\circ f=\id_V$.\qed

\begin{teo}
Si $V$ tiene dimensi\'on finita igual a $n$, la colecci\'on de operadores unitarios de $V$ est\'a en correspondencia biyectiva con la colecci\'on de $n$-tuplas $(v_1,\ldots,v_n)\in V\times\ldots\times V$ tales que $\{v_1,\ldots,v_n\}$ es una base ortonormal de $V$.
\end{teo}

\dem El argumento es similar al de la demostraci\'on de Teorema \ref{ortotorsor}.\qed

\section{Estructura compleja}

Sea $V$ un espacio vectorial sobre $\mathbb{R}$.

\begin{pro}
Suponga que $f\in\Hom_{\mathbb{R}}(V,V)$ es un operador simple, entonces: o bien,
\begin{enumerate}
\item $f=c\id_V$, para alg\'un $c\in\mathbb{R}$ y en tal caso $\dim(V)=1$; o bien,
\item $f=a\id_V+bj$, donde $j\in\Hom_{\mathbb{R}}(V,V)$ es tal que $j^2=-\id_V$, $a,b\in\mathbb{R}$, y en tal caso $\dim(V)=R^2$.
\end{enumerate}
\end{pro}

\dem Como $f$ es simple, su polinomio minimal es irreducible, en particular este es de la forma $t-c$, con $c\in\mathbb{R}$, \'o $(t-a)^2+b^2$, con $a,b\in\mathbb{R}$. En el primer caso $f-c\id_V=0$, es decir $f=c\id_V$; en el segundo, $(f-a\id_V)^2+b^2\id_V=0$, y si $j=\frac{1}{b}(f-a\id_V)$, tenemos $f=a\id_V+bj$ con
\begin{eqnarray*}
j^2 & = & \left(\frac{1}{b}(f-a\id_V)\right)\circ\left(\frac{1}{b}(f-a\id_V)\right)\\
 & = & \frac{1}{b^2}\left(f-a\id_V\right)^2\\
 & = & \frac{1}{b^2}\left(-b^2\id_V\right)=-\id_V.
\end{eqnarray*}
Como $f$ es simple, en el primer caso, dado $v\in V$, $v\ne 0$, $\langle v\rangle$ es invariante bajo $f$ y as\'i $V=\langle v\rangle$; en el segundo, dado $v\in V$, $v\ne 0$, $\{v,j(v)\}$ es linealmente independiente pues si
\[
\alpha v+\beta j(v)=0,
\]
$\alpha,\beta\in\mathbb{R}$, entonces operando por $j$,
\[
-\beta v+\alpha j(v)=0;
\]
combinando las dos relaciones obtenemos
\[
0=\alpha\left(\alpha v+\beta j(v)\right)-\beta\left(-\beta v+\alpha j(v)\right)=(\alpha^2+\beta^2)v
\]
pero $v\ne 0$, luego $\alpha^2+\beta^2=0$, de donde $\alpha=\beta=0$; ahora $\langle v,j(v)\rangle$ es invariante bajo $f$, as\'i $V=\langle v,j(v)\rangle$.\qed

\begin{obs}
Considere $\mathbb{C}$ como espacio vectorial sobre $\mathbb{R}$, con la base $\mathcal{B}=\{1,i\}$ (en particular $\mathbb{C}\simeq_\mathbb{R}\mathbb{R}^2$). Dado $a+bi\in\mathbb{C}$, con $a,b\in\mathbb{R}$, la funci\'on:
\begin{eqnarray*}
m_{a+bi}:\mathbb{C} & \longrightarrow & \mathbb{C}\\
 z & \longrightarrow & (a+bi)z
\end{eqnarray*}
es un operador en $\Hom_{\mathbb{R}}(\mathbb{C},\mathbb{C})$, de hecho para todo $z,z_1,z_2\in\mathbb{C}$ y $c\in\mathbb{R}$
\[
(a+bi)\left(z_1+z_2\right)=(a+bi)z_1+(a+bi)z_2, (a+bi)cz=c(a+bi)z.
\]
Tenemos entonces que si $f=m_{a+ib}$, entonces
\[
\Big[f\Big]^\mathcal{B}_\mathcal{B}=\left[\begin{array}{rr} a & -b\\b & a\end{array}\right]
\]
y $f=a\id_\mathbb{C}+bj$ donde $j=m_i$. Note que
\[
\Big[j\Big]^\mathcal{B}_\mathcal{B}=\left[\begin{array}{rr} 0 & -1\\1 & 0\end{array}\right]
\]
\end{obs}

\begin{obs}\label{estcomest}
La observaci\'on anterior se puede generalizar a $\mathbb{C}^n$ el cual visto como espacio vectorial sobre $\mathbb{R}$ tiene dimensi\'on $2n$. Tome la base
\[
\mathcal{B}=\{e_1,\ldots,e_n,f_1,\ldots,f_n\}
\]
donde $\{e_1,\ldots,e_n\}$ es la base can\'onica de $\mathbb{C}^n$, visto
como espacio vectorial sobre $\mathbb{C}$, y, $f_1=ie_i, \ldots, f_n=ie_n$. Tome $j\in\Hom_\mathbb{R}(\mathbb{C}^n,\mathbb{C}^n)$ definida por
\begin{eqnarray*}
j:\mathbb{C}^n &\longrightarrow &\mathbb{C}^n\\
(z_1,\ldots,z_n) &\longmapsto & i(z_1,\ldots,z_n).
\end{eqnarray*}
Entonces $j^2=-\id_{\mathbb{C}^n}$ y
\[
\Big[j\Big]^\mathcal{B}_\mathcal{B}=\left[\begin{array}{cc} 0 & -I_n\\I_n & 0\end{array}\right]
\]
donde $0$ denota el origen de $M_{n\times n}(\mathbb{R})$ y $I_n\in\mathbb{M}_{n\times n}(\mathbb{R})$ es la matriz con unos en la diagonal y ceros en el resto de entradas.
\end{obs}

\begin{defn}
Una \emph{estructura compleja} en $V$ es un operador $j\in\Hom_\mathbb{R}(V,V)$ tal que $j^2=-\id_V$.
\end{defn}

\begin{pro}\label{proestcom}
Suponga que $V$ tiene dimensi\'on finita. Si $V$ admite una estructura compleja $j$, entonces la dimensi\'on de $V$ es par. En tal caso $V$ es un espacio vectorial sobre $\mathbb{C}$ mediante el producto por escalar definido por
\[
(a+bi)v=\left(a\id_V+bj\right)(v)
\]
para todo $a,b\in\mathbb{R}$ y $v\in V$. M\'as a\'un, existe una base de la forma $T=\{v_1,\ldots,v_n,j(v_1),\ldots,j(v_n)\}$ donde $2n=\dim(V)$. En particular
\[
\Big[j\Big]^\mathcal{B}_\mathcal{B}=\left[\begin{array}{cc} 0 & -I_n\\I_n & 0\end{array}\right]
\]
\end{pro}

\dem Si $m$ es la dimensi\'on de $V$ entonces 
\[
0\le\left(\det(j)\right)^2=\det\left(j^2\right)=\det(-\id_V)=(-1)^m
\]
luego $m$ es par, es decir $m=2n$ para alg\'un $n\in\mathbb{Z}_{\ge 0}$. Para verificar que bajo la multiplicaci\'on por escalar definida $V$ es un espacio vectorial bajo $\mathbb{C}$ basta verificar que esta es unitaria, asociativa y que es distributiva. Lo cual se sigue de las siguientes igualdades
\begin{eqnarray*}
1v & = & \id_V(v)=v \\
(a+bi)\left((c+di)v\right) & = &  \left(a\id_V+bj\right)\circ\left(c\id_V+dj\right)(v) \\
                     & = & \left(ac\id_V+adj+bcj+bdj^2\right)(v)\\
                     & = & \left((ac-bd)\id_V+(ad+bc)j\right)(v)\\
                     & = & \left((a+bi)(c+di)\right)v\\
(a+bi)(v+w) & = & \left(a\id_V+bj\right)(v+w)\\
                   & = & \left(a\id_V+bj\right)(v)+\left(a\id_V+bj\right)(w)\\
                   & = & (a+bi)v+(a+bi)w\\
\left((a+bi)+(c+di)\right)v & = & \left((a+c)\id_V+(b+d)j\right)v\\
    & = & \left(a\id_V+bj+c\id_V+dj\right)v\\
    & = & \left(a\id_V+bj\right)(v)+\left(c\id_V+dj\right)(v)\\
    & = & (a+bi)v+(c+di)v
\end{eqnarray*}
validas para todo $a,b,c,d\in\mathbb{R}$ y $v,w\in V$.\\
Sea $\{v_1,\ldots,v_{n'}\}$ una base de $V$ como espacio vectorial sobre $\mathbb{C}$, donde $n'$ es su dimensi\'on. Entonces $V$ es generado como espacio vectorial sobre $\mathbb{R}$, por $\{v_1,\ldots,v_{n'},j(v_1),\ldots,j(v_{n'})\}$ pues
\[
\sum_{k=1}^{n'}z_kv_k=\sum_{k=1}^{n'}a_kv_k+b_kj(v_k),
\]
donde para $k=1,\ldots,n'$, $z_k=a_k+b_ki$ con $a_k,b_k\in\mathbb{R}$. M\'as este conjunto de generadores es linealmente independiente pues si $a_1,\ldots,a_{n'},b_1,\ldots,b_{n'}\in\mathbb{R}$ son tales que $\sum_{k=1}^{n'}a_kv_k+b_kj(v_k)=0$, entonces $\sum_{k=1}^{n'}z_kv_k=0$, con $z_k=a_k+b_ki$; luego todo $z_k=0$ y as\'i todo $a_k=b_k=0$. Tenemos $2n'=2n$ y $\mathcal{B}=\{v_1,\ldots,v_n,j(v_1),\ldots,j(v_n)\}$ es una base de $V$ como espacio vectorial sobre $\mathbb{R}$.\qed

\begin{pro}\label{proestlcom}
Sean $j$ una estructura compleja en $V$ y $f\in\Hom_{\mathbb{R}}(V,V)$. Entonces, tomando $V$ como un espacio vectorial sobre $\mathbb{C}$ mediante el producto por escalar definido por $(a+bi)v=\left(a\id_V+bj\right)(v)$, para todo $a,b\in\mathbb{R}$ y $v\in V$, tenemos $f\in\Hom_{\mathbb{C}}(V,V)$ si y solo si  $f\circ j=j\circ f$, o, equivalentemente, $-j\circ f\circ j=f$.
\end{pro}

\dem Suponga que $f\in\Hom_{\mathbb{C}}(V,V)$, entonces para todo $v\in V$
\[
f\circ j(v)=f(iv)=if(v)=j\circ f(v).
\]
Rec\'iprocamente, si $f\circ j=j\circ f$, para todo $a,b\in\mathbb{R}$ y $v\in V$,
\begin{eqnarray*}
f\left((a+bi)v\right) & = & f\circ(a\id_V+bj)(v)\\
 & = & \left(af+bf\circ j\right)(v)\\
 & = & \left(af+bj\circ f\right)(v)\\
 & = & (a\id_V+bj)\circ f(v)\\
 & = & (a+bi)f(v)
\end{eqnarray*}
luego $f\in\Hom_{\mathbb{C}}(V,V)$. Finalmente componemos ambos lados de la igualdad $f\circ j=j\circ f$ por $-j=j^{-1}$ para obtener  $-j\circ f\circ j=f$.\qed

\begin{teo}
Suponga que $V$ tiene dimensi\'on finita igual a $2n$ y sea $j\in\Hom_\mathbb{R}(\mathbb{C}^n,\mathbb{C}^n)$ el operador que corresponde a multiplicaci\'on por $i$. Entonces cada estructura complejas en $V$ es de la forma
$$j_f=f\circ j\circ f^{-1}$$
para alg\'un isomorfismos $f\in\Hom_{\mathbb{R}}(\mathbb{C}^n,V)$. M\'as a\'un $j_f=j_g$ si y solo si $g^{-1}\circ f\in\Hom_{\mathbb{C}}(\mathbb{C}^n,\mathbb{C}^n)$.
\end{teo}

\dem Dado un isomorfismo $f\in\Hom_{\mathbb{R}}(\mathbb{C}^n,V)$, el operador
\[
j_f=f\circ j\circ f^{-1}\in\Hom_\mathbb{R}(V,V)
\]
es una estructura compleja (ver Figura \ref{estcomp}), pues
\begin{eqnarray*}
j_f^2 & = & f\circ j\circ f^{-1}\circ f\circ j\circ f^{-1}\\
         & = & f\circ j^2\circ f^{-1}\\
         & = & f\circ-\id_{\mathbb{C}^n}\circ f^{-1}\\
         & = & -f\circ f^{-1}\\
         & = & -\id_V.
\end{eqnarray*}
\begin{figure}[!hbp]
\centering
\begin{tikzpicture}[auto, node distance=2cm,>=latex']
    \node (C1){$\mathbb{C}^n$};
    \node (C2) [below of=C1] {$\mathbb{C}^n$};
    \node (V1) [right of=C1] {$V$};
    \node (V2) [below of=V1] {$V$};
    
    \path[->] (C1) edge node {$f$} (V1);
    \path[->] (C1) edge  node {$j$} (C2);
    \path[->] (C2) edge  node {$f$} (V2);
    \path[dashed,->] (V1) edge  node {$j_f$} (V2);
\end{tikzpicture}
\caption{Estructura compleja}
\label{estcomp}
\end{figure}
Rec\'iprocamente, dada una estructura compleja $j_V$ en $V$, usando la notaci\'on en Observaci\'on \ref{estcomest}, Propiedad \ref{proestcom} implica que
\begin{eqnarray*}
f:\mathbb{C}^n & \longrightarrow & V\\
e_k &\longmapsto & v_k\\
f_k &\longmapsto & j_V(v_k)
\end{eqnarray*}
donde $\{v_1,\ldots,v_n,j_V(v_1),\ldots,j_V(v_n)\}$ es una base de $V$, es un isomorfismo de espacios vectoriales sobre $\mathbb{R}$ tal que $j_f=j_V$. Finalmente, suponga que $f,g\in\Hom_{\mathbb{R}}(\mathbb{C}^n,V)$ son isomorfismos tales que $j_f=j_g$, es decir
$$f\circ j\circ f^{-1}=g\circ j\circ g^{-1}.$$
Entonces, como $(g^{-1}\circ f)\circ j=j\circ (g^{-1}\circ f)$, por Propiedad \ref{proestlcom} tenemos que $g^{-1}\circ f\in\Hom_{\mathbb{C}}(\mathbb{C}^n,\mathbb{C}^n)$. \qed

\begin{pro}
Suponga que $V$ es un espacio euclideo con producto interno denotado por $\langle\bullet,\bullet\rangle$ y $j$ es una estructura compleja en $V$. Entonces, tomando $V$ como un espacio vectorial sobre $\mathbb{C}$ mediante el producto por escalar definido por $(a+bi)v=\left(a\id_V+bj\right)(v)$, para todo $a,b\in\mathbb{R}$ y $v\in V$,
\begin{eqnarray*}
\langle\bullet;\bullet\rangle_j: V\times V & \longrightarrow & \mathbb{C}\\
(v_1,v_2) & \longmapsto & \langle v_1;v_2\rangle+\langle v_1;j(v_2)\rangle i
\end{eqnarray*}
es un producto de herm\'itico sobre $V$ si y solo si $j$ es ortogonal.
\end{pro}

\dem Suponga primero que $j$ es ortogonal. Entonces $j^*=j^{-1}=-j$. Note que $\langle\bullet,\bullet\rangle_f$ es sesquilineal y adem\'as herm\'itica pues
\begin{eqnarray*}
\langle v_2;v_1\rangle_j & = & \langle v_2;v_1\rangle+\langle v_2;j(v_1)\rangle i\\
 & = & \langle v_2;v_1\rangle-\langle j(v_2);v_1\rangle i\\
 & = & \langle v_1;v_2\rangle-\langle v_1;j(v_2)\rangle i\\
 & = & \overline{\langle v_1;v_2\rangle_j}.
\end{eqnarray*}
Ahora, dado $v\in V$, 
\[
\langle v;j(v)\rangle=-\langle j(v);v\rangle=-\langle v;j(v)\rangle
\]
luego $\langle v;j(v)\rangle=0$. Si adem\'as $v\ne 0$,
\[
\langle v;v \rangle_j=\langle v;v \rangle+\langle v;j(v) \rangle i= \langle v;v \rangle>0
\]
luego $\langle\bullet;\bullet\rangle_f$ es definitivamente positiva y as\'i es un producto herm\'itico.\\
Rec\'iprocamente suponga que $\langle\bullet;\bullet\rangle_f$ es producto herm\'itico, entonces para todo $v_1,v_2\in V$
\begin{eqnarray*}
\langle j(v_1);j(v_2)\rangle & = & \rea\left(\langle j(v_1);j(v_2)\rangle_j\right)\\
 & = & \left(\langle j(v_1);j(v_2)\rangle_j+\langle j(v_2);j(v_1)\rangle_j\right)/2\\
 & = & \left(\langle iv_1;iv_2\rangle_j+\langle iv_2;iv_1\rangle_j\right)/2\\
 & = & \left(\langle v_1;v_2\rangle_j+\langle v_2;v_1\rangle_j\right)/2\\
 & = & \rea\left(\langle v_1;v_2\rangle_j\right)\\
 & = & \langle v_1;v_2\rangle
\end{eqnarray*}
luego $j$ es ortogonal.\qed

\begin{obs}
Note que la norma definida por $\langle\bullet;\bullet\rangle$ y por $\langle\bullet;\bullet\rangle_j$ coinciden.
\end{obs}

\begin{pro}
Suponga que $V$ es un espacio euclideo con producto interno denotado por $\langle\bullet;\bullet\rangle$, $j$ es una estructura compleja ortogonal en $V$ y $f\in\Hom_\mathbb{R}(V,V)$. Entonces, tomando $V$ como un espacio unitario mediante el producto por escalar definido por $(a+bi)v=\left(a\id_V+bj\right)(v)$, para todo $a,b\in\mathbb{R}$ y $v\in V$, y el producto herm\'itico
\[
\langle\bullet;\bullet\rangle_j = \langle\bullet;\bullet\rangle+\langle\bullet;j(\bullet)\rangle i,
\]
tenemos $f$ es unitario si y solo si $f\circ j = j\circ f$ y $f$ es ortonormal.
\end{pro}

\dem Suponga primero que $f$ es unitaria, entonces $f\in\Hom_\mathbb{C}(V,V)$ y as\'i $f\circ j=j\circ f$; adem\'as para todo $v\in V$
\[
\|j(v)\|^2=\langle j(v);j(v)\rangle=\langle j(v);j(v)\rangle_j=\langle v;v\rangle_j=\langle v;v\rangle=\|v\|^2
\]
luego $f$ es ortonormal. Rec\'iprocamente, si $f\circ j=j\circ f$ y $f$ es ortogonal, 
para todo $v_1,v_2\in V$,
\begin{eqnarray*}
\langle f(v_1);f(v_2)\rangle_j & = & \langle f(v_1);f(v_2)\rangle+\langle f(v_1);j\circ f(v_2)\rangle i\\
 & = & \langle f(v_1);f(v_2)\rangle+\langle f(v_1);f\circ j(v_2)\rangle i\\
 & = & \langle v_1;v_2\rangle+\langle v_1;j(v_2)\rangle i\\
 & = & \langle v_1;v_2\rangle_j,
\end{eqnarray*}
es decir, $f$ es unitaria.\qed

\begin{obs}
Bajo las hip\'otesis de la propiedad, suponga adem\'as que $V$ tiene dimensi\'on finita. Si $f$ es unitaria, entonces
\[
j=f^*\circ j \circ f
\]
pues, como $f$ es ortogonal, $f^*=f^{-1}$ y, como $f\in\Hom_{\mathbb{C}}(V,V)$, $f\circ j=j\circ f$, as\'i $f^*\circ j \circ f=f^{-1}\circ f\circ j=j$. Rec\'iprocamente,
suponga que $j=f^*\circ j \circ f$, entonces para todo $v_1,v_2\in V$
\begin{eqnarray*}
\langle f(v_1),j\circ f(v_2)\rangle & = & \langle v_1,f^*\circ j\circ f(v_2)\rangle\\
  & = & \langle v_1,j(v_2)\rangle
\end{eqnarray*}
luego $f$ preserva la parte imaginaria del producto herm\'itico $\langle\bullet;\bullet\rangle_j$. Si ademas $f$ es ortogonal, $f$ preserva la parte real y como en tal caso $f^*=f^{-1}$,
\[
f\circ j=f\circ f^*\circ j\circ f=j\circ f,
\]
luego $f\in\Hom_{\mathbb{C}}(V,V)$ y $f$ es unitaria. Por otro lado si $f$ conmuta con $j$, entonces $j=f^*\circ j \circ f=f^*\circ f\circ j$ y as\'i $\id_V=f^*\circ f$, es decir $f$ es ortogonal y a su vez $f$ es entonces unitaria.\\
El hecho que $j=f^*\circ j\circ f$ sea equivalente a que $f$ preserve la parte imaginaria del producto herm\'itico, $\langle\bullet;j(\bullet)\rangle$, es la motivaci\'on para el contenido del pr\'oximo capitulo donde entraremos a estudiar este tipo de estructuras que se llaman espacios simpl\'ecticos, las cuales generalizan est\'a intersecci\'on entre espacios ortogonales, espacios unitarios y estructuras complejas.
\end{obs} 
\chapter{Espacios simpl\'ecticos}

Sea $K$ un cuerpo y $V$ un espacio vectorial sobre $K$.

\section{Forma simpl\'ectica}

\begin{defn}
Una \emph{forma simpl\'ectica} en $V$ es una funci\'on
\begin{eqnarray*}
\sigma: V\times V & \longrightarrow & K\\
(v_1,v_2) & \longmapsto & \sigma(v_1,v_2)
\end{eqnarray*}
tal que:
\begin{enumerate}
\item \emph{es bilineal}: para todo $v,v_1,v_2\in V$ y $c\in\mathbb{C}$
\begin{eqnarray*}
\sigma(v_1+v_2,v) & = & \sigma(v_1,v)+\sigma(v_2,v)\\
\sigma(cv_1,v_2) & = & c\sigma(v_1,v_2)\\
\sigma(v,v_1+v_2) & = & \sigma(v,v_1) + \sigma(v,v_2)\\
\sigma(v_1,cv_2) & = & c\sigma(v_1,v_2);
\end{eqnarray*}
\item \emph{es alternante}: para todo $v\in V$
\[
\sigma(v,v)=0;
\]
\item \emph{es no-degenerada} Si $v\in V$ es tal que $\sigma(v,w)=0$ para todo $w\in V$ entonces $v=0$.
\end{enumerate}
Un \emph{espacio simpl\'ectico} es un espacio vectorial provisto de una forma simpl\'ectica. 
\end{defn}

\begin{obs}
Para todo $v_1,v_2\in V$, 
\[
\sigma(v_2,v_1)=-\sigma(v_1,v_2).
\]
De hecho, la condici\'on alternante implica que
\begin{eqnarray*}
0 & = & \sigma(v_1+v_2,v_1+v_2)\\
   & = & \sigma(v_1,v_1)+\sigma(v_1,v_2)+\sigma(v_2,v_1)+\sigma(v_2,v_2)\\
   & = & \sigma(v_1,v_2)+\sigma(v_2,v_1)
\end{eqnarray*}
\end{obs}

\begin{ejem}\label{ejem0simp}
\begin{enumerate}
\item Sobre $V=K^{2n}=K^n\times K^n$
\[
\sigma\left((\overline{q},\overline{p}),(\overline{q}',\overline{p}'\right)=\sum_{i=1}^n p_iq'_i-p'_iq_i
\]
donde $\overline{q}=(q_1,\ldots,q_n)$, $\overline{p}=(p_1,\ldots,p_n)$, $\overline{q}'=(q'_1,\ldots,q'_n)$ y $\overline{p}'=(p'_1,\ldots,p'_n)$.
\item Sobre $V\times V^*$
\[
\sigma\left((v,\lambda),(w,\mu)\right)=\lambda(w)-\mu(v)
\]
donde $v,w\in V$ y $\lambda,\mu\in V^*$.
\item Suponga que $K=\mathbb{R}$ y $V$ es un espacio eucl\'ideo con una estructura compleja ortogonal $j$. Sobre $V$
\[
\sigma(v_1,v_2)=\langle v_1,j(v_2)\rangle
\]
\end{enumerate}
\end{ejem}

\begin{obs}
Una forma simpl\'ectica $\sigma$, al ser bilineal, induce una transformaci\'on lineal
\begin{eqnarray*}
s: V & \longrightarrow & V^*\\
v & \longmapsto & s_v=\sigma(v,\bullet): w\mapsto \sigma(v,w).
\end{eqnarray*}
El hecho que $\sigma$ sea no-degenerada implica que $s$ es inyectiva; y es un isomorfismo si $V$ tiene dimensi\'on finita. Pues, la condici\'on de que $\sigma$ sea no degenerada quiere decir que $s_v=0$ si y solo si $v=0$.\\
Con este mapa, tenemos
\[
s(v)(w)=s_v(w)=\sigma(v,w)
\]
En particular $s$ es una transformaci\'on lineal $V\rightarrow V^*$ y su dual $s^*$ es una transformaci\'on lineal $\left(V^*\right)^*\rightarrow V^*$.
\end{obs}


\begin{prop}
Sea $V$ un espacio simpl\'ectico de dimensi\'on finita y
\begin{eqnarray*}
\widehat{\bullet}: V & \longrightarrow & \left(V^*\right)^*\\
                            v &\longmapsto &\widehat{v}:\lambda\mapsto\lambda(v).
\end{eqnarray*}
el isomorfismo can\'onico. Entonces para todo $v\in V$
\[
s^*(\widehat{v})=-s(v)
\]
\end{prop}

\dem Para todo $w\in V$
\begin{eqnarray*}
s^*(\widehat{v})(w)  & = & \widehat{v}(s(w))\\
  & = & s(w)(v)\\
  & = & \sigma(w,v)\\
  & = & -\sigma(v,w)\\
  & = & -s(v)(w).
\end{eqnarray*}
\qed

\begin{pro}
Si $V$ es un espacio simpl\'ectico y tiene dimensi\'on finita, entonces su dimensi\'on es par. 
\end{pro}

\dem Sea $T=\{v_1,\ldots,v_m\}$ una base de $V$ y $T^*=\{\lambda_1,\ldots,\lambda_m\}$ la base de $V^*$ dual de $T$. Tomamos la imagen de $T$ mediante el isomorfismo can\'onico $V\mapsto \left(V^*\right)^*$, la cual es la base $\widehat{T}=\{\widehat{v_1},\ldots,\widehat{v_m}\}$ de $\left(V^*\right)^*$ dual de $T^*$. La proposici\'on anterior implica que si tomamos las representaciones matriciales en $M_{m\times m}(K)$
\[
A=\Big[s\Big]^{T^*}_T,\textrm{ y } B=\Big[s^*(\widehat{\bullet})\Big]^{T^*}_{T}=\Big[s^*\Big]^{T^*}_{\widehat{T}}\Big[\widehat{\bullet}\Big]^{\widehat{T}}_T=\Big[s^*\Big]^{T^*}_{\widehat{T}}, 
\]
entonces $B=-A$, pero por otro lado $B=A^\intercal$, luego $A^\intercal=-A$. De donde
\[
\det(A)=\det(A^\intercal)=\det(-A)=(-1)^m\det(A).
\]
Ahora como $\sigma: V\rightarrow V^*$ es inyectiva, $\det(A)\ne 0$ y as\'i $1=(-1)^m$, en particular $m=2n$ para alg\'un $n\in\mathbb{Z}_{\ge 0}$.\qed

\begin{defn}
Sean $V$ un espacio simpl\'ectico y $S\subseteq V$, el \emph{conjunto $\sigma$-ortogonal a $S$} est\'a definido por
\[
S^\sigma=\{v\in V|\ \sigma(w,v)=0, \textrm{ para todo } w\in S\}
\]
\end{defn}

\begin{obs}
Note que $S^\sigma=(s(S))_0$. De hecho
\begin{eqnarray*}
S^\sigma & = & \{v\in V|\ \sigma(w,v)=0, \textrm{ para todo } w\in S\}\\
  & = & \{v\in V|\ s(w)(v)=0,\textrm{ para todo } w\in S\}\\
  & = & \{v\in V|\ \lambda(v)=0,\textrm{ para todo } \lambda\in s(S)\}\\
  & = & (s(S))_0  
\end{eqnarray*}
Esto implica la siguiente propiedad.
\end{obs}

\begin{pro}\label{ortsubespsimp}
Sean $V$ un espacio simpl\'ectico y $S\subseteq V$, entonces $S^\sigma\le V$. Si $S'\subseteq S$ entonces $S^\sigma\le S'^\sigma$. Si $V$ tiene dimensi\'on finita y $U\le V$ entonces
\[
\dim(U)+\dim (U^\sigma)=\dim(V)
\]
\end{pro}

\section{Subespacios isotr\'opicos y bases de Darboux}

Sea $V$ un espacio simpl\'ectico sobre $K$.

\begin{defn}
Sea $U\le V$, entonces decimos que
\begin{enumerate}
\item $U$ es un \emph{subespacio simpl\'ectico} si la restricci\'on de $\sigma$ a $U\times U$ es una forma simpl\'ectica;
\item $U$ es un \emph{subespacio isotr\'opico} si $U\le U^\sigma$; 
\item $U$ es un \emph{subespacio lagrangiano} si $U=U^\sigma$;
\end{enumerate}
\end{defn}

\begin{obs}
Suponga que $V$ tiene dimensi\'on finita. Si $U\le V$ es un subespacio lagrangiano y $\dim(V)=2n$ entonces $\dim(U)=n$, de hecho como $U=U^\sigma$, 
\[
2n=\dim(V)=\dim(U)+\dim(U^\sigma)=2\dim(U).
\]
\end{obs}

\begin{ejem}
\begin{enumerate}
\item En el espacio simpl\'ectico de Ejemplo \ref{ejem0simp}.1, $V=K^n\times K^n$, denote, para $i=1,\ldots,n$, $e_i$ el elemento cuya $i$-\'esima entrada es $1$ y el resto ceros, y $f_i$ el elemento cuya $n+i$-\'esima entrada es $1$ y el resto ceros. Note que para todo $i,j\in\{1,\ldots,n\}$,
\[
\sigma(e_i,f_j)=-\delta_{ij}\quad\sigma(e_i,e_j)=\sigma(f_i,f_j)=0.
\]
Entonces para cualquier subconjunto de indices $J\subset I=\{1,\ldots,n\}$,
\[
V_J=\Sp\left(\{e_j,f_j\}_{j\in J}\right)
\]
es un subespacio simpl\'ectico,
\[
E_J=\Sp\left(\{e_j\}_{j\in J}\right),\textrm{ y } F_J=\Sp\left(\{f_j\}_{j\in J}\right)
\]
son isotr\'opicos, y si $J=I$, $E_I$ y $F_I$ son subespacios lagrangianos.
\item  En el espacio simpl\'ectico de Ejemplo \ref{ejem0simp}.2, $V\times V^*$, sea $\{v_i\}_{i\in I}$ una base de $V$ y $\{\lambda_i\}_{i\in I'}$ una base de $V^*$ donde $I\subseteq I'$ y $\lambda_i(v_j)=\delta_{ij}$ para todo $i,j\in I$. Note que para todo $i,j\in I$
\[
\sigma\left((v_i,0),(0,\lambda_j)\right)=-\delta_{ij}\quad\sigma\left((v_i,0),(v_j,0)\right)=0
\]
y para todo $i,j\in I'$
\[
\sigma\left((0,f_i),(0,f_j)\right)=0.
\]
Entonces para cualquier subconjunto de indices $J\subset I$,
\[
(V\times V^*)_J=\Sp\left(\{(v_j,0),(0,\lambda_j)\}_{j\in J}\right)
\]
es un subespacio simpl\'ectico,
\[
E_J=\Sp\left(\{(v_j,0)\}_{j\in J}\right),\textrm{ y } F_J=\Sp\left(\{(0,\lambda_j)\}_{j\in J}\right)
\]
son isotr\'opicos, y si $J=I$, $E_I$ y $F_I$ son subespacios lagrangianos.
\end{enumerate}
\end{ejem}

\begin{prop}\label{propsimisolan}
Sea $U\le V$, entonces
\begin{enumerate}
\item $U$ es un subespacio simpl\'ectico si y solo si la restricci\'on de $s$ a $U$ es inyectiva. En particular, $U$ es un subesapcio simpl\'ectico si y solo si $U\cap U^\sigma=\{0\}$.
\item $U$ es un subespacio isotr\'opico si y solo si $\sigma(u,u')=0$ para todo $u,u'\in U$ (es decir $s(U)=0$).
\item $U$ es un subespacio lagrangiano si y solo si $U$ es un subespacio isotr\'opico maximal.
\end{enumerate}
\end{prop}

\dem
\begin{enumerate}
\item Si $U$ es subespacio simpl\'ectico, entonces la restricci\'on de $\sigma$ a $U\times U$ es no-degenerada, en particular dado $u\in U$, $u\ne 0$, existe $w\in U$ tal que $\sigma(u,w)\ne 0$. As\'i pues la imagen en $U^*$, $s(u)$, es diferente de $0$ pues $s(u)(w)\ne 0$. Es decir el n\'ucleo de $s$ restringido a $U$ es $\{0\}$, luego es inyectiva. Reciprocamente, si la restricci\'on $s$ a $U$ es inyectiva, la restricci\'on de $\sigma$ a $U\times U$ es bilineal, alternante y no-degenerada, luego $U$ es un subespacio simplectico. Para establecer la segunda afirmaci\'on basta con observar que $u\in U\cap U^\sigma$ si y solo si $s(u)(w)=0$ para todo $w\in U$, es decir si y solo si $u$ pertenece al n\'ucleo de la restricci\'on de $s$ a $U$.
\item Suponga que $U$ es isotr\'opico, luego para todo $u,u'\in U$, como $U\le U^\sigma$, $u'\in U^\sigma$ y $\sigma(u,u')=0$. Reciprocamente, si $\sigma(u,u')=0$ para todo $u,u'\in U$, entonces $U\le U^\sigma$.
\item Suponga que $U$ es un subespacio lagrangiano, entonces $U$ es isotr\'opico. Suponga que existe $U'\le V$ isotr\'opico tal que $U\le U'$. Sea $u'\in U'$, entonces $\sigma(u,u')=0$ para todo $u\in U$, en particular $u'\in U^\sigma=U$. Luego $U'=U$. Rec\'iprocamente, si $U$ es isotr\'opico m\'aximal, dado $u'\in U^\sigma$,
\[
\sigma(u_1+au',u_2+bu')=\sigma(u_1,u_2)+b\sigma(u_1,u')+a\sigma(u',u_2)+\sigma(u',u')=0,
\]
luego $U+\Sp(\{u'\})$ es isotr\'opico y as\'i $u'\in U$. Luego $U=U^\sigma$.\qed 
\end{enumerate}

\begin{pro}
Suponga que $V$ tiene dimensi\'on finita y sea $U_0\le V$ un subespacio isotr\'opico. Entonces existe un subespacio lagrangiano $U\le V$ que contiene a $U_0$.
\end{pro}

\dem Tenemos $U_0\le U_0^\sigma$. Si $U_0=U_0^\sigma$, entonces $U=U_0$ es un subespacio lagrangiano, de lo contrario existe $u\in U_0^\sigma\setminus U_0$. Entonces, $u\ne 0$ y para todo $u_1+au,u_2+bu\in U'=U_0+\Sp(\{u\})$, $u_1,u_2\in U$ y $a,b\in K$,
\[
\sigma(u_1+au,u_2+bu)=\sigma(u_1,u_2)+b\sigma(u_1,u)+a\sigma(u,u_2)+\sigma(u,u)=0,
\]
luego $U'$ es isotr\'opico y 
\[
U_0<U'\le U'^\sigma<U_0^\sigma.
\]
Reemplazamos $U_0$ por $U'$ y continuamos recursivamente. Por monotonia de la dimensi\'on, como $V$ tiene dimensi\'on finita, eventualmente obtenemos $U=U'$ subespacio lagrangiano.\qed

\begin{obs}[Extensi\'on de subespacios isotr\'opicos a lagrangianos en dimensi\'on infinita]
El mismo resultado de la proposici\'on anterior, se puede generalizar a espacios simpl\'ecticos de dimensi\'on infinita usando Lema de Zorn. De hecho, dado $U_0\le V$ subespacio isotr\'opico,  consideramos la colecci\'on $P$ de subespacios isotr\'opicos que contienen a $U_0$, ordenados por contenencia. Como $U_0\in P$, $P\ne\emptyset$. Tambi\'en, la uni\'on de elementos en una cadena de $P$ est\'a en $P$ y es una cota superior de la cadena. Entonces $U$ m\'aximal en $P$ por la proposi\'on anterior es lagrangiano y contiene a $U_0$.
\end{obs}

\begin{prop}
Suponga que $V$ tiene dimensi\'on finita y sea $V_1\le V$ un subespacio lagrangiano. Entonces dado $U\le V$, isotr\'opico, tal que $U\cap V_1=\{0\}$, exite $V_2\le V$ lagrangiano, tal que $U\le V_2$ y $V_1\cap V_2=\{0\}$, en particular
\[
V=V_1\oplus V_2.
\]
\end{prop}

\dem Suponga que $\dim(V)=2n$. Si $U=U^\sigma$, entonces $V_2=U$ es lagrangiano, de lo contrario $\dim(U)=k<n$ y $\dim(U^\sigma)=2n-k>n$, y, como $\dim(V_1)=n$,
\[
\dim(U+V_1)=\dim(U)+\dim(V_1)-\dim(U\cap V_1)=k+n
\]
Suponga por contradicci\'on que $U^\sigma\subseteq U+V_1$, entonces tomando los espacios $\sigma$-ortogonales,
\[
U^\sigma\cap V_1=U^\sigma\cap V_1^\sigma=(U+V_1)^\sigma\subseteq \left(U^\sigma\right)^\sigma=U
\]
de donde $U^\sigma\cap V_1\subseteq U\cap V_1=\{0\}$. Entonces
\[
\dim(U^\sigma+V_1)=\dim(U^\sigma)+\dim(V_1)>n+n=2n=\dim(V)
\]
lo cual es una contradicci\'on. As\'i pues, existe $u\in U^\sigma\setminus (U+V_1)$. Entonces, $u\not\in V_1$ y para todo $u_1+au,u_2+bu\in U'=U+\Sp(\{u\})$, $u_1,u_2\in U$ y $a,b\in K$,
\[
\sigma(u_1+au,u_2+bu)=\sigma(u_1,u_2)+b\sigma(u_1,u)+a\sigma(u,u_2)+\sigma(u,u)=0,
\]
luego $U'$ es isotr\'opico, $U'\cap V_1=\{0\}$ y 
\[
U<U'\le U'^\sigma<U^\sigma.
\]
Reemplazamos $U_0$ por $U'$ y continuamos recursivamente. Por monotonia de la dimensi\'on, como $V$ tiene dimensi\'on finita, eventualmente obtenemos $V_2=U'$ lagrangiano con $V_2\cap V_1=\{0\}$. Ahora como $\dim(V_1)=\dim(V_2)=n$ entonces $\dim(V_1+V_2)=2n$ y $V=V_1\oplus V_2$.\qed


\begin{pro}
Suponga que $V$ tiene dimensi\'on finita y sean $V_1,V_2\le V$ lagrangianos tales que 
\[
V=V_1\oplus V_2,
\]
entonces si $\pi_1:V\rightarrow V_1$ y $\pi_2:V\rightarrow V_2$ son respectivamente las proyecciones sobre $V_1$ y $V_2$ dadas por la descomposici\'on $V=V_1\oplus V_2$,
\[
\pi_1^*(V_1^*)=s(V_2)\qquad\textrm{y}\qquad\pi_2^*(V_2^*)=s(V_1)
\]
\end{pro}
\begin{figure}[!hbp]
\centering
\begin{tikzpicture}[node distance=2cm, auto, >=latex']
    \node (V1) {$V_1$};
    \node (V0) [below of=V1] {$V_1\oplus V_2$};
    \node (V2) [below of=V0] {$V_2$};
    \node (V) [right of=V0] {$V$};
    \node (A) [right of=V] {};
    \node (VD) [right of=A] {$V^*$};
    \node (V0D) [right of=VD] {$\pi_2^*(V_2^*)\oplus\pi_1^*(V_1^*)$};
    \node (V2D) [above of=V0D] {$\pi_2^*(V_2^*)$};
    \node (V1D) [below of=V0D] {$\pi_1^*(V_1^*)$};
    
    \path[-] (V) edge node {$s$} (VD);
    \path[-] (V1) edge node {$s$} (V2D);
    \path[-] (V2) edge node {$s$} (V1D);
    \path[<-] (V1) edge node {$\pi_1$} (V);
    \path[->] (V) edge node {$\pi_2$} (V2);
    \path[<-] (VD) edge node {} (V2D);
    \path[->] (V1D) edge  node {} (VD);
    \draw[double equal sign distance,shorten <=6pt,shorten >=6pt] (V0) to (V);
    \draw[double equal sign distance,shorten <=6pt,shorten >=6pt] (VD) to (V0D);
\end{tikzpicture}
\caption{Descomposici\'on lagrangiana}
\label{desclag}
\end{figure}

\dem (Ver Figura \ref{desclag}) Como $\pi_1$ y $\pi_2$ son sobreyectivas, $\pi_1^*$ y $\pi_2^*$ son  inyectivas. Luego dado $\lambda\in\pi^*(V_1)$, existe un \'unico $\lambda_1\in V_1^*$ tal que $\pi^*(\lambda_1)=\lambda$. En particular, para todo $v_2\in V_2$, como $\pi_1(v_2)=0$,
\[
\lambda(v_2)=\pi^*(\lambda_1)(v_2)=\lambda_1\left(\pi_1(v_2)\right)=0.
\]
Ahora, como $V$ tiene dimensi\'on finita $\sigma: V\rightarrow V^*$ es un isomorfismo, luego existe un \'unico $v\in V$ tal que $\sigma(v)=\lambda$, pero $\sigma(v,v_2)=\lambda(v_2)=0$ para todo $v_2\in V_2$, entonces $v\in V_2^\sigma=V_2$. As\'i $\pi_1^*(V_1^*)\subseteq S(V_2)$. Reciprocamente, dado $v_2\in V_2$, defina $\lambda_1\in V_1^*$ por
\[
\lambda_1:v_1\mapsto s(v_2)(v_1)=\sigma(v_2,v_1)
\]
el mapa
\begin{eqnarray*}
V_2 & \longrightarrow & \pi_1^*(V_1^*)\\
v & \longmapsto & s(v)=\lambda=\pi_1^*(\lambda_1)
\end{eqnarray*}
es inyectivo, y como $V_1^*$ y $\sigma(V_2)$ tienen la misma dimensi\'on, es un isomorfismo.\qed

\begin{defn}
Suponga que $V$ tiene dimensi\'on finita y sea $T=\{v_1,\ldots,v_n,w_1,\ldots,w_n\}$ una base de $V$. Decimos que $T$ es una base de Darboux si para todo $i,j\in\{1,\ldots,n\}$
\[
\begin{array}{rcccl}
\sigma(v_i,v_j) & = & 0 & = & \sigma(w_i,w_j)\\
\sigma(w_i,v_j) & = &\delta_{ij}.
\end{array}
\]
\end{defn}

\begin{obs}
Note que si $T=\{v_1,\ldots,v_n,w_1,\ldots,w_n\}$ es una base de Darboux entonces $V_1=\Sp(\{v_1,\ldots,v_n\})$ y $V_2=\Sp(\{w_1,\ldots,w_n\})$ son lagrangianos. M\'as a\'un si $T_1=\{v_1,\ldots,v_n\}$, $T_2=\{w_1,\ldots,w_n\}$ y $T_1^*=\{\lambda_1,\ldots,\lambda_n\}$ y $T_2^*=\{\mu_1,\ldots,\mu_n\}$ son las bases de $V_1^*$ y $V_2^*$, respectivamente, duales de $T_1$ y $T_2$. Entonces para $i=1,\ldots,n$,
\[
s(w_i)=\pi_1^*(\lambda_i)\qquad s(-v_i)=\pi_2^*(\mu_i).
\]
\end{obs}

\begin{teo}
Suponga que $V$ tiene dimensi\'on finita, entonces $V$ admite un base de Darboux.
\end{teo}

\dem Como $V$ tiene dimensi\'on finita, existen $V_1,V_2\le V$ subespacios lagrangianos, tales que $V=V_1\oplus V_2$. Sea $T_1=\{v_1,\ldots,v_n\}$ una base de $V_1$ y $T_1^*=\{\lambda_1,\ldots,\lambda_n\}$ la base de $V_1^*$ dual de $T_1$. Como $s(V_2)=\pi_1^*(V_1^*)$, donde $\pi_1:V\rightarrow V_1$ es la proyecci\'on en el primer sumando de la descomposici\'on $V=V_1\oplus V_2$, entonces existe $T_2=\{w_1,\ldots,w_n\}$ base de $V_2$ tal que para $i=1,\ldots,n$,
\[
s(w_i)=\pi_1^*(\lambda_i).
\]
Entonces como $V_1$ y $V_2$ son subespacios lagrangianos, en particular isotr\'opicos,
\[
\sigma(v_i,v_j) =  0  =  \sigma(w_i,w_j)
\]
para todo $i,j\in\{1,\ldots,n\}$; adem\'as
\[
\sigma(w_i,v_j)=s(w_i)(v_j)=\pi_1^*(\lambda_i)(v_j)=\lambda_i\left(\pi_1(v_j)\right)=\lambda_i(v_j)=\delta_{ij}.
\]
\qed

\begin{pro}\label{coorsimp}
Suponga que $V$ tiene dimensi\'on finita y sea $T=\{v_1,\ldots,v_n,w_1,\ldots,w_n\}$ una base de Darboux. Entonces para todo $v\in V$
\[
v=\sum_{i=1}^n\sigma(w_i,v)v_i-\sigma(v_i,v)w_i.
\]
En particular, si $v_1,v_2\in V$ son tales que
\[
v_1=\sum_{i=1}^nq_iv_i+p_iw_i \qquad v_2=\sum_{i=1}^nq'_iv_i+p'_iw_i
\]
entonces
\[
\sigma(v_1,v_2)=\sum_{i=1}^np_iq'_i-p'_iq_i
\]
\end{pro}

\dem Sean $a_1,\ldots,a_n,b_1,\ldots,b_n\in K$ tales que
\[
v=\sum_{i=1}^na_iv_i+b_iw_i,
\]
de forma que para $j=1,\ldots,n$,
\[
\sigma(v_j,v)=\sum_{i=1}^na_i\sigma(v_j,v_i)+b_i\sigma(v_j,w_i)=-b_j
\]
y
\[
\sigma(w_j,v)=\sum_{i=1}^na_i\sigma(w_j,v_i)+b_i\sigma(w_j,w_i)=a_j.
\]
Finalmente
\begin{eqnarray*}
\sigma(v_1,v_2) & = & \sigma(\sum_{i=1}^nq_iv_i+p_iw_i,\sum_{j=1}^nq'_jv_j+p'_jw_j)\\
 & = & \sum_{i,j=1}^nq_iq'_j\sigma(v_i,v_j)+p_iq'_j\sigma(w_i,v_j)+q_ip'_j\sigma(v_i,w_j)+p_ip'_j\sigma(w_i,w_j)\\
 & = & \sum_{i,j=1}^n(p_iq'_j-q_ip'_j)\delta_{ij}\\
 & = & \sum_{i=1}^np_iq'_i-p'_iq_i
\end{eqnarray*}
\qed

\begin{obs}
Note que si $T=\{v_1,\ldots,v_n,w_1,\ldots,w_n\}$, es una base de Darboux, para todo $J\subseteq\{1,\ldots,n\}$,
\[
V_J=\Sp\left(\{v_j,w_j\}_{j\in J}\right)
\]
es tambi\'en un subespacio simpl\'ectico.
\end{obs}

\begin{teo}[Ortogonalizaci\'on de Gram-Schmidt]
Suponga que $\dim(V)=2n$ y sean $U\le V$ subespacios simpl\'ectico con $\dim(U)=2m$, $m\le n$. Entonces existe una base de Darboux $T=\{v_1,\ldots,v_n,w_1,\ldots,w_n\}$ de $V$, tal que
\[
U=\Sp\left(\{v_j,w_j\}_{j=1,\ldots,m}\right)
\]
\end{teo}

\dem Sea $T'=\{v_1,\ldots,v_m,w_1,\ldots,w_m\}$ una base de Darboux de $U$. Si $m=n$ hemos terminado. De lo contrario, $m+1\le n$ y tome $v'_{m+1}\in V\setminus U$. Defina
\[
v_{m+1}=v'_{m+1}-\left(\sum_{i=1}^m\sigma(w_i,v'_{m+1})v_i-\sigma(v_i,v'_{m+1})w_i\right)
\]
de forma tal que para $j=1,\ldots,m$,  
\begin{eqnarray*}
\sigma(v_j,v_{m+1}) & = & \sigma(v_j,v'_{m+1})-\left(\sum_{i=1}^m\sigma(w_i,v'_{m+1})\sigma(v_j,v_i)-\sigma(v_i,v'_{m+1})\sigma(v_j,w_i)\right)\\
 & = & \sigma(v_j,v'_{m+1})-\sum_{i=1}^m\sigma(v_i,v'_{m+1})\delta_{ij}\\
 & = & \sigma(v_j,v'_{m+1})-\sigma(v_j,v'_{m+1})\\
 & = & 0;
\end{eqnarray*}
y,
\begin{eqnarray*}
\sigma(w_j,v_{m+1}) & = & \sigma(w_j,v'_{m+1})-\left(\sum_{i=1}^m\sigma(w_i,v'_{m+1})\sigma(w_j,v_i)-\sigma(v_i,v'_{m+1})\sigma(w_j,w_i)\right)\\
 & = & \sigma(w_j,v'_{m+1})-\sum_{i=1}^m\sigma(w_i,v'_{m+1})\delta_{ji}\\
 & = & \sigma(w_j,v'_{m+1})-\sigma(w_j,v'_{m+1})\\
 & = & 0.
\end{eqnarray*}
Ahora, existe $w''_{m+1}\in V$ tal que $\sigma(w''_{m+1},v_{m+1})\ne 0$. Sea
\[
w'_{m+1}=\frac{1}{\sigma(w''_{m+1},v_{m+1})}w''_{m+1}
\]
de forma que $\sigma(w'_{m+1},v_{m+1})=1$. Defina
\[
w_{m+1}=w'_{m+1}-\left(\sum_{i=1}^m\sigma(w_i,w'_{m+1})v_i-\sigma(v_i,w'_{m+1})w_i\right),
\]
as\'i $\sigma(w_{m+1},v_{m+1})=1$ y al igual que con $v_{m+1}$, para $j=1,\ldots,m$,
\[
\sigma(w_{m+1},v_j)=0=\sigma(w_{m+1},w_j).
\]
En particular $\{v_1,v_2,\ldots,v_{m+1},w_1,\ldots,w_{m+1}\}$ es base de Darboux de $U'=U+\Sp\left(\{v_{m+1},w_{m+1}\right)$. Reemplazamos $U$ por $U'$ y continuamos recursivamente.\qed

\begin{pro}
Suponga que $V$ tiene dimensi\'on finita y sea $U<V$ subespacio simplectico. Entonces $U^\sigma<V$ es un subespacio simpl\'ectico tal que
\[
V=U\oplus U^\sigma.
\]
\end{pro}

\dem Sea $v\in U$. Como $U$ es un espacio simpl\'ectico, si $v\ne 0$, existe $w\in U$ tal que $\sigma(w,v)\ne 0$, luego $v\not\in U^\sigma$. Luego
\[
U\cap U^\sigma=\{0\}
\]
Ahora como $\dim(U)+\dim(U^\sigma)=\dim(V)$, entonces $V=U\oplus U^\sigma$. Finalmente, sea $T=\{v_1,\ldots,v_n,w_1,\ldots,w_n\}$ una base de Darboux de $V$, tal que
\[
U=\Sp\left(\{v_j,w_j\}_{j=1,\ldots,m}\right)
\]
donde $2n=\dim(V)$ y $2m=\dim(U)$, $m<n$. Entonces si
\[
U'=\Sp\left(\{v_j,w_j\}_{j=m+1,\ldots,n}\right),
\]
$\dim(U')=\dim(U^\sigma)$ y $U'\subseteq U^\sigma$, luego $U'=U^\sigma$ es un subespacio simpl\'ectico de $V$.\qed

\begin{defn}
Suponga que $V$ tiene dimensi\'on finita y sea $U\le V$ un subespacio simpl\'ectico de $V$. Llamamos a $U^\sigma$ el \emph{complemento simpl\'ectico de $U$}. A la proyecci\'on
\[
p_U^\sigma: V\longrightarrow V
\]
sobre $U$, definida por la descomposici\'on $V=U\oplus U^\sigma$ la llamamos \emph{proyecci\'on simpl\'ectica sobre $U$}.
\end{defn}

\begin{pro}
Suponga que $V$ tiene dimensi\'on finita. Sean $V_1,V_2<V$ subespacios lagrangianos tales que
\[
V=V_1\oplus V_2
\]
y sean $p_1:V\rightarrow V$ y $p_2: V\rightarrow V$ las respectivas proyecciones sobre $V_1$ y $V_2$ definidas por esta descomposici\'on. Entonces para todo $v,w\in V$,
\[
\sigma\left(p_1(v),w\right)=\sigma\left(v,p_2(w)\right).
\]
Por otro lado, sea $U<V$ subespacio simpl\'ectico, entonces para todo $v,w\in V$,
\[
\sigma\left(p_U^\sigma(v),w\right)=\sigma\left(v,p_U^\sigma(w)\right).
\]
\end{pro}

\dem Sean $v_1,w_1\in V_1$ y $v_2,w_2\in V_2$ tales que
\[
v=v_1+w_1\qquad w=w_1+w_2,
\]
entonces
\begin{eqnarray*}
\sigma\left(p_1(v),w\right) & = & \sigma (v_1,w_1)+\sigma(v_1,w_2)=\sigma(v_1,w_2)\\
\sigma\left(v,p_2(w)\right) & = & \sigma(v_1,w_2)+\sigma(v_2,w_2)=
\sigma(v_1,w_2).
\end{eqnarray*}
Considere ahora $v',w'\in U^\sigma$ tales que
\[
v=p_U^\sigma(v)+v'\qquad w=p_U^\sigma(w)+w'.
\]
Entonces
\begin{eqnarray*}
\sigma\left(p_U^\sigma(v),w\right) & = & \sigma\left(p_U^\sigma(v),p_U^\sigma(w)\right)+\sigma\left(p_U^\sigma(v),w'\right)= \sigma\left(p_U^\sigma(v),p_U^\sigma(w)\right)\\
\sigma\left(v,p_U^\sigma(w)\right) & = & \sigma\left(p_U^\sigma(v),p_U^\sigma(w)\right)+\sigma\left(v',p_U^\sigma(w)\right)=\sigma\left(p_U^\sigma(v),p_U^\sigma(w)\right).
\end{eqnarray*}
\qed

\section{Operadores adjuntos}

Sea $V$ un espacio simpl\'ectico y $f\in\Hom_\mathbb{K}(V,V)$ un operador.

\begin{defn}
Sea $g\in\Hom_\mathbb{K}(V,V)$, decimos que $g$ es un \emph{operador adjunto de $f$} si para todo $v,w\in V$
\[
\sigma\left( f(v),w \right)=\sigma\left( v,g(w)\right).
\]
Decimos que $f$ es \emph{auto-adjunto} si $f$ es un operador adjunto de $f$. 
\end{defn}

\begin{obs}
Note que si $g$ es adjunto de $f$, entonces $f$ es adjunto de $g$. De hecho
\[
\sigma\left( g(v),w\right)= -\sigma\left( w,g(v)\right)= -\sigma\left( f(w),v\right)= \sigma\left(v,f(w)\right).
\]
\end{obs}

\begin{defn}
Sea $n\in\mathbb{Z}_{>0}$ y $A\in M_{2n\times 2n}(K)$, definimos la \emph{matriz adjunta simpl\'ectica} de $A$ por $A^{sigma}\in M_{2n\times 2n}(K)$ tal que
\[
A^\sigma=J^{-1}A^\intercal J
\]
con $J\in M_{2n\times 2n}(K)$ tal que
\[
J=\left[\begin{array}{cc} 0 & -I_n\\I_n & 0\end{array}\right]
\]
donde $0$ denota el origen de $M_{n\times n}(K)$ y $I_n\in\mathbb{M}_{n\times n}(K)$ es la matriz con unos en la diagonal y ceros en el resto de entradas. Es decir si $A_{11},A_{12},A_{21},A_{22}\in M_{n\times n}(K)$ son tales que
\[
A=\left[\begin{array}{cc} A_{11} & A_{12}\\A_{21} & A_{22}\end{array}\right],
\]
entonces
\[
A^\sigma=\left[\begin{array}{cc} A_{22}^\intercal & -A_{12}^\intercal\\-A_{21}^\intercal & A_{11}^\intercal\end{array}\right].
\]
Decimos que $A$ es \emph{auto-adjunta simpl\'ectica} si $A^\sigma=A$. Es decir si $A_{11}=A_{22}^\intercal$, $A_{12}=-A_{12}^\intercal$ y $A_{21}=-A_{12}^\intercal$.
\end{defn}


\begin{prop}\label{adjtrassim}
Suponga que $V$ tiene dimensi\'on finita, entonces existe un \'unico operador $g\in\Hom_{K}(V,V)$ adjunto de $f$. M\'as a\'un, si $T=\{v_1,\ldots,v_n,w_1,\ldots,w_n\}$ es una base de Darboux de $V$, entonces
\[
\Big[g\Big]^T_T=\left(\Big[f\Big]^T_T\right)^\sigma
\]
\end{prop}

\dem Defina el operador $g\in\Hom_{K}(V,V)$ por la imagen de la base $T$:
\begin{eqnarray*}
g(v_j) & = & \sum_{i=1}^n\sigma(f(w_i),v_j)v_i-\sigma(f(v_i),v_j)w_i,\\
g(w_j) & = & \sum_{i=1}^n\sigma(f(w_i),w_j)v_i-\sigma(f(v_i),w_j)w_i.
\end{eqnarray*}
De esta forma
\begin{eqnarray*}
\sigma\left(v_i,g(v_j)\right) & = & \sigma\left(f(v_i),v_j\right)\\
\sigma\left(v_i,g(w_j)\right) & = & \sigma\left(f(v_i),w_j\right)\\
\sigma\left(w_i,g(v_j)\right) & = &  \sigma\left(f(w_i),v_j\right)\\
 \sigma\left(w_i,g(w_j)\right) & = & \sigma\left(f(w_i),w_j\right)
\end{eqnarray*}
y por Propiedad \ref{coorsimp}
\begin{eqnarray*}
\sigma\left(v,g(w)\right) & = & \sum_{i=1}^n \sigma\left(v_i,g(w)\right)\sigma\left(w_i,v\right)-\sigma\left(v_i,v\right)\sigma\left(w_i,g(w)\right)\\
 & = & \sum_{i,j=1}^n \Big(\sigma\left(w_j,w\right)\sigma\left(v_i,g(v_j)\right)-\sigma\left(v_j,w\right)\sigma\left(v_i,g(w_j)\right)\Big)\sigma\left(w_i,v\right)\\
 & & \quad-\sigma\left(v_i,v\right)\Big(\sigma\left(w_j,w\right)\sigma\left(w_i,g(v_j)\right)-\sigma\left(v_j,w\right)\sigma\left(w_i,g(w_j)\right)\Big)\\
  & = & \sum_{i,j=1}^n \Big(\sigma\left(w_j,w\right)\sigma\left(f(v_i),v_j\right)-\sigma\left(v_j,w\right)\sigma\left(f(v_i),w_j\right)\Big)\sigma\left(w_i,v\right)\\
 & & \quad-\sigma\left(v_i,v\right)\Big(\sigma\left(w_j,w\right)\sigma\left(f(w_i),v_j\right)-\sigma\left(v_j,w\right)\sigma\left(f(w_i),w_j\right)\Big)\\
  & = & \sum_{i,j=1}^n \sigma\left(v_j,w\right)\Big(\sigma\left(v_i,v\right)\sigma\left(f(w_i),w_j\right)-\sigma\left(w_i,v\right)\sigma\left(f(v_i),w_j\right)\Big)\\
 & & \quad-\Big(\sigma\left(v_i,v\right)\sigma\left(f(w_i),v_j\right)-\sigma\left(w_i,v\right)\sigma\left(f(v_i),v_j\right)\Big)\sigma\left(w_j,w\right)\\
  & = & \sum_{i,j=1}^n \sigma\left(v_j,w\right)\Big(\sigma\left(w_i,v\right)\sigma\left(w_j,f(v_i)\right)-\sigma\left(v_i,v\right)\sigma\left(w_j,f(w_i)\right)\Big)\\
 & & \quad-\Big(\sigma\left(w_i,v\right)\sigma\left(v_j,f(v_i)\right)-\sigma\left(v_i,v\right)\sigma\left(v_j,f(w_i)\right)\Big)\sigma\left(w_j,w\right)\\
 & = & \sum_{j=1}^n \sigma\left(v_j,w\right)\sigma\left(w_j,f(v)\right)-\sigma\left(v_j,f(v)\right)\sigma\left(w_j,w\right)\\
 & = & \sigma\left(f(v),w\right)
\end{eqnarray*}
Por otro lado si, $h\in\Hom_\mathbb{K}(V,V)$ es adjunto de $f$, por Propiedad \ref{coorortonorher},
\begin{eqnarray*}
h(v_j) & = & \sum_{i=1}^n \sigma\left(w_i,h(v_j)\right)v_i-\sigma\left(v_i,h(v_j)\right)w_i\\
         & = & \sum_{i=1}^n \sigma\left(f(w_i),v_j\right)v_i-\sigma\left(f(v_i),v_j\right)w_i\\
         & = & g(u_j),\\
h(w_j) & = & \sum_{i=1}^n \sigma\left(w_i,h(w_j)\right)v_i-\sigma\left(v_i,h(w_j)\right)w_i\\
         & = & \sum_{i=1}^n \sigma\left(f(w_i),w_j\right)v_i-\sigma\left(f(v_i),w_j\right)w_i\\
         & = & g(w_j).       
\end{eqnarray*}
luego $h=g$.\\
Ahora, para ver que la representaci\'on matricial de $g$ respecto a $T$ es la adjunta simpl\'ectica de la de $f$ basta observar que para $i,j=1,\ldots,n$
\begin{eqnarray*}
\Big[g\Big]^T_{T,(i,j)} & = & \Big[g(v_j)\Big]^T_i\\
  & = & \sigma\left(w_i,g(v_j)\right)\\
  & = & \sigma\left(f(w_i),v_j\right)\\
  & = & -\sigma\left(v_j,f(w_i)\right)\\
  & = & \Big[f(w_i)\Big]^T_{n+j}\\
  & = & \Big[f\Big]^T_{T,(n+j,n+i)},\\
\Big[g\Big]^T_{T,(n+i,n+j)} & = & \Big[g(w_j)\Big]^T_{n+i}\\
  & = & -\sigma\left(v_i,g(w_j)\right)\\
  & = & -\sigma\left(f(v_i),w_j\right)\\
  & = & \sigma\left(w_j,f(v_i)\right)\\
  & = & \Big[f(v_i)\Big]^T_{j}\\
  & = & \Big[f\Big]^T_{T,(j,i)},\\
\Big[g\Big]^T_{T,(n+i,j)} & = & \Big[g(v_j)\Big]^T_{n+i}\\
  & = & -\sigma\left(v_i,g(v_j)\right)\\
  & = & -\sigma\left(f(v_i),v_j\right)\\
  & = & \sigma\left(v_j,f(v_i)\right)\\
  & = & -\Big[f(v_i)\Big]^T_{n+j}\\
  & = & -\Big[f\Big]^T_{T,(n+j,i)},\\
\Big[g\Big]^T_{T,(i,n+j)} & = & \Big[g(w_j)\Big]^T_i\\
  & = & \sigma\left(w_i,g(w_j)\right)\\
  & = & \sigma\left(f(w_i),w_j\right)\\
  & = & -\sigma\left(w_j,f(w_i)\right)\\
  & = & -\Big[f(w_i)\Big]^T_{j}\\
  & = & -\Big[f\Big]^T_{T,(j,n+i)}.
\end{eqnarray*}
\qed

\begin{nota}
Si $V$ tiene dimensi\'on finita, a la adjunta de $f$ la denotaremos por $f^*$.
\end{nota}

\begin{obs}
Note que si $V$ tiene dimensi\'on finita, para todo $v,w\in V$,
\begin{eqnarray*}
f^*\left(s(v))(w)\right) & = & s(v)\left(f(w)\right)\\
 & = & \sigma(v,f(w))\\
 & = & \sigma(f^*(v),w)\\
 & = & s\left(f^*(v)\right)(w)
\end{eqnarray*}
luego
\[
f^*\circ s=s\circ f^*
\]
donde a la izquierda en la igualdad tenemos el dual y a la derecha el adjunto.
\end{obs}

\begin{obs}
Si $V$ tiene dimensi\'on finita $f^*\circ f$ es auto-adjunta, de hecha para todo $v,w\in V$
\[
\sigma(v,f^*\circ f(w))=\sigma(f(v),f(w))=\sigma(f^*\circ f(v),w).
\]
\end{obs}

\begin{prop}
Si $V$ tiene dimensi\'on finita, las siguientes propiedades son equivalentes:
\begin{enumerate}
\item $f$ es auto-adjunta; y,
\item la representaci\'on matricial de $f$ respecto a una base de Darboux es auto-adjunta sympl\'ectica.
\end{enumerate}
\end{prop}

\dem Proposi\'on \ref{adjtrassim} implica que si $f$ es auto-adjunta, su representaci\'on matricial respecto a una base de Darboux es auto-adjunta sympl\'ectica. Para establecer el converso, tomamos una base de Darboux $T=\{v_1,\ldots,v_n,w_1,\ldots,w_n\}$ asumimos que $\Big[f\Big]^T_T$ es auto-adjunta simpl\'ectica, es decir para todo $i,j\in\{1,\ldots,n\}$
\begin{eqnarray*}
\sigma\left(w_i,f(v_j)\right) & = & \Big[f\Big]^T_{T,(i,j)}\\
  & = & \Big[f\Big]^T_{T,(n+j,n+i)}\\
  & = & -\sigma\left(v_j,f(w_i)\right)\\
-\sigma\left(v_i,f(w_j)\right) & = & \Big[f\Big]^T_{T,(n+i,n+j)}\\
  & = & \Big[f\Big]^T_{T,(j,i)}\\
  & = & \sigma\left(w_j,f(v_i)\right)\\
-\sigma\left(v_i,f(v_j)\right) & = & \Big[f\Big]^T_{T,(n+i,j)}\\
  & = & -\Big[f\Big]^T_{T,(n+j,i)}\\
  & = & \sigma\left(v_j,f(v_i)\right)\\
\sigma\left(w_i,f(w_j)\right) & = & \Big[f\Big]^T_{T,(i,n+j)}\\
  & = & -\Big[f\Big]^T_{T,(j,n+i)}\\
  & = & -\sigma\left(w_j,f(w_i)\right),
\end{eqnarray*}
en particular
\begin{eqnarray*}
\sigma\left(v_i,f(v_j)\right) & = & \sigma\left(f(v_i),v_j\right)\\
\sigma\left(v_i,f(w_j)\right) & = & \sigma\left(f(v_i),w_j\right)\\
\sigma\left(w_i,f(v_j)\right) & = & \sigma\left(f(w_i),v_j\right)\\
\sigma\left(w_i,f(w_j)\right) & = & \sigma\left(f(w_i),w_j\right).
\end{eqnarray*}
As\'i, por la demostraci\'on de Proposici\'on \ref{adjtrassim}, est\'as igualdades implican que el adjunto de $f$ es \'el mismo.\qed

\begin{ejem}\label{ejemautoadj}
Suponga que $V=U\times U^*$ y
\[
\sigma\left((v,\lambda),(w,\mu)\right)=\lambda(w)-\mu(v).
\]
Sea ahora $g\in\Hom_K(U,U)$ y tome
\[
f(v,\lambda)=(g(v),g^*(\lambda))
\]
de forma que
\begin{eqnarray*}
\sigma\left((v,\lambda),f(w,\mu)\right) & = & \lambda(g(w))-g^*(\mu)(v)\\
 & = & g^*(\lambda)(w)-\mu(g(v))\\
 & = & \sigma\left(f(v,\lambda),(w,\mu)\right),
\end{eqnarray*}
luego $f$ es auto-adjunto.
\end{ejem}

\begin{obs}
El operador del ejemplo anterior es de hecho la forma m\'as general de operador auto-adjunto sobre un espacio simpl\'ectico. Es decir, dado un operador auto-adjunto, existe una descomposici\'on del espacio en subespacios invariantes compatibles con el operador tales que este toma la forma como el operador $f$ en Ejemplo \ref{ejemautoadj}. El resto de este cap\'itulo tiene como objetivo establecer ese resultado para el caso en el que el polinomio caracter\'istico del operador se factoriza en factores lineales en $K$.
\end{obs}

\begin{lema}
Suponga que $V$ tiene dimensi\'on finita y que $f$ es auto-adjunto, entonces:
\begin{enumerate}
\item si $f$ es una proyecci\'on, es decir $f^2=f$, entonces $f(V)$ es un subespacio simpl\'ectico;
\item para todo $P(t)\in K[t]$, $P(f)$ es auto-adjunto.
\end{enumerate}
\end{lema}

\dem 
\begin{enumerate}
\item Como $f$ es una proyecci\'on $V=f(V)\oplus\ker(f)$, y para $v\in V$
\[
v=f(v)+(v-f(v))
\]
con $f(v)\in f(V)$ y $v-f(v)\in\ker(f)$. Para probar el lema basta con establecer que $\ker(f)=f(V)^\sigma$, pues en tal caso, como $f(V)\cap\ker(f)=\{0\}$ tendr\'iamos $f(V)\cap f(V)^\sigma=\{0\}$, y la conclusi\'on se sigue de Proposici\'on \ref{propsimisolan}.1. Ahora, dado $v\in\ker(f)$, para todo $w\in V$,
\[
\sigma(f(w),v)=\sigma(w,f(v))=0
\]
luego $v\in f(V)^\sigma$, y as\'i $\ker(f)\subseteq f(V)^\sigma$. Pero $\dim(\ker(f))=V-\dim(f(V))=\dim(f(V)^\sigma)$, entonces $\ker(f)=f(V)^\sigma$.
\item Si $P(t)=\sum_{i=0}^da_it^i$, para todo $v,w\in V$ tenemos
\begin{eqnarray*}
\sigma\big(v,P(f)(w)\big) & = & \sum_{i=0}^da_i\sigma\big(v,f^i(w)\big)\\
  & = & \sum_{i=1}^da_i\sigma\big(f^i(v),w\big)\\
  & = & \sigma\big(P(f)(v),w\big).
\end{eqnarray*}\qed
\end{enumerate}

\begin{prop}
Suponga que $V$ tiene dimensi\'on finita, que $f$ es auto-adjunto y que
\[
P_f(t)=(t-\lambda_1)^{m_1}(t-\lambda_2)^{m_2}\ldots(t-\lambda_r)^{m_r}, \quad \lambda_1,\lambda_2,\ldots,\lambda_r\in \mathbb{K}.
\]
con $\lambda_i\ne \lambda_j$ si $i\ne j$. Entonces para $i=1,\ldots,r$, $V_i=\ker(P_i(f)^{m_i})$ es un subespacio simpl\'ectico invariante bajo $f$, y
\[
V=V_1\oplus\ldots\oplus V_r.
\]
\end{prop}

\dem La descomposici\'on $V=V_1\oplus\ldots\oplus V_r$ como suma directa de subespacios invariantes bajo $f$ es consecuencia directa de Propiedad \ref{prodescomp}, y al considerar tambi\'en la afirmaci\'on 2. del lema anterior obtenemos que la proyecci\'on sobre cada uno de estos subespacios es auto-adjunta. La primera afirmaci\'on del mismo lema implica que cada uno de estos $V_i$, $i=1,\ldots,r$, es un subespacio simpl\'ectico.\qed


\chapter{\'Algebra Multilineal}

Sea $K$ un cuerpo y $V, W$ espacios vectoriales sobre $K$.

\begin{nota}
Dado $k\in\mathbb{Z}_{> 0}$ denotamos
$$V^k=\underbrace{V\times\cdots\times V}_{k-\text{veces}}$$
y para $k=0$ usamos la convenci\'on $V^0=K$.
\end{nota}

\section{Tensores}

\begin{defn}
Sea $k\in\mathbb{Z}_{\ge 0}$ y $T:V^k\longrightarrow K$ una funci\'on. Decimos que $T$ es \emph{multilineal} si para $i=1\ldots k$ tenemos
$$ T(v_1,\ldots,c_iv_i+c_i'v_i',\ldots,v_k)=c_iT(v_1,\ldots,v_i,\ldots,v_k)+c_i'T(v_1,\ldots,v_i',\ldots,v_k)$$
para todo $v_1,\ldots,v_i,v_i',\ldots,v_k\in V$ y $c_i,c_i'\in K$. Si $T$ es multilineal decimos que $T$ es un \emph{$k$-tensor}. Al conjunto de $k$-tensores lo denotamos por $T^k(V)$.
\end{defn}

\begin{obs}
Para todo $k\in\mathbb{Z}_{\ge 0}$ el conjunto $T^k(V)$ es un espacio vectorial sobre $K$ bajo las operaciones:
$$
\begin{array}{rccl}
S+T:& V^k &\longrightarrow & K\\
&(v_1,\ldots,v_k) & \longmapsto & S(v_1,\ldots,v_k)+T(v_1,\ldots,v_k)\\
\\
cT:& V^k &\longrightarrow & K\\
&(v_1,\ldots,v_k) & \longmapsto & cT(v_1,\ldots,v_k)\\
\end{array}
$$
para todo $S,T\in T^k(V)$ y $c\in K$.
\end{obs}

\begin{obs}
$T^0(V)\simeq K$ y $T^1(V)=V^*$.
\end{obs}

\begin{ejem} Sea $n\in\mathbb{Z}_{>0}$.
\begin{enumerate}
\item La funci\'on
$$\begin{array}{rccl}
\langle\bullet;\bullet\rangle: & K^n\times K^n &\longrightarrow & K\\
&\Big((x_1,\ldots,x_n),(y_1,\ldots,y_n)\Big) & \longmapsto & \sum_{i=1}^nx_iy_i
\end{array}
$$
define un $2$-tensor.
\item Sea $a_{ij}\in K$, $i,j=1,\ldots,n$. La funci\'on
$$\begin{array}{rccl}
T: & K^n\times K^n&\longrightarrow & K\\
&\Big((x_1,\ldots,x_n),(y_1,\ldots,y_n)\Big) & \longmapsto & \sum_{i,j=1}^nx_ia_{ij}y_j
\end{array}
$$
define un $2$-tensor.
\item La funci\'on
$$\begin{array}{rccl}
\det: & \Big(K^n\Big)^n &\longrightarrow & K\\
&\big(\overline{x}_1,\ldots,\overline{x}_n\big) & \longmapsto & \det(x_{ij})=\sum_{\sigma\in S_n}\sgn(\sigma)x_{\sigma(1)1}\ldots x_{\sigma(n)n}
\end{array}
$$
donde $\overline{x}_j=(x_{1j},\ldots,x_{nj})$ para $j=1,\ldots,n$ define un $n$-tensor.
\end{enumerate}
\end{ejem}

\begin{defn}
Sean $k,l\in\mathbb{Z}_{\ge 0}$. Dados $S\in T^k(V)$ y $T\in T^l(V)$ definimos su \emph{producto tensorial} $S\otimes T\in T^{k+l}(V)$ por
$$ S\otimes T(v_1,\ldots,v_k,v_{k+1},\ldots,v_{k+l})= S(v_1,\ldots,v_k)T(v_{k+1},\ldots,v_{k+l}) $$
para todo $v_1,\ldots,v_{k+l}\in V$.
\end{defn}

\begin{obs}
El producto tensorial no es conmutativo pues no siempre es cierto que $S\otimes T$ y $S\otimes T$ coincidan.
\end{obs}

\begin{pro}\label{ptbya}
El producto tensorial es bilineal y asociativo. Es decir, dados $k,l,m\in\mathbb{Z}_{\ge 0}$ y $S,S'\in T^k(V)$, $T,T'\in T^l(V)$, $U\in T^m(V)$, $c\in K$ tenemos:
\begin{enumerate}
\item $(S+S')\otimes T=S\otimes T + S'\otimes T$,
\item $S\otimes (T+T')=S\otimes T + S\otimes T'$,
\item $(cS)\otimes T=c(S\otimes T)=S\otimes(cT)$,
\item $(S\otimes T)\otimes U=S\otimes (T\otimes U)$.
\end{enumerate} 
\end{pro}

\dem La demostraci\'on es una verificaci\'on directa.\qed

\begin{nota}
Por la Proposici\'on \ref{ptbya} 4. denotamos $S\otimes T\otimes U= (S\otimes T)\otimes U$ y as\'i podemos definir producto tensorial de m\'as de dos tensores.
\end{nota}

\begin{teo}\label{bt}
Suponga que $\dim_K(V)=n<\infty$. Sea $\{v_1,\ldots,v_n\}$ una base de $V$ y $\{\lambda_1,\ldots,\lambda_n\}$ la base dual. Para todo $k\in\mathbb{Z}_{>0}$, la colecci\'on de $k$-tensores
$$\Big\{\lambda_{i_1}\otimes\cdots\otimes\lambda_{i_k}\Big\}_{i_1,\ldots,i_k=1}^n$$
es una base de $T^k(V)$. En particular $$\dim_K(T^k)=n^k.$$
\end{teo}

\dem Veamos primero que $$T^k(V)=\langle \lambda_{i_1}\otimes\cdots\otimes\lambda_{i_k} \rangle_{i_1,\ldots,i_k=1}^n.$$ Note que para todo $i_1\ldots,i_k,j_1,\ldots,j_k\in\{1,\ldots,n\}$
$$\lambda_{i_1}\otimes\cdots\otimes\lambda_{i_k}(v_{j_1},\ldots,v_{j_k})= \delta_{i_1j_1}\cdots\delta_{i_kj_k}=\left\{\begin{array}{rl}
1 & \text{si } i_1=j_1,\ldots, i_k=j_k \\
0 & \text{si no}
\end{array}\right. .$$
As\'i, si $w_1,\dots,w_k\in V$ y $w_j=\sum_{i=1}^n a_{ij}v_i$, $a_{ij}\in K$, $j=1,\ldots,k$ entonces
\begin{align*}
\lambda_{i_1}\otimes\cdots\otimes\lambda_{i_k}(w_1,\ldots,w_k) &= \sum_{j_1,\ldots,j_k=1}^n a_{j_11}\cdots a_{j_kk} \lambda_{i_1}\otimes\cdots\otimes\lambda_{i_k}(v_{j_1},\ldots,v_{j_k})\\
&= \sum_{j_1,\ldots,j_k=1}^n a_{j_11}\cdots a_{j_kk}\delta_{i_1j_1}\cdots\delta_{i_kj_k}\\
&= a_{i_11}\cdots a_{i_kk}\\
\end{align*}
Ahora, dado $T\in T^k(V)$
\begin{align*}
T(w_1,\ldots,w_k) &= \sum_{i_1,\ldots,i_k=1}^n a_{i_11}\cdots a_{i_kk} T(v_{i_1},\ldots,v_{i_k})\\
&= \sum_{i_1,\ldots,i_k=1}^n T(v_{i_1},\ldots,v_{i_k})\lambda_{i_1}\otimes\cdots\otimes\lambda_{i_k}(w_1,\ldots,w_k)
\end{align*}
luego
$$T=\sum_{i_1,\ldots,i_k=1}^n T(v_{i_1},\ldots,v_{i_k})\lambda_{i_1}\otimes\cdots\otimes\lambda_{i_k},$$
y $T^k(V)=\langle \lambda_{i_1}\otimes\cdots\otimes\lambda_{i_k} \rangle_{i_1,\ldots,i_k=1}^n$.

Establezcamos ahora la independencia lineal. Suponga que
$$0=\sum_{i_1,\ldots,i_k=1}^n c_{i_1\ldots i_k}\lambda_{i_1}\otimes\cdots\otimes\lambda_{i_k}.$$
Si evaluamos en $(v_{j_1},\ldots,v_{j_k})$ obtenemos
\begin{align*}
0 &=\sum_{i_1,\ldots,i_k=1}^n c_{i_1\ldots i_k}\lambda_{i_1}\otimes\cdots\otimes\lambda_{i_k}(v_{j_1},\ldots,v_{j_k})\\
 &=c_{j_1,\ldots,j_k}.
\end{align*}
\qed

\begin{ejem} Sea $n\in\mathbb{Z}_{>0}$, $\{e_i\}_{i=1}^n$ la base can\'onica de $K^n$ y $\{f_1,\ldots,f_n\}$ la base dual.
\begin{enumerate}
\item Si
$$\begin{array}{rccl}
\langle\bullet;\bullet\rangle: & K^n\times K^n &\longrightarrow & K\\
&\Big((x_1,\ldots,x_n),(y_1,\ldots,y_n)\Big) & \longmapsto & \sum_{i=1}^nx_iy_i
\end{array}
$$
entonces $\langle\bullet;\bullet\rangle=\sum_{i=1}^n f_i\otimes f_i$.
\item Sea $a_{ij}\in K$, $i,j=1,\ldots,n$. Si
$$\begin{array}{rccl}
T: & K^n\times K^n&\longrightarrow & K\\
&\Big((x_1,\ldots,x_n),(y_1,\ldots,y_n)\Big) & \longmapsto & \sum_{i,j=1}^nx_ia_{ij}y_j
\end{array}
$$
entonces $T=\sum_{i,j=1}^na_{ij}f_i\otimes f_j$.
\item Si
$$\begin{array}{rccl}
\det: & \Big(K^n\Big)^n &\longrightarrow & K\\
&\big(\overline{x}_1,\ldots,\overline{x}_n\big) & \longmapsto & \sum_{\sigma\in S_n}\sgn(\sigma)x_{\sigma(1)1}\ldots x_{\sigma(n)n}
\end{array}
$$
donde $\overline{x}_j=(x_{1j},\ldots,x_{nj})$ para $j=1,\ldots,n$ entonces $$\det=\sum_{\sigma\in S_n}\sgn(\sigma)f_{\sigma(1)}\otimes\cdots\otimes f_{\sigma(n)}.$$
\end{enumerate}
\end{ejem}

\begin{defn}
Sea $f\in\Hom_K(V,W)$. Dado $k\in\mathbb{Z}_{\ge 0}$ definimos
\begin{eqnarray*}
f^*: T^k(W) & \longrightarrow & T^k(V)\\
  T & \longmapsto & f^*T: (v_1,\ldots,v_k) \mapsto T\big(f(v_1),\ldots,f(v_k)\big).
\end{eqnarray*}
\end{defn}

\begin{prop}
Sea $f\in\Hom_K(V,W)$ y $k,l\in\mathbb{Z}_{\ge 0}$. Para todo $S\in T^k(W)$ y $T\in T^l(W)$ tenemos
$$f^*(S\otimes T)=f^*S\otimes f^*T.$$
\end{prop}

\dem La demostraci\'on es una verificaci\'on inmediata.\qed

\section{Tensores alternantes}

\begin{defn}
Sea $k\in\mathbb{Z}_{\ge 0}$ y $\omega\in T^k(V)$, decimos que $\omega$ es \emph{alternante} si
$$\omega(v_1,\ldots,v_i,\ldots,v_j,\ldots,v_k)=0,\quad v_1,\ldots,v_k\in V$$
siempre que $v_i=v_j$ para alg\'un par $i,j\in\{1,\ldots,k\}$, $i\ne j$. Denotamos por $\Lambda^k(V)$ al subespacio de $T^k(V)$ de $k$-tensores alternantes.  
\end{defn}

\begin{pro}
Sea $k\in\mathbb{Z}_{\ge 0}$ y $\omega\in T^k(V)$ un $k$-tensor alternante, entonces 
$$\omega(v_1,\ldots,v_i,\ldots,v_j,\ldots,v_k)=-\omega(v_1,\ldots,v_j,\ldots,v_i,\ldots,v_k)$$
para todo $v_1,\ldots,v_k\in V$ y todo $1\le i<j\le k$.
\end{pro}

\dem \begin{align*}
0 =&  \omega(v_1,\ldots,v_i+v_j,\ldots,v_i+v_j,\ldots,v_k)\\
 =& \omega(v_1,\ldots,v_i,\ldots,v_i,\ldots,v_k)+\omega(v_1,\ldots,v_i,\ldots,v_j,\ldots,v_k)\\
 & \quad +\omega(v_1,\ldots,v_j,\ldots,v_i,\ldots,v_k)+\omega(v_1,\ldots,v_j,\ldots,v_j,\ldots,v_k)\\
 = &\omega(v_1,\ldots,v_i,\ldots,v_j,\ldots,v_k)+\omega(v_1,\ldots,v_j,\ldots,v_i,\ldots,v_k)\\
\end{align*}\qed

\begin{obs}
Sea $k\in\mathbb{Z}_{\ge 0}$ y $\omega\in \Lambda^k(V)$, para toda permutaci\'on $\sigma\in S_k$ tenemos
$$\omega(v_{\sigma(1)},\ldots,v_{\sigma(k)})=-\sgn(\sigma)\omega(v_1,\ldots,v_k).$$
Esta \'ultima igualdad es la que justifica el nombre de alternante.
\end{obs}

\begin{obs}
$\Lambda^0(V)=T^0(V)\simeq K$ y $\Lambda^1(V)=T^1(V)=V^*$.
\end{obs}

\begin{ejem} Sea $n\in\mathbb{Z}_{>0}$.
\begin{enumerate}
\item La funci\'on determinante
$$\begin{array}{rccl}
\det: & \Big(K^n\Big)^n &\longrightarrow & K\\
&\big(\overline{x}_1,\ldots,\overline{x}_n\big) & \longmapsto & \sum_{\sigma\in S_n}\sgn(\sigma)x_{\sigma(1)1}\ldots x_{\sigma(n)n}
\end{array}
$$
donde $\overline{x}_j=(x_{1j},\ldots,x_{nj})$ para $j=1,\ldots,n$ define un $n$-tensor alternante.
\item Sea $k\in\{1,\ldots,n\}$ y $1\le i_1<\ldots<i_k\le n$. La funci\'on determinante menor
$$\begin{array}{rccl}
\det_{i_1\ldots i_k}: & \Big(K^n\Big)^k &\longrightarrow & K\\
&\big(\overline{x}_1,\ldots,\overline{x}_k\big) & \longmapsto & \sum_{\sigma\in S_k}\sgn(\sigma)x_{i_{\sigma(1)}1}\ldots x_{i_{\sigma(k)}k}
\end{array}
$$
donde $\overline{x}_j=(x_{1j},\ldots,x_{nj})$ para $j=1,\ldots,n$ define un $k$-tensor alternante.
\end{enumerate}
\end{ejem}

\begin{obs}
Sea $k,l\in\mathbb{Z}_{\ge 0}$ y $\omega\in\Lambda^k(V)$, $\eta\in\Lambda^l(V)$, el $k+l$-tensor $\omega\otimes\eta$ no es necesariamente alternante. Para obtener un tensor alternante hace falta proyectar sobre el subespacio $\Lambda^{k+l}(V)\le T^{k+l}(V)$.
\end{obs}

\begin{defn}
Sea $k\in\mathbb{Z}_{\ge 0}$ tal que si $\chara(K)>0$ entonces $k<\chara(K)$. Definimos $\Alt\in\Hom_K\left(T^k(V),T^k(V)\right)$ por
$$\Alt(T)(v_1,\ldots,v_k)=\dfrac{1}{k!}\sum_{\sigma\in S_k}\sgn(\sigma)T(v_{\sigma(1)},\ldots,v_{\sigma(k)})$$
\end{defn}

\begin{ejem} Sea $n\in\mathbb{Z}_{>0}$, $\{e_i\}_{i=1}^n$ la base can\'onica de $K^n$ y $\{f_1,\ldots,f_n\}$ la base dual.
\begin{enumerate}
\item Para $\chara(K)>0$ suponga que $n<\chara(K)$. Si $T=f_1\otimes\cdots\otimes f_n$, tenemos que para todo $\overline{x}_j=(x_{1j},\ldots,x_{nj})\in K^n$, $j=1,\ldots,n$,
\begin{align*}
\Alt(T)(\overline{x}_1,\ldots,\overline{x}_n) &= \dfrac{1}{n!}\sum_{\sigma\in S_n}\sgn(\sigma)f_1\otimes\cdots\otimes f_n(v_{\sigma(1)},\ldots,v_{\sigma(n)})\\
&= \dfrac{1}{n!}\sum_{\sigma\in S_n}\sgn(\sigma)x_{1\sigma(1)}\cdots x_{n\sigma(n)}\\
&= \dfrac{1}{n!}\sum_{\sigma\in S_n}\sgn(\sigma)x_{\sigma^{-1}(1)1}\cdots x_{\sigma^{-1}(n)n}\\
&= \dfrac{1}{n!}\sum_{\sigma\in S_n}\sgn(\sigma^{-1})x_{\sigma^{-1}(1)1}\cdots x_{\sigma^{-1}(n)n}\\
&= \dfrac{1}{n!}\sum_{\sigma\in S_n}\sgn(\sigma)x_{\sigma(1)1}\cdots x_{\sigma(n)n}\\
&=\dfrac{1}{n!}\det\big(\overline{x}_1,\ldots,\overline{x}_n\big)
\end{align*}
as\'i $\Alt(f_1\otimes\cdots\otimes f_n)=\frac{1}{n!}\det$.
\item Sea $k\in\{1,\ldots,n\}$ y $1\le i_1<\ldots<i_k\le n$. Para $\chara(K)>0$ suponga que $k<\chara(K)$. Si $T=f_{i_1}\otimes\cdots\otimes f_{i_k}$, tenemos que para todo $\overline{x}_j=(x_{1j},\ldots,x_{nj})\in K^n$, $j=1,\ldots,k$,
\begin{align*}
\Alt(T)(\overline{x}_1,\ldots,\overline{x}_k) &= \dfrac{1}{k!}\sum_{\sigma\in S_k}\sgn(\sigma)f_{i_1}\otimes\cdots\otimes f_{i_k}(v_{\sigma(1)},\ldots,v_{\sigma(k)})\\
&= \dfrac{1}{k!}\sum_{\sigma\in S_k}\sgn(\sigma)x_{i_1\sigma(1)}\cdots x_{i_k\sigma(k)}\\
&= \dfrac{1}{k!}\sum_{\sigma\in S_k}\sgn(\sigma)x_{i_{\sigma^{-1}(1)}1}\cdots x_{i_{\sigma^{-1}(k)}k}\\
&= \dfrac{1}{k!}\sum_{\sigma\in S_k}\sgn(\sigma^{-1})x_{i_{\sigma^{-1}(1)}1}\cdots x_{i_{\sigma^{-1}(k)}k}\\
&= \dfrac{1}{k!}\sum_{\sigma\in S_k}\sgn(\sigma)x_{i_{\sigma(1)}1}\cdots x_{i_{\sigma(k)}k}\\
&=\dfrac{1}{k!}\det{}_{i_1\ldots i_k}\big(\overline{x}_1,\ldots,\overline{x}_k\big)
\end{align*}
as\'i $\Alt(f_{i_1}\otimes\cdots\otimes f_{i_k})=\frac{1}{k!}\det_{i_1\ldots i_k}$.
\end{enumerate}
\end{ejem}

\begin{prop} Para todo $k\in\mathbb{Z}_{\ge 0}$ el operador $\Alt$ es una proyecci\'on sobre $\Lambda^k(V)$. Es decir:
\begin{enumerate}
\item para todo $T\in T^k(V)$, $\Alt(T)\in\Lambda^k(V)$,
\item para todo $\omega\in\Lambda^k(V)$, $\Alt(\omega)=\omega$,
\item $\Alt\circ\Alt=\Alt$. 
\end{enumerate}
\end{prop}

\dem \begin{enumerate}
\item Dados $1\le i<j \le k$, defina $\tau\in S_k$ la transposici\'on que intercambia $i$ y $j$ (y deja al resto de elementos en $\{1,\ldots,k\}$ fijos). Sea $A_k$ el subgrupo de $S_k$ formado por las permutaciones con signo $1$, de forma que si $\tau A_k=\{\tau\sigma\in S_k\|\ \sigma\in A_k\}$ entonces $S_k=A_k\cup \tau A_k$ es una partici\'on de $S_k$. Nota que $\sgn(\tau\sigma)=-\sgn(\sigma)$ para todo $\sigma\in S_k$. Sean $v_1,\ldots,v_k\in V$ tales que $v_i=v_j$, entonces para todo $\sigma\in S_k$ tenemos
$$(v_{\sigma(1)},\ldots,v_{\sigma(k)})=(v_{\tau\sigma(1)},\ldots,v_{\tau\sigma(k)})$$
y as\'i
\begin{IEEEeqnarray*}{rCl}
  \IEEEeqnarraymulticol{3}{l}{
\Alt(T)(v_1,\ldots,v_k)}\\
 &=& \dfrac{1}{k!}\sum_{\sigma\in S_k}\sgn(\sigma)T(v_{\sigma(1)}\ldots,v_{\sigma(k)})\\
 &=& \dfrac{1}{k!}\left(\sum_{\sigma\in A_k}\sgn(\sigma)T(v_{\sigma(1)},\ldots,v_{\sigma(k)})+\sum_{\sigma\in \tau A_k}\sgn(\sigma)T(v_{\sigma(1)},\ldots,v_{\sigma(k)})\right)\\
 &=&\dfrac{1}{k!}\left(\sum_{\sigma\in A_k}\sgn(\sigma)T(v_{\sigma(1)},\ldots,v_{\sigma(k)})+\sum_{\sigma\in A_k}\sgn(\tau\sigma)T(v_{\tau\sigma(1)},\ldots,v_{\tau\sigma(k)})\right)\\
 &=&\dfrac{1}{k!}\left(\sum_{\sigma\in A_k}\sgn(\sigma)T(v_{\sigma(1)},\ldots,v_{\sigma(k)})-\sum_{\sigma\in A_k}\sgn(\sigma)T(v_{\tau\sigma(1)},\ldots,v_{\tau\sigma(k)})\right)\\
 &=& 0
\end{IEEEeqnarray*}
luego $\Alt(T)\in\Lambda^k(V)$.
\item Sea $\omega\in\Lambda^k(V)$, entonces para todo $v_1,\ldots,v_k\in V$ tenemos que
\begin{align*}
\Alt(\omega)(v_1,\ldots,v_k) &=\dfrac{1}{k!}\sum_{\sigma\in S_k}\sgn(\sigma)\omega(v_{\sigma(1)},\ldots,v_{\sigma(k)})\\
&=\dfrac{1}{k!}\sum_{\sigma\in S_k}\sgn(\sigma)\sgn(\sigma)\omega(v_1,\ldots,v_k)\\
&=\dfrac{1}{k!}\sum_{\sigma\in S_k}\omega(v_1,\ldots,v_k)\\
&=\dfrac{1}{k!}k!\ \omega(v_1,\ldots,v_k)\\
&=\omega(v_1,\ldots,v_k)
\end{align*}
luego $\Alt(\omega)=\omega$.
\item Se sigue inmediatamente de 1. y 2.
\end{enumerate}\qed

\begin{defn}
Sea $k,l\in\mathbb{Z}_{\ge 0}$ tal que si $\chara(K)>0$ entonces $k+l<\chara(K)$. Sea $\omega\in\Lambda^k{V}$ y $\eta\in\Lambda^l(V)$, definimos el \emph{producto exterior} $\omega\wedge\eta\in\Lambda^{k+l}(V)$ por
$$\omega\wedge\eta=\dfrac{(k+l)!}{k!\ l!}\Alt(\omega\otimes\eta).$$
\end{defn}

\begin{lema}\label{lemaalt}
Sea $k,l,m\in\mathbb{Z}_{\ge 0}$  tal que si $\chara(K)>0$ entonces $k+l+m<\chara(K)$.  Sea $S\in T^k(V)$, $T\in T^l(V)$ y $U\in T^m(V)$ tenemos que
\begin{enumerate}
\item si $\Alt(S)=0$ entonces $\Alt(S\otimes T)=0=\Alt(T\otimes S)$,
\item $\Alt(\Alt(S\otimes T)\otimes U)=\Alt(S\otimes T\otimes U)=\Alt(S\otimes \Alt(T\otimes U))$
\end{enumerate}
\end{lema}

\dem
\begin{enumerate}
\item Tome $S_k$ como el subgrupo de $S_{k+l}$ formado por la permutaciones que dejan fijos a $k+1,\ldots,k+l$ y sean $\sigma_1,\ldots,\sigma_{N}\in S_{k+l}$, $N=(k+l)!/(k!)$, representantes de los coconjuntos $S_k\sigma=\{\tau\sigma\in S_{k+l}|\ \tau\in S_{k}\}$, $\sigma\in S_{k+l}$, de forma que 
$$ S_{k+l}=\cup_{i=1}^N \sigma_iS_k $$
es una partici\'on de $S_{k+l}$. Entonces para todo $v_1,\ldots,v_{k+l}\in V$
\begin{IEEEeqnarray*}{rCl}
  \IEEEeqnarraymulticol{3}{l}{
\Alt (S\otimes T)(v_1,\ldots,v_{k+l})} \\
& = & \dfrac{1}{(k+l)!}\sum_{\sigma\in S_{k+l}} \sgn(\sigma) S\otimes T(v_{\sigma(1)},\ldots,v_{\sigma(k+l)})\\
& = & \dfrac{1}{(k+l)!}\sum_{i=1}^N\sum_{\tau\in S_{k}} \sgn(\tau\sigma_i) S\otimes T(v_{\tau\sigma_i(1)},\ldots,v_{\tau\sigma_i(k+l)})\\
& = & \dfrac{1}{(k+l)!}\sum_{i=1}^N\sgn(\sigma_i)\sum_{\tau\in S_{k}} \sgn(\tau) S(v_{\tau\sigma_i(1)},\ldots,v_{\tau\sigma_i(k)}) T(v_{\tau\sigma_i(k+1)},\ldots,v_{\tau\sigma_i(k+l)}).
\end{IEEEeqnarray*}
Para $i=1,\ldots,N$ y $j=1,\ldots,k+l$, denote $w^{(i)}_j=v_{\sigma_i(j)}$, as\'i, como $\Alt=0$ entonces
\begin{IEEEeqnarray*}{rCl}
  \IEEEeqnarraymulticol{3}{l}{
\Alt (S\otimes T)(v_1,\ldots,v_{k+l})} \\
& = & \dfrac{1}{(k+l)!}\sum_{i=1}^N\sgn(\sigma_i)\sum_{\tau\in S_{k}} \sgn(\tau) S(w^{(i)}_{\tau(1)},\ldots,w^{(i)}_{\tau(k)}) T(w^{(i)}_{\tau(k+1)},\ldots,w^{(i)}_{\tau(k+l)})\\
& = & \dfrac{1}{(k+l)!}\sum_{i=1}^N\sgn(\sigma_i)\sum_{\tau\in S_{k}} \sgn(\tau) S(w^{(i)}_{\tau(1)},\ldots,w^{(i)}_{\tau(k)}) T(w^{(i)}_{k+1},\ldots,w^{(i)}_{k+l})\\
& = & \dfrac{1}{(k+l)!}\sum_{i=1}^N\sgn(\sigma_i)\left(\sum_{\tau\in S_{k}} \sgn(\tau) S(w^{(i)}_{\tau(1)},\ldots,w^{(i)}_{\tau(k)})\right) T(w^{(i)}_{k+1},\ldots,w^{(i)}_{k+l})\\
& = & \dfrac{k!}{(k+l)!}\sum_{i=1}^N\sgn(\sigma_i)\Alt(S)(w^{(i)}_1,\ldots,w^{(i)}_k)T(w^{(i)}_{k+1},\ldots,w^{(i)}_{k+l})\\
& = & 0
\end{IEEEeqnarray*}
Similarmente, si tomamos $S_k$ como el subgrupo de $S_{k+l}$ formado por la permutaciones que dejan fijos a $1,\ldots,l$, obtenemos $\Alt(T\otimes S)=0$.
\item $\Alt\left(\Alt(S\otimes T)-S\otimes T\right)=\Alt(S\otimes T)-\Alt(S\otimes T)=0$, luego por 1.
\begin{align*}
0 &= \Alt\left(\left(\Alt(S\otimes T)-S\otimes T\right)\otimes U\right)\\
  &= \Alt\left(\Alt(S\otimes T)\otimes U\right)-\Alt(S\otimes T\otimes U)
\end{align*}
y as\'i $\Alt\left(\Alt(S\otimes T)\otimes U\right)=\Alt(S\otimes T\otimes U)$. Similarmente obtenemos $\Alt(S\otimes \Alt(T\otimes U))=\Alt(S\otimes T\otimes U)$.
\end{enumerate}\qed


\begin{pro}\label{pebya}
El producto exterior es bilineal, asociativo y  anticonmutativo. Es decir, dados $k,l,m\in\mathbb{Z}_{\ge 0}$ tal que si $\chara(K)>0$ entonces $k+l+m<\chara(K)$ y $\omega,\omega'\in\Lambda^k(V)$, $\eta,\eta'\in\Lambda^l(V)$, $\theta\in\Lambda^m(V)$, $c\in K$ tenemos:
\begin{enumerate}
\item $(\omega+\omega')\wedge\eta=\omega\wedge\eta+\omega'\wedge\eta$,
\item $\omega\wedge(\eta+\eta')=\omega\wedge\eta+\omega\wedge\eta'$,
\item $(c\omega)\wedge\eta=c(\omega\wedge\eta)=\omega\wedge(c\eta)$,
\item $\omega\wedge\eta=(-1)^{kl}\eta\wedge\omega$,
\item $(\omega\wedge\eta)\wedge\theta=\omega\wedge(\eta\wedge\theta)=\frac{(k+l+m)!}{k!\ l!\ m!}\Alt(\omega\otimes\eta\otimes\theta)$.
\end{enumerate}
\end{pro}

\dem Las propiedades 1., 2. y 3. se siguen inmediatamente de la bilinearidad de $\otimes$ y de la linearidad de $\Alt$.
\begin{enumerate} \setcounter{enumi}{3}
\item Sea $\tau\in S_{k+l}$ definida por
$$\tau(i)=\left\{\begin{array}{rl}
i+k & \text{ si } 1\le i\le l\\
i-l & \text{ si } l+1\le i\le l+k
\end{array}\right.$$
Como $\sgn(\tau)=(-1)^{kl}$ y $\sgn(\sigma\tau)=(-1)^{kl}\sgn(\sigma)$ para toda $\sigma\in S_{k+l}$ entonces para todo $v_1,\ldots,v_{k+l}\in V$
\begin{IEEEeqnarray*}{rCl}
  \IEEEeqnarraymulticol{3}{l}{
\omega\wedge\eta(v_1,\ldots,v_{k+l})}\\
 &=&\dfrac{(k+l)!}{k!\ l!} \Alt(\omega\otimes\eta)(v_1,\ldots,v_{k+l})\\
 &=& \dfrac{1}{k!\ l!}\sum_{\sigma\in S_{k+l}} \sgn(\sigma)\omega\otimes\eta(v_{\sigma(1)},\ldots,v_{\sigma(k)},v_{\sigma(k+1)},\ldots,v_{\sigma(k+l)})\\
 &=& \dfrac{(1}{k!\ l!}\sum_{\sigma\in S_{k+l}} \sgn(\sigma)\omega(v_{\sigma(1)},\ldots,v_{\sigma(k)})\eta(v_{\sigma(k+1)},\ldots,v_{\sigma(k+l)})\\
 &=& \dfrac{1}{k!\ l!}\sum_{\sigma\in S_{k+l}} \sgn(\sigma)\omega(v_{\sigma\tau(l+1)},\ldots,v_{\sigma\tau(l+k)})\eta(v_{\sigma\tau(1)},\ldots,v_{\sigma\tau(l)})\\
 &=& \dfrac{1}{k!\ l!}\sum_{\sigma\in S_{k+l}} (-1)^{kl}\sgn(\sigma\tau)\eta(v_{\sigma\tau(1)},\ldots,v_{\sigma\tau(l)})\omega(v_{\sigma\tau(l+1)},\ldots,v_{\sigma\tau(l+k)})\\
 &=& (-1)^{kl}\dfrac{1}{k!\ l!}\sum_{\sigma\tau\in S_{k+l}} \sgn(\sigma\tau)\eta(v_{\sigma\tau(1)},\ldots,v_{\sigma\tau(l)})\omega(v_{\sigma\tau(l+1)},\ldots,v_{\sigma\tau(l+k)})\\
 &=& (-1)^{kl}\dfrac{1}{k!\ l!}\sum_{\sigma\tau\in S_{k+l}} \sgn(\sigma\tau)\eta\otimes\omega(v_{\sigma\tau(1)},\ldots,v_{\sigma\tau(k)},v_{\sigma\tau(k+1)},\ldots,v_{\sigma\tau(k+l)})\\
 &=&(-1)^{kl}\dfrac{(k+l)!}{k!\ l!} \Alt(\eta\otimes\omega)(v_1,\ldots,v_{k+l})\\
 &=&(-1)^{kl}\eta\wedge\omega(v_1,\ldots,v_{k+l}).
\end{IEEEeqnarray*}
as\'i $\omega\wedge\eta=(-1)^{kl}\eta\wedge\omega$.

\item Por Lema \ref{lemaalt} 
\begin{eqnarray*}
(\omega\wedge\eta)\wedge\theta & = & \dfrac{(k+l+m)!}{(k+l)!\ m!}\Alt\left((\omega\wedge\eta)\otimes\theta\right)\\
 & = & \dfrac{(k+l+m)!}{(k+l)!\ m!}\Alt\left(\dfrac{(k+l)!}{k!\ l!}\Alt(\omega\otimes\eta)\otimes\theta\right)\\
 & = & \frac{(k+l+m)!}{k!\ l!\ m!}\Alt(\omega\otimes\eta\otimes\theta)
\end{eqnarray*}
Similarmente obtenemos $\omega\wedge(\eta\wedge\theta)=\frac{(k+l+m)!}{k!\ l!\ m!}\Alt(\omega\otimes\eta\otimes\theta)$.
\end{enumerate}\qed

\begin{obs}\label{pe0}
Suponga que $\chara(K)\ne 2$ y que $\lambda$ es un $1$-tensor, entonces por Propiedad \ref{pebya} 4. $$\lambda\wedge\lambda=0.$$
De hecho $\lambda\wedge\lambda=-\lambda\wedge\lambda$ luego $2\lambda\wedge\lambda=0$ y como $2\ne 0$ tenemos $\lambda\wedge\lambda=0$. Adem\'as si $\lambda_1,\lambda_2$ son $1$-tensores entonces por definici\'on
$$\lambda_1\wedge\lambda_2=\lambda_1\otimes\lambda_2-\lambda_2\otimes\lambda_1,$$
y en general para $\lambda_1,\ldots,\lambda_k\in T^1(V)$, $k\in\mathbb{Z}_{\ge 0}$ tal que si $\chara(K)>0$ entonces $k<\chara(K)$,
$$\lambda_1\wedge\ldots\wedge\lambda_k=\sum_{\sigma\in S_k} \sgn(\sigma)\lambda_{\sigma(1)}\otimes\cdots\otimes\lambda_{\sigma(k)}.$$
\end{obs}

\begin{prop}
Sea $f\in\Hom_K(V,W)$ y $k,l\in\mathbb{Z}_{\ge 0}$. Para todo $\omega\in \Lambda^k(W)$ y $\eta\in \Lambda^l(W)$ tenemos
$$f^*(\omega\wedge \eta)=f^*\omega\wedge f^*\eta.$$
\end{prop}

\dem La demostraci\'on es una verificaci\'on inmediata.\qed

\begin{nota}
Por Propiedad \ref{pebya} 5. denotamos $(\omega\wedge\eta)\wedge\theta=\omega\wedge\eta\wedge\theta$ y as\'i podemos definir producto exterior de m\'as de dos tensores alternantes.
\end{nota}

\begin{coro}
Sean $k_1,\ldots,k_r\in\mathbb{Z}_{\ge 0}$ tal que si $\chara(K)>0$ entonces $k_1+\ldots+k_r<\chara(K)$ y $\omega_i\in\Lambda^{k_i}(V)$, $i=1,\ldots,r$, entonces
$$\omega_1\wedge\cdots\wedge\omega_r=\dfrac{(k_1+\ldots+k_r)!}{k_1!\cdots k_r!}\Alt(\omega_1\otimes\cdots\otimes\omega_r).$$
\end{coro}

\dem Se sigue de inmediatamente de Propiedad \ref{pebya} 5. por inducci\'on en $r$.\qed

\begin{ejem}\label{peej} Sea $n\in\mathbb{Z}_{>0}$, $\{e_i\}_{i=1}^n$ la base can\'onica de $K^n$ y $\{f_1,\ldots,f_n\}$ la base dual.
\begin{enumerate}
\item Para $\chara(K)>0$ suponga que $n<\chara(K)$. Como $\Alt(f_1\otimes\cdots\otimes f_n)=\frac{1}{n!}\det$ entonces
$$f_1\wedge\cdots\wedge f_n=\det.$$
\item Sea $k\in\{1,\ldots,n\}$ y $1\le i_1<\ldots<i_k\le n$. Para $\chara(K)>0$ suponga que $k<\chara(K)$. Como $\Alt(f_{i_1}\otimes\cdots\otimes f_{i_k})=\frac{1}{k!}\det_{i_1\ldots i_k}$ entonces
$$f_{i_1}\wedge\cdots\wedge f_{i_k}=\det_{i_1\ldots i_k}.$$
\end{enumerate}
\end{ejem}

\begin{teo}
Suponga que $\chara(K)\ne 2$ y $\dim_K(V)=n<\infty$. Sea $\{v_1,\ldots,v_n\}$ una base de $V$ y $\{\lambda_1,\ldots,\lambda_n\}$ la base dual. Sea $k\in\mathbb{Z}_{>0}$ tal que si $\chara(K)>0$ entonces $k<\chara(K)$. La colecci\'on de $k$-tensores
$$\Big\{\lambda_{i_1}\wedge\cdots\wedge\lambda_{i_k}\Big\}_{1\le i_1<\ldots<i_k\le n}$$
es una base de $\Lambda^k(V)$. En particular $$\dim_K\left(\Lambda^k(V)\right)=\binom{n}{k}.$$
\end{teo}

\dem Sea $k\in\mathbb{Z}_{\ge 0}$ y $\omega\in\Lambda^k(V)$, como $\omega\in T^k(V)$ y $\big\{\lambda_{i_1}\otimes\cdots\otimes\lambda_{i_k}\big\}_{i_1,\ldots,i_k=1}^n$ es un base de $T^k(V)$ entonces
$$\omega=\sum_{j_1,\ldots,j_k=1}^n c_{j_1\ldots j_k}\lambda_{j_1}\otimes\cdots\otimes\lambda_{j_k}$$
con $c_{j_1\ldots j_k}\in K$. As\'i
\begin{eqnarray*}
\omega & = & \Alt(\omega)\\
 & = & \sum_{j_1,\ldots,j_k=1}^n c_{j_1\ldots j_k}\Alt(\lambda_{j_1}\otimes\cdots\otimes\lambda_{j_k})\\
 & = & \sum_{j_1,\ldots,j_k=1}^n c_{j_1\ldots j_k}k!\lambda_{j_1}\wedge\cdots\wedge\lambda_{j_k}.\\
\end{eqnarray*}
Ahora, dados $j_1,\ldots,j_k\in\{1,\ldots,n\}$ por Propiedad \ref{pebya} 4. y Observaci\'on \ref{pe0}, si dos de los subindices $j_i$ coinciden entonces $\lambda_{j_1}\wedge\cdots\wedge\lambda_{j_k}=0$, luego podemos asumir que son todos distintos; en tal caso $\lambda_{j_1}\wedge\cdots\wedge\lambda_{j_k}=\sgn(\sigma_{j_1,\ldots,j_k})\lambda_{i_1}\wedge\cdots\wedge\lambda_{i_k}$ donde $i_1<\ldots<i_k$ son los mismos subindices $j_1,\ldots,j_k$ ordenados en forma creciente y $\sigma_{j_1,\ldots,j_k}\in S_k$ es la permutaci\'on que reorganiza los $j_1,\ldots,j_k$.
Luego
$$\omega=\sum_{1\le i_1<\ldots<i_k\le n} a_{i_1,\ldots,i_k}\lambda_{i_1}\wedge\cdots\wedge\lambda_{i_k}$$
con $a_{i_1,\ldots,i_k}=k!\sum_{\sigma\in S_k}c_{i_{\sigma(1)}\ldots i_{\sigma(k)}}$, y as\'i  $\Lambda^k(V)=\langle \lambda_{i_1}\wedge\cdots\wedge\lambda_{i_k} \rangle_{1\le i_1<\ldots<i_k\le n}$.

Para establecer la independencia lineal, note que si $1\le i_1<\ldots<i_k\le n$ y $1\le j_1<\ldots<j_k\le n$ entonces (ver Ejemplo \ref{peej} 2.)
$$\lambda_{i_1}\wedge\cdots\wedge\lambda_{i_k}(v_{j_1},\ldots,v_{j_k})=\left\{\begin{array}{rl}
1 & \text{si } i_1=j_1,\ldots, i_k=j_k \\
0 & \text{si no}
\end{array}\right. $$
luego si $0=\sum_{1\le i_1<\ldots<i_k\le n} a_{i_1,\ldots,i_k}\lambda_{i_1}\wedge\cdots\wedge\lambda_{i_k}$ entonces
\begin{align*}
0 &=\sum_{1\le i_1<\ldots<i_k\le n} a_{i_1,\ldots,i_k}\lambda_{i_1}\wedge\cdots\wedge\lambda_{i_k}(v_{j_1},\ldots,v_{j_k})\\
 &=a_{j_1,\ldots,j_k}.
\end{align*}\qed

\section{$(l,k)$-Tensores}

En esta secci\'on asumiremos que $V$ tienen dimensi\'on finita.

\begin{defn}
Sean $l,k\in\mathbb{Z}_{\ge 0}$. Un \emph{$(l,k)$-tensor} es una funci\'on multilineal:
$$T^{(l,k)}: (V^*)^l\times V^k=\underbrace{V^*\times\cdots\times V^*}_{l-\text{veces}}\times\underbrace{V\times\cdots\times V}_{k-\text{veces}}\longrightarrow K.$$
Al espacio de $(l,k)$-tensores lo denotamos $T^{(l,k)}(V)$ \'o $T^l_k(V)$. 
\end{defn}

\begin{nota}
Como $V$ tiene dimensi\'on finita, por Teorema \ref{dualdual} tenemos $V\simeq (V^*)^*$ can\'onicamente, en particular todo $v\in V$ lo identificaremos con el $(1,0)$-tensor
\begin{eqnarray*}
\widehat{v}: V^* & \longrightarrow & K\\
\lambda & \longmapsto & \lambda(v) 
\end{eqnarray*}
y $T^(1,0)(V)\simeq V$ can\'onicamente.
\end{nota}

\begin{defn}
Sean $l_1,l_2,k_1,k_2\in\mathbb{Z}_{\ge 0}$, $S\in T^{(l_1,k_1)}(V)$ y $T\in T^{(l_2,k_2)}(V)$ definimos su \emph{producto tensorial} $S\otimes T\in T^{(l_1+l_2,k_1+k_2)}(V)$ por
\begin{IEEEeqnarray*}{rCl}
  \IEEEeqnarraymulticol{3}{l}{
S\otimes T(\lambda_1,\ldots,\lambda_{l_1+l_2},v_1,\ldots,v_{k_1+k_2})}\\
&=&S(\lambda_1,\ldots,\lambda_{l_1},v_1,\ldots,v_{k_1})T(\lambda_{l_1+1},\ldots,\lambda_{l_1+l_2},v_{k_1+1},\ldots,v_{k_1+k_2})
\end{IEEEeqnarray*}
para todo $\lambda_1,\ldots,\lambda_{l_1+l_2}\in V^*$, $v_1,\ldots,v_{k_1+k_2}\in V$.
\end{defn}

\begin{pro}\label{ptgbya}
El producto tensorial es bilineal y asociativo. Es decir, dados $l_1,l_2,l_3,k_1,k_2,k_3\in\mathbb{Z}_{\ge 0}$ y $S,S'\in T^{(l_1,k_1)}(V)$, $T,T'\in T^{(l_2,k_2)}(V)$, $U\in T^{(l_3,k_3)}(V)$, $c\in K$ tenemos:
\begin{enumerate}
\item $(S+S')\otimes T=S\otimes T + S'\otimes T$,
\item $S\otimes (T+T')=S\otimes T + S\otimes T'$,
\item $(cS)\otimes T=c(S\otimes T)=S\otimes(cT)$,
\item $(S\otimes T)\otimes U=S\otimes (T\otimes U)$.
\end{enumerate} 
\end{pro}

\dem La demostraci\'on es una verificaci\'on directa.\qed

\begin{nota}
Por la Proposici\'on \ref{ptgbya} 4. denotamos $S\otimes T\otimes U= (S\otimes T)\otimes U$ y as\'i podemos definir producto tensorial de m\'as de dos tensores.
\end{nota}

\begin{teo}
Suponga que $\dim_K(V)=n$. Sea $\{v_1,\ldots,v_n\}$ una base de $V$ y $\{\lambda_1,\ldots,\lambda_n\}$ la base dual. Para todo $l,k\in\mathbb{Z}_{>0}$, la colecci\'on de $(l,k)$-tensores
$$\Big\{v_{i_1}\otimes\cdots\otimes v_{i_l}\otimes\lambda_{j_1}\otimes\cdots\otimes\lambda_{j_k}\Big\}_{i_1,\ldots,i_l,j_1,\ldots,j_k=1}^n$$
es una base de $T^{(l,k)}(V)$. En particular $$\dim_K\left(T^{(l,k)}(V)\right)=n^{(l+k)}.$$
\end{teo}

\dem Similar a la demostraci\'on de Teorema \ref{bt}. Note que para todo $T\in T^{(l,k)}(V)$
$$T=\sum_{i_1,\ldots,i_l=1}^n\sum_{j_1,\ldots,j_k=1}^nT(\lambda_{i_1},\ldots,\lambda_{i_l},v_{j_1},\ldots,v_{j_k})v_{i_1}\otimes\cdots\otimes v_{i_l}\otimes\lambda_{j_1}\otimes\cdots\otimes\lambda_{j_k}.$$\qed

\begin{ejem}\label{tlkej}
Sea $\{e_1,\ldots,e_n\}$ la base can\'onica de $K^n$ y $\{f_1,\ldots,f_n\}$ la base dual.
\begin{enumerate}
\item Suponga que $\chara(K)\ne 2$. Para $n=3$ sea $T_{\times}\in T^{(1,2)}(K^3)$ el tensor
$$T_\times=e_1\otimes(f_2\wedge f_3)-e_2\otimes(f_1\wedge f_3)+e_3\otimes(f_1\wedge f_2).$$
\item Sea $A=(a_{ij})\in M_{n\times n}(K)$, y $T_A\in T^{(1,1)}(V)$ el tensor
$$T_A=\sum_{i,j=1}^n a_{ij}e_i\otimes f_j.$$ 
\end{enumerate}
\end{ejem}

\begin{defn}
Suponga que $\dim_K(V)=n$. Sea $\{v_1,\ldots,v_n\}$ una base de $V$ y $\{\lambda_1,\ldots,\lambda_n\}$ la base dual. Sea $l,k\in\mathbb{Z}_{\ge 0}$ y
$$t=v_{\alpha_1}\otimes\cdots\otimes v_{\alpha_l}\otimes\lambda_{\beta_1}\otimes\cdots\otimes\lambda_{\beta_k}\in T^{(l,k)}(V).$$
Dados $l',k'\in\mathbb{Z}_{\ge 0}$ tales que $l'\ge k$ y $k'\ge l$ definimos la \emph{contracci\'on por $t$} como la transformaci\'on lineal
\begin{eqnarray*}
\widehat{t}: T^{(l',k')}(V) & \longrightarrow & T^{(l'-k,k'-l)}(V)\\
T & \longmapsto & T(t):=\widehat{t}(T)
\end{eqnarray*} 
donde si
$$T=v_{i_1}\otimes\cdots\otimes v_{i_{l'}}\otimes\lambda_{j_1}\otimes\cdots\otimes\lambda_{j_{k'}}$$
entonces
$$T(t)=\left(\prod_{s=1}^k\lambda_{\beta_s}(v_{i_s})\prod_{s=1}^l\lambda_{j_s}(v_{\alpha_s})\right)v_{i_{k+1}}\otimes\cdots\otimes v_{i_{l'}}\otimes\lambda_{j_{l+1}}\otimes\cdots\otimes\lambda_{j_{k'}}.$$
Extendemos linealmente a todo los $(l,k)$-tensores $t\in T^{(l,k)}(V)$ para obtener
\begin{eqnarray*}
\widehat{\bullet}: T^{(l,k)}(V) & \longrightarrow & \Hom_K\left(T^{(l',k')}(V), T^{(l'-k,k'-l)}(V)\right)\\
t & \longmapsto & \widehat{t}
\end{eqnarray*}
\end{defn}

\begin{obs}
Note que $$\left(\prod_{s=1}^k\lambda_{\beta_s}(v_{i_s})\prod_{s=1}^l\lambda_{j_s}(v_{\alpha_s})\right)=v_{i_1}\otimes\cdots\otimes v_{i_k}\otimes\lambda_{j_1}\otimes\cdots\otimes\lambda_{j_l}(v_{\alpha_1},\ldots, v_{\alpha_l},\lambda_{\beta_1},\ldots,\lambda_{\beta_k}).$$
\end{obs}

\begin{ejem}
Sea $\{e_1,\ldots,e_n\}$ la base can\'onica de $K^n$ y $\{f_1,\ldots,f_n\}$ la base dual.
\begin{enumerate}
\item Sea $T_\times\in T^{(1,2)}(K^3)$ como en Ejemplo \ref{tlkej} 1., y
$$v_1=\sum_{i=1}^3a_ie_i,\quad v_2=\sum_{i=1}^3 b_ie_i$$
con $a_i,b_i\in K$, $i=1,\ldots,3$. Entonces $v_1\otimes v_2\in T^{(2,0)}(K^3)$ y $T_\times(v_1\otimes v_2)\in T^{(1,0)}(K^3)\simeq K^3$ con
\begin{align*}
T_\times(v_1\otimes v_2)&= (f_2\wedge f_3)(v_1,v_2)e_1-(f_1\wedge f_3)(v_1,v_2)e_2+(f_1\wedge f_2)(v_1,v_2)e_3\\
&=(a_2b_3-a_3b_2)e_1-(a_1b_3-a_3b_1)e_2+(a_1b_2-a_2b_1)e_3\\
&=: v_1\times v_2
\end{align*}
\item Sea $T_A\in T^{(1,1)}(K^n)$ como en Ejemplo \ref{tlkej} 2. y
$$v=\sum_{i=1}^n c_ie_i$$
con $c_i\in K$, $i=1,\ldots,n$. Entonces $T_A(v)\in T^{(1,0)}(K^n)\simeq K^n$ con
\begin{align*}
T_A(v) &=\sum_{i,j=1}^n a_{ij}f_j(v)e_i\\
 &=\sum_{i,j=1}^n a_{ij}c_je_i\\
 &=\sum_{i=1}^n\left(\sum_{j=1}^n a_{ij}c_j\right)e_i\\
 &=f_A(v)
\end{align*}
donde $f_A\in\Hom_K(K^n,K^n)$ est\'a definida por $f_A(e_j)=\sum_{i=1}^na_{ij}e_i$.
\item Sea $T_AT^{(1,1)}(K^n)$ como en Ejemplo \ref{tlkej} 2. y
$$\lambda=\sum_{j=1}^n d_jf_j$$
con $d_i\in K$, $i=1,\ldots,n$. Entonces $T_A(\lambda)\in T^{(0,1)}(K^n)\simeq (K^n)^*$ con
\begin{align*}
T_A(\lambda) &=\sum_{i,j=1}^n a_{ij}\lambda(e_i)f_j\\
 &=\sum_{i,j=1}^n a_{ij}d_if_j\\
 &=\sum_{j=1}^n\left(\sum_{i=1}^n a_{ij}d_i\right)f_j\\
 &=f^*_A(\lambda)
\end{align*}
donde $f_A\in\Hom_K(K^n,K^n)$ est\'a definida por $f_A(e_j)=\sum_{i=1}^na_{ij}e_i$.
\end{enumerate}
\end{ejem}

\section{Convenciones en notaci\'on de tensores}

En esta secci\'on asumiremos que $V$ tienen dimensi\'on finita.

\begin{nota}
En esta secci\'on usaremos las siguientes convenciones.
\begin{enumerate}
\item Los elementos de $V$ los denotaremos con sub\'indices.
\item Los elementos de $V^*$ los denotaremos con super\'indices.
\item Al espacio de $(l,k)$-tensores lo denotaremos por $T^l_k(V)$.
\item Convenci\'on de Einstein: si un indice se repite como sub\'indice y super\'indice se asume sumatoria sobre este.
\end{enumerate}
\end{nota}

\begin{ejem}
Sea $n=\dim(V)$, $\{v_1,\ldots,v_n\}$ una base de $V$ y $\{\lambda^1,\ldots,\lambda^n\}$ la base dual.
\begin{enumerate}
\item Dado $v\in V$ y $\lambda\in V^*$ si denotamos
\begin{align*}
v^i &= \lambda^i(v)\\
\lambda_i &=\lambda(v_i)
\end{align*}
entonces
\begin{align*}
v &= v^iv_i\\
\lambda &=\lambda_i\lambda^i
\end{align*}
\item Dado $T\in T^l_k(V)$ si denotamos
$$T^{i_1\ldots i_l}_{j_1\ldots j_k}=T(\lambda^{i_1},\ldots,\lambda^{i_l},v_{j_1},\ldots,v_{j_k})$$
entonces
$$T=T^{i_1\ldots i_l}_{j_1\ldots j_k}v_{i_1}\otimes\cdots\otimes v_{i_l}\otimes\lambda^{j_1}\otimes\cdots\otimes\lambda^{j_k}$$
\item Dado $f\in\Hom_K(V,V)$ si denotamos
$$f^i_j=\lambda^i(f(v_j))$$
entonces el tensor $T_f=f^i_jv_i\otimes\lambda^j\in T^1_1(V)$ es tal que
\begin{align*}
T_f(v) &= f^i_j\left(v_i\otimes\lambda^j\right)(v)\\
 &= f^i_j\lambda^j(v)v_i\\
 &= f^i_jv^jv_i\\
 &= f(v)
\end{align*}
y
\begin{align*}
T_f(\lambda) &= f^i_j\left(v_i\otimes\lambda^j\right)(\lambda)\\
 &= f^i_j\lambda(v_i)\lambda^j\\
 &= f^i_j\lambda_i\lambda^j\\
 &= f^*(\lambda)
\end{align*}
\end{enumerate}
\end{ejem}


\appendix

\chapter{Cuerpos}


En este ap\'endice definimos cuerpos y polinomios. Tambi\'en establecemos los elementos necesarios en nuestro estudio de operadores. 

\begin{defn}\label{defcuerpo}
Un \emph{cuerpo} $K$ es un conjunto con dos operaciones $+$ y $\cdot$ (e.d. funciones $K\times K\rightarrow K$), que llamamos respectivamente \emph{suma} y \emph{producto} (o \emph{adici\'on} y \emph{multiplicaci\'on}) , y dos elementos distintos $0$ y $1$, que llamamos respectivamente \emph{cero} y \emph{uno}, los cuales satisfacen las siguientes propiedades.
\begin{enumerate}
\item \emph{Commutatividad}: Para todo $a,b,c\in K$, se tiene $a+b=b+a$ y $a\cdot b=b\cdot a$.
\item \emph{Asociatividad}: Para todo $a,b,c\in K$, se tiene $a+(b+c)=(a+b)+c$ y $a\cdot(b\cdot c)=(a\cdot b)\cdot c$,
\item \emph{Neutralidad de $0$ y $1$}: Para todo $a\in K$, se tiene $0+a=a$ y $1\cdot a=a$,
\item \emph{Existencia de opuesto y de inverso}: Para todo $a\in K$, existe $-a\in K$ para el cual se tiene $-a+a=0$ y, si tenemos $a\ne 0$, entonces existe $a^{-1}\in K$ para el cual se tiene $a\cdot a^{-1}=1$,
\item \emph{Distributividad del producto sobre la suma}: Para todo $a,b,c\in K$, se tiene $a\cdot(b+c)=a\cdot b+a\cdot c$. 
\end{enumerate}
\end{defn}

\begin{nota}
Es usual omitir el s\'imbolo $\cdot$ en la operaci\'on de multiplicaci\'on, de tal forma que $a\cdot b$ se denota tambi\'en por $ab$.
\end{nota}

\begin{ejem}
Los siguientes conjuntos junto con sus respectivas operaciones son cuerpos.
\begin{enumerate}
\item El conjunto de los n\'umeros reales $\mathbb{R}$ con sus operaciones usuales de suma y producto.
\item El conjunto de los n\'umeros complejos $\mathbb{C}$ con sus operaciones usuales de suma y producto.
\item El conjunto de los n\'umeros racionales $\mathbb{Q}$ con sus operaciones usuales de suma y producto.
\item El subconjunto de los n\'umeros reales $\mathbb{Q}[\sqrt{2}]$, el cual est\'a formado por los n\'umeros de la forma $a+b\sqrt{2}$ donde $a,b\in\mathbb{Q}$, con las operaciones heredades de $\mathbb{R}$. 
\item El subconjunto de los n\'umeros complejos $\mathbb{Q}[i]$, el cual est\'a formado por los n\'umeros de la forma $a+bi$ donde $a,b\in\mathbb{Q}$, con las operaciones heredadas de $\mathbb{C}$.
\item El conjunto $\mathbb{F}_p$ de clases de equivalencia m\'odulo $p$ en los n\'umeros enteros $\mathbb{Z}$, donde $p$ es un n\'umero primo, con las operaciones heredadas de las operaciones usuales de suma y multiplicaci\'on de $\mathbb{Z}$.
\end{enumerate}
\end{ejem}

\begin{ejem}
Los siguientes conjuntos no son cuerpos.
\begin{enumerate}
\item El conjunto de los n\'umeros naturales $\mathbb{N}$ con sus operaciones usuales, pues los elementos diferentes de $0$ no tienen opuesto.
\item El conjunto de los n\'umeros enteros $\mathbb{Z}$ con sus operaciones usuales, pues los elementos diferentes de $0$, aparte de $-1$ y de $1$ no tienen inverso.
\end{enumerate}
\end{ejem}

\begin{prop}[Ley de cancelaci\'on]
Sea $K$ un cuerpo y sean $a,b,c\in K$. Si se tiene $a+b=c+b$ \'o, entonces se tiene $a=b$. Si $b$ es diferente de $0$ y se tiene $ab=cb$, entonces se tiene $a=c$. 
\end{prop}

\dem Basta con sumar el opuesto de $b$ a ambos lados de la igualdad en el caso de la suma, o multiplicar por el inverso de $b$ en el caso de la multiplicaci\'on. \qed

\begin{prop}[Unicidad de $0$, $1$, del opuesto y del inverso]
Si $K$ es un cuerpo, entonces los elementos neutros de la suma y del producto son \'unicos. Igualmente, para todo $a\in K$ su opuesto, y, si $a$ es diferente de $0$, su inverso son \'unicos.
\end{prop}

\dem Si $e\in K$ es tal que se tiene $a+e=a$ para alg\'un $a\in K$, por la neutralidad de $0$ se obtiene $a+0=a=a+e$. La ley de cancelaci\'on implica que $0$ es igual a $e$. Similarmente, se puede verificar la unicidad de $1$ como neutro del producto.

\noindent Para verificar la unicidad del opuesto, observe que si $a\in K$ y $b,c\in K$ son tales que se tiene $a+b=0=a+c$, la ley de cancelaci\'on implica que $b$ y $c$ son iguales. Similarmente se establece la unicidad del inverso, cuando este existe. \qed


\begin{nota}
Si $a\in K$ es diferente de $0$, a su inverso $a^{-1}$ tambi\'en lo denotaremos por $1/a$. Es usual denotar las operaciones $a+(-b)$ por $a-b$ y $a\cdot b^{-1}$ por $\frac{a}{b}$.
\end{nota}

\begin{pro}\label{propa00}
Sea $K$ un cuerpo, y sean $a,b\in K$, entonces se tienen las siguientes igualdades.
\begin{enumerate}
\item $a\cdot 0=0$
\item $-1\cdot a=-a$
\item $(-a)\cdot b=a\cdot(-b)=-(a\cdot b)$
\item $(-a)\cdot (-b)=a\cdot b$
\end{enumerate}
\end{pro}

\dem
\begin{enumerate}
\item Tenemos
\[
0+a\cdot 0=a\cdot 0=a\cdot (0+0)=a\cdot 0+ a\cdot 0,
\]
luego por Ley de cancelaci\'on se obtiene $0=a\cdot 0$.
\item Tenemos
\[
-1\cdot a+a=-1\cdot a+1\cdot a=(-1+1)a=0\cdot a=0.
\]
\item Por unicidad del opuesto basta verificar que $(-a)b$ y $a(-b)$ son el opuesto de $ab$. De las igualdades
\[
0=0\cdot b=\left( a+(-a)\right) b=ab+(-a)b
\]
se obtiene que $(-a)b$ es el opuesto de $ab$. Similarmente se establece $a\cdot(-b)=-(a\cdot b)$.
\item Usando la igualdad $-(-b)=b$ y la propiedad \ref{propa00}.3 obtenemos
\[
(-a)(-b)=a\left(-(-b)\right) =ab
\]
\end{enumerate}\qed

\begin{defn}
Sea $K$ un cuerpo. Si existe un n\'umero natural $k$ para el cual se tiene
\[
\underbrace{1+1+\ldots+1}_{k \textrm{ sumandos}}=0,
\]
al m\'inimo entre estos lo llamamos la \emph{caracter\'istica} de $K$ y lo denotamo por $\chara(K)$. En caso de que no existe tal $k$, definimos la caracter\'istica de $K$ como $0$.   
\end{defn}

\begin{ejem}
La caracter\'istica de $\mathbb{F}_p$ es $p$ y las de $\mathbb{Q}$, $\mathbb{R}$, $\mathbb{C}$ son todas iguales a $0$.
\end{ejem}

\begin{obs}
Un cuerpo es la m\'inima estructura $K$ para la cual, para todo $a,b,c\in K$, con $a\ne 0$, la ecuaci\'on lineal $ax+b=c$ tiene una soluci\'on, a saber $x=(c-b)/a$.
\end{obs}

\begin{ejem}
En $\mathbb{F}_5$ las operaciones de suma y producto est\'an dadas por las siguientes tablas.
{\small
$$\begin{array}{c||c|c|c|c|c|}
+ & 0 & 1 & 2 & 3 & 4\\
\hline
\hline
0 & 0 & 1 & 2 & 3 & 4\\
\hline
1 & 1 & 2 & 3 & 4 & 0\\
\hline 
2 & 2 & 3 & 4 & 0 & 1\\
\hline 
3 & 3 & 4 & 0 & 1 & 2\\
\hline 
4 & 4 & 0 & 1 & 2 & 3\\
\hline 
\end{array} \qquad
\begin{array}{c||c|c|c|c|c|}
\cdot & 0 & 1 & 2 & 3 & 4\\
\hline
\hline
0 & 0 & 0 & 0 & 0 & 0\\
\hline
1 & 0 & 1 & 2 & 3 & 4\\
\hline 
2 & 0 & 2 & 4 & 1 & 3\\
\hline 
3 & 0 & 3 & 1 & 4 & 2\\
\hline 
4 & 0 & 4 & 3 & 2 & 1\\
\hline 
\end{array}$$}
La ecuaci\'on $3x+4=1$ es equivalente a
\begin{align*}
3x & = 1+(-4), & 3x & = 1+1, & 3x & = 2, & x & = 2/3, & x & = 2\cdot 2
\end{align*}
luego la soluci\'on es $x=4$.
\end{ejem}

\begin{defn}
Sea $K$ un cuerpo. Un \emph{polinomio con coeficientes en $K$ en la variable $t$} es una expresi\'on de la forma
$$a_nt^n+a_{n-1}t^{n-1}+\ldots+a_1t+a_0$$
donde $a_n,\ldots,a_1,a_0$ son elementos en $K$. Denote por $P(t)$ a este polinomio. Dado $c\in K$, el \emph{valor de $P(t)$ en $c$} es
$$a_nc^n+a_{n-1}c^{n-1}+\ldots+a_1c+a_0,$$
el cual denotaremos por $P(c)$. Cuando tenemos $P(c)=0$ decimos que $c$ es una \emph{ra\'iz} de $P(t)$.
\end{defn}

\begin{defn}
Sea $K$ un cuerpo. Decimos que $K$ es \emph{algebraicamente cerrado} si todo polinomio no constante tiene una ra\'iz.
\end{defn}

\begin{teo}[Teorema fundamental del \'algebra]
El cuerpo de los n\'umeros complejos $\mathbb{C}$ es algebraicamente cerrado.
\end{teo}

\dem Presentamos una prueba usando an\'alisis complejo, en particular usamos el Teorema de Liouville que indica que si una funci\'on es anal\'itica y acotada en todo el plano complejo, entonces es constante.

Sea $P(t)$ un  polinomio  con coeficientes en $\mathbb{C}$. Suponga que tenemos $P(t)=a_nt^n+\ldots+a_1t+a_0$, con $a_n\ne 0$y considere la funci\'on $f(z)$ dada por 
$$f(z)=P(z)/(a_nz^n)=1+\sum_{k=1}^{n-1}\dfrac{a_k}{a_n}\dfrac{1}{z^{n-k}}$$
la cual est\'a definida para todo $z\in\mathbb{C}\setminus\{0\}$. Si tomamos $z=re^{\theta i}$, entonces por la desigualdad triangular, obtenemos
$$ |f(z)| \ge 1-\sum_{k=0}^{n-1}\left|\dfrac{a_k}{a_n}\right|\dfrac{1}{r^{n-k}}.$$
El l\'imite cuando $r$ tiende a infinito del lado derecho de la desigualdad es $1$, luego existe $R>0$ tal que se cumple $|f(z)|>1/2$ para $|z|=r>R$. As\'i, se tiene $|P(z)|>|a_n|R^n/2>0$ para $|z|=r>R$. Sea $D$ el disco cerrado centrado en el origen de radio $R$. Como $D$ es compacto y la funci\'on $|P(z)|$ es continua, esta alcanza un m\'inimo $m$.

Suponga que $P(t)$ no tiene ra\'ices, luego la funci\'on $1/P(z)$ es anal\'itica sobre todo el plano complejo y se tiene $m>0$. Tenemos que $|1/P(z)|$ est\'a acotada por $2/(|a_n|R^n)$ fuera de $D$ y por $1/m$ en $D$, luego por el teorema de Louiville $1/P(z)$ es una funci\'on constante y as\'i $P(t)$ es un polinomio constante. Luego todo polinomio no constante con coeficientes complejos tiene ra\'ices.\qed

\section*{Polinomios con coeficientes en un cuerpo}

\begin{defn}
Al conjunto de polinomios con coeficientes en $K$ en la variable $t$ lo denotamos $K[t]$. Si $P(t)\in K[t]$ es el polinomio $a_nt^n+a_{n-1}t^{n-1}+\ldots+a_1t+a_0$, con $a_n\ne 0$, decimos que el \emph{grado de $P(t)$} es $n$ y lo denotamos $\deg\left(P(t)\right)$. En tal caso llamamos a $a_n$ el \emph{coeficiente l\'ider}. Cuando tenemos $P(t)=0$, definimos $\deg\left(P(t)\right)$ como $-\infty$ y convenimos que se tiene $-\infty<n$ para todo $n\in\mathbb{Z}$. Si $P(t)$ y $Q(t)$ son los polinomios $a_nt^n+a_{n-1}t^{n-1}+\ldots+a_1t+a_0$ y $b_mt^m+\ldots+b_1t+b_0$, su producto $P(t)Q(t)$ es el polinomio $\sum_{i=0}^{n+m}\left(\sum_{j=0}^{i}b_ia_{i-j}\right)t^i$.
\end{defn}

\begin{obs}
Para todo $P(t),Q(t)\in K[t]$, tenemos
\begin{align*}
\deg\left(P(t)+Q(t)\right) &  \le \max\{\deg\left(P(t)\right),\deg\left(Q(t)\right)\}\\
\deg\left(P(t)Q(t)\right) & = \deg(P(t))+\deg(Q(t)).
\end{align*}
Si $P(t)$ y $Q(t)$ tiene grado diferente, entonces se tiene
\[
\deg\left(P(t)+Q(t)\right)=\max\{\deg\left(P(t)\right),\deg\left(Q(t)\right)\}.
\]
Si $Q(t)$ no es constante, entonces se tiene
\[
\deg\left(P(t)Q(t)\right) >\deg\left(P(t)\right).
\]
\end{obs}

\begin{teo}[Algoritmo de la divisi\'on]
Dados $P(t),Q(t)\in K[t]$, con $Q(t)\ne 0$, existe un \'unico par de polinomios $S(t),R(t)\in K[t]$ tales que se tiene
\[
P(t)=S(t)Q(t)+R(t),\textrm{ y } \deg\left(R(t)\right)<\deg\left(Q(t)\right).
\]
\end{teo}

\dem Sea $N$ el conjunto $\{P(t)-T(t)Q(t)\}_{T(t)\in K[t]}$. Como $N$ no es vacio, contiene un polinomio $R(t)$ de grado m\'inimo. Sea $S(t)\in K[t]$ para el cual se tiene $R(t)=P(t)-S(t)Q(t)$. Obtenemos $\deg\left(R(t)\right)<\deg\left(Q(t)\right)$, pues de lo contrario, si escribimos
$R(t)=at^{n_R}+\ldots$ y $Q(t)=bt^{n_Q}+\ldots$ donde $a,b\in K$ son los respectivos coeficientes l\'ideres de $R(t)$ y $Q(t)$ entonces el polinomio $R(t)-\frac{a}{b}t^{n_R-n_Q}Q(t)$, que es igual a $P(t)-\left(S(t)+\frac{a}{b}t^{n_R-n_Q}\right)Q(t)$, ser\'ia un polinomio en $N$ de grado estrictamente menor que $R(t)$, contradiciendo la minimalidad del grado de este.

Para establecer la unicidad de $S(t)$ y $R(t)$, tomamos $S'(t),R'(t)\in K[t]$ para los cuales se tiene $P(t)=S'(t)Q(t)+R'(t)$ con $\deg\left(R'(t)\right)<\deg\left(Q(t)\right)$. Tenemos $S(t)Q(t)+R(t)=P(t)=S'(t)Q(t)+R'(t)$ y as\'i $\left(S(t)-S'(t)\right)Q(t)=R'(t)-R(t)$. De las desigualdades $\deg\left(R(t)-R'(t)\right)\le \max\{\deg\left(R(t)\right),\deg\left(R'(t)\right)\}<\deg\left(Q(t)\right)$, se obtiene $\deg(\left(S(t)-S'(t)\right)Q(t))<\deg\left(Q(t)\right)$. Por ende $S(t)-S'(t)$ es igual $0$ y as\'i $R(t)-R'(t)$ tambi\'en es $0$, es decir $R'(t)=R(t)$ y $S'(t)=S(t)$.\qed

\begin{coro}
Sea $P(t)\in K[t]$. Si $\lambda\in K$ es una ra\'iz de $P(t)$, entonces $(t-\lambda)$ divide a $P(t)$.
\end{coro}

\dem Sea $Q(t)$ el polinomio $t-\lambda$ y sean $S(t),R(t)\in K[t]$ como en el algoritmo de la divisi\'on. En particular, se tiene $\deg\left(R(t)\right)=0$ es decir $R(t)=a$ para alg\'un $a\in K$. As\'i, obtenemos $0=P(\lambda)=(\lambda-\lambda)S(\lambda)+a=a$ y $P(t)=S(t)Q(t)$. \qed

\begin{defn}
Sean $P(t),Q(t)\in K[t]$. Decimos que $D(t)\in K[t]$ es un \emph{m\'aximo divisor com\'un} de $P(t)$ y $Q(t)$ si satisface las siguiente dos propiedades.
\begin{enumerate}
\item El polinomio $D(t)$ divide a $P(t)$ y a $Q(t)$.
\item Si $D_0(t)\in K[t]$ divide a $P(t)$ y a $Q(t)$, entonces $D_0(t)$ divide a $D(t)$. 
\end{enumerate}
\end{defn}

\begin{prop}
Para todo $P(t),Q(t)\in K[t]$, con uno de ellos no nulo, existe un m\'aximo com\'un divisor $D(t)$ de $P(t)$ y $Q(t)$. Mas a\'un, existen $P_0(t),Q_0(t)\in K[t]$ para los cuales se tiene $D(t)=Q_0(t)P(t)+P_0(t)Q(t)$.
\end{prop}

\dem Sea $N$ el conjunto $\left\{Q_1(t)P(t)+P_1(t)Q(t)\in K[t]\setminus\{0\}\right\}_{P_1(t),Q_1(t)\in K[t]}$.
Como $N$ no es vac\'io, contiene un un polinomio $D(t)$ de grado m\'inimo. Sean $Q_0(t),P_0(t)\in K[t]$ para los cuales se tiene $D(t)=Q_0(t)P(t)+P_0(t)Q(t)$.

Veamos que $D(t)$ es un divisor com\'un de $P(t)$ y $Q(t)$. De hecho, si para $S(t),R(t)\in K[t]$ se tiene $P(t)=S(t)D(t)+R(t)$ con $\deg\left(R(t)\right)<\deg\left(D(t)\right)$, entonces obtenemos
$$R(t)=P(t)-S(t)D(t)=\Big(1-S(t)Q_0(t)\Big)P(t)-S(t)P_0(t)Q(t),$$
y as\'i, como $D(t)$ tiene grado m\'inimo en $N$, entonces $R(t)$ no pertenece a $N$, luego es igual a $0$. Por ende $D(t)$ divide a $P(t)$. Por un argumento similar, se obtiene que $D(t)$ divide a $Q(t)$.

Veamos que $D(t)$ es m\'aximal entre los divisores comunes de $P(t)$ y $Q(t)$. Si $D_0(t)\in K[t]$ divide a $P(t)$ y $Q(t)$, es decir si existen
$T_1(t),T_2(t)\in K[t]$ para los cuales se tiene $P(t)=T_1(t)D_0(t)$ y $Q(t)=T_2(t)D_0(t)$, entonces $D_0(t)$ tambi\'en divide a $D(t)$, pues se tiene
\[
D(t)=Q_0(t)P(t)+P_0(t)Q(t)=\Big(Q_0(t)S'_1(t)+P_0(t)S'_2(t)\Big)D'(t).
\]
\qed

\begin{nota}
Sean $P(t),Q(t)\in K[t]$, con $P(t)\ne 0$ \'o $Q(t)\ne 0$. Al m\'aximo divisor com\'un de $P(t)$ y $Q(t)$ que es m\'onico lo denotamos $(P(t),Q(t))$.
\end{nota}

\begin{obs}
Si $\lambda_1,\lambda_2\in K$ son distintos, entonces $(t-\lambda_1,t-\lambda_2)$ es $1$, pues se tiene la igualdad
\[
\frac{1}{\lambda_2-\lambda_1}\Big( (t-\lambda_1)-(t-\lambda_2) \Big)=1.
\]
\end{obs}

\begin{pro}[Algoritmo de Euclides]
Sean $P(t),Q(t)\in K[t]$. Sean $S(t),R(t)\in K[t]$ como en el algoritmo de la divisi\'on, es decir $P(t)=S(t)Q(t)+R(t)$ con $\deg\left(R(t)\right)<\deg\left(Q(t)\right)$, entonces se tiene
$$ (P(t),Q(t))=(Q(t),R(t)) $$
\end{pro}

\dem Sea $D(t)=(P(t),Q(t))$. Veamos que $D(t)$ es un divisor com\'un de $Q(t)$ y $R(t)$. Para ello basta verificar que $D(t)$ divide a $R(t)$. Para ello, note que si $P(t)=P_1(t)D(t)$ y $Q(t)=Q_1(t)D(t)$, entonces $$R(t)=P(t)-S(t)Q(t)=\Big(P_1(t)-S(t)Q_1(t)\Big)D(t).$$

Veamos ahora que $D(t)$ es maximal entre los divisores comunes de $Q(t)$ y $R(t)$. Como $D(t)=(P(t),Q(t))$, existen $P_0(t),Q_0(t)\in K[t]$ tales que $Q_0(t)P(t)+P_0(t)Q(t)=D(t)$.
Suponga que $D'(t)$ divide a $Q(t)$ y a $R(t)$, entonces si $Q(t)=S_1(t)D'(t)$ y $R(t)=S_2(t)D'(t)$, tenemos
\begin{align*}
D(t) & = Q_0(t)P(t)+P_0(t)Q(t)\\
 & =Q_0(t)\Big(S(t)Q(t)+R(t)\Big)+P_0(t)Q(t)\\
 & =\Big(\big(Q_0(t)S(t)+P_0(t)\big)S_1(t)+Q_0(t)S_2(t)\Big)D'(t)
\end{align*}
luego $D'(t)$ divide a $D(t)$.\qed

\begin{obs}
La utilidad del \'algoritmo de Euclides es que en la busqueda del m\'aximo com\'un divisor de $P(t)$ y $Q(t)$ podemos reemplazar esta pareja por la pareja de menor grado $Q(t)$ y $R(t)$. De esta forma seguimos iterativamente reduciendo los grados en la pareja hasta el caso trivial $(R_0(t),0)=R_0(t)$.
\end{obs}

\begin{ejem}
Considere los polinomios $P(t)=t^3+t^2-t-1$ y $Q(t)=t^3-t^2$  en $\mathbb{Q}[t]$, tenemos
\begin{align*}
P(t) & = Q(t) + 2t^2-t-1\\
Q(t) & =\Big(\frac{1}{2}t-\frac{1}{4}\Big)(2t^2-t-1)+\frac{1}{4}t-\frac{1}{4}\\
2t^2-t-1 & = (8t+4)\Big(\frac{1}{4}t-\frac{1}{4}\Big)
\end{align*}
as\'i
\begin{align*}
(P(t),Q(t)) & = (Q(t),2t^2-t-1) = (2t^2-t-1,\frac{1}{4}t-\frac{1}{4}) = (\frac{1}{4}t-\frac{1}{4},0)\\
 & = t-1. 
\end{align*}
y
\begin{align*}
t-1 & = 4Q(t)-(2t-1)(2t^2-t-1)\\
  & = 4Q(t)-(2t-1)\big(P(t)-Q(t)\big)\\
  & = -(2t-1)P(t)+(2t+3)Q(t)
\end{align*}
\end{ejem}

\begin{obs}
Similarmente a como definimos m\'aximo com\'un divisor de un par de polinomios en $K[t]$, podemos definir  \emph{m\'aximo com\'un divisor de una familia finita de polinomios}. Si denotamos al m\'aximo com\'un divisor de $\{P_1(t),\ldots,P_n(t)\}$ que es m\'onico por $\left(P_1(t),\ldots,P_n(t)\right)$, tenemos
\[
\left(P_1(t),\ldots,P_n(t)\right)=\Big(\big(P_1(t),\ldots,P_{n-1}(t)\big), P_n(t)\Big).
\]
\end{obs}

\begin{pro}[Relaci\'on de Bezout]
Dados $P_1(t),\ldots,P_n(t)\in K[t]$, con uno de ellos no nulo, existen polinomios $Q_1(t),\ldots,Q_n(t)\in K[t]$ tales que
\[
Q_1(t)P_1(t)+\ldots+Q_n(t)P_n(t)=\left(P_1(t),\ldots,P_n(t)\right)
\]
\end{pro}

\dem Hacemos inducci\'on en $n$, donde el caso base $n=2$ ya fue demostrado. Para obtener el paso inductivo, asumimos que el resultado es cierto cuando nos son dados $n-1$ polinomios, uno de ellos no nulo. Por la hip\'otesis de inducci\'on existen $R_1(t),\ldots,R_{n-1}(t)\in K[t]$ para los cuales se tiene
\[
R_1(t)P_1(t)+\ldots+R_{n-1}(t)P_{n-1}(t)=\big(P_1(t),\ldots,P_{n-1}(t)\big),
\]
y, por el caso base, existen $Q(t),Q_n(t)\in K[t]$ para los cuales se tiene
\[
Q(t)\left(P_1(t),\ldots,P_{n-1}(t)\right)+Q_n(t)P_n(t)=\Big(\big(P_1(t),\ldots,P_{n-1}(t)\big), P_n(t)\Big).
\]
Entonces, si definimos $Q_i(t)=Q(t)R_i(t)$, para $i\{1,\ldots,n-1\}$, obtenemos
\begin{eqnarray*}
Q_1(t)P_1(t)+\ldots+Q_n(t)P_n(t) & = & \Big(\big(P_1(t),\ldots,P_{n-1}(t)\big), P_n(t)\Big)\\
 & = & \left(P_1(t),\ldots,P_n(t)\right)
\end{eqnarray*}
y se sigue la propiedad.
\qed

\begin{ejem}
Considere los polinomios $P(t)=t^3+t^2-t-1$, $Q(t)=t^3-t^2$ y $R(t)=t^2+t$ en $\mathbb{Q}[t]$, tenemos $$\Big(P(t),Q(t),R(t)\Big)=\Big(\big(P(t),Q(t)\big),R(t)\Big)=\big(t-1,R(t)\big).$$
De las igualdades
\begin{align*}
R(t)& =t(t-1)+2t\\
t-1& =\frac{1}{2}(2t)-1\\
2t & = (-2t)(-1)+0
\end{align*}
y el \'algoritmo de Euclides, se sigue
\begin{align*}
(R(t),t-1) & =(t-1,2t) =(2t,-1) =(-1,0)\\
  & =1
\end{align*}
y obtenemos
\begin{align*}
1 &= \frac{1}{2}(2t)-(t-1)\\
   &= \frac{1}{2}(R(t)-t(t-1))-(t-1)\\
   &= \frac{1}{2}R(t)-(\frac{1}{2}t+1)(t-1)
\end{align*}
al igual que
\begin{align*}
1   &= \frac{1}{2}R(t)-(\frac{1}{2}t+1)\big(-(2t-1)P(t)+(2t+3)Q(t)\big)\\
   &= (\frac{1}{2}t+1)(2t-1)P(t)-(\frac{1}{2}t+1)(2t+3)Q(t)+\frac{1}{2}R(t).
\end{align*}
\end{ejem}

\begin{defn}
Sea $P(t)\in K[t]$, con $\deg(P(t))>0$. Decimos que $P(t)$ es un polinomio irreducible si para toda factorizaci\'on $P(t)=S(t)Q(t)$ con $S(t),Q(t)\in K[t]$ tenemos que $\deg(S(t))=0$ \'o $\deg(Q(t))=0$ (e.d. $S(t)=c$ \'o $Q(t)=c$ para alg\'un $c\in K$, con $c\ne 0$).
\end{defn}

\begin{lema}
Sea $P(t)\in K[t]$ un polinomio irreducible y sean $R(t),S(t)\in K[t]$ tales que $P(t)$ divide a $R(t)S(t)$, entonces $P(t)$ divide a $R(t)$ o a $S(t)$.
\end{lema}

\dem Sea $Q(t)$ tal que se tiene $P(t)Q(t)=R(t)S(t)$ y suponga que $P(t)$ no divide a $R(t)$. Como $P(t)$ es irreducible tenemos $(P(t),R(t))=1$. Sean $R_0(t),P_0(t)\in K[t]$ para los cuales se tiene $1=R_0(t)P(t)+P_0(t)R(t)$, obtenemos entonces
\begin{align*}
S(t) & =(R_0(t)P(t)+P_0(t)R(t))S(t)\\
& =R_0(t)P(t)S(t)+P_0(t)R(t)S(t)\\
& =R_0(t)P(t)S(t)+P_0(t)P(t)Q(t)\\
& =P(t)\left(R_0(t)S(t)+P_0(t)Q(t)\right)
\end{align*}
y as\'i $P(t)$ divide a $S(t)$.

\begin{teo}[Factorizaci\'on \'unica]
Si $P(t)\in K[t]$ es un polinomio con $\deg(P(t))>0$, entonces existen polinomios irreducibles $P_1(t),\ldots,P_n(t)\in K[t]$ para los cuales se tiene la factorizaci\'on $P(t)=P_1(t)\cdots P_n(t)$. M\'as a\'un esta factorizaci\'on es \'unica en el siguiente sentido. Si tenemos $P(t)=Q_1(t)\cdots Q_m(t)$ con $Q_1(t),\ldots,Q_m(t)\in K[t]$ irreducibles, entonces $m$ es igual a $n$ y existe una permutaci\'on $\sigma$ de $\{1,\ldots,n\}$ tal que, para $i\in\{1,\ldots,n\}$ se tiene $Q_i(t)=c_iP_{\sigma(i)}(t)$ para alg\'un $c_i\in K$.
\end{teo}

\dem Demostremos primero la existencia de la factorizaci\'on. Lo haremos por inducci\'on en $\deg(P(t))$, siendo el caso base $\deg(P(t))=1$ evidente pues en tal caso $P(t)$ es irreducible. Suponga que la factorizaci\'on por irreducibles ha sido demostrada para polinomios de grado estrictamente menor a $d$ y sea $P(t)\in K[t]$ con $\deg(P(t))=d$. Si $P(t)$ es irreducible, no hay nada que demostrar. Suponga entonces que existen $S(t),Q(t)\in K[t]$ tales que se tiene $P(t)=S(t)Q(t)$ con $\deg(S(t))>0$ y $deg(Q(t))>0$. En particular tenemos $\deg(S(t))<d$ y $\deg(Q(t))<d$. Por hip\'otesis de inducci\'on existen polinomios irreducibles $P_1(t),\ldots,P_{n_1}(t)$, $P_{n_1+1}(t),\ldots,P_{n_1+n_2}(t)\in K[t]$ para los cuales se tiene $S(t)=P_1(t)\cdots P_{n_1}(t)$ y $Q(t)=P_{n_1+1}(t)\cdots P_{n_1+n_2}(t)$.
As\'i, obtenemos la factorizaci\'on $P(t)=P_1(t)\cdots P_{n_1+n_2}(t)$.\\
Para establecer la unicidad de la factorizaci\'on procedemos por inducci\'on en el n\'umero de factores, siendo el caso base de un \'unico factor inmediato pues en tal caso $P(t)$ es irreducible. Asuma por inducci\'on que la unicidad ha sido demostrada cuando el n\'umero de factores es estrictamente menor a $n$. Si tenemos $P(t)=P_1(t)\cdots P_n(t)=Q_1(t)\cdots Q_m(t)$, entonces el lema anterior implica que $P_n(t)$ divide a alg\'un $Q_i(t)$, que podemos asumir, sin p\'erdida de generalidad, que es $Q_m(t)$. Pero como $Q_m(t)$ es irreducible entonces obtenemos $Q_m(t)=cP_m(t)$ para alg\'un $c\in K$ con $c\ne 0$. Tenemos as\'i $cP_1(t)\cdots P_{n-1}(t)=Q_1(t)\cdots Q_{m-1}(t)$. Como $cP_1(t)$ es irreducible, aplicamos la hip\'otesis de inducci\'on para establecer la igualdad $n-1=m-1$ y la existencia de una biyecci\'on $\sigma_0$ de $\{1,\ldots,n-1\}$ tal que, para $i\in\{1,\ldots,n-1\}$, se tiene $Q_i(t)=c_iP_{\sigma(i)}(t)$ para alg\'un $c_i\in K$. El teorema se sigue al tomar la biyecci\'on $\sigma$ definida por $\sigma(i)=\sigma_0(i)$ para $1\le i\le n-1$ y $\sigma(n)=n$, y notar que $n-1=m-1$ implica $n=m$.

\begin{teo}
Si $P(t)\in\mathbb{R}[t]$ es un polinomio m\'onico irreducible, entonces $P(t)=t-a$, con $a\in\mathbb{R}$, \'o $P(t)=(t-a)^2+b^2$, con $a,b\in\mathbb{R}$. 
\end{teo}

\dem Sea $P(t)\in\mathbb{R}[t]$ un polinomio m\'onico irreducible. Por el teorema fundamental del \'algebra $P(t)$ tiene una ra\'iz $w=a+bi$ en $\mathbb{C}$, donde $a,b\in\mathbb{R}$. Si $w$ es un n\'umero real, es decir $b=0$ y $w=a$, entonces $t-a$ divide a $P(t)$ y como $P(t)$ es m\'onico, obtenemos $P(t)=t-a$. De lo contrario, tenemos $w=a+bi$ con $b\ne 0$, y en tal caso $\overline{w}$, el conjudado de $w$, tambi\'en es una ra\'iz de $P(t)$. Luego $t-w$ y $t-\overline{w}$ dividen a $P(t)$ en $\mathbb{C}[t]$ y as\'i el m\'inimo multiplo com\'un $(t-w)(t-\overline{w})$ tambi\'en divide a $P(t)$ (ver ejercicio \ref{ejmcm}). Por ende, como se tiene $(t-w)(t-\overline{w})=(t-a-bi)(t-a+bi)=(t-a)^2+b^2$ entonces $(t-w)(t-\overline{w})$ es un divisor de $P(t)$ en $\mathbb{R}[t]$ y al ser $P(t)$ m\'onico, obtenemos $P(t)=(t-a)^2+b^2$.

\section*{Ejercicios}

\begin{enumerate}
\item Para cada una de las siguientes familias de polinomios $\{P_1(t),\ldots,P_n(t)\}\subseteq\mathbb{Q}[t]$:
\begin{itemize}
\item[i)] $P_1(t)=(t-3)(t-5)$, $P_2(t)=(t-1)(t-5)$, $P_3(t)=(t-1)(t-3)$;
\item[ii)] $P_1(t)=(t-3)$, $P_2(t)=(t-1)^2$;
\item[iii)] $P_1(t)=(t-1)(t^2-2)$, $P_2(t)=(t+1)(t^2-2)$, $P_3(t)=t^2-1$;
\item[iv)] $P_1(t)=t^2-2t+10$, $P_2(t)=t^2-2t+2$;
\end{itemize}
\begin{itemize}
\item[(a)] Encuentre el m\'aximo com\'un divisor $\big(P_1(t),\ldots,P_n(t)\big)$ en $\mathbb{Q}(t)$.
\item[(b)] Encuentre polinomios $Q_1(t),\ldots,Q_n(t)\in\mathbb{Q}[t]$ para los cuales se tiene
$$\big(P_1(t),\ldots,P_n(t)\big)=Q_1(t)P_1(t)+\ldots+Q_n(t)P_n(t).$$
\end{itemize}

\item\label{ejmcm} Sea $K$ un cuerpo y sean $P(t),Q(t)\in K[t]$. Decimos que $M(t)\in K[t]$ es un m\'inimo m\'ultiplo com\'un de $P(t)$ si satisface las siguientes dos propiedades.
\begin{itemize}
\item[i.] $P(t)$ y $Q(t)$ dividen a $M(t)$.
\item[ii.] Si $P(t)$ y $Q(t)$ dividen a $M_1(t)\in K[t]$, entonces $M(t)$ divide a $M_1(t)$.
\end{itemize}
Demuestre las siguientes dos afirmaciones.
\begin{itemize}
\item[(a)] Para todo $P(t),Q(t)\in K[t]$, con $P(t)\ne 0$ y $Q(t)\ne 0$, existe un m\'inimo m\'ultiplo com\'un de $P(t)$ y $Q_(t)$.
\item[(b)] Si $[P(t),Q(t)]$ denota al m\'inimo m\'ultiplo com\'un de $P(t)$ y $Q(t)$ que es m\'onico, entonces 
$$P(t)Q(t)=ab(P(t),Q(t))[P(t),Q(t)]$$
donde $a,b\in K$ son los respectivos coeficientes l\'ideres de $P(t)$ y $Q(t)$.
\end{itemize} 

\item Sea $K$ un cuerpo y sea $Q(t)\in K[t]$ con $\deg(Q(t))>0$. Dado $P(t)\in K[t]$, demuestre que existen unos \'unicos $P_0(t),P_1(t),\ldots,P_m(t)\in K[t]$ con $\deg(P_i(t))<\deg(Q(t))$ para $i=0,1,\ldots,m$ para los cuales se tiene
$$P(t)=P_0(t)+P_1(t)Q(t)+\ldots+P_m(t)Q(t)^m.$$

\item Sea $K$ un cuerpo y sean $c_0,c_1,\ldots,c_n\in K$ distintos. Para $i\in\{0,1,\ldots,n\}$ defina
$$P_i(t)=\prod_{j\ne i} \dfrac{t-c_i}{c_j-c_i}.$$
Dado $P(t)\in K[t]$ con $\deg(P(t))\le n$ demuestre que
$$P(t)=\sum_{i=0}^n P(c_i)P_i(t).$$
\end{enumerate}

\backmatter
%\include{biblio}
%\include{notacion}
\printindex

\end{document}
 
