\chapter{Espacio dual}

Sea $K$ un cuerpo y $V$, $W$ espacios vectoriales sobre $K$.

\begin{nota}
Dada una colecci\'on de indices $I$, definimos para cada $i,j\in I$ el s\'imbolo \emph{delta de Kronecker}:
\[
\delta_{ij}=\left\{\begin{array}{rl}1 & \textrm{si } i=j\\ 0 & \textrm{si } i\ne j\end{array}\right.
\]
\end{nota}

\section{Funcionales lineales}

\begin{defn}
El \emph{espacio dual de $V$} es el espacio vectorial $V^*=\Hom_K(V,K)$, es decir la colecci\'on de transformaciones lineales
\[
\lambda: V\longrightarrow K.
\]
A los elementos $\lambda\in V^*$ los llamamos \emph{funcionales lineales}. 
\end{defn}

\begin{prop}
Si $V$ tiene dimensi\'on finita $\dim(V)=\dim(V^*)$.
\end{prop}

\dem Sea $\{v_1,\ldots,v_n\}$ una base de $V$, donde $n=\dim(V)$. Por Proposici\'on \ref{unitrlin}.2, $\lambda\in V^*$ est\'a un\'ivocamente por los valores $\lambda(v_1),\ldots,\lambda(v_n)$. Defina $\lambda_1,\ldots,\lambda_n\in V*$ por
\[
\lambda_i(v_j)=\delta_{ij}.
\]
Veamos que $\{\lambda_1,\ldots,\lambda_n\}$ es una base de $V^*$ probando que es linealmente independiente y que genera a $V^*$; y as\'i obtenemos $\dim(V^*)=n$. Para la independencia lineal, tome $a_1,\ldots,a_n\in K$ tales que
\[
\sum_{i=1}^na_i\lambda_i=0.
\]
De forma que, para $j=1,\ldots,n$,
\[
0=\left(\sum_{i=1}^na_i\lambda_i\right)(v_j)=\sum_{i=1}^na_i\lambda_i(v_j)=\sum_{i=1}^ja_i\delta_{ij}=a_j.
\]
Para ver que $V^*=\langle\lambda_1,\ldots,\lambda_n\rangle$, dado $\lambda\in V^*$, defina $a_i=\lambda(v_i)$ y sea 
\[
\mu=\sum_{i=1}^na_i\lambda_i.
\]
De forma que, para $j=1,\ldots,n$,
\[
\mu(v_j)=\left(\sum_{i=1}^na_i\lambda_i\right)(v_j)=\sum_{i=1}^ja_i\delta_{ij}=a_j=\lambda(v_j),
\]
luego $\mu=\lambda$.\qed

\begin{defn}
Suponga que $V$ tiene dimensi\'on finita y sea $\mathcal{B}=\{v_1,\ldots,v_n\}$ una base de $V$, donde $n=\dim(V)$. A la base $\mathcal{B}^*=\{\lambda_1,\ldots,\lambda_n\}$ de $V^*$ donde
\[
\lambda_i(v_j)=\delta_{ij}.
\]
la llamamos \emph{base dual} de $\mathcal{B}$.
\end{defn}

\begin{obs}\label{basedualinfinita}
Si $V$ tiene dimensi\'on finita, $\mathcal{B}=\{v_1,\ldots,v_n\}$ es una base de $V$ y $\mathcal{B}^*=\{\lambda_1,\ldots,\lambda_n\}$ es la base dual, entonces para todo $v\in V$
\[
v=\sum_{i=1}^n\lambda_i(v)v_i.
\]
De hecho si $a_1,\ldots,a_n\in K$ son tales que $a_1v_1+\ldots+a_nv_n=v$,
\[
\lambda_i(v)=\lambda_i\left(\sum_{j=1}^na_jv_j\right)=\sum_{j=1}^n a_j\delta_{ij}=a_i.
\]
Es decir, $\lambda_i$ arroja la coordenada en $v_i$.
\end{obs}

\begin{obs}\label{basedualinfinita}
Si $V$ tiene dimensi\'on infinita y $\{v_i\}_{i\in I}$ es una base de $V$, igualmente podemos definir la colecci\'on $\{\lambda_i\}_{i\in I}\subseteq V^*$ por
\[
\lambda_i(v_j)=\delta_{ij}.
\]
e igualmente tenemos que para todo $v\in V$
\[
v=\sum_{i\in I}\lambda_i(v)v_i.
\]
Note que $\lambda_i(v)=0$ para todo $i\in I$ salvo para una subcolecci\'on finita de indices. La diferencia con el caso en dimensi\'on infinita es que $\{\lambda_i\}_{i\in I}$ no es una base de $V^*$, pues en tal caso el funcional lineal $\lambda$ definido por $\lambda(v_i)=1$, para todo $i\in I$, no es una combinaci\'on lineal de $\{\lambda_i\}_{i\in I}$.
\end{obs}

\begin{defn}
Sean $S\subseteq V$ y $L\subseteq V^*$, definimos:
\begin{enumerate}
\item el \emph{anulador} de $S$ por
\[
S^0=\{\lambda\in V^*\ |\ \lambda(v)=0, \forall v\in S\};
\]
\item el \emph{cero} de $L$ por
\[
L_0=\{v\in V\ |\ \lambda(v)=0, \forall \lambda\in L\}.
\]
\end{enumerate} 
\end{defn}

\begin{pro}
Sean $S\subseteq V$ y $L\subseteq V^*$. Tenemos:
\begin{enumerate}
\item $S^0\le V^*$ y $L_0\le V$;
\item si $S_1,S_2\subseteq V$ y $L_1,L_2\subseteq V^*$ son tales que
\[
S_1\subseteq S_2\qquad\textrm{ y }\qquad L_1\subseteq L_2,
\]
entonces
\[
S_2^0\le S_1^0\qquad\textrm{ y }\qquad \left(L_2\right)_0\le\left(L_1\right)_0;
\]
\item $\langle S^0\rangle_0=\Sp(S)$;
\item si $V_1,V_2\le V$ y $V_1^*,V_2^*\le V^*$, entonces
\[
\left(V_1+V_2\right)^0=V_1^0\cap V_2^0,\textrm{ y } \left(V_1^*+V_2^*\right)_0=\left(V_1^*\right)_0\cap \left(V_2^*\right)_0;
\]
\item si $V$ tiene dimensi\'on finita,
\[
\dim\left(\langle S\rangle\right)+\dim(S^0)=\dim(V),\textrm{ y } \dim(L_0)+\dim\left(\langle L\rangle\right)=\dim(V^*).
\]
\end{enumerate}
\end{pro}

\dem
\begin{enumerate}
\item Si $\lambda_1,\lambda_2\in S^0$ y $c\in K$, entonces para todo $v\in S$
\[
(\lambda_1+\lambda_2)(v)=\lambda_1(v)+\lambda_2(v)=0,\textrm{ y } (c\lambda_1)(v)=c\lambda_1(v)=0, 
\]
es decir $\lambda_1+\lambda_2\in S^0$ y $c\lambda_1\in S^0$. Luego $S_0$ es un subespacio de $V^*$. Similarmente $L_0$ es un subespacio de $V$.
\item Sea $\lambda\in S_2^0$, luego, si $v\in S_1$, como $v\in S_2$, entonces $\lambda(v)=0$; en particular $\lambda\in S_1^0$. Similarmente, sea $v\in\left(L_2\right)^0$, luego, si $\lambda\in L_1$, como $\lambda\in L_2$, entonces $\lambda(v)=0$; en particular $v\in\left(L_1\right)^0$.
\item Sea $v\in\langle S\rangle$, entonces existen $v_1,\ldots,v_m\in S$ y $a_1,\ldots,a_m\in K$ tales que
\[
v=a_1v_1+\ldots+a_mv_m
\]
as\'i, si $\lambda\in S^0$, $\lambda(v)=a_1\lambda(v_1)+\ldots+a_m\lambda(v_m)=0$. Luego $\langle S\rangle\le \left(S^0\right)_0$.\\
Tome ahora un subconjunto $S'\subseteq S$ linealmente independiente tal que $\langle S'\rangle=\langle S\rangle$, el cual extendemos a una base $\mathcal{B}=\{v_i\}_{i\in I}$ de $V$. Defina, para cada $i\in I$, $\lambda_i\in V^*$ por
\[
\lambda_i(v_j)=\delta_{ij}
\]
para todo $j\in I$. De esta forma $\lambda_i\in S^0$ si y solo si $v_i\not\in S'$. Sea $J\subset I$ la subcolecci\'on de indices definida por $j\in J$ si $v_j\in S'$. Entonces $L=\{\lambda_i\}_{i\in I\setminus J}\subseteq S^0$ y $\left(S^0\right)_0\le L_0$. Ahora si $v\in V$, como
\[
v=\sum_{i\in J}\lambda_i(v)v_i+\sum_{i\in I\setminus J}\lambda_i(v)v_i
\] 
entonces $v\in L_0$ si y solo si $v\in\langle S'\rangle$, es decir $L_0=\langle S'\rangle$. Luego $\left(S^0\right)_0\le\langle S'\rangle=\langle S\rangle$.
\item Suponga que $\lambda\in (V_1+V_2)^0$, luego, si $v\in V_i$, con $i=1$ \'o $i=2$, entonces $v\in V_1+V_2$ y $\lambda(v)=0$, en particular $\lambda\in V_1^0\cap V_2^0$. Rec\'iprocamente, si $\lambda\in V_1^0\cap V_2^0$ y $v\in V_1+V_2$, con $v=v_1+v_2$, $v_1\in V_1$ y $v_2\in V_2$, $\lambda(v)=\lambda(v_1)+\lambda(v_2)=0$, en particular $\lambda\in (V_1+V_2)^0$.\\
Similarmente, suponga que $v\in\left(V_1^*+V_2^*\right)_0$, luego, si $\lambda\in V_i^*$, con $i=1$ \'o $i=2$, entonces $\lambda\in V_1^*+V_2^*$ y $\lambda(v)=0$, en particular $v\in \left(V_1^*\right)_0\cap \left(V_2^*\right)_0$. Rec\'iprocamente, si $v\in \left(V_1^*\right)_0\cap \left(V_2^*\right)_0$ y $\lambda\in V_1^*+V_2^*$, con $\lambda=\lambda_1+\lambda_2$, $\lambda_1\in V_1$ y $\lambda_2\in V_2$, $\lambda(v)=\lambda_1(v)+\lambda_2(v)=0$, en particular $v\in \left(V_1^*+V_2^*\right)_0$.
\item Tome un subconjunto $\{v_1,\ldots,v_k\}=S'\subseteq S$ linealmente independiente tal que $\langle S'\rangle=\langle S\rangle$, el cual extendemos a una base $\mathcal{B}=\{v_1\ldots,v_n\}$ de $V$. Sea $\mathcal{B}^*=\{\lambda_1\ldots,\lambda_n\}$ la base dual a $\mathcal{B}$. Defina $f\in\Hom_K(V,V)$ por
\[
f(v)=\sum_{i=1}^k\lambda_i(v)v_i.
\]
Por construcci\'on, $\im(f)=\langle S'\rangle=\langle S\rangle$ y $\ker(f)=\langle\lambda_{k+1},\ldots,\lambda_n\rangle_0$. Pero $\langle\lambda_{k+1}\ldots\lambda_{n}\rangle=S^0$, luego $\dim\left(\langle S\rangle\right)+\dim(S^0)=n$.\\
Similarmente, tome un subconjunto $\{\lambda_1,\ldots,\lambda_k\}=L'\subseteq L$ linealmente independiente tal que $\langle L'\rangle=\langle L\rangle$, el cual extendemos a una base $\mathcal{B}^*=\{\lambda_1,\ldots,\lambda_n\}$ de $V$. Sea $\{v_1,\ldots,v_n\}$ una base de $V$. Defina $f\in\Hom_K(V,V)$ por
\[
f(v)=\sum_{i=1}^k\lambda_i(v)v_i.
\]
Por construcci\'on, $\ker(f)=L'_0=L_0$ y $\im(f)=\langle v_1,\ldots,v_k\rangle$. Luego $\dim\im(f)=\dim\left(\langle L\rangle\right)$ y $\dim(L_0)+\dim\left(\langle L\rangle\right)=n$.\qed
\end{enumerate}

\begin{prop}
Sea $V_1\le V$ entonces $\left(V/V_1\right)^*=V_1^0$.
\end{prop}

\dem Tome $\pi_{V_1}:V\rightarrow V/V_1$ con $\pi_{V_1}(v)=v+V_1$; y, defina la transformaci\'on lineal $f:\left(V/V_1\right)^*\rightarrow V^*$ por $f(\lambda)=\lambda\circ\pi_{V_1}$. Veamos que $f$ es un isomorfismo. Primero es inyectiva pues si $\lambda\circ\pi_{V_1}=0$, entonces, para todo $v+V_1\in V/V_1$, $\lambda(v+V_1)=\lambda\circ\pi_{V_1}(v)=0$. Es decir $\lambda=0$. Por otro lado $f$ es sobreyectiva, pues dado $\mu\in V_1^0$, si $v-v'\in V_1$ entonces $\mu(v)-\mu(v')=\mu(v-v')=0$, luego la funci\'on $\lambda:V/V_1\rightarrow K$ tal que $\lambda(v+V_1)=\mu(v)$ es un funcional lineal tal que $f(\lambda)=\mu$. \qed

\begin{teo}\label{dualdual}
Existe una transformaci\'on lineal can\'onica inyectiva
\begin{eqnarray*}
\widehat{\bullet}: V & \longrightarrow & \left(V^*\right)^*\\
                            v &\longmapsto &\widehat{v}:\lambda\mapsto\lambda(v).
\end{eqnarray*}
Si $V$ tiene dimensi\'on finita, $\widehat{\bullet}$ es un isomorfismo.
\end{teo}

\dem Por definici\'on $\widehat{\bullet}$ es lineal, pues 
\[
\widehat{v_1+v_2}(\lambda)=\lambda(v_1+v_2)=\lambda(v_1)+\lambda(v_2)=\left(\widehat{v_1}+\widehat{v_2}\right)(\lambda).
\]
Ahora sea $\{v_i\}_{i\in I}$ una base de $V$, y $\{\lambda_i\}_{i\in I}\subseteq V^*$ la colecci\'on tal que $\lambda_i(v_j)=\delta_{ij}$. Como para todo $v\in V$, $v=\sum_{i\in I}\lambda_i(v)v_i$, si $\widehat{v}=0$ entonces $\lambda_i(v)=0$ para todo $i$ y as\'i $v=0$. Luego $\widehat{\bullet}$ es inyectiva.\\
Si $V$ tiene dimensi\'on finita, $\dim(V)=\dim(V^*)=\dim\left(\left(V^*\right)^*\right)$, entonces $\widehat{\bullet}$ es un isomorfismo.\qed

\section{Transformaci\'on dual}

\begin{defn}
Sea $f\in\Hom_K(V,W)$. Definimos la \emph{transformaci\'on dual}, $f^*\in\Hom_K(W^*,V^*)$, por
\[
f^*(\lambda)=\lambda\circ f
\]
para todo $\lambda\in W^*$.
\end{defn}

\begin{figure}[!hbp]
\centering
\begin{tikzpicture}[auto, node distance=2cm,>=latex']
    \node (V) {$V$};
    \node (W) [right of=V] {$W$};
    \node (K) [below of=W] {$K$};
    
    \path[->] (V) edge node {$f$} (W);
    \path[->] (W) edge  node {$\lambda$} (K);
    \path[dashed,<-] (K) edge  node {$f^*(\lambda)$} (V);
\end{tikzpicture}
\caption{Transformaci\'on dual}
\label{trdual}
\end{figure}

\begin{obs}
La linearidad del mapa $f^*$ se sigue de las siguientes igualdades, validas para todo $f\in\Hom_K(V,W)$, $\lambda_1,\lambda_2\in W^*$, $c\in K$:
\begin{eqnarray*}
(\lambda_1+\lambda_2)\circ f & = & \lambda_1\circ f+\lambda_2\circ f\\
(c\lambda_1)\circ f & = & c(\lambda_1\circ f)
\end{eqnarray*}
\end{obs}

\begin{pro}
Sean $U$ un espacio vectorial sobre $K$ y $f\in\Hom_K(V,W)$ y $g\in\Hom_K(W,U)$, entonces
\[
(g\circ f)^*=f^*\circ g^*
\]
\end{pro}

\dem Para $\lambda\in U^*$, tenemos
\[
(g\circ f)^*\lambda=\lambda\circ (g\circ f)=(\lambda\circ g)\circ f=g^*(\lambda)\circ f=f^*\circ g^* (\lambda)
\]
\qed

\begin{pro}
Sea $f\in\Hom_K(V,W)$, entonces
\begin{enumerate}
\item Si $f$ es sobreyectiva, $f^*$ es inyectiva; y,
\item Si $f$ es inyectiva, $f^*$ es sobreyectiva. 
\end{enumerate}
\end{pro}

\dem
\begin{enumerate}
\item Sea $\lambda\in W^*$ tal que $f^*(\lambda)=0$, entonces, dado $w\in W$, como $f$ es sobreyectiva, existe $v\in V$ tal que $w=f(v)$, as\'i
\[
\lambda(w)=\lambda\left(f(v)\right)=f^*(\lambda)(v)=0
\]
luego $\lambda=0$ y $f^*$ es inyectiva.
\item Sea $W_1,W_2\le W$ tales que $W=W_1\oplus W_2$ y $W_1=f(V)$. Tome $\mu\in V$, y  defina $\lambda: W \rightarrow K$ por
\[
\lambda(w)=\mu(v_1)
\]
donde $w=w_1+w_2$, $w_1\in W_1$ y $w_2\in W_2$ y $f(v_1)=w_1$. Como $f$ es inyectiva, $v_1$ es \'unico, y la funci\'on $\lambda$ est\'a bien definida. Como la descomposici\'on de $w=w_1+w_2$ es lineal y $\mu$ y $f$ son lineales, entonces $\lambda$ lineal, es decir $\lambda\in W^*$. Por contrucci\'on $\mu=f^*\lambda$ pues
\[
f^*(\lambda)(v_1)=\lambda(f(v_1))=\lambda(w_1)=\mu(v_1),
\]
luego $f^*$ es sobreyectiva.\qed
\end{enumerate}

\begin{pro}
Sean $V_1,V_2\le V$ tales que $V=V_1\oplus V_2$; y, $\pi_1:V\rightarrow  V_1$ y $\pi_2:V\rightarrow V_2$ respectivamente las proyecciones sobre $V_1$ y $V_2$ dadas por la descomposici\'on $V=V_1\oplus V_2$. Entonces
\[
V^*=\pi_1^*(V^*_1)\oplus\pi_2^*(V^*_2)
\]
\end{pro}

\dem Dado $\lambda\in V^*$, defina $\lambda_1\in V_1^*$ y $\lambda_2\in V_2$ por
\[
\lambda_1(v_1)=\lambda(v_1)\qquad\lambda_2(v_2)=\lambda(v_2).
\]
De tal forma que si $v=v_1+v_2\in V$ con $v_1=\pi_1(v)$ y $v_2=\pi_2(v)$, entonces
\begin{eqnarray*}
\lambda(v) & = & \lambda(v_1)+\lambda(v_2)\\
 & = & \lambda_1\left(\pi_1(v)\right)+\lambda_2\left(\pi_2(v)\right)\\
 & = & \left(\pi_1^*(\lambda_1)+\pi_2^*(\lambda_2)\right)(v)
\end{eqnarray*}  
luego $V^*=\pi_1^*(V_1)+\pi_2^*(V_2)$. Ahora, si $\lambda\in \pi_1^*(V_1)\cap\pi_2^*(V_2)$, existen $\lambda_1\in V_1^*$ y $\lambda_2\in V_2^*$ tales que $\lambda=\pi_1^*(\lambda_1)=\pi_2^*(\lambda_2)$, de esta forma, para todo $v=v_1+v_2\in V$ con $v_1=\pi_1(v)$ y $v_2=\pi_2(v)$
\[
\lambda(v)=\lambda(v_1)+\lambda(v_2)=\lambda_2\left(\pi_2(v_1)\right)+\lambda_1\left(\pi_1(v_2)\right)=\lambda_2(0)+\lambda_1(0)=0.
\]
Luego $\pi_1^*(V_1)\cap\pi_2^*(V_2)=\{0\}$.\qed

\begin{teo}
El mapa
\begin{eqnarray*}
\bullet^*:\Hom_K(V,W) & \longrightarrow & \Hom_K(W^*,V^*)\\
f & \longmapsto & f^*
\end{eqnarray*}
es una transformaci\'on lineal inyectiva. Si $W$ tiene dimensi\'on finita entonces es un isomorfismo.
\end{teo}

\dem Sean $f,g\in\Hom_K(V,W)$, y $c\in K$. Dado $\lambda\in W^*$ tenemos:
\begin{eqnarray*}
(f+g)^*(\lambda) & = & \lambda\circ(f+g)\\
   & = & \lambda\circ f+\lambda\circ g\\
   & = & f^*(\lambda)+g^*(\lambda)\\
   & = & (f^*+g^*)(\lambda)\\
(cf)^*(\lambda) & = &  \lambda\circ (cf)\\
   & = & c(\lambda\circ f)\\
   & = & cf^*(\lambda).
\end{eqnarray*}
Es decir $(f+g)^*=f^*+g^*$ y $(cf)^*=cf^*$, y as\'i $\bullet^*$es lineal.\\
Ahora suponga que $f^*=0$, es decir $f^*(\lambda)=0$ para todo $\lambda\in W^*$. Tomamos una base $\{w_i\}_{i\in I}$ de $W$ y definimos $\{\lambda_i\}_{i\in I}\subseteq W^*$ por
\[
\delta_i(v_j)=\delta_{ij}
\]
As\'i, como en Observaci\'on \ref{basedualinfinita}, tenemos para todo $v\in V$
\[
f(v)=\sum_{i\in I}\lambda_i\left(f(v)\right)w_i=\sum_{i\in I} f^*(\lambda_i)(v)w_i=\sum_{i\in I} 0w_i=0.
\]
Es decir $f=0$ y as\'i $\bullet^*$ es inyectiva.\\
Si adem\'as asumimos que $W$ tiene dimensi\'on finita, entonces $\{\lambda_i\}_{i\in I}$ es la base de $W^*$, dual de $\{w_i\}_{i\in I}$. En particular $\phi\in\Hom^*(W^*,V^*)$ est\'a un\'ivocamente determinado por la im\'agen $\{\phi(\lambda_i)\}_{i\in I}\subseteq V^*$. Defina $f\in\Hom_K(V,W)$ por la suma finita
\[
f(v)=\sum_{j\in I}\left[\phi(\lambda_j)(v)\right]w_j,
\]
de forma tal que para todo $i\in I$, $v\in V$,
\begin{eqnarray*}
f^*(\lambda_i)(v) & = & \lambda_i(f(v))\\
 & = & \lambda_i\left(\sum_{j\in I}\left[\phi(\lambda_j)(v)\right]w_j\right)\\
 & = & \sum_{j\in I}\left[\phi(\lambda_j)(v)\right]\lambda_i(w_j)\\
 & = & \sum_{i\in I}\left[\phi(\lambda_j)(v)\right]\delta_{ij}=\phi(\lambda_i)(v).
\end{eqnarray*}
Es decir $f^*(\lambda_i)=\phi(\lambda_i)$, para todo $i\in I$. Luego $f^*=\phi$, de donde $\bullet^*$ es tambi\'en sobreyectiva, as\'i es un isomorfismo.\qed

\begin{obs}
Suponga que $W$ tiene dimensi\'on finita, sea $\mathcal{B}_W=\{w_1,\ldots,w_m\}$ una base de $W$ y $\lambda\in W^*=\Hom_K(W,K)$. Si tomamos la base $\{1\}$ de $K$, entonces la representaci\'on matricial de $\lambda$ respecto a las base $\mathcal{B}_W$ y $\{1\}$ es
\[
\Big[\lambda\Big]^{\{1\}}_{\mathcal{B}_W}=\Big[\lambda(w_1)\cdots\lambda(w_m)\Big].
\]
Ahora, si $\mathcal{B}_V$ es una base de $V$ y $f\in\Hom_K(V,W)$, tenemos
\[
\Big[f^*(\lambda)\Big]^{\{1\}}_{\mathcal{B}_V}=\Big[\lambda\circ f\Big]^{\{1\}}_{\mathcal{B}_V}=\Big[\lambda\Big]^{\{1\}}_S\Big[f\Big]^{\mathcal{B}_V}_{\mathcal{B}_W}.
\]
Por otro lado, podemos tomar las coordenadas de $\lambda$ en la base $\mathcal{B}_W^*=\{\lambda_1,\ldots,\lambda_m\}$ de $W^*$ dual de $\mathcal{B}_W$:
\[
\Big[\lambda\Big]^{\mathcal{B}_W^*}=\left[\begin{array}{c} \lambda(w_1)\\ \vdots \\ \lambda(w_m)\end{array}\right]
\]
Si ademas asumimos que $V$ tiene tambi\'en dimensi\'on finita y tomamos la base $\mathcal{B}_V^*$  de $V^*$ dual de $T$, entonces
\[
\Big[f^*(\lambda)\Big]^{\mathcal{B}_V^*}=\Big[f^*\Big]^{\mathcal{B}_V^*}_{\mathcal{B}_W^*}\Big[\lambda\Big]^{\mathcal{B}_W^*}.
\]
La pregunta inmediata es: ?`Cu\'al es la relaci\'on entre $\Big[f\Big]^{\mathcal{B}_W}_{\mathcal{B}_V}$ y $\Big[f^*\Big]^{\mathcal{B}_V^*}_{\mathcal{B}_W^*}$?
\end{obs}

\begin{defn}
Sean $I,J$ conjuntos y $A\in M_{I\times J}(K)$, definimos la \emph{matriz traspuesta} de $A$ por $A^\intercal\in M_{J\times I}(K)$ tal que
\[
A^\intercal(j,i)=A(i,j)
\]
para todo $(j,i)\in J\times I$. Es decir el valor en $(j,i)$ de $A^\intercal$ es el valor en $(i,j)$ de $A$. Similarmente si $m,n\in\mathbb{Z}_{>0}$, y $A\in M_{m\times n}(K)$, definimos su traspuesta por $A^\intercal\in M_{n\times m}(K)$ tal que
\[
A^\intercal(j,i)=A(i,j)
\]
Sea $A\in M_{I\times I}(K)$, o $A\in M_{n\times n}(K)$, decimos que $A$ es \emph{sim\'etrica} si $A^\intercal=A$.
\end{defn}

\begin{teo}
Suponga que $V$ y $W$ tienen dimensi\'on finita, $n=\dim(V)>0$ y $m=\dim(W)>0$, y $f\in\Hom_K(V,W)$. Sean $\mathcal{B}_W=\{w_1,\ldots,w_m\}$ y $\mathcal{B}_V=\{v_1,\ldots,v_n\}$ respectivamente bases de $W$ y $V$; y, sean $\mathcal{B}_W^*=\{\lambda_1,\ldots,\lambda_m\}$ y $\mathcal{B}_V^*=\{\mu_1,\ldots,\mu_n\}$ respectivamente las bases de $W^*$ y $V^*$ duales de $S$ y $T$. Sea $A\in M_{m\times n}(K)$ la representaci\'on matricial de $f$ respecto a las bases $\mathcal{B}_V$ y $\mathcal{B}_W$, entonces $A^\intercal\in M_{n\times m}(K)$ es la representaci\'on matricial de $f^*$ respecto a las bases $\mathcal{B}_V^*$ y $\mathcal{B}_W^*$. Es decir,
\[
\Big[f^*\Big]^{\mathcal{B}_V^*}_{\mathcal{B}_W^*}=\left(\Big[f\Big]^{\mathcal{B}_W}_{\mathcal{B}_V}\right)^\intercal 
\] 
\end{teo}

\dem Si $a_{ij}\in K$ es la $ij$-\'esima entrada de $A=\Big[f\Big]^{\mathcal{B}_W}_{\mathcal{B}_V}$, entonces
\[
a_{ij}=\lambda_i\left(f(v_j)\right)=f^*(\lambda_i)(v_j);
\]
y,
\[
f^*(\lambda_i)=\sum_{l=1}^{n}\left[f^*(\lambda_i)(v_l)\right]\mu_l,
\]
luego la $ji$-esima entrada de $\Big[f^*\Big]^{\mathcal{B}_V^*}_{\mathcal{B}_W^*}$ es $f^*(\lambda_i)(v_j)=a_{ij}$.\qed